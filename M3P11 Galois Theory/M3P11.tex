\def\module{M3P11 Galois Theory}
\def\lecturer{Prof Alessio Corti}
\def\term{Spring 2019}

\def\thm{section}

\documentclass{article}

% Packages

\usepackage{amssymb}
\usepackage{amsthm}
\usepackage[UKenglish]{babel}
\usepackage{commath}
\usepackage{enumitem}
\usepackage{etoolbox}
\usepackage{fancyhdr}
\usepackage[margin=1in]{geometry}
\usepackage{graphicx}
\usepackage[hidelinks]{hyperref}
\usepackage[utf8]{inputenc}
\usepackage{listings}
\usepackage{mathtools}
\usepackage{stmaryrd}
\usepackage{tikz-cd}
\usepackage{csquotes}

% Formatting

\addto\captionsUKenglish{\renewcommand{\abstractname}{Syllabus}}
\delimitershortfall5pt
\ifx\thm\undefined\newtheorem{n}{}\else\newtheorem{n}{}[\thm]\fi
\newcommand\newoperator[1]{\ifcsdef{#1}{\cslet{#1}{\relax}}{}\csdef{#1}{\operatorname{#1}}}
\setlength{\parindent}{0cm}

% Environments

\theoremstyle{plain}
\newtheorem{algorithm}[n]{Algorithm}
\newtheorem*{algorithm*}{Algorithm}
\newtheorem{algorithm**}{Algorithm}
\newtheorem{conjecture}[n]{Conjecture}
\newtheorem*{conjecture*}{Conjecture}
\newtheorem{conjecture**}{Conjecture}
\newtheorem{corollary}[n]{Corollary}
\newtheorem*{corollary*}{Corollary}
\newtheorem{corollary**}{Corollary}
\newtheorem{lemma}[n]{Lemma}
\newtheorem*{lemma*}{Lemma}
\newtheorem{lemma**}{Lemma}
\newtheorem{proposition}[n]{Proposition}
\newtheorem*{proposition*}{Proposition}
\newtheorem{proposition**}{Proposition}
\newtheorem{theorem}[n]{Theorem}
\newtheorem*{theorem*}{Theorem}
\newtheorem{theorem**}{Theorem}

\theoremstyle{definition}
\newtheorem{aim}[n]{Aim}
\newtheorem*{aim*}{Aim}
\newtheorem{aim**}{Aim}
\newtheorem{axiom}[n]{Axiom}
\newtheorem*{axiom*}{Axiom}
\newtheorem{axiom**}{Axiom}
\newtheorem{condition}[n]{Condition}
\newtheorem*{condition*}{Condition}
\newtheorem{condition**}{Condition}
\newtheorem{definition}[n]{Definition}
\newtheorem*{definition*}{Definition}
\newtheorem{definition**}{Definition}
\newtheorem{example}[n]{Example}
\newtheorem*{example*}{Example}
\newtheorem{example**}{Example}
\newtheorem{exercise}[n]{Exercise}
\newtheorem*{exercise*}{Exercise}
\newtheorem{exercise**}{Exercise}
\newtheorem{fact}[n]{Fact}
\newtheorem*{fact*}{Fact}
\newtheorem{fact**}{Fact}
\newtheorem{goal}[n]{Goal}
\newtheorem*{goal*}{Goal}
\newtheorem{goal**}{Goal}
\newtheorem{law}[n]{Law}
\newtheorem*{law*}{Law}
\newtheorem{law**}{Law}
\newtheorem{plan}[n]{Plan}
\newtheorem*{plan*}{Plan}
\newtheorem{plan**}{Plan}
\newtheorem{problem}[n]{Problem}
\newtheorem*{problem*}{Problem}
\newtheorem{problem**}{Problem}
\newtheorem{question}[n]{Question}
\newtheorem*{question*}{Question}
\newtheorem{question**}{Question}
\newtheorem{warning}[n]{Warning}
\newtheorem*{warning*}{Warning}
\newtheorem{warning**}{Warning}
\newtheorem{acknowledgements}[n]{Acknowledgements}
\newtheorem*{acknowledgements*}{Acknowledgements}
\newtheorem{acknowledgements**}{Acknowledgements}
\newtheorem{annotations}[n]{Annotations}
\newtheorem*{annotations*}{Annotations}
\newtheorem{annotations**}{Annotations}
\newtheorem{assumption}[n]{Assumption}
\newtheorem*{assumption*}{Assumption}
\newtheorem{assumption**}{Assumption}
\newtheorem{conclusion}[n]{Conclusion}
\newtheorem*{conclusion*}{Conclusion}
\newtheorem{conclusion**}{Conclusion}
\newtheorem{claim}[n]{Claim}
\newtheorem*{claim*}{Claim}
\newtheorem{claim**}{Claim}
\newtheorem{notation}[n]{Notation}
\newtheorem*{notation*}{Notation}
\newtheorem{notation**}{Notation}
\newtheorem{note}[n]{Note}
\newtheorem*{note*}{Note}
\newtheorem{note**}{Note}
\newtheorem{remark}[n]{Remark}
\newtheorem*{remark*}{Remark}
\newtheorem{remark**}{Remark}

% Lectures

\newcommand{\lecture}[3]{ % Lecture
  \marginpar{
    Lecture #1 \\
    #2 \\
    #3
  }
}

% Blackboard

\renewcommand{\AA}{\mathbb{A}} % Blackboard A
\newcommand{\BB}{\mathbb{B}}   % Blackboard B
\newcommand{\CC}{\mathbb{C}}   % Blackboard C
\newcommand{\DD}{\mathbb{D}}   % Blackboard D
\newcommand{\EE}{\mathbb{E}}   % Blackboard E
\newcommand{\FF}{\mathbb{F}}   % Blackboard F
\newcommand{\GG}{\mathbb{G}}   % Blackboard G
\newcommand{\HH}{\mathbb{H}}   % Blackboard H
\newcommand{\II}{\mathbb{I}}   % Blackboard I
\newcommand{\JJ}{\mathbb{J}}   % Blackboard J
\newcommand{\KK}{\mathbb{K}}   % Blackboard K
\newcommand{\LL}{\mathbb{L}}   % Blackboard L
\newcommand{\MM}{\mathbb{M}}   % Blackboard M
\newcommand{\NN}{\mathbb{N}}   % Blackboard N
\newcommand{\OO}{\mathbb{O}}   % Blackboard O
\newcommand{\PP}{\mathbb{P}}   % Blackboard P
\newcommand{\QQ}{\mathbb{Q}}   % Blackboard Q
\newcommand{\RR}{\mathbb{R}}   % Blackboard R
\renewcommand{\SS}{\mathbb{S}} % Blackboard S
\newcommand{\TT}{\mathbb{T}}   % Blackboard T
\newcommand{\UU}{\mathbb{U}}   % Blackboard U
\newcommand{\VV}{\mathbb{V}}   % Blackboard V
\newcommand{\WW}{\mathbb{W}}   % Blackboard W
\newcommand{\XX}{\mathbb{X}}   % Blackboard X
\newcommand{\YY}{\mathbb{Y}}   % Blackboard Y
\newcommand{\ZZ}{\mathbb{Z}}   % Blackboard Z

% Brackets

\renewcommand{\eval}[1]{\left. #1 \right|}          % Evaluation
\newcommand{\br}{\del}                              % Brackets
\newcommand{\abr}[1]{\left\langle #1 \right\rangle} % Angle brackets
\newcommand{\fbr}[1]{\left\lfloor #1 \right\rfloor} % Floor brackets
\newcommand{\lbr}[1]{\left\lfloor #1 \right\rfloor} % Ceiling brackets
\newcommand{\st}{\ \middle| \ }                     % Such that

% Calligraphic

\newcommand{\AAA}{\mathcal{A}} % Calligraphic A
\newcommand{\BBB}{\mathcal{B}} % Calligraphic B
\newcommand{\CCC}{\mathcal{C}} % Calligraphic C
\newcommand{\DDD}{\mathcal{D}} % Calligraphic D
\newcommand{\EEE}{\mathcal{E}} % Calligraphic E
\newcommand{\FFF}{\mathcal{F}} % Calligraphic F
\newcommand{\GGG}{\mathcal{G}} % Calligraphic G
\newcommand{\HHH}{\mathcal{H}} % Calligraphic H
\newcommand{\III}{\mathcal{I}} % Calligraphic I
\newcommand{\JJJ}{\mathcal{J}} % Calligraphic J
\newcommand{\KKK}{\mathcal{K}} % Calligraphic K
\newcommand{\LLL}{\mathcal{L}} % Calligraphic L
\newcommand{\MMM}{\mathcal{M}} % Calligraphic M
\newcommand{\NNN}{\mathcal{N}} % Calligraphic N
\newcommand{\OOO}{\mathcal{O}} % Calligraphic O
\newcommand{\PPP}{\mathcal{P}} % Calligraphic P
\newcommand{\QQQ}{\mathcal{Q}} % Calligraphic Q
\newcommand{\RRR}{\mathcal{R}} % Calligraphic R
\newcommand{\SSS}{\mathcal{S}} % Calligraphic S
\newcommand{\TTT}{\mathcal{T}} % Calligraphic T
\newcommand{\UUU}{\mathcal{U}} % Calligraphic U
\newcommand{\VVV}{\mathcal{V}} % Calligraphic V
\newcommand{\WWW}{\mathcal{W}} % Calligraphic W
\newcommand{\XXX}{\mathcal{X}} % Calligraphic X
\newcommand{\YYY}{\mathcal{Y}} % Calligraphic Y
\newcommand{\ZZZ}{\mathcal{Z}} % Calligraphic Z

% Fraktur

\newcommand{\aaa}{\mathfrak{a}}   % Fraktur a
\newcommand{\bbb}{\mathfrak{b}}   % Fraktur b
\newcommand{\ccc}{\mathfrak{c}}   % Fraktur c
\newcommand{\ddd}{\mathfrak{d}}   % Fraktur d
\newcommand{\eee}{\mathfrak{e}}   % Fraktur e
\newcommand{\fff}{\mathfrak{f}}   % Fraktur f
\renewcommand{\ggg}{\mathfrak{g}} % Fraktur g
\newcommand{\hhh}{\mathfrak{h}}   % Fraktur h
\newcommand{\iii}{\mathfrak{i}}   % Fraktur i
\newcommand{\jjj}{\mathfrak{j}}   % Fraktur j
\newcommand{\kkk}{\mathfrak{k}}   % Fraktur k
\renewcommand{\lll}{\mathfrak{l}} % Fraktur l
\newcommand{\mmm}{\mathfrak{m}}   % Fraktur m
\newcommand{\nnn}{\mathfrak{n}}   % Fraktur n
\newcommand{\ooo}{\mathfrak{o}}   % Fraktur o
\newcommand{\ppp}{\mathfrak{p}}   % Fraktur p
\newcommand{\qqq}{\mathfrak{q}}   % Fraktur q
\newcommand{\rrr}{\mathfrak{r}}   % Fraktur r
\newcommand{\sss}{\mathfrak{s}}   % Fraktur s
\newcommand{\ttt}{\mathfrak{t}}   % Fraktur t
\newcommand{\uuu}{\mathfrak{u}}   % Fraktur u
\newcommand{\vvv}{\mathfrak{v}}   % Fraktur v
\newcommand{\www}{\mathfrak{w}}   % Fraktur w
\newcommand{\xxx}{\mathfrak{x}}   % Fraktur x
\newcommand{\yyy}{\mathfrak{y}}   % Fraktur y
\newcommand{\zzz}{\mathfrak{z}}   % Fraktur z

% Geometry

\newcommand{\CP}{\mathbb{CP}}                                              % Complex projective space
\newcommand{\iintd}[4]{\iint_{#1} \, #2 \, \dif #3 \, \dif #4}             % Double integral
\newcommand{\RP}{\mathbb{RP}}                                              % Real projective space
\newcommand{\intd}[4]{\int_{#1}^{#2} \, #3 \, \dif #4}                     % Single integral
\newcommand{\iiintd}[5]{\iint_{#1} \, #2 \, \dif #3 \, \dif #4 \, \dif #5} % Triple integral

% Logic

\newcommand{\iffb}[2]{\br{#1 \leftrightarrow #2}} % Biconditional
\newcommand{\andb}[2]{\br{#1 \land #2}}           % Conjunction
\newcommand{\orb}[2]{\br{#1 \lor #2}}             % Disjunction
\newcommand{\nib}[2]{\br{#1 \notin #2}}           % Element of
\newcommand{\eqb}[2]{\br{#1 = #2}}                % Equal to
\newcommand{\teb}[1]{\br{\exists #1}}             % Existential quantifier
\newcommand{\impb}[2]{\br{#1 \rightarrow #2}}     % Implication
\newcommand{\ltb}[2]{\br{#1 < #2}}                % Less than
\newcommand{\leb}[2]{\br{#1 \le #2}}              % Less than or equal to
\newcommand{\notb}[1]{\br{\neg #1}}               % Negation
\newcommand{\inb}[2]{\br{#1 \in #2}}              % Not element of
\newcommand{\neb}[2]{\br{#1 \ne #2}}              % Not equal to
\newcommand{\subb}[2]{\br{#1 \subseteq #2}}       % Subset
\newcommand{\fab}[1]{\br{\forall #1}}             % Universal quantifier

% Maps

\newcommand{\bijection}[7][]{    % Bijection
  \ifx &#1&
    \begin{array}{rcl}
      #2 & \longleftrightarrow & #3 \\
      #4 & \longmapsto         & #5 \\
      #6 & \longmapsfrom       & #7
    \end{array}
  \else
    \begin{array}{ccrcl}
      #1 & : & #2 & \longrightarrow & #3 \\
         &   & #4 & \longmapsto     & #5 \\
         &   & #6 & \longmapsfrom   & #7
    \end{array}
  \fi
}
\newcommand{\correspondence}[2]{ % Correspondence
  \cbr{
    \begin{array}{c}
      #1
    \end{array}
  }
  \qquad
  \leftrightsquigarrow
  \qquad
  \cbr{
    \begin{array}{c}
      #2
    \end{array}
  }
}
\newcommand{\function}[5][]{     % Function
  \ifx &#1&
    \begin{array}{rcl}
      #2 & \longrightarrow & #3 \\
      #4 & \longmapsto     & #5
    \end{array}
  \else
    \begin{array}{ccrcl}
      #1 & : & #2 & \longrightarrow & #3 \\
         &   & #4 & \longmapsto     & #5
    \end{array}
  \fi
}
\newcommand{\functions}[7][]{    % Functions
  \ifx &#1&
    \begin{array}{rcl}
      #2 & \longrightarrow & #3 \\
      #4 & \longmapsto     & #5 \\
      #6 & \longmapsto     & #7
    \end{array}
  \else
    \begin{array}{ccrcl}
      #1 & : & #2 & \longrightarrow & #3 \\
         &   & #4 & \longmapsto     & #5 \\
         &   & #6 & \longmapsto     & #7
    \end{array}
  \fi
}

% Matrices

\newcommand{\onebytwo}[2]{      % One by two matrix
  \begin{pmatrix}
    #1 & #2
  \end{pmatrix}
}
\newcommand{\onebythree}[3]{    % One by three matrix
  \begin{pmatrix}
    #1 & #2 & #3
  \end{pmatrix}
}
\newcommand{\twobyone}[2]{      % Two by one matrix
  \begin{pmatrix}
    #1 \\
    #2
  \end{pmatrix}
}
\newcommand{\twobytwo}[4]{      % Two by two matrix
  \begin{pmatrix}
    #1 & #2 \\
    #3 & #4
  \end{pmatrix}
}
\newcommand{\threebyone}[3]{    % Three by one matrix
  \begin{pmatrix}
    #1 \\
    #2 \\
    #3
  \end{pmatrix}
}
\newcommand{\threebythree}[9]{  % Three by three matrix
  \begin{pmatrix}
    #1 & #2 & #3 \\
    #4 & #5 & #6 \\
    #7 & #8 & #9
  \end{pmatrix}
}
\newcommand{\twobytwosmall}[4]{ % Two by two small matrix
  \begin{psmallmatrix}
    #1 & #2 \\
    #3 & #4
  \end{psmallmatrix}
}

% Number theory

\renewcommand{\symbol}[2]{\br{\tfrac{#1}{#2}}} % Power residue symbol
\newcommand{\unit}[1]{\br{\ZZ / #1\ZZ}^\times} % Unit group

% Operators

\newoperator{ab}    % Abelian
\newoperator{AG}    % Affine geometry
\newoperator{alg}   % Algebraic
\newoperator{Ann}   % Annihilator
\newoperator{area}  % Area
\newoperator{Aut}   % Automorphism
\newoperator{card}  % Cardinality
\newoperator{ch}    % Characteristic
\newoperator{Cl}    % Class
\newoperator{col}   % Column
\newoperator{Corr}  % Correspondence
\newoperator{diam}  % Diameter
\newoperator{Disc}  % Discriminant
\newoperator{dom}   % Domain
\newoperator{Em}    % Embedding
\newoperator{End}   % Endomorphism
\newoperator{fin}   % Finite
\newoperator{Fix}   % Fixed
\newoperator{Frac}  % Fraction
\newoperator{Frob}  % Frobenius
\newoperator{Fun}   % Function
\newoperator{Gal}   % Galois
\newoperator{GL}    % General linear
\newoperator{Ham}   % Hamming
\newoperator{Homeo} % Homeomorphism
\newoperator{Hom}   % Homomorphism
\newoperator{id}    % Identity
\newoperator{Im}    % Image
\newoperator{Ind}   % Index
\newoperator{Ker}   % Kernel
\newoperator{lcm}   % Least common multiple
\newoperator{Mat}   % Matrix
\newoperator{mult}  % Multiplicity
\newoperator{new}   % New
\newoperator{Nm}    % Norm
\newoperator{old}   % Old
\newoperator{ord}   % Order
\newoperator{Pay}   % Payley
\newoperator{PG}    % Projective geometry
\newoperator{PGL}   % Projective general linear
\newoperator{PSL}   % Projective special linear
\newoperator{rad}   % Radical
\newoperator{ran}   % Range
\newoperator{Res}   % Residue
\newoperator{rk}    % Rank
\newoperator{Re}    % Real
\newoperator{row}   % Row
\newoperator{sgn}   % Sign
\newoperator{Sing}  % Singular
\newoperator{sp}    % Span
\newoperator{SL}    % Special linear
\newoperator{SO}    % Special orthogonal
\newoperator{Spec}  % Spectrum
\newoperator{Stab}  % Stabiliser
\newoperator{star}  % Star
\newoperator{srg}   % Strongly regular graph
\newoperator{Sym}   % Symmetric
\newoperator{tors}  % Torsion
\newoperator{Tr}    % Trace
\newoperator{vol}   % Volume
\newoperator{wt}    % Weight

% Roman

\newcommand{\A}{\mathrm{A}}   % Roman A
\newcommand{\B}{\mathrm{B}}   % Roman B
\newcommand{\C}{\mathrm{C}}   % Roman C
\newcommand{\D}{\mathrm{D}}   % Roman D
\newcommand{\E}{\mathrm{E}}   % Roman E
\newcommand{\F}{\mathrm{F}}   % Roman F
\newcommand{\G}{\mathrm{G}}   % Roman G
\renewcommand{\H}{\mathrm{H}} % Roman H
\newcommand{\I}{\mathrm{I}}   % Roman I
\newcommand{\J}{\mathrm{J}}   % Roman J
\newcommand{\K}{\mathrm{K}}   % Roman K
\renewcommand{\L}{\mathrm{L}} % Roman L
\newcommand{\M}{\mathrm{M}}   % Roman M
\newcommand{\N}{\mathrm{N}}   % Roman N
\renewcommand{\O}{\mathrm{O}} % Roman O
\renewcommand{\P}{\mathrm{P}} % Roman P
\newcommand{\Q}{\mathrm{Q}}   % Roman Q
\newcommand{\R}{\mathrm{R}}   % Roman R
\renewcommand{\S}{\mathrm{S}} % Roman S
\newcommand{\T}{\mathrm{T}}   % Roman T
\newcommand{\U}{\mathrm{U}}   % Roman U
\newcommand{\V}{\mathrm{V}}   % Roman V
\newcommand{\W}{\mathrm{W}}   % Roman W
\newcommand{\X}{\mathrm{X}}   % Roman X
\newcommand{\Y}{\mathrm{Y}}   % Roman Y
\newcommand{\Z}{\mathrm{Z}}   % Roman Z

\renewcommand{\a}{\mathrm{a}} % Roman a
\renewcommand{\b}{\mathrm{b}} % Roman b
\renewcommand{\c}{\mathrm{c}} % Roman c
\renewcommand{\d}{\mathrm{d}} % Roman d
\newcommand{\e}{\mathrm{e}}   % Roman e
\newcommand{\f}{\mathrm{f}}   % Roman f
\newcommand{\g}{\mathrm{g}}   % Roman g
\newcommand{\h}{\mathrm{h}}   % Roman h
\renewcommand{\i}{\mathrm{i}} % Roman i
\renewcommand{\j}{\mathrm{j}} % Roman j
\renewcommand{\k}{\mathrm{k}} % Roman k
\renewcommand{\l}{\mathrm{l}} % Roman l
\newcommand{\m}{\mathrm{m}}   % Roman m
\renewcommand{\n}{\mathrm{n}} % Roman n
\renewcommand{\o}{\mathrm{o}} % Roman o
\newcommand{\p}{\mathrm{p}}   % Roman p
\newcommand{\q}{\mathrm{q}}   % Roman q
\renewcommand{\r}{\mathrm{r}} % Roman r
\newcommand{\s}{\mathrm{s}}   % Roman s
\renewcommand{\t}{\mathrm{t}} % Roman t
\renewcommand{\u}{\mathrm{u}} % Roman u
\renewcommand{\v}{\mathrm{v}} % Roman v
\newcommand{\w}{\mathrm{w}}   % Roman w
\newcommand{\x}{\mathrm{x}}   % Roman x
\newcommand{\y}{\mathrm{y}}   % Roman y
\newcommand{\z}{\mathrm{z}}   % Roman z

% Tikz

\tikzset{
  arrow symbol/.style={"#1" description, allow upside down, auto=false, draw=none, sloped},
  subset/.style={arrow symbol={\subset}},
  cong/.style={arrow symbol={\cong}}
}

% Fancy header

\pagestyle{fancy}
\lhead{\module}
\rhead{\nouppercase{\leftmark}}

% Make title

\title{\module}
\author{Lectured by \lecturer \\ Typed by David Kurniadi Angdinata}
\date{\term}

\begin{document}

% Title page

\maketitle

\cover

\vfill

\begin{abstract}
\noindent\syllabus
\end{abstract}

\pagebreak

% Contents page

\tableofcontents

\pagebreak

\setcounter{section}{-1}

\section{What is Galois theory?}

\lecture{1}{Thursday}{10/01/19}

References.
\begin{itemize}
\item E Artin, Galois theory, 1994
\item A Grothendieck and M Raynaud, Rev\^etements \'etales et groupe fondamental, 2002
\item I N Herstein, Topics in algebra, 1975
\item M Reid, Galois theory, 2014
\end{itemize}

\begin{notation*}
If $ K $ is a field, or a ring, I denote
$$ K\sb{x} = \cb{a_0 + \dots + a_nx^n \mid a_i \in K}, $$
the ring of polynomials with coefficients in $ K $.
\end{notation*}

\begin{example*}
\hfill
\begin{itemize}
\item $ \Q, \R, \C $.
\item Quadratic fields
$$ \Q\rb{\sqrt{2}} = \cb{a + b\sqrt{2} \mid a, b \in \Q} = \dfrac{\Q\sb{x}}{\ab{x^2 - 2}}. $$
It is also a field, since
$$ \dfrac{1}{\rb{a + b\sqrt{2}}} = \dfrac{a - b\sqrt{2}}{a^2 - 2b^2}. $$
\item If $ p $ is prime, $ \Z / p\Z = \F_p $ is a finite field. If $ f\rb{x} \in K\sb{x} $ is irreducible, $ K\sb{x} / \ab{f\rb{x}} $ is a field. For example, $ x^2 - 2 $. Both $ \Z $ and $ K\sb{x} $ have a division algorithm. For example, let $ \sb{a} \in \Z / p\Z $ and $ \sb{a} \ne 0 $, that is $ p \mid a $. Since $ p $ is prime, $ \gcd\rb{p, a} = 1 $, so there exist $ x, y \in \Z $ such that $ ax + py = 1 $. Thus $ \sb{a} \cdot \sb{x} = 1 $ in $ \Z / p\Z $.
\item For $ K $ a field, either for all $ m \in \Z $, $ m \ne 0 $ in $ K $, so $ K $ has characteristic $ ch\rb{K} = 0 $, or there exists $ p $ prime such that $ m = 0 $ if and only if $ p \mid m $, so $ K $ has characteristic $ ch\rb{K} = p $.
\item For $ K $ a field,
$$ K\rb{x} = Frac\rb{K\sb{x}} = \cb{\phi\rb{x} = \dfrac{f\rb{x}}{g\rb{x}} \ \Bigg| \ f, g \in K\sb{x}, \ g \ne 0}. $$
is also a field, the field of rational functions with coefficients in $ K $. For example, $ \F_p\rb{x, Y} = \F_p\rb{x}\rb{Y} $.
\end{itemize}
\end{example*}

\begin{example*}
Consider algebraic equations in a field $ K $.
\begin{itemize}
\item Let $ ax^2 + bx^2 + c = 0 $ for $ a, b, c \in K $ be a quadratic. There is a formula
$$ x = \dfrac{-b + \sqrt{b^2 - 4ac}}{2a}. $$
\item For a cubic $ y^3 + 3py + 2q = 0 $,
$$ y = \sqrt[3]{-q + \sqrt{q^2 + p^3}} + \sqrt[3]{-q - \sqrt{q^2 + p^3}}. $$
\item There is a formula for quartic equations.
\item It is a theorem that there can be no such formula for equations of degree at least five.
\end{itemize}
Galois theory deals with these easily.
\end{example*}

\lecture{2}{Friday}{11/01/19}

\begin{definition}
A \textbf{field homomorphism} is a function $ \phi : K_1 \to K_2 $ that preserves the field operations, for all $ a, b \in K_1 $,
\begin{align*}
\phi\rb{a + b} & = \phi\rb{a} + \phi\rb{b}, \\
\phi\rb{ab} & = \phi\rb{a}\phi\rb{b},
\end{align*}
and $ \phi\rb{0_{K_1}} = 0_{K_2} $ and $ \phi\rb{1_{K_1}} = 1_{K_2} $.
\end{definition}

\begin{remark*}
All field homomorphisms are injective. If $ a \in K_1 \setminus \cb{0} $, then there exists $ b \in K_1 $ such that $ ab = 1 $, then $ \phi\rb{a}\phi\rb{b} = 1 $, so $ \phi\rb{a} \ne 0 $. This easily implies $ \phi $ is injective. If $ a_1 \ne a_2 $, then $ a_1 - a_n \ne 0 $, so $ 0 \ne \phi\rb{a_1 - a_2} = \phi\rb{a_1} - \phi\rb{a_2} $. Then $ \phi\rb{a_1} \ne \phi\rb{a_2} $.
\end{remark*}

We concern ourselves with field extensions $ k \subset K $, and every homomorphism is an extension. Consider a field extension $ k \subset K $ and $ \alpha \in K $. Then $ k\rb{\alpha} \subset K $ denotes the smallest subfield of $ K $ that contains $ k, \alpha $. Not to be confused with $ k\rb{x} $.

\begin{example*}
There are two very different cases exemplified in $ \Q \subset \C $.
\begin{itemize}
\item $ \alpha = \sqrt{2} $, $ \Q\rb{\sqrt{2}} $.
\item $ \alpha = \pi $, $ \Q\rb{\pi} $.
\end{itemize}
\end{example*}

\begin{definition}
\hfill
\begin{itemize}
\item $ \alpha $ is \textbf{algebraic} over $ k $ if $ f\rb{\alpha} = 0 $ for some $ 0 \ne f \in k\sb{x} $. Otherwise we say that $ \alpha $ is \textbf{transcendental} over $ k $.
\item The extension $ k \subset K $ is \textbf{algebraic} if for all $ \alpha \in K $, $ \alpha $ is algebraic over $ k $.
\end{itemize}
\end{definition}

\begin{definition}
Consider a field $ k $ and $ f \in k\sb{x} $. We say that $ k \subset K $ is a \textbf{splitting field} for $ f $ if
\begin{itemize}
\item $ f\rb{x} = a\prod_{i = 1}^n \rb{x - \lambda_i} \in K\sb{x} $ for $ a \in k \setminus \cb{0} $, and
\item $ K = k\rb{\lambda_1, \dots, \lambda_n} $.
\end{itemize}

\end{definition}

\begin{example*}
\hfill
\begin{itemize}
\item If $ f\rb{x} = x^2 - 2 \in \Q\sb{x} $, then $ K = \Q\rb{\sqrt{2}} $ is a splitting field for $ f $. Indeed
$$ x^2 - 2 =\rb{x + \sqrt{2}}\rb{x - \sqrt{2}} \in \Q\rb{\sqrt{2}}\sb{x}. $$
\item If $ f\rb{x} = x^2 + 2 $, then $ K = \Q\rb{\sqrt{-2}} $.
\item If $ f\rb{x} = x^3 - 2 $, then
$$ \Q\rb{\sqrt[3]{2}} = \cb{a + b\sqrt[3]{2} + c\sqrt[3]{4} \mid a, b, c \in \Q} $$
is not a splitting field. $ K = \Q\rb{\sqrt[3]{2}, \omega} $, where $ \omega = \tfrac{-1 + \sqrt{3}}{2} $, is a splitting field.
$$ x^3 - 2 = \rb{x - \sqrt[3]{2}}\rb{x - \omega\sqrt[3]{2}}\rb{x - \omega^2\sqrt[3]{2}}. $$
\end{itemize}
\end{example*}

\pagebreak

\begin{theorem}[Fundamental theorem of Galois theory]
\label{thm:galoiscorrespondence}
Assume characteristic zero. Let $ k \subset K $ be the splitting field of $ f\rb{x} \in k\sb{x} $. Let
$$ G = \cb{\sigma : K \to K \mid \sigma \ \text{field automorphism}, \ \sigma\mid_{k} = id_k}. $$
We call this group the \textbf{Galois group}. There is a one-to-one correspondence
$$
\begin{array}{rcl}
\cb{k \subset K_1 \subset K \mid K_1 \ \text{subfield}} & \leftrightarrow & \cb{H \le G \mid H \ \text{subgroup}} \\
K_1 & \mapsto & \cb{\sigma \in G \mid \forall \lambda \in K_1, \ \sigma\rb{\lambda} = \lambda} \\
\cb{\lambda \in K \mid \forall \sigma \in H, \ \sigma\rb{\lambda} = \lambda} & \mapsfrom & H \le G
\end{array}.
$$
\end{theorem}

Why is this cool? Fields are hard, groups are easy. We will see that there is a good formula for the roots of $ f\rb{x} $ if and only if $ G $ is a soluble group.

\lecture{3}{Tuesday}{15/01/19}

\begin{example*}
Let $ \deg\rb{f} = 2 $ and $ f\rb{x} = x^2 + 2Ax + B \in K\sb{x} $. If $ K $ already contains the roots then $ L = K $ and $ G = \cb{id} $. Suppose $ K $ does not contain the roots. We still have quadratic formula
$$ \lambda_{1, 2} = -A \pm \sqrt{A^2 - B}. $$
If $ \Delta = A^2 - B $ then $ \sqrt{\Delta} $ does not exist in $ K $. We must have
$$ L = K\rb{\sqrt{\Delta}} = \cb{a + b\sqrt{\Delta} \mid a, b \in K}. $$
Then $ K \subset L $ and
$$ G = \cb{\sigma : L \to L \mid \sigma \mid_K = id_K} = C_2 $$
is generated by
$$ \sigma : a + b\sqrt{\Delta} \mapsto a - b\sqrt{\Delta}. $$
The following is further specialisation.
\begin{itemize}
\item Let $ K = \R $ and $ \Delta = -1 $. Then
$$ L = \C = \cb{a + b\sqrt{-1} \mid a, b \in \R}, $$
and $ G = C_2 $ is generated by
$$ \sigma : a + b\sqrt{-1} \mapsto a - b\sqrt{-1}, $$
complex conjugation.
\item Let $ K = \Q $ and $ \Delta = 2 $. Then
$$ L = \cb{a + b\sqrt{2} \mid a, b \in \Q}, $$
and $ G = C_2 $ is generated by
$$ \sigma : a + b\sqrt{2} \mapsto a - b\sqrt{2}. $$
\end{itemize}
The fundamental theorem implies there does not exist
$$ K \subsetneq K_1 \subsetneq K\rb{\sqrt{\Delta}} = L. $$
Is this obvious? Consider $ x \in L \setminus K $, so $ x = a + b\sqrt{\Delta} $, and $ b \ne 0 $, and then
$$ \sqrt{\Delta} = \dfrac{x - a}{b}, $$
so $ K\rb{x} = L $.
\end{example*}

\begin{example*}
Let $ f\rb{x} = x^3 - 2 \in \Q\sb{x} $ and $ L = \Q\rb{\sqrt[3]{2}, \omega} $, where $ \omega = \tfrac{-1 + i\sqrt{3}}{2} $ is a solution of $ x^2 + x + 1 = 0 $. Then
$$ \Q\rb{\omega} = \Q\rb{\sqrt{-3}}, \qquad \Q\rb{\sqrt[3]{2}} = \cb{a + b\sqrt[3]{2} + c\sqrt[3]{4} \mid a, b, c \in \Q}. $$
\end{example*}

\begin{remark*}
For any splitting field of $ f $, there is always a natural inclusion group homomorphism
$$ \rho : G \hookrightarrow S\rb{\lambda_1, \dots, \lambda_n}, $$
where $ S\rb{\lambda_1, \dots, \lambda_n} $ is the group of permutations of the roots of $ f = x^n + a_1x^{n - 1} + \dots + a_n $.
\begin{itemize}
\item If $ \sigma \in G $, $ f\rb{\lambda} = 0 $, so $ \lambda^n + a_1\lambda^{n - 1} + \dots + a_n = 0 $.
$$ 0 = \sigma\rb{0} = \sigma\rb{\lambda^n + a_1\lambda^{n - 1} + \dots + a_n} = \sigma\rb{\lambda}^n + a_1\sigma\rb{\lambda}^{n - 1} + \dots + a_n. $$
\item $ \rho $ is injective. If for all $ i $, $ \sigma\rb{\lambda_i} = \lambda_i $, then $ \sigma = id $ on $ K\rb{\lambda_1, \dots, \lambda_n} = L $.
\end{itemize}
\end{remark*}

The fundamental theorem and remark gives $ G = \mathfrak{S}_3 $.

\lecture{4}{Thursday}{17/01/19}

\begin{definition}
$ K \subset L $ is \textbf{finite} if $ L $ is finite-dimensional as a vector space over $ K $. The \textbf{degree} of $ L $ over $ K $ is $ \sb{L : K} = \dim_{K}\rb{L} $.
\end{definition}

Two things about this.

\begin{theorem}[Tower law]
\label{thm:towerlaw}
Let $ K \subset L \subset F $. Then $ \sb{F : K} = \sb{F : L}\sb{L : K} $.
\end{theorem}

\begin{theorem}
\label{thm:degree}
Suppose $ f\rb{x} \in K\sb{x} $ is irreducible of degree $ d = \deg\rb{f} $ and $ L = K\rb{\lambda} $ where $ f\rb{\lambda} = 0 $, then $ \sb{K\rb{\lambda} : K} = d $.
\end{theorem}

\begin{example*}
$$ K = \Q\rb{\sqrt[3]{2}} = \cb{a + b\sqrt[3]{2} + c\sqrt[3]{4} \mid a, b, c \in \Q} $$
is a field, and $ \sb{K : \Q} = 3 $.
\end{example*}

\begin{example*}
Let $ L = \Q\rb{\sqrt[3]{2}, \omega} $ be the splitting field of $ x^3 - 2 $ over $ \Q $. The lattice of subfields is
$$
\begin{tikzcd}
& & L & \\
\Q\rb{\sqrt[3]{2}} \arrow[dash]{urr}{2} & \Q\rb{\omega\sqrt[3]{2}} \arrow[dash]{ur}{2} & \Q\rb{\omega^2\sqrt[3]{2}} \arrow[dash]{u}{2} & \\
& & & \Q\rb{\omega} \arrow[dash, swap]{uul}{3} \\
& & \Q \arrow[dash]{uull}{3} \arrow[dash]{uul}{3} \arrow[dash]{uu}{3} \arrow[dash, swap]{ur}{2} &
\end{tikzcd}.
$$
Then
$$ \Q\rb{\sqrt[3]{2} + \omega} = L, \qquad \Q\rb{\omega^2\sqrt[3]{2}} \cap \Q\rb{\omega\sqrt[3]{2}} = \Q, \qquad \Q\rb{\sqrt[3]{2}, \omega\sqrt[3]{2}} = L. $$
(Exercise) What is $ \sb{L : \Q\rb{\sqrt[3]{2}}} $? Note that $ L = \Q\rb{\sqrt[3]{2}}\rb{\sqrt{-3}} $. Could $ \sqrt{-3} \in \Q\rb{\sqrt[3]{2}} $? Consider $ x^2 + 3 \in \Q\rb{\sqrt[3]{2}}\sb{x} $. By the tower law,
$$
\begin{cases}
\sb{L : \Q} = \sb{L : \Q\rb{\omega}}\sb{\Q\rb{\omega} : \Q} = 2\sb{L : \Q\rb{\omega}} & \implies 2 \mid \sb{L : \Q} \\
\sb{L : \Q} = \sb{L : \Q\rb{\sqrt[3]{2}}}\sb{\Q\rb{\sqrt[3]{2}} : \Q} = 3\sb{L : \Q\rb{\sqrt[3]{2}}} & \implies 3 \mid \sb{L : \Q}
\end{cases}
\qquad \implies \qquad 6 \mid \sb{L : \Q}.
$$
\begin{itemize}
\item Either $ x^2 + 3 $ is irreducible over $ \Q\rb{\sqrt[3]{2}} $, so by Theorem \ref{thm:degree} $ \sb{L : \Q\rb{\sqrt[3]{2}}} = 2 $ and $ \sb{L : \Q} = 6 $.
\item Or $ x^2 + 3 $ is not irreducible, so $ \Q\rb{\sqrt[3]{2}} = L $ and $ \sb{L : \Q} = 3 $, a contradiction.
\end{itemize}
Are there any other fields? Claim that there are no other fields. Suppose $ \Q \subsetneq K \subsetneq L $ is such a field. By the tower law $ \sb{K : \Q} = 2 $ or $ \sb{K : \Q} = 3 $.
\begin{itemize}
\item Suppose $ \sb{K : \Q} = 2 $.
$$
\begin{tikzcd}
& L & \\
& K\rb{\omega} \arrow[dash]{u} & \\
K \arrow[dash]{uur}{3} \arrow[dash]{ur} & & \Q\rb{\omega} \arrow[dash]{ul} \\
& \Q \arrow[dash]{ul}{2} \arrow[dash, swap]{ur}{2} &
\end{tikzcd}.
$$
\begin{itemize}
\item Either $ \omega \in K $, that is $ \Q\rb{\omega} \subset K $, so by the tower law $ \Q\rb{\omega} = K $.
\item Or $ \omega \notin K $ gives $ \sb{K\rb{\omega} : K} = 2 $, so $ \sb{K\rb{\omega} : \Q} = 4 $ contradicts the tower law for $ \Q \subset K\rb{\omega} \subset L $.
\end{itemize}
\item Suppose $ \sb{K : \Q} = 3 $.
$$
\begin{tikzcd}
L & \\
& K\rb{\sqrt[3]{2}} \arrow[dash]{ul} \\
K \arrow[dash]{uu}{2} \arrow[dash]{ur} & \\
\Q \arrow[dash]{u}{3} &
\end{tikzcd}.
$$
Claim that $ x^3 - 2 \in K\sb{x} $ splits. Suppose that it were irreducible, then $ \sb{K\rb{\sqrt[3]{2}} : K} = 3 $, which contradicts the tower law for $ K \subset K\rb{\sqrt[3]{2}} \subset L $. So it has a root in $ K $. Either $ \sqrt[3]{2} \in K $, $ \omega\sqrt[3]{2} \in K $, or $ \omega^2\sqrt[3]{2} \in K $. Thus $ \Q\rb{\sqrt[3]{2}} = K $, $ \Q\rb{\omega\sqrt[3]{2}} = K $, or $ \Q\rb{\omega^2\sqrt[3]{2}} = K $.
\end{itemize}
I want to prove that
$$ G = Aut_\Q\rb{L} = \cb{\sigma : L \to L \mid \sigma \mid_\Q = id_\Q} = \mathfrak{S}_3. $$
\end{example*}

\lecture{5}{Friday}{18/01/19}

\begin{proof}[Proof of Theorem \ref{thm:towerlaw}]
Suppose $ y_1, \dots, y_m \in F $ is a basis of $ F $ as a vector space over $ L $. Suppose $ x_1, \dots, x_n \in L $ is a basis of $ L $ as a vector space over $ K $. Claim that $ \cb{x_iy_j} $ is a basis of $ F $ over $ K $.
\begin{itemize}
\item $ \cb{x_iy_j} $ generates $ F $. Let $ z \in F $. There exist $ \mu_1, \dots, \mu_n \in L $ such that
\begin{equation}
\label{eq:1}
z = \mu_1y_1 + \dots + \mu_ny_n.
\end{equation}
$ \mu_j \in L $ so for all $ j $ there exists $ \lambda_{ij} \in K $ such that
\begin{equation}
\label{eq:2}
\mu_j = x_1\lambda_{1j} + \dots + x_m\lambda_{mj}.
\end{equation}
Plug in $ \rb{\ref{eq:2}} $ into $ \rb{\ref{eq:1}} $,
$$ z = \sum_{i, j} \lambda_{ij}x_iy_j. $$
\item $ \cb{x_iy_j} $ are linearly independent over $ K $. Suppose there exists $ \lambda_{ij} \in K $ such that
$$ 0 = \sum_{i, j} \lambda_{ij}x_iy_j = \sum_j \rb{\sum_i \lambda_{ij}x_i}y_j, $$
so for all $ j $, $ \sum_i \lambda_{ij}x_i = 0 $, so for all $ j $ and all $ i $, $ \lambda_{ij} = 0 $.
\end{itemize}
\end{proof}

\begin{example*}
To show $ G = \mathfrak{S}_3 $. Let $ \sigma = \onebytwo{1}{2} $. A basis of $ L / \Q $ is
$$ 1, \sqrt[3]{2}, \sqrt[3]{4}, \omega, \omega\sqrt[3]{2}, \omega\sqrt[3]{4}. $$
\begin{itemize}
\item $ \sigma\rb{1} = 1 $.
\item $ \sigma\rb{\sqrt[3]{2}} = \omega\sqrt[3]{2} $.
\item $ \sigma\rb{\omega\sqrt[3]{2}} = \sqrt[3]{2} $.
\item $ \sigma\rb{\sqrt[3]{4}} = \sigma\rb{\sqrt[3]{2} \cdot \sqrt[3]{2}} = \omega\sqrt[3]{2} \cdot \omega\sqrt[3]{2} = \omega^2\sqrt[3]{4} = \rb{-\omega - 1}\sqrt[3]{4} = -\omega\sqrt[3]{4} - \sqrt[3]{4} $.
\item $ \sigma\rb{\omega} = \sigma\rb{\omega\sqrt[3]{2} / \sqrt[3]{2}} = \sigma\rb{\omega\sqrt[3]{2}} / \sigma\rb{\sqrt[3]{2}} = \sqrt[3]{2} / \omega\sqrt[3]{2} = 1 / \omega = -1 - \omega $.
\item $ \sigma\rb{\omega\sqrt[3]{4}} = \sigma\rb{\omega\sqrt[3]{2} \cdot \sqrt[3]{2}} = \sigma\rb{\omega\sqrt[3]{2}} \cdot \sigma\rb{\sqrt[3]{2}} = \sqrt[3]{2} \cdot \omega\sqrt[3]{2} = \omega\sqrt[3]{4} $.
\end{itemize}
Thus
$$ \sigma =
\begin{pmatrix}
1 & 0 & 0 & -1 & 0 & 0 \\
0 & 0 & 0 & 0 & 1 & 0 \\
0 & 0 & -1 & 0 & 0 & 0 \\
0 & 0 & 0 & -1 & 0 & 0 \\
0 & 1 & 0 & 0 & 0 & 0 \\
0 & 0 & -1 & 0 & 0 & 1
\end{pmatrix}.
$$
A question is if there were $ \sigma \in G $ such that $ \rho\rb{\sigma} = \onebytwo{1}{2} $ then we have written the matrix of $ \sigma $ as a $ \Q $-linear map of $ L $ in a basis. But how to check that this $ \Q $-linear map is a field homomorphism? We know the Galois correspondence for extensions of degree two.
$$ Gal_{\Q\rb{\sqrt[3]{2}}}\rb{L}, Gal_{\Q\rb{\omega^2\sqrt[3]{2}}}\rb{L}, Gal_{\Q\rb{\omega\sqrt[3]{2}}}\rb{L} \subset G $$
contain an element of order two, and
$$
\begin{array}{rrl}
\rho : & Gal_{\Q\rb{\sqrt[3]{2}}}\rb{L} & \mapsto \onebytwo{2}{3} \\
& Gal_{\Q\rb{\omega^2\sqrt[3]{2}}}\rb{L} & \mapsto \onebytwo{1}{2} \\
& Gal_{\Q\rb{\omega\sqrt[3]{2}}}\rb{L} & \mapsto \onebytwo{1}{3}.
\end{array}
$$
The lattice of subgroups is
$$
\begin{tikzcd}
& & \cb{e} & \\
\ab{\onebytwo{2}{3}} \arrow[dash]{urr}{2} & \ab{\onebytwo{1}{2}} \arrow[dash]{ur}{2} & \ab{\onebytwo{1}{3}} \arrow[dash]{u}{2} & \\
& & & \onebythree{1}{2}{3} \arrow[dash, swap]{uul}{3} \\
& & G \arrow[dash]{uull}{3} \arrow[dash]{uul}{3} \arrow[dash]{uu}{3} \arrow[dash, swap]{ur}{2} &
\end{tikzcd}.
$$
$ \Q\rb{\omega} / \Q $ is the splitting field of $ x^2 + x + 1 $ and of $ x^2 + 3 $.
\end{example*}

We can learn the following. Let $ k \subset L $ be a splitting field. Consider $ k \subset K \subset L $. Then $ K \subset L $ is also a splitting field. The corresponding $ H \le G $ is the Galois group $ Gal_K\rb{L} $. On the other hand $ k \subset K $ is not always a splitting field. It is a splitting field if and only if the corresponding $ H \le G $ is a normal subgroup and in that case $ Gal_k\rb{K} = G / H $.

\pagebreak

\section{Elementary facts}

\lecture{6}{Tuesday}{22/01/19}

Let $ K \subset L $ and $ a \in L $. The \textbf{evaluation homomorphism}
$$ \function[e_a]{K\sb{x}}{K\sb{a} \subset L}{f\rb{x}}{f\rb{a}} $$
is a surjective ring homomorphism, where $ K\sb{a} $ is the smallest subring of $ L $ containing $ K $ and $ a $.

\begin{definition}
$ f\rb{x} = a_0x^n + \dots + a_n \in K\sb{x} $ is \textbf{monic} if $ a_0 = 1 $.
\end{definition}

\begin{lemma}
\hfill
\begin{itemize}
\item If $ a $ is transcendental, $ e_a $ is injective and it extends to $ \widetilde{e_a} : K\rb{x} \to K\rb{a} $, by
$$
\begin{tikzcd}
K\rb{x} \arrow{dr}{\widetilde{e_a}} & \\
K\sb{x} \arrow[subset]{u} \arrow[swap]{r}{e_a} & L
\end{tikzcd}.
$$
\item If $ a $ is algebraic, then $ Ker\rb{e_a} = \ab{f_a} $, where $ f_a \in K\sb{x} $ is irreducible, or prime, and unique if monic, then called the minimal polynomial of $ a \in L / K $. In this case
$$
\begin{tikzcd}
K\sb{x} \arrow{r}{e_a} & K\sb{a} \cong K\rb{a} \subset L \\
\dfrac{K\sb{x}}{\ab{f_a}} \arrow[subset]{u} \arrow{ur}{\sim}[swap]{\sb{e_a}}
\end{tikzcd}.
$$
\end{itemize}
\end{lemma}

\begin{proof}
There is nothing to prove.
\end{proof}

\begin{remark*}
Let $ g\rb{x} \in K\sb{x} $ and $ g\rb{a} \ne 0 $. Claim that $ 1 / g\rb{a} \in K\sb{a} $. Indeed $ \gcd\rb{f, g} = 1 $ in $ K\sb{x} $ and $ f \nmid g $. There exists $ \phi, \psi \in K\sb{x} $ such that $ f\phi + g\psi = 1 $ and $ g\rb{a}\psi\rb{a} = 1 $. All of this is saying
\begin{itemize}
\item $ K\sb{a} \cong K\rb{a} $, and
\item $ K\sb{x} / \ab{f_a} \cong K\rb{a} $.
\end{itemize}
\end{remark*}

Let
$$ Em_K\rb{K\rb{a}, F} = \cb{\sigma : K\rb{a} \to F \mid \sigma \ \text{field homomorphism}, \ \sigma_K = id_K}, $$
where
$$
\begin{tikzcd}
& K\rb{a} \arrow[dotted]{dd}{\sigma} \\
k \arrow[subset]{ur} \arrow[subset]{dr} & \\
& F
\end{tikzcd}.
$$

\begin{corollary}
For $ K \subset L $ and $ a \in L $ algebraic over $ K $,
\begin{itemize}
\item $ \sb{K\rb{a} : K} = \deg\rb{f_a} $, and
\item If $ K \subset F $ is an extension,
$$ Em_K\rb{K\rb{a}, F} = \cb{b \in F \mid f_a\rb{b} = 0}. $$
\end{itemize}
\end{corollary}

\begin{proof}
Since $ K\rb{a} = K\sb{a} $, $ \sb{K\rb{a} : K} = \dim_K\rb{K\rb{a}} = \dim_K\sb{K\rb{a}} $. Suppose
$$ f\rb{x} = x^n + \mu_1x^{n - 1} + \dots + \mu_n \in K\sb{x} $$
is the minimal polynomial of $ a $ over $ K $. Claim that $ 1, \dots, a^{n - 1} $ is a basis of $ K\sb{a} $ over $ K $.
\begin{itemize}
\item The set generates $ K\sb{a} $. Let $ c \in K\sb{a} $. There exists $ g \in K\sb{x} $ such that $ g\rb{a} = c $. Long division gives
$$ g\rb{x} = f\rb{x}q\rb{x} + r\rb{x}, \qquad m = \deg\rb{r\rb{x}} < n. $$
Then $ r\rb{x} = \lambda_0 + \dots + \lambda_mx^m $ and $ g\rb{a} = r\rb{a} = \lambda_0 + \dots + \lambda_ma^m $.
\item The set is linearly independent, otherwise there exists
$$ g\rb{x} = \lambda_0 + \dots + \lambda_{n - 1}x^{n- 1} \in K\sb{x}, g\rb{a} = 0, $$
and $ f $ was not the minimal polynomial.
\end{itemize}
$ \sigma\rb{a} $ is a root of $ f $, since applying $ \sigma $ to $ f\rb{a} = 0 $ gives
$$ 0 = \sigma\rb{a^n + \mu_1a^{n - 1} + \dots + \mu_n} = \sigma\rb{a}^n + \mu_1^{n - 1}\sigma\rb{a}^{n - 1} + \dots + \mu_n = f\rb{\sigma\rb{a}}. $$
Vice versa, if $ b \in F $ is a root of $ f $,
$$ K\rb{b} \xleftarrow[\sim]{\sb{e_b}} \dfrac{K\sb{x}}{\ab{f}} \xrightarrow[\sim]{\sb{e_a}} K\rb{a}, $$
then $ \sigma = \sb{e_b}\sb{e_a}^{-1} $. Thus there is a one-to-one correspondence
$$
\begin{array}{rcl}
Em_K\rb{K\rb{a}, F} & \leftrightarrow & \cb{b \in F \mid f\rb{b} = 0} \\
\sigma & \mapsto & \sigma\rb{a} \\
\sb{e_b}\sb{e_a}^{-1} & \mapsfrom & b
\end{array}.
$$
\end{proof}

\begin{corollary}
Let $ K $ be a field and $ f \in K\sb{x} $. Then there exists $ K \subset L $ such that $ f $ has a root in $ L $.
\end{corollary}

\begin{proof}
Take $ g $ a prime factor of $ f $. Take $ L = K\sb{x} / \ab{g} $. In here $ a = \sb{x} $ is a root of $ g $ hence a root of $ f $.
\end{proof}

\lecture{7}{Thursday}{24/01/19}

From now on in this course, we study field extensions $ K \subset L $, always assumed to be finite, so $ \sb{L : K} = \dim_K\rb{L} < \infty $.

\begin{remark*}
$ K \subset L $ is finite if and only if
\begin{itemize}
\item it is algebraic, that is for all $ a \in L $, $ a $ is algebraic over $ K $, and
\item it is finitely generated, that is there exist $ a_1, \dots, a_m \in L $ such that $ L = K\rb{a_1, \dots, a_m} $.
\end{itemize}
\end{remark*}

An important point of view is that we study all possible field homomorphisms
$$ Em\rb{K, L} = \cb{\sigma : K \to L \mid \sigma \ \text{field homomorphism}}. $$
Often there is a field $ k \subset K, L $ in the background which we want to stay fixed, so
$$ Em_k\rb{K, L} = \cb{\sigma : K \to L \mid \sigma \ \text{field homomorphism}, \ \sigma \mid_k = id_k}. $$

\begin{example*}
Let $ K = \Q\rb{\sqrt[3]{2}} $. The minimal polynomial of $ \sqrt[3]{2} $ is $ x^3 - 2 $. Let $ L = \Q\rb{\sqrt[3]{2}, \sqrt{-3}} $ be the splitting field of $ x^3 - 2 $. Then
$$ Em_\Q\rb{K, L} = Em\rb{K, L} = \cb{\text{roots of} \ x^3 - 2 \ \text{in} \ L} = \cb{\sqrt[3]{2}, \omega\sqrt[3]{2}, \omega^2\sqrt[3]{2}}. $$
\end{example*}

\begin{remark*}
Suppose $ k \subset K $. $ Em_k\rb{K, K} = G = Gal_k\rb{K} $. Indeed every $ k $-homomorphism $ \sigma : K \to K $ is automatically invertible. We know $ \sigma $ is injective. $ \sigma $ is also surjective because $ \sigma $ is a $ k $-linear endomorphism of a finite-dimensional $ k $-vector space.
\end{remark*}

\section{Axiomatics}

\begin{proposition}
\label{prop:1}
Fix $ k \subset K $ and $ k \subset L $. Then $ \#Em_k\rb{K, L} \le \sb{K : k} $.
\end{proposition}

\begin{proof}
\hfill
\begin{itemize}[leftmargin=1in]
\item[Special case.] If $ K = k\rb{a} $, let $ f\rb{x} \in k\sb{x} $ be the minimal polynomial of $ a $. Then $ Em_k\rb{k\rb{a}, L} $ is the roots of $ f\rb{x} $ in $ L $, so
$$ \#Em_k\rb{K, L} = \#\cb{\text{roots}} \le \deg\rb{f} = \sb{k\rb{a} : k}, $$
as proved last time.
\item[General case.] If $ k = K $, nothing to do. Otherwise choose $ a \in K \setminus k $.
$$
\begin{tikzcd}
& & L \\
k \arrow[subset]{urr} \arrow[subset]{r} & k\rb{a} \arrow[dotted, swap]{ur}{y} \arrow[subset]{r} & K \arrow[dotted]{u}
\end{tikzcd}.
$$
Consider the restriction map
$$ \rho : Em_k\rb{K, L} \to Em_k\rb{k\rb{a}, L}. $$
Fix $ y \in Em_k\rb{k\rb{a}, L} $. Then
$$ \rho^{-1}\rb{y} = \cb{x : K \to L \mid x \mid_{k\rb{a}} = id_{k\rb{a}}}. $$
Since $ \sb{k\rb{a} : k} > 1 $, by the tower law $ \sb{K : k\rb{a}} < \sb{K : k} $. By induction we may assume $ \#\rho^{-1}\rb{y} \le \sb{K : k\rb{a}} $. So
$$ \#Em_k\rb{K, L} \le \sum_{y \in Em_k\rb{k\rb{a}, L}} \#\rho^{-1}\rb{y} \le \sb{k\rb{a} : k}\sb{K : k\rb{a}} = \sb{K : k}, $$
by the tower law.
\end{itemize}
\end{proof}

\begin{proposition}
Suppose given two field extensions $ k \subset K $ and $ k \subset L $. Then there is a non-unique bigger common field
$$
\begin{tikzcd}
& K \arrow[hookrightarrow]{dr}{\phi_1} & \\
k \arrow[hookrightarrow]{ur}{\sigma_1} \arrow[hookrightarrow, swap]{dr}{\sigma_2} & & \Omega \\
& L \arrow[hookrightarrow, swap]{ur}{\phi_2} &
\end{tikzcd}
$$
that contains both.
\end{proposition}

\begin{remark*}
\hfill
\begin{itemize}
\item More formally, suppose given $ \sigma_1 \in Em\rb{k, K} $ and $ \sigma_2 \in Em\rb{k, L} $, then there exists $ \Omega $, $ \phi_1 \in Em\rb{K, \Omega} $, and $ \phi_2 \in Em\rb{L, \Omega} $ such that $ \phi_1 \circ \sigma_1 = \phi_2 \circ \sigma_2 $.
\item I never said that $ \Omega $ is unique. For example, let $ K = \Q\rb{\sqrt[3]{2}} $ and $ L = \Q\rb{\sqrt[3]{2}} $. One choice is $ \Omega = k $. Another choice is $ \Omega = \Q\rb{\sqrt[3]{2}, \sqrt{-3}} $, where
$$
\begin{tikzcd}
& \Q\rb{\sqrt[3]{2}} \arrow[hookrightarrow]{dr}{\phi_1 : \sqrt[3]{2} \mapsto \sqrt[3]{2}} & \\
\Q \arrow[hookrightarrow]{ur}{\sigma} \arrow[hookrightarrow, swap]{dr}{\sigma} & & \Q\rb{\sqrt[3]{2}, \sqrt{-3}} \\
& \Q\rb{\sqrt[3]{2}} \arrow[hookrightarrow, swap]{ur}{\phi_2 : \sqrt[3]{2} \mapsto \omega\sqrt[3]{2}} &
\end{tikzcd}.
$$
\end{itemize}
\end{remark*}

Another more precise way to state this is there exists $ k \subset \Omega $ such that $ Em_k\rb{K, \Omega} $ and $ Em_k\rb{L, \Omega} $ are both non-empty.

\lecture{8}{Friday}{25/01/19}

\begin{proof}
\hfill
\begin{itemize}[leftmargin=1in]
\item[Special case.] If $ K = k\rb{a} $, let $ f\rb{x} \in k\sb{x} $ be the minimal polynomial of $ a $ over $ k $. Let $ L \subset E $ be such that $ f\rb{x} \in L\sb{x} $ has a root $ \alpha \in E $. Then there exists $ \sigma \in Em_k\rb{k\rb{a}, E} $ such that $ \sigma\rb{a} = \alpha $.
$$
\begin{tikzcd}
& k\rb{a} \arrow[dotted]{dr}{\sigma} & \\
k \arrow[subset]{ur} \arrow[subset]{dr} & & E \\
& L \arrow[subset]{ur} &
\end{tikzcd}.
$$
\item[General case.] By induction on $ \sb{K : k} $. If $ \sb{K : k} = 1 $, take $ \Omega = L $. If $ \sb{K : k} > 1 $, take $ a \in K \setminus k $.
$$
\begin{tikzcd}
& & K \arrow[subset]{dr} & \\
& k\rb{a} \arrow[subset]{ur} \arrow[subset]{dr} & & \Omega \\
k \arrow[subset]{ur} \arrow[subset]{dr} & & E \arrow[subset]{ur} & \\
& L \arrow[subset]{ur} & &
\end{tikzcd}.
$$
By special case there exists $ E $ as in the diagram. By tower law $ \sb{K : k\rb{a}} < \sb{K : k} $ hence by induction find $ \Omega $ as in the diagram. $ \Omega $ solves the original problem.
\end{itemize}
\end{proof}

\begin{proposition}
\label{prop:3}
Let $ L $ be any field and $ G $ be a finite group acting on $ L $ as automorphisms. Let
$$ K = G^* = Fix\rb{G} = L^G = \cb{\lambda \in L \mid \forall \sigma \in G, \ \sigma\rb{\lambda} = \lambda}. $$
Consider $ Aut_K\rb{L} = K^\dagger $. Then the obvious inclusion $ G \subset K^\dagger = \rb{G^*}^\dagger $ is an equality, so $ G $ is all of $ K^\dagger $.
\end{proposition}

\begin{remark*}
Contextualising, this thing is half of the Galois correspondence.
$$
\begin{array}{rcl}
\cb{F \mid k \subset F \subset \Omega} & \leftrightarrow & \cb{G \mid G \le Aut_k\rb{\Omega}} \\
F & \mapsto & Aut_F\rb{\Omega} = F^\dagger \\
Fix\rb{G} = G^* & \mapsfrom & G
\end{array}.
$$
Then to prove the Galois correspondence, we need for all $ G $, $ G = \rb{G^*}^\dagger $. We also need for all $ F $, $ F = \rb{F^\dagger}^* $.
\end{remark*}

Proposition \ref{prop:3} follows from the following lemma.

\begin{lemma}
\label{lem:3}
$ K \subset L $ is a finite extension of degree $ \sb{L : K} \le \abs{G} $.
\end{lemma}

\begin{proof}[Proof of Proposition \ref{prop:3}]
From Proposition \ref{prop:1}, $ Aut_K\rb{L} = Em_K\rb{L, L} $ because $ K \subset L $ is finite, and $ \#Em_K\rb{L, L} \le \sb{L : K} $. By the lemma,
$$ \sb{L : K} \le \#Em_K\rb{L, L} \le \sb{L : K}, $$
so $ \abs{G} = \#Em_K\rb{L, L} $. By what we said, $ G \subset Em_K\rb{L, L} $, so $ G = Em_K\rb{L, L} $.
\end{proof}

\lecture{9}{Tuesday}{29/01/19}

Lecture 9 is a problem class.

\lecture{10}{Thursday}{31/01/19}

\begin{proof}[Proof of Lemma \ref{lem:3}]
Write $ G = \cb{\sigma_1, \dots, \sigma_n} $ for $ n = \abs{G} $. Want that all $ \rb{n + 1} $-tuples $ a_1, \dots, a_{n + 1} \in L $ are linearly dependent over $ K $. Let $ a_1, \dots, a_{n + 1} \in L $. Consider the $ n + 1 $ vectors in $ L^n $. Let
$$ \overline{a_1} = \threebyone{\sigma_1\rb{a_1}}{\vdots}{\sigma_n\rb{a_1}}, \ \dots, \ \overline{a_{n + 1}} = \threebyone{\sigma_1\rb{a_{n + 1}}}{\vdots}{\sigma_n\rb{a_{n + 1}}} \in L^n. $$
These are linearly dependent over $ L $. There exist $ x_1, \dots, x_{n + 1} \in L $ not all zero such that
$$ x_1\overline{a_1} + \dots + x_{n + 1}\overline{a_{n + 1}} = \overline{0}. $$
By reordering the $ \overline{a_i} $, may assume
\begin{equation}
\label{eq:3}
x_1\overline{a_1} + \dots + x_k\overline{a_k} = \overline{0},
\end{equation}
for some $ 1 \le k \le n + 1 $ with
\begin{itemize}
\item for all $ i \in \cb{1, \dots, k} $, $ x_i \ne 0 $,
\item such $ k $ is the smallest, and
\item $ x_1 = 1 $.
\end{itemize}
Claim that all these $ x_i \in K $. This does it, by reading $ j $-th row where $ \sigma_j = id_G $. We need to show for all $ i $ $ x_i \in L^G $. Take $ \sigma \in G $.
$$ \sigma\rb{x_1}\threebyone{\sigma\rb{\sigma_1\rb{a_1}}}{\vdots}{\sigma\rb{\sigma_n\rb{a_1}}} + \dots + \sigma\rb{x_k}\threebyone{\sigma\rb{\sigma_1\rb{a_k}}}{\vdots}{\sigma\rb{\sigma_n\rb{a_k}}} = \overline{0} \in L^n. $$
Note that
$$ \function{G}{G}{\tau}{\sigma \circ \tau} $$
is a bijective function and $ \cb{\sigma \circ \sigma_1, \dots, \sigma \circ \sigma_n} = G $. Multiplying by $ \sigma $ reshuffles the rows. So in fact,
\begin{equation}
\label{eq:4}
\sigma\rb{x_1}\overline{a_1} + \dots + \sigma\rb{x_k}\overline{a_k} = \overline{0}.
\end{equation}
Claim that for all $ i $ $ \sigma\rb{x_i} = x_i $. Otherwise $ \rb{\ref{eq:3}} - \rb{\ref{eq:4}} $,
$$ \rb{x_2 - \sigma\rb{x_2}}\overline{a_2} + \dots + \rb{x_k - \sigma\rb{x_k}}\overline{a_k} = \overline{0} $$
is a shorter solution, contradicting $ k $ minimal.
\end{proof}

\pagebreak

\section{Galois correspondence}

\begin{definition}
$ k \subset K $ is \textbf{normal} if
\begin{equation}
\label{eq:5}
\forall k \subset \Omega, \ \forall \sigma_1, \sigma_2 \in Em_k\rb{K, \Omega}, \ \exists \sigma \in Em_k\rb{K, K}, \ \sigma_2 = \sigma_1 \circ \sigma.
\end{equation}
$$
\begin{tikzcd}
& \Omega & \\
\sigma_1\rb{K} \arrow[subset]{ur} \arrow[bend left=15, dotted]{rr}{\sigma} & K \arrow{l}{\sigma_1} \arrow[swap]{r}{\sigma_2} & \sigma_2\rb{K} \arrow[subset]{ul} \\
& k \arrow[subset]{ul} \arrow[subset, u] \arrow[subset]{ur} &
\end{tikzcd}.
$$
Equivalently, $ k \subset K $ is normal if
\begin{equation}
\label{eq:6}
\forall k \subset \Omega, \ \forall \sigma_1, \sigma_2 \in Em_k\rb{K, \Omega}, \ \sigma_2\rb{K} \subset \sigma_1\rb{K}.
\end{equation}
\end{definition}

\begin{example*}
$ \Q \subset \Q\rb{\sqrt[3]{2}} $ is not normal. Take $ \Omega = \Q\rb{\sqrt[3]{2}, \sqrt{-3}} $.
\end{example*}

\begin{itemize}[leftmargin=1in]
\item[$ \rb{\ref{eq:5}} \implies \rb{\ref{eq:6}} $] Indeed for all $ \lambda \in K $, $ \sigma_2\rb{\lambda} = \sigma_1\rb{\sigma\rb{\lambda}} \in \sigma_1\rb{K} $, so $ \sigma_2\rb{K} \subset \sigma_1\rb{K} $.
\item[$ \rb{\ref{eq:6}} \implies \rb{\ref{eq:5}} $] Work inside $ \Omega $.
$$ k \subset \sigma_2\rb{K} \subset \sigma_1\rb{K} \subset \Omega. $$
Tower law gives
$$ \sb{K : k} = \sb{\sigma_1\rb{K} : k} = \sb{\sigma_1\rb{K} : \sigma_2\rb{K}}\sb{\sigma_2\rb{K} : k} = \sb{\sigma_1\rb{K} : \sigma_2\rb{K}}\sb{K : k}. $$
So $ \sb{\sigma_1\rb{K} : \sigma_2\rb{K}} = 1 $, so $ \sigma_1\rb{K} = \sigma_2\rb{K} $. Take $ \sigma = \sigma_1^{-1} \circ \sigma_2 $. $ \sigma $ is clearly bijective and it is more or less obvious that $ \sigma \in Em_k\rb{K, K} $.
\end{itemize}

\lecture{11}{Friday}{01/02/19}

Equivalently, $ k \subset K $ is normal if for all $ K \subset \Omega $, for all $ \sigma \in Em_k\rb{K, \Omega} $, $ \sigma\rb{K} \subset K $.
$$
\begin{tikzcd}
& \Omega & \\
K \arrow[subset]{ur} \arrow[dotted]{rr}{\sigma} & & \sigma\rb{K} \arrow[subset]{ul} \\
& k \arrow[subset]{ul} \arrow[subset]{ur} &
\end{tikzcd}
$$

\begin{remark*}
We will see that $ k \subset K $ is normal if and only if there exists $ f\rb{x} \in K\sb{x} $ such that $ K $ is a splitting field of $ f $.
\end{remark*}

\begin{lemma}
Suppose $ k \subset K $ is normal. Consider $ k \subset L \subset K $. Then also $ L \subset K $ is normal.
\end{lemma}

\begin{proof}
If $ \sigma \in Em_L\rb{K, \Omega} $, then $ \sigma \in Em_k\rb{K, \Omega} $.
\end{proof}

Warning.
\begin{itemize}
\item It is not true in general that $ k \subset K $ is normal gives that $ k \subset L $ is normal. For example, let
$$ k = \Q \subset \Q\rb{\sqrt[3]{2}} \subset \Q\rb{\sqrt[3]{2}, \sqrt{-3}} = K. $$
$ k \subset K $ is normal because it is a splitting field but $ \Q \subset \Q\rb{\sqrt[3]{2}} $ is not normal.
\item Suppose $ k \subset L $ is normal and $ L \subset K $ is normal. This does not imply $ k \subset K $ is normal. This will be in an example sheet.
\end{itemize}

\begin{definition}
$ k \subset K $ is \textbf{separable} if for all $ k \subset K_1 \subset K_2 \subset K $, if $ K_1 \ne K_2 $, there exist $ k \subset \Omega $ and embeddings $ x \in Em_k\rb{K_1, \Omega} $ and $ y_1, y_2 \in Em_k\rb{K_2, \Omega} $ such that
$$
\begin{tikzcd}
K & & \Omega & \\
K_2 \arrow[dotted]{r}{y_1} \arrow[bend right=15, dotted, near start, swap]{rrr}{y_2} \arrow[subset]{u} & y_1\rb{K_2} \arrow[subset]{ur} & & y_2\rb{K_2} \arrow[subset]{ul} \\
K_1 \arrow[subset]{u} \arrow{rr}{x} & & x\rb{K_1} \arrow[subset]{ul} \arrow[subset]{ur} & \\
k \arrow[subset]{u} \arrow[subset]{urr} & & &
\end{tikzcd}.
$$
That is, $ y_1 \mid_{K_1} = y_2 \mid_{K_1} = x $ but $ y_1 \ne y_2 $.
\end{definition}

Slogan is that embeddings separate fields. We will see that
\begin{itemize}
\item in characteristic zero everything is separable, and
\item in characteristic $ p $ we will have good ways to decide if something is separable.
\end{itemize}

\begin{lemma}
Suppose $ k \subset K \subset L $. Then $ k \subset L $ is separable if and only if $ k \subset K $ and $ K \subset L $ are separable.
\end{lemma}

\begin{proof}
\hfill
\begin{itemize}
\item[$ \implies $] Obvious. $ K \subset K_1 \subset K_2 \subset L $ leads to $ k \subset K_1 \subset K_2 \subset L $.
\item[$ \impliedby $] I will do later.
\end{itemize}
\end{proof}

\begin{theorem}[Fundamental theorem of Galois theory]
Let $ k \subset K $ be normal and separable. Let $ G = Em_k\rb{K, K} $. Then there is a one-to-one correspondence
$$
\begin{array}{rcl}
\cb{k \subset L \subset K} & \leftrightarrow & \cb{H \le G} \\
L & \mapsto & L^\dagger = \cb{\sigma \in G \mid \forall \lambda \in L, \ \sigma\rb{\lambda} = \lambda}\\
H^* = \cb{\lambda \in K \mid \forall \sigma \in H, \ \sigma\rb{\lambda} = \lambda} & \mapsfrom & H
\end{array}.
$$
\end{theorem}

\begin{proof}
We show that for all $ H \le G $, $ \rb{H^*}^\dagger = H $ and for all $ k \subset L \subset K $, $ \rb{L^\dagger}^* = L $. We already did the former. We just prove the latter now. Note that $ L \subset K $ is normal and separable so all I need to show is $ \rb{k^\dagger}^* = k $, that is $ k = G^* $ is the fixed field of $ G $. That is, if $ \lambda \notin k $, there exists $ \sigma : K \to K $ in $ G $ such that $ \sigma\rb{\lambda} \ne \lambda $. By separability, there exists $ \Omega $ and $ x_1 \ne x_2 \in Em_k\rb{k\rb{\lambda}, \Omega} $ such that
$$
\begin{tikzcd}
K \arrow[dotted]{dr}{\widetilde{x_1}, \widetilde{x_2}} & \\
k\rb{\lambda} \arrow[dash]{u} \arrow[swap]{r}{x_1, x_2} & \Omega \\
k \arrow[dash]{u} \arrow[dash]{ur} &
\end{tikzcd},
$$
so $ x_1\rb{\lambda} \ne x_2\rb{\lambda} $. Two steps.
\begin{itemize}
\item There exist $ \widetilde{x_1}, \widetilde{x_2} : K \to \Omega $ extending $ x_1, x_2 : k\rb{\lambda} \to \Omega $, by the following lemma.
\item Because $ k \subset K $ is normal there exists $ \sigma \in Em_k\rb{K, K} $ such that $ \widetilde{x_2} = \widetilde{x_1} \circ \sigma $ then clearly $ \sigma\rb{\lambda} \ne \lambda $.
\end{itemize}
\end{proof}

\begin{lemma}
Suppose $ k \subset K $ is normal. Then for all towers $ k \subset F \subset K \subset \Omega $, the natural restriction $ \rho : Em_k\rb{K, \Omega} \to Em_k\rb{F, \Omega} $ is surjective.
\end{lemma}

\lecture{12}{Tuesday}{05/02/19}

The statement says for all $ \sigma \in Em_k\rb{F, \Omega} $, there exists $ \widetilde{\sigma} \in Em_k\rb{K, \Omega} $ such that $ \widetilde{\sigma} \mid_F = \sigma $.
$$
\begin{tikzcd}
K \arrow[dotted]{dr}{\widetilde{\sigma}} & \\
F \arrow[dash]{u} \arrow[swap]{r}{\sigma} & \Omega \\
k \arrow[dash]{u} \arrow[dash]{ur} &
\end{tikzcd}.
$$

\begin{proof}
We know that there exists $ \widetilde{\Omega} $ as follows.
$$
\begin{tikzcd}
K \arrow{r}{\phi_2} \arrow[dotted, swap]{dr}{\widetilde{\sigma}} & \widetilde{\Omega} \\
F \arrow[dash]{u} \arrow[swap]{r}{\sigma} & \Omega \arrow[swap]{u}{\psi} \\
k \arrow[dash]{u} \arrow[dash]{ur} &
\end{tikzcd}.
$$
There are two $ K \subset \widetilde{\Omega} $,
$$
\phi_1 : K \subset \Omega \xhookrightarrow{\psi} \widetilde{\Omega}, \qquad \phi_2 : K \hookrightarrow \widetilde{\Omega}.
$$
Because $ k \subset K $ is normal $ \phi_2\rb{K} \subset \phi_1\rb{K} \subset \psi\rb{\Omega} $. That proves that $ \widetilde{\sigma} $ exists.
\end{proof}

\begin{corollary}
Suppose $ k \subset K $ is normal. Then for all towers $ k \subset F \subset K \subset \Omega $, $ Em_k\rb{F, K} \to Em_k\rb{F, \Omega} $ is also surjective.
\end{corollary}

The corollary states that for all $ \sigma \in Em_k\rb{F, \Omega} $, $ \sigma\rb{F} \subset K $.
$$
\begin{tikzcd}
\Omega & \\
K \arrow[dash]{u} \arrow[dotted]{r}{\widetilde{\sigma}} & \widetilde{\sigma}\rb{K} \arrow[dash]{ul} \\
F \arrow[dash]{u} \arrow[swap]{r}{\sigma} & \sigma\rb{F} \arrow[dash]{ul} \arrow[dash]{u} \\
k \arrow[dash]{u} \arrow[dash]{ur} &
\end{tikzcd}.
$$

\begin{proof}
This clearly follows from the lemma. $ \sigma\rb{F} \subset \widetilde{\sigma}\rb{K} \subset K $ by definition of normal.
\end{proof}

\pagebreak

\section{Normal extensions}

\begin{theorem}
For finite $ k \subset K $, the following are equivalent.
\begin{enumerate}
\item For all $ f \in k\sb{x} $ irreducible either $ f $ has no root in $ K $ or $ f $ splits completely in $ K $.
\item There exists $ f \in k\sb{x} $ not necessarily irreducible such that $ K $ is a splitting field of $ f $.
\item $ k \subset K $ is normal.
\end{enumerate}
\end{theorem}

\begin{proof}
\hfill
\begin{itemize}[leftmargin=0.5in]
\item[$ 1 \implies 2 $] There are $ \lambda_1, \dots, \lambda_m \in K $ such that $ K = k\rb{\lambda_1, \dots, \lambda_m} $. For all $ i $ let $ f_i \in k\sb{x} $ be the minimal polynomial of $ \lambda_i $. $ f_i $ is irreducible and by $ 1 $ it splits completely. $ K $ is the splitting field of $ f\rb{x} = \prod_{i = 1}^m f_i\rb{x} $.
\item[$ 2 \implies 3 $] Suppose $ K \subset \Omega $. Let $ \sigma : K \to \Omega $ be another embedding. For all $ \lambda_i $, $ \sigma\rb{\lambda_i} $ is a root of $ f $, so $ \sigma\rb{\lambda_i} \subset K $ hence $ \sigma\rb{K} \subset K $.
\item[$ 3 \implies 1 $] Let $ f\rb{x} \in k\sb{x} $ be irreducible. Suppose there exists $ \lambda \in K $ such that $ f\rb{\lambda} = 0 $. Let $ \Omega $ be a splitting field of $ f\rb{x} \in K\sb{x} $. Let $ \mu \in \Omega $ be a root of $ f $. There exists a unique $ \sigma \in Em_k\rb{k\rb{\lambda}, \Omega} $ such that $ \sigma\rb{\lambda} = \mu $.
$$
\begin{tikzcd}
K & \\
F = k\rb{\lambda} \arrow[dash]{u} \arrow{r}{\sigma} & \sigma\rb{F} \subset \Omega \ni \mu \\
k \arrow[dash]{u} \arrow[dash]{ur} &
\end{tikzcd}.
$$
By corollary, $ \sigma\rb{F} \subset K $, so $ \mu \in K $.
\end{itemize}
\end{proof}

(Exercise: prove that any two splitting fields of $ f \in k\sb{x} $ are $ k $-isomorphic, not necessarily in a unique way)

\lecture{13}{Thursday}{07/02/19}

\begin{proposition}
Let $ k \subset L $ be a field extension. Then there exists a tower $ k \subset L \subset K $ such that $ k \subset K $ is normal.
\end{proposition}

\begin{proof}
We use normal if and only if splitting field. Pick $ \lambda_1, \dots, \lambda_n \in L $ such that $ L = k\rb{\lambda_1, \dots, \lambda_n} $. Let $ f_i \in k\sb{x} $ be the minimal polynomial of $ \lambda_i $ over $ k $. Let $ K $ be the splitting field of $ f = \prod_{i = 1}^n f_i \in L\sb{x} $. Claim that $ K $ is the splitting field of $ f $ over $ k $. Key point is argue that $ K $ is generated by the roots of $ f $ over $ k $.
\end{proof}

\pagebreak

\section{Separable extensions}

\begin{definition}
A polynomial $ f \in k\sb{x} $ is \textbf{separable} if it has $ n = \deg\rb{f} $ distinct roots in any field $ k \subset K $ such that $ f \in K\sb{x} $ splits completely.
\end{definition}

\begin{remark*}
It is not completely obvious that this definition is independent of $ K $. To see this, use the fact that any two splitting fields are isomorphic.
\end{remark*}

\begin{example*}
\hfill
\begin{itemize}
\item Let $ k = \F_p = \Z / p\Z $. Then $ x^p - a = \rb{x - a}^p $ is not separable, since in characteristic $ p $, $ \rb{a + b}^p = a^p + b^p $.
\item Let $ k = \F_p\rb{t} $. Then $ x^p - t $ is an irreducible polynomial. Why? Let
$$ K = \dfrac{\F_p\rb{t}\sb{u}}{\ab{u^p - t}} = \F_p\rb{u}. $$
In $ K\sb{x} $, $ x^p - t = \rb{x - u}^p $.
\end{itemize}
\end{example*}

For all $ k $, define the \textbf{derivation} as
$$ \function[D]{k\sb{x}}{k\sb{x}}{x^n}{nx^{n - 1}}, $$
and extend linearly to all of $ k\sb{x} $. The following are some properties.
\begin{itemize}
\item $ D $ is $ k $-linear, that is for all $ \lambda, \mu \in k $, for all $ f, g \in k\sb{x} $,
$$ D\rb{\lambda f + \mu g} = \lambda Df + \mu Dg. $$
\item Leibnitz rule. For all $ f, g \in k\sb{x} $,
$$ D\rb{fg} = fDg + gDf. $$
\end{itemize}
Most important thing to know in characteristic $ p $, if $ p \mid n $ then $ Dx^n = nx^{n - 1} = 0 $. If $ Df = 0 $ that does not mean $ f $ is constant. This just means that there exists $ h \in k\sb{x} $ such that $ f\rb{x} = h\rb{x^p} $.

\begin{proposition}
\label{prop:separable}
$ f\rb{x} \in k\sb{x} $ is separable if and only if $ \gcd\rb{f, Df} = 1 $.
\end{proposition}

In $ \R\sb{x} $, $ f $ is inseparable if and only if there exists a multiple root, a critical point, which is a root of $ Df $.

\lecture{14}{Friday}{08/02/19}

\begin{lemma}
Let $ f, g \in k\sb{x} $ and $ c = \gcd\rb{f, g} $ in $ k\sb{x} $. Let $ k \subset L $ be an extension. Then $ c = \gcd\rb{f, g} $ in $ L\sb{x} $.
\end{lemma}

\begin{proof}
Indeed, if $ c \mid f $, $ c \mid g $ in $ k\sb{x} $ then also in $ L\sb{x} $. We also know that there exists $ \phi, \psi \in k\sb{x} $ such that
\begin{equation}
\label{eq:7}
f\phi + g\psi = c
\end{equation}
in $ k\sb{x} $, and hence also in $ L\sb{x} $. Suppose that $ u \mid f $, $ u \mid g $ in $ L\sb{x} $, so $ u \mid c $ in $ L\sb{x} $ by $ \rb{\ref{eq:7}} $.
\end{proof}

\begin{proof}[Proof of Proposition \ref{prop:separable}]
Let $ k \subset L $ be any field where $ f $ splits completely. We can do the proof in $ L\sb{x} $. That is, we may assume that $ f $ splits completely, so $ f\rb{x} = \prod_i \rb{x - \lambda i} $.
\begin{itemize}
\item[$ \impliedby $] Assume for a contradiction that $ f $ is not separable then $ f\rb{x} = \rb{x - \lambda}^2g\rb{x} $.
$$ Df\rb{x} = 2\rb{x - \lambda}g\rb{x} + \rb{x - \lambda}^2Dg\rb{x} = \rb{x - \lambda}\rb{2y\rb{x} + \rb{x - \lambda}Dg\rb{x}}. $$
That is, $ \rb{x - \lambda} \mid f $ and $ \rb{x - \lambda} \mid Df $, so $ \gcd\rb{f, Df} \ne 1 $.
\item[$ \implies $] For all $ i \ne j $, $ \lambda_i \ne \lambda_j $.
$$ Df = \sum_{i = 1}^j \rb{\prod_{j \ne i} \rb{x - \lambda_j}}. $$
Claim that for all $ i $, $ \rb{x - \lambda_i} \nmid Df $. I hope you see this. This shows $ \gcd\rb{f, Df} = 1 $.
\end{itemize}
\end{proof}

\begin{theorem}
$ f \in k\sb{x} $ irreducible is inseparable if and only if
\begin{itemize}
\item $ ch\rb{k} = p > 0 $, and
\item there exists $ h \in k\sb{x} $ such that $ f\rb{x} = h\rb{x^p} $.
\end{itemize}
\end{theorem}

\begin{proof}
Indeed $ f $ is inseparable if and only if $ \gcd\rb{f, Df} \ne 1 $, if and only if $ Df = 0 $, since $ f $ is irreducible so $ \gcd\rb{f, Df} \ne 1 $ if and only if $ f \mid Df $, and $ \deg\rb{Df} < \deg\rb{f} $.
\end{proof}

\begin{definition}
A field $ k $ in $ ch\rb{k} = p > 0 $ is \textbf{perfect} if for all $ a \in k $ there exists $ b \in k $ such that $ b^p = a $.
\end{definition}

\begin{proposition}
If $ k $ is perfect then $ f \in k\sb{x} $ is irreducible gives that $ f\rb{x} $ is separable.
\end{proposition}

\begin{proof}
If $ f $ were inseparable then $ f\rb{x} = h\rb{x^p} $. For all $ i $, find $ b_i^p = a_i $,
$$ h\rb{x} = x^n + a_1x^{n - 1} \dots + a_n = x^n + b_1^px^{n - 1} \dots + b_n^p. $$
Thus
$$ f\rb{x} = h\rb{x^p} = \rb{x^n + b_1x^{n - 1} + \dots + b_n}^p, $$
so $ f $ is not irreducible.
\end{proof}

\begin{example*}
All finite fields are perfect. Suppose $ F $ is a finite field. Then $ ch\rb{F} = p > 0 $ so $ \F_p \subset F $ therefore $ \sb{\F : \F_p} = n < 0 $. $ \dim_{\F_p}\rb{F} = n < \infty $, so $ F \cong \rb{\F_p}^n $ as a vector space over $ \F_p $ gives that $ F $ has $ p^n $ elements. The group $ F^\times = F \setminus \cb{0} $ has $ p^n - 1 $ elements. So for all $ a \in F^\times $, $ a^{p^n - 1} = 1 $. For all $ a \in F $, $ a^{p^n} = a $, so
$$ \rb{a^{p^{n - 1}}}^p = a, $$
and this shows $ F $ is perfect.
\end{example*}

\begin{definition}
Consider $ k \subset K $. An element $ a \in L $ is \textbf{separable} over $ k $ if the minimal polynomial $ f\rb{x} \in k\sb{x} $ of $ a $ is a separable polynomial.
\end{definition}

\pagebreak

\section{Separable degree}

\begin{definition}
Let $ K \subset L $. Then
$$ \sb{L : K}_s = \abs{Em_K\rb{L, \Omega}}, $$
where $ K \subset L \subset \Omega $ is any such that $ K \subset \Omega $ is normal.
\end{definition}

\begin{proposition}
$ \sb{L : K}_s $ does not depend on $ \Omega $.
\end{proposition}

\begin{theorem}
\hfill
\begin{itemize}
\item $ k \subset L $ is separable if and only if $ \sb{L : K}_s = \sb{L : K} $.
\item For all $ k \subset K \subset L $, $ k \subset K $ and $ K \subset L $ are separable gives that $ k \subset L $ is separable.
\item $ k \subset L $ is separable if and only if for all $ \lambda \in L $, $ \lambda $ is separable over $ k $.
\end{itemize}
\end{theorem}

\lecture{15}{Tuesday}{12/02/19}

Lecture 15 is a problem class.

\lecture{16}{Thursday}{14/02/19}

Lecture 16 is a problem class.

\end{document}