\def\module{M4P32 Number Theory: Elliptic Curves}
\def\lecturer{Prof Toby Gee}
\def\term{Autumn 2018}

\def\thm{section}

\documentclass{article}

% Packages

\usepackage{amssymb}
\usepackage{amsthm}
\usepackage[UKenglish]{babel}
\usepackage{commath}
\usepackage{enumitem}
\usepackage{etoolbox}
\usepackage{fancyhdr}
\usepackage[margin=1in]{geometry}
\usepackage{graphicx}
\usepackage[hidelinks]{hyperref}
\usepackage[utf8]{inputenc}
\usepackage{listings}
\usepackage{mathtools}
\usepackage{stmaryrd}
\usepackage{tikz-cd}
\usepackage{csquotes}

% Formatting

\addto\captionsUKenglish{\renewcommand{\abstractname}{Syllabus}}
\delimitershortfall5pt
\ifx\thm\undefined\newtheorem{n}{}\else\newtheorem{n}{}[\thm]\fi
\newcommand\newoperator[1]{\ifcsdef{#1}{\cslet{#1}{\relax}}{}\csdef{#1}{\operatorname{#1}}}
\setlength{\parindent}{0cm}

% Environments

\theoremstyle{plain}
\newtheorem{algorithm}[n]{Algorithm}
\newtheorem*{algorithm*}{Algorithm}
\newtheorem{algorithm**}{Algorithm}
\newtheorem{conjecture}[n]{Conjecture}
\newtheorem*{conjecture*}{Conjecture}
\newtheorem{conjecture**}{Conjecture}
\newtheorem{corollary}[n]{Corollary}
\newtheorem*{corollary*}{Corollary}
\newtheorem{corollary**}{Corollary}
\newtheorem{lemma}[n]{Lemma}
\newtheorem*{lemma*}{Lemma}
\newtheorem{lemma**}{Lemma}
\newtheorem{proposition}[n]{Proposition}
\newtheorem*{proposition*}{Proposition}
\newtheorem{proposition**}{Proposition}
\newtheorem{theorem}[n]{Theorem}
\newtheorem*{theorem*}{Theorem}
\newtheorem{theorem**}{Theorem}

\theoremstyle{definition}
\newtheorem{aim}[n]{Aim}
\newtheorem*{aim*}{Aim}
\newtheorem{aim**}{Aim}
\newtheorem{axiom}[n]{Axiom}
\newtheorem*{axiom*}{Axiom}
\newtheorem{axiom**}{Axiom}
\newtheorem{condition}[n]{Condition}
\newtheorem*{condition*}{Condition}
\newtheorem{condition**}{Condition}
\newtheorem{definition}[n]{Definition}
\newtheorem*{definition*}{Definition}
\newtheorem{definition**}{Definition}
\newtheorem{example}[n]{Example}
\newtheorem*{example*}{Example}
\newtheorem{example**}{Example}
\newtheorem{exercise}[n]{Exercise}
\newtheorem*{exercise*}{Exercise}
\newtheorem{exercise**}{Exercise}
\newtheorem{fact}[n]{Fact}
\newtheorem*{fact*}{Fact}
\newtheorem{fact**}{Fact}
\newtheorem{goal}[n]{Goal}
\newtheorem*{goal*}{Goal}
\newtheorem{goal**}{Goal}
\newtheorem{law}[n]{Law}
\newtheorem*{law*}{Law}
\newtheorem{law**}{Law}
\newtheorem{plan}[n]{Plan}
\newtheorem*{plan*}{Plan}
\newtheorem{plan**}{Plan}
\newtheorem{problem}[n]{Problem}
\newtheorem*{problem*}{Problem}
\newtheorem{problem**}{Problem}
\newtheorem{question}[n]{Question}
\newtheorem*{question*}{Question}
\newtheorem{question**}{Question}
\newtheorem{warning}[n]{Warning}
\newtheorem*{warning*}{Warning}
\newtheorem{warning**}{Warning}
\newtheorem{acknowledgements}[n]{Acknowledgements}
\newtheorem*{acknowledgements*}{Acknowledgements}
\newtheorem{acknowledgements**}{Acknowledgements}
\newtheorem{annotations}[n]{Annotations}
\newtheorem*{annotations*}{Annotations}
\newtheorem{annotations**}{Annotations}
\newtheorem{assumption}[n]{Assumption}
\newtheorem*{assumption*}{Assumption}
\newtheorem{assumption**}{Assumption}
\newtheorem{conclusion}[n]{Conclusion}
\newtheorem*{conclusion*}{Conclusion}
\newtheorem{conclusion**}{Conclusion}
\newtheorem{claim}[n]{Claim}
\newtheorem*{claim*}{Claim}
\newtheorem{claim**}{Claim}
\newtheorem{notation}[n]{Notation}
\newtheorem*{notation*}{Notation}
\newtheorem{notation**}{Notation}
\newtheorem{note}[n]{Note}
\newtheorem*{note*}{Note}
\newtheorem{note**}{Note}
\newtheorem{remark}[n]{Remark}
\newtheorem*{remark*}{Remark}
\newtheorem{remark**}{Remark}

% Lectures

\newcommand{\lecture}[3]{ % Lecture
  \marginpar{
    Lecture #1 \\
    #2 \\
    #3
  }
}

% Blackboard

\renewcommand{\AA}{\mathbb{A}} % Blackboard A
\newcommand{\BB}{\mathbb{B}}   % Blackboard B
\newcommand{\CC}{\mathbb{C}}   % Blackboard C
\newcommand{\DD}{\mathbb{D}}   % Blackboard D
\newcommand{\EE}{\mathbb{E}}   % Blackboard E
\newcommand{\FF}{\mathbb{F}}   % Blackboard F
\newcommand{\GG}{\mathbb{G}}   % Blackboard G
\newcommand{\HH}{\mathbb{H}}   % Blackboard H
\newcommand{\II}{\mathbb{I}}   % Blackboard I
\newcommand{\JJ}{\mathbb{J}}   % Blackboard J
\newcommand{\KK}{\mathbb{K}}   % Blackboard K
\newcommand{\LL}{\mathbb{L}}   % Blackboard L
\newcommand{\MM}{\mathbb{M}}   % Blackboard M
\newcommand{\NN}{\mathbb{N}}   % Blackboard N
\newcommand{\OO}{\mathbb{O}}   % Blackboard O
\newcommand{\PP}{\mathbb{P}}   % Blackboard P
\newcommand{\QQ}{\mathbb{Q}}   % Blackboard Q
\newcommand{\RR}{\mathbb{R}}   % Blackboard R
\renewcommand{\SS}{\mathbb{S}} % Blackboard S
\newcommand{\TT}{\mathbb{T}}   % Blackboard T
\newcommand{\UU}{\mathbb{U}}   % Blackboard U
\newcommand{\VV}{\mathbb{V}}   % Blackboard V
\newcommand{\WW}{\mathbb{W}}   % Blackboard W
\newcommand{\XX}{\mathbb{X}}   % Blackboard X
\newcommand{\YY}{\mathbb{Y}}   % Blackboard Y
\newcommand{\ZZ}{\mathbb{Z}}   % Blackboard Z

% Brackets

\renewcommand{\eval}[1]{\left. #1 \right|}          % Evaluation
\newcommand{\br}{\del}                              % Brackets
\newcommand{\abr}[1]{\left\langle #1 \right\rangle} % Angle brackets
\newcommand{\fbr}[1]{\left\lfloor #1 \right\rfloor} % Floor brackets
\newcommand{\lbr}[1]{\left\lfloor #1 \right\rfloor} % Ceiling brackets
\newcommand{\st}{\ \middle| \ }                     % Such that

% Calligraphic

\newcommand{\AAA}{\mathcal{A}} % Calligraphic A
\newcommand{\BBB}{\mathcal{B}} % Calligraphic B
\newcommand{\CCC}{\mathcal{C}} % Calligraphic C
\newcommand{\DDD}{\mathcal{D}} % Calligraphic D
\newcommand{\EEE}{\mathcal{E}} % Calligraphic E
\newcommand{\FFF}{\mathcal{F}} % Calligraphic F
\newcommand{\GGG}{\mathcal{G}} % Calligraphic G
\newcommand{\HHH}{\mathcal{H}} % Calligraphic H
\newcommand{\III}{\mathcal{I}} % Calligraphic I
\newcommand{\JJJ}{\mathcal{J}} % Calligraphic J
\newcommand{\KKK}{\mathcal{K}} % Calligraphic K
\newcommand{\LLL}{\mathcal{L}} % Calligraphic L
\newcommand{\MMM}{\mathcal{M}} % Calligraphic M
\newcommand{\NNN}{\mathcal{N}} % Calligraphic N
\newcommand{\OOO}{\mathcal{O}} % Calligraphic O
\newcommand{\PPP}{\mathcal{P}} % Calligraphic P
\newcommand{\QQQ}{\mathcal{Q}} % Calligraphic Q
\newcommand{\RRR}{\mathcal{R}} % Calligraphic R
\newcommand{\SSS}{\mathcal{S}} % Calligraphic S
\newcommand{\TTT}{\mathcal{T}} % Calligraphic T
\newcommand{\UUU}{\mathcal{U}} % Calligraphic U
\newcommand{\VVV}{\mathcal{V}} % Calligraphic V
\newcommand{\WWW}{\mathcal{W}} % Calligraphic W
\newcommand{\XXX}{\mathcal{X}} % Calligraphic X
\newcommand{\YYY}{\mathcal{Y}} % Calligraphic Y
\newcommand{\ZZZ}{\mathcal{Z}} % Calligraphic Z

% Fraktur

\newcommand{\aaa}{\mathfrak{a}}   % Fraktur a
\newcommand{\bbb}{\mathfrak{b}}   % Fraktur b
\newcommand{\ccc}{\mathfrak{c}}   % Fraktur c
\newcommand{\ddd}{\mathfrak{d}}   % Fraktur d
\newcommand{\eee}{\mathfrak{e}}   % Fraktur e
\newcommand{\fff}{\mathfrak{f}}   % Fraktur f
\renewcommand{\ggg}{\mathfrak{g}} % Fraktur g
\newcommand{\hhh}{\mathfrak{h}}   % Fraktur h
\newcommand{\iii}{\mathfrak{i}}   % Fraktur i
\newcommand{\jjj}{\mathfrak{j}}   % Fraktur j
\newcommand{\kkk}{\mathfrak{k}}   % Fraktur k
\renewcommand{\lll}{\mathfrak{l}} % Fraktur l
\newcommand{\mmm}{\mathfrak{m}}   % Fraktur m
\newcommand{\nnn}{\mathfrak{n}}   % Fraktur n
\newcommand{\ooo}{\mathfrak{o}}   % Fraktur o
\newcommand{\ppp}{\mathfrak{p}}   % Fraktur p
\newcommand{\qqq}{\mathfrak{q}}   % Fraktur q
\newcommand{\rrr}{\mathfrak{r}}   % Fraktur r
\newcommand{\sss}{\mathfrak{s}}   % Fraktur s
\newcommand{\ttt}{\mathfrak{t}}   % Fraktur t
\newcommand{\uuu}{\mathfrak{u}}   % Fraktur u
\newcommand{\vvv}{\mathfrak{v}}   % Fraktur v
\newcommand{\www}{\mathfrak{w}}   % Fraktur w
\newcommand{\xxx}{\mathfrak{x}}   % Fraktur x
\newcommand{\yyy}{\mathfrak{y}}   % Fraktur y
\newcommand{\zzz}{\mathfrak{z}}   % Fraktur z

% Geometry

\newcommand{\CP}{\mathbb{CP}}                                              % Complex projective space
\newcommand{\iintd}[4]{\iint_{#1} \, #2 \, \dif #3 \, \dif #4}             % Double integral
\newcommand{\RP}{\mathbb{RP}}                                              % Real projective space
\newcommand{\intd}[4]{\int_{#1}^{#2} \, #3 \, \dif #4}                     % Single integral
\newcommand{\iiintd}[5]{\iint_{#1} \, #2 \, \dif #3 \, \dif #4 \, \dif #5} % Triple integral

% Logic

\newcommand{\iffb}[2]{\br{#1 \leftrightarrow #2}} % Biconditional
\newcommand{\andb}[2]{\br{#1 \land #2}}           % Conjunction
\newcommand{\orb}[2]{\br{#1 \lor #2}}             % Disjunction
\newcommand{\nib}[2]{\br{#1 \notin #2}}           % Element of
\newcommand{\eqb}[2]{\br{#1 = #2}}                % Equal to
\newcommand{\teb}[1]{\br{\exists #1}}             % Existential quantifier
\newcommand{\impb}[2]{\br{#1 \rightarrow #2}}     % Implication
\newcommand{\ltb}[2]{\br{#1 < #2}}                % Less than
\newcommand{\leb}[2]{\br{#1 \le #2}}              % Less than or equal to
\newcommand{\notb}[1]{\br{\neg #1}}               % Negation
\newcommand{\inb}[2]{\br{#1 \in #2}}              % Not element of
\newcommand{\neb}[2]{\br{#1 \ne #2}}              % Not equal to
\newcommand{\subb}[2]{\br{#1 \subseteq #2}}       % Subset
\newcommand{\fab}[1]{\br{\forall #1}}             % Universal quantifier

% Maps

\newcommand{\bijection}[7][]{    % Bijection
  \ifx &#1&
    \begin{array}{rcl}
      #2 & \longleftrightarrow & #3 \\
      #4 & \longmapsto         & #5 \\
      #6 & \longmapsfrom       & #7
    \end{array}
  \else
    \begin{array}{ccrcl}
      #1 & : & #2 & \longrightarrow & #3 \\
         &   & #4 & \longmapsto     & #5 \\
         &   & #6 & \longmapsfrom   & #7
    \end{array}
  \fi
}
\newcommand{\correspondence}[2]{ % Correspondence
  \cbr{
    \begin{array}{c}
      #1
    \end{array}
  }
  \qquad
  \leftrightsquigarrow
  \qquad
  \cbr{
    \begin{array}{c}
      #2
    \end{array}
  }
}
\newcommand{\function}[5][]{     % Function
  \ifx &#1&
    \begin{array}{rcl}
      #2 & \longrightarrow & #3 \\
      #4 & \longmapsto     & #5
    \end{array}
  \else
    \begin{array}{ccrcl}
      #1 & : & #2 & \longrightarrow & #3 \\
         &   & #4 & \longmapsto     & #5
    \end{array}
  \fi
}
\newcommand{\functions}[7][]{    % Functions
  \ifx &#1&
    \begin{array}{rcl}
      #2 & \longrightarrow & #3 \\
      #4 & \longmapsto     & #5 \\
      #6 & \longmapsto     & #7
    \end{array}
  \else
    \begin{array}{ccrcl}
      #1 & : & #2 & \longrightarrow & #3 \\
         &   & #4 & \longmapsto     & #5 \\
         &   & #6 & \longmapsto     & #7
    \end{array}
  \fi
}

% Matrices

\newcommand{\onebytwo}[2]{      % One by two matrix
  \begin{pmatrix}
    #1 & #2
  \end{pmatrix}
}
\newcommand{\onebythree}[3]{    % One by three matrix
  \begin{pmatrix}
    #1 & #2 & #3
  \end{pmatrix}
}
\newcommand{\twobyone}[2]{      % Two by one matrix
  \begin{pmatrix}
    #1 \\
    #2
  \end{pmatrix}
}
\newcommand{\twobytwo}[4]{      % Two by two matrix
  \begin{pmatrix}
    #1 & #2 \\
    #3 & #4
  \end{pmatrix}
}
\newcommand{\threebyone}[3]{    % Three by one matrix
  \begin{pmatrix}
    #1 \\
    #2 \\
    #3
  \end{pmatrix}
}
\newcommand{\threebythree}[9]{  % Three by three matrix
  \begin{pmatrix}
    #1 & #2 & #3 \\
    #4 & #5 & #6 \\
    #7 & #8 & #9
  \end{pmatrix}
}
\newcommand{\twobytwosmall}[4]{ % Two by two small matrix
  \begin{psmallmatrix}
    #1 & #2 \\
    #3 & #4
  \end{psmallmatrix}
}

% Number theory

\renewcommand{\symbol}[2]{\br{\tfrac{#1}{#2}}} % Power residue symbol
\newcommand{\unit}[1]{\br{\ZZ / #1\ZZ}^\times} % Unit group

% Operators

\newoperator{ab}    % Abelian
\newoperator{AG}    % Affine geometry
\newoperator{alg}   % Algebraic
\newoperator{Ann}   % Annihilator
\newoperator{area}  % Area
\newoperator{Aut}   % Automorphism
\newoperator{card}  % Cardinality
\newoperator{ch}    % Characteristic
\newoperator{Cl}    % Class
\newoperator{col}   % Column
\newoperator{Corr}  % Correspondence
\newoperator{diam}  % Diameter
\newoperator{Disc}  % Discriminant
\newoperator{dom}   % Domain
\newoperator{Em}    % Embedding
\newoperator{End}   % Endomorphism
\newoperator{fin}   % Finite
\newoperator{Fix}   % Fixed
\newoperator{Frac}  % Fraction
\newoperator{Frob}  % Frobenius
\newoperator{Fun}   % Function
\newoperator{Gal}   % Galois
\newoperator{GL}    % General linear
\newoperator{Ham}   % Hamming
\newoperator{Homeo} % Homeomorphism
\newoperator{Hom}   % Homomorphism
\newoperator{id}    % Identity
\newoperator{Im}    % Image
\newoperator{Ind}   % Index
\newoperator{Ker}   % Kernel
\newoperator{lcm}   % Least common multiple
\newoperator{Mat}   % Matrix
\newoperator{mult}  % Multiplicity
\newoperator{new}   % New
\newoperator{Nm}    % Norm
\newoperator{old}   % Old
\newoperator{ord}   % Order
\newoperator{Pay}   % Payley
\newoperator{PG}    % Projective geometry
\newoperator{PGL}   % Projective general linear
\newoperator{PSL}   % Projective special linear
\newoperator{rad}   % Radical
\newoperator{ran}   % Range
\newoperator{Res}   % Residue
\newoperator{rk}    % Rank
\newoperator{Re}    % Real
\newoperator{row}   % Row
\newoperator{sgn}   % Sign
\newoperator{Sing}  % Singular
\newoperator{sp}    % Span
\newoperator{SL}    % Special linear
\newoperator{SO}    % Special orthogonal
\newoperator{Spec}  % Spectrum
\newoperator{Stab}  % Stabiliser
\newoperator{star}  % Star
\newoperator{srg}   % Strongly regular graph
\newoperator{Sym}   % Symmetric
\newoperator{tors}  % Torsion
\newoperator{Tr}    % Trace
\newoperator{vol}   % Volume
\newoperator{wt}    % Weight

% Roman

\newcommand{\A}{\mathrm{A}}   % Roman A
\newcommand{\B}{\mathrm{B}}   % Roman B
\newcommand{\C}{\mathrm{C}}   % Roman C
\newcommand{\D}{\mathrm{D}}   % Roman D
\newcommand{\E}{\mathrm{E}}   % Roman E
\newcommand{\F}{\mathrm{F}}   % Roman F
\newcommand{\G}{\mathrm{G}}   % Roman G
\renewcommand{\H}{\mathrm{H}} % Roman H
\newcommand{\I}{\mathrm{I}}   % Roman I
\newcommand{\J}{\mathrm{J}}   % Roman J
\newcommand{\K}{\mathrm{K}}   % Roman K
\renewcommand{\L}{\mathrm{L}} % Roman L
\newcommand{\M}{\mathrm{M}}   % Roman M
\newcommand{\N}{\mathrm{N}}   % Roman N
\renewcommand{\O}{\mathrm{O}} % Roman O
\renewcommand{\P}{\mathrm{P}} % Roman P
\newcommand{\Q}{\mathrm{Q}}   % Roman Q
\newcommand{\R}{\mathrm{R}}   % Roman R
\renewcommand{\S}{\mathrm{S}} % Roman S
\newcommand{\T}{\mathrm{T}}   % Roman T
\newcommand{\U}{\mathrm{U}}   % Roman U
\newcommand{\V}{\mathrm{V}}   % Roman V
\newcommand{\W}{\mathrm{W}}   % Roman W
\newcommand{\X}{\mathrm{X}}   % Roman X
\newcommand{\Y}{\mathrm{Y}}   % Roman Y
\newcommand{\Z}{\mathrm{Z}}   % Roman Z

\renewcommand{\a}{\mathrm{a}} % Roman a
\renewcommand{\b}{\mathrm{b}} % Roman b
\renewcommand{\c}{\mathrm{c}} % Roman c
\renewcommand{\d}{\mathrm{d}} % Roman d
\newcommand{\e}{\mathrm{e}}   % Roman e
\newcommand{\f}{\mathrm{f}}   % Roman f
\newcommand{\g}{\mathrm{g}}   % Roman g
\newcommand{\h}{\mathrm{h}}   % Roman h
\renewcommand{\i}{\mathrm{i}} % Roman i
\renewcommand{\j}{\mathrm{j}} % Roman j
\renewcommand{\k}{\mathrm{k}} % Roman k
\renewcommand{\l}{\mathrm{l}} % Roman l
\newcommand{\m}{\mathrm{m}}   % Roman m
\renewcommand{\n}{\mathrm{n}} % Roman n
\renewcommand{\o}{\mathrm{o}} % Roman o
\newcommand{\p}{\mathrm{p}}   % Roman p
\newcommand{\q}{\mathrm{q}}   % Roman q
\renewcommand{\r}{\mathrm{r}} % Roman r
\newcommand{\s}{\mathrm{s}}   % Roman s
\renewcommand{\t}{\mathrm{t}} % Roman t
\renewcommand{\u}{\mathrm{u}} % Roman u
\renewcommand{\v}{\mathrm{v}} % Roman v
\newcommand{\w}{\mathrm{w}}   % Roman w
\newcommand{\x}{\mathrm{x}}   % Roman x
\newcommand{\y}{\mathrm{y}}   % Roman y
\newcommand{\z}{\mathrm{z}}   % Roman z

% Tikz

\tikzset{
  arrow symbol/.style={"#1" description, allow upside down, auto=false, draw=none, sloped},
  subset/.style={arrow symbol={\subset}},
  cong/.style={arrow symbol={\cong}}
}

% Fancy header

\pagestyle{fancy}
\lhead{\module}
\rhead{\nouppercase{\leftmark}}

% Make title

\title{\module}
\author{Lectured by \lecturer \\ Typed by David Kurniadi Angdinata}
\date{\term}

\begin{document}

% Title page

\maketitle

\cover

\vfill

\begin{abstract}
\noindent\syllabus
\end{abstract}

\pagebreak

% Contents page

\tableofcontents

\pagebreak

\setcounter{section}{-1}

\section{Introduction}

\lecture{1}{Thursday}{04/10/18}

The following is an outline.
\begin{itemize}
\item Introduction
\item Conics
\begin{itemize}
\item The $ p $-adic numbers
\item Basic algebraic geometry
\item Plane curves
\item The Hasse principle
\end{itemize}
\item Cubics
\begin{itemize}
\item Definitions
\item Elliptic curves over $ \Q_p $
\item Elliptic curves over $ \Q $
\item The Mordell-Weil theorem
\end{itemize}
\end{itemize}
The following are references.
\begin{itemize}
\item J W S Cassels, Lectures on elliptic curves, 1991
\item J H Silverman, The arithmetic of elliptic curves, 1986
\item J H Silverman and J Tate, Rational points on elliptic curves, 1992
\end{itemize}
The theme of the course will be studying polynomial equations over $ \Z $ or $ \Q $.

\begin{example*}
For $ x^7 + y^7 + z^7 = 1 $, what are the solutions with $ x, y, z \in \Q $? Answer is hard. However $ x^2 - 1 = 0 $ is easy. In fact any equation in one variable is easy to solve over $ \Q $. For example, $ 3x^5 - 9x^3 + x^2 + 148 / 81 = 0 $ iff $ 243x^5 - 729x^3 + 81x^2 + 148 = 0 $. Letting $ x = r / s $ for $ \rb{r, s} = 1 $ and $ s \ge 1 $, such as $ r = -2 $ and $ s = 5 $ in $ -2 / 5 $, gives $ 243r^5 - 729r^3s^2 + 81r^2s^3 + 148s^5 = 0 $. So $ r^2 \mid 148s^5 $, but $ \rb{r, s} = 1 $, so $ r^2 \mid 148 $. Similarly $ s^2 \mid 243 $. Now check the finitely many possibilities to get $ x = 2 / 3 $ as the only solution in $ \Q $.
\end{example*}

If $ k $ is a field, such as $ \Q $, $ \R $, $ \C $, $ \F_p $, $ \Q_p $, we write $ k\sb{x_1, \dots, x_n} $ for the polynomial ring in $ n $ variables $ x_1, \dots, x_n $. A \textbf{monomial} is an expression $ x_1^{\alpha_1}, \dots, x_n^{\alpha_n} $ for $ a_i \in \Z_{> 0} $. The \textbf{degree} of $ x_1^{\alpha_1}, \dots, x_n^{\alpha_n} $ is $ a_1 + \dots + a_n $. An element of $ k\sb{x_1, \dots, x_n} $ is just a finite sum of multiples of monomials with coefficients in $ k $. Its degree is the largest degree of any monomial occurring in it. For example, $ 5x_{10}^3 + x_2x_3^{10} - \rb{2 / 3}x_1^7 $ has degree $ 11 $. Typically we will be looking at the case of two variables $ x_1, x_2 $, which we will usually call $ x, y $.

\begin{theorem}[Falting's theorem]
The general equation in two variables of degree greater than four over $ \Q $ is known to only have finitely many solutions in $ \Q $.
\end{theorem}

\begin{example*}
$ x^4 + y^4 = 17 $ only has $ \rb{\pm 1, \pm 2} $ and $ \rb{\pm 2, \pm 1} $ as solutions, which is proved only in 2001. $ \rb{x^{100} + 5y^{100} - 7}\rb{x - y} = 0 $ is not a general equation and has infinitely many solutions.
\end{example*}

\begin{example*}
$ x^2 + y^2 = -1 $ has no solutions in $ \Q $. $ x^2 + y^2 = 0 $ has finitely many solutions in $ \Q $ $ \rb{x = y = 0} $. $ x^2 + y^2 = 1 $ has infinitely many solutions in $ \Q $. Let $ x = a / c $ and $ y = b / c $ for $ a, b, c \in \Z $ gives Pythagorean triples $ a^2 + b^2 = c^2 $, such as $ 3^2 + 4^2 = 5^2 $, $ 5^2 + 12^2 = 13^2 $, etc.
\end{example*}

\begin{example*}
$ x^2 + y^2 = 3 $ has solutions in $ \R $ but no solutions in $ \Q $. Equivalently, $ a^2 + b^2 = 3c^2 $ has no solutions with $ a, b, c \in \Z $ except $ a = b = c = 0 $. Suppose we have a solution. Then $ a^2 + b^2 \equiv 0 \mod 3 $. A fact is that if $ n \in \Z $, then $ n^2 \equiv 0 \mod 3 $ if $ 3 \mid n $, or $ n^2 \equiv 1 \mod 3 $ if $ 3 \nmid n $. So $ a \equiv b \equiv 0 \mod 3 $, or $ 3 \mid a, b $. Write $ a = 2A $ and $ b = 3B $. Then $ \rb{3A}^2 + \rb{3B}^2 = 3c^2 $ iff $ 9\rb{A^2 + B^2} = 3c^2 $, iff $ 3\rb{A^2 + B^2} = c^2 $. So $ 3 \mid c^2 $ and $ 3 \mid c $. Write $ c = 3C $. Then $ 3\rb{A^2 + B^2} = \rb{3C}^2 $ iff $ 3\rb{A^2 + B^2} = 9C^2 $, iff $ A^2 + B^2 = 3C^2 $. Thus $ a = b = c = 0 $.
\end{example*}

$ x^2 + y^2 = -1 $ has an obstruction in $ \R $. $ x^2 + y^2 = 3 $ has an obstruction in $ \Q_3 $. Hasse principle will tell us that for general equations of degree two in $ x, y $, there are either infinitely many solutions in $ \Q $ or no solutions, and furthermore either no solutions in $ \R $ or no solutions in $ \Q_p $ for some prime $ p $.

\lecture{2}{Friday}{05/10/18}

We study plane conics and plane cubics. Plane refers to 2 variables $ x, y $. Conic refers to degree two while cubic refers to degree three.

\begin{example*}
$ x^2 + 2y^2 = 6 $ has a rational solution $ \rb{2, 1} $. Drawing lines at $ \rb{2, 1} $ with rational slope $ y - 1 = m\rb{x - 2} $ will get all rational solutions by intersecting with $ x^2 + 2y^2 = 6 $. Then $ x^2 + 2\rb{m\rb{x - 2} + 1}^2 = 6 $ gives $ \rb{2m^2 + 1}x^2 + 4m\rb{1 - 2m}x + 2\rb{2m^2 - 1}^2 = 0 $. The sum of the roots of this equation is $ 4m\rb{2m - 1} / \rb{2m^2 + 1} $. Since $ x = 2 $ is a root, the other root is
$$ x_0 = \dfrac{4m\rb{2m - 1}}{2m^2 + 1} - 2 = \dfrac{4m^2 - 4m - 2}{2m^2 + 1}, \qquad y_0 = m\rb{x_0 - 2} + 1 = m\rb{\dfrac{-4m - 4}{2m^2 + 1}} + 1 = \dfrac{-2m^2 - 4m + 1}{2m^2 + 1}. $$
If $ m \in \Q $, $ \rb{x_0, y_0} \in \Q^2 $ and conversely, which is easy. For example, $ m = 1 $ gives $ \rb{x_0, y_0} = \rb{-2 / 3, -5 / 3} $ and $ m \to \infty $ gives $ \rb{x_0, y_0} \to \rb{2, -1} $. (Exercise: for $ x^2 + y^2 = 1 $, $ x^2 + y^2 = 0 $, $ xy = 0 $, $ x^2 - y^2 = 0 $)
\end{example*}

\pagebreak

\section{The $ p $-adic numbers}

\begin{definition}
A \textbf{norm} on a field $ k $ is a function $ \abs{\cdot} : k \to \R $ such that:
\begin{enumerate}
\item $ \abs{x} \ge 0 $ with equality iff $ x = 0 $,
\item $ \abs{xy} = \abs{x}\abs{y} $, and
\item $ \abs{x + y} \le \abs{x} + \abs{y} $.
\end{enumerate}
\end{definition}

\begin{note*}
$ \abs{1} = \abs{-1} = 1 $ and $ \abs{-x} = \abs{-1}\abs{x} = \abs{x} $.
\end{note*}

\begin{example*}
Let $ k = \Q $ or $ \R $ and $ \abs{x} $ is the absolute value of $ x $, that is $ \abs{x} = x $ if $ x \ge 0 $ and $ \abs{x} = -x $ if $ x < 0 $.
\end{example*}

\begin{example*}
Discrete norm $ \abs{x} = 0 $ if $ x = 0 $ and $ \abs{x} = 1 $ if $ x \ne 0 $.
\end{example*}

\begin{remark}
Any norm on $ k $ defines a metric by $ d\rb{x, y} = \abs{x - y} $. In particular, a norm determines a topology.
\end{remark}

\begin{example*}
The discrete norm determines the discrete topology.
\end{example*}

\begin{definition}
Let $ p $ be a prime number. Let $ a / b \in \Q $ with $ \rb{a, b} = 1 $. Let $ p^n $ be the biggest power of $ p $ dividing $ a / b $, so $ a / b = p^n \rb{c / d} $ with $ \rb{c, p} = \rb{d, p} = 1 $. Then the \textbf{$ p $-adic norm} is $ \abs{a / b}_p = p^{-n} $.
\end{definition}

\begin{example*}
$ \abs{1 / 6}_5 = 1 $, $ \abs{1 / 6}_3 = 3 $. $ p^n \to 0 $ into the $ p $-adic topology as $ n \to \infty $.
\end{example*}

\begin{lemma}
$ \abs{\cdot}_p $ is a norm on $ \Q $.
\end{lemma}

\begin{proof}
\hfill
\begin{enumerate}
\item Trivial.
\item Unique factorisation.
\item Prove $ 3' : \abs{x + y}_p \le \max\rb{\abs{x}_p, \abs{y}_p} $. Without loss of generality, $ x, y, x + y \ne 0 $. Multiplying $ x, y $ by any power of $ p $ does not affect the truth of $ 3' $, so without loss of generality $ x, y \in \Z $. Then we have to show that $ p^r \mid x $ and $ p^r \mid x + y $, which is obvious.
\end{enumerate}
\end{proof}

$ 3' $ is the \textbf{ultrametric inequality}.

\begin{definition}
If $ \abs{\cdot} $ satisfies $ 3' $, we say that $ \abs{\cdot} $ is \textbf{nonarchimedean}. Otherwise $ \abs{\cdot}_p $ is \textbf{archimedean}.
\end{definition}

Say that two norms $ \abs{\cdot}_1 $, $ \abs{\cdot}_2 $ on a field $ k $ are equivalent if $ \abs{\cdot}_1 = \abs{\cdot}_2^\alpha $ for some $ \alpha > 0 $. If two norms are equivalent, they define the same topology. \textbf{Ostrowski's theorem} states that up to equivalence, the only norms on $ \Q $ are the usual archimedean norm, the discrete norm, and the $ p $-adic norms.

\lecture{3}{Tuesday}{09/10/18}

\begin{lemma}
\label{lem:1.6}
If $ \abs{\cdot} $ is nonarchimedean and $ \abs{x} \ne \abs{y} $, then $ \abs{x + y} = \max\rb{\abs{x}, \abs{y}} $.
\end{lemma}

\begin{proof}
Without loss of generality, $ \abs{x} > \abs{y} $. Write $ x = \rb{x + y} + \rb{-y} $, so $ \abs{x} \le \max\rb{\abs{x + y}, \abs{-y}} $. So equality holds in all inequalities, so $ \abs{x} = \abs{x + y} $.
\end{proof}

\begin{example*}
Think about $ D\rb{a, r} = \cb{x \mid \abs{x - a} < r} $ and $ D\rb{b, r} = \cb{x \mid \abs{x - b} < r} $. What are the possibilities for $ D\rb{a, r} \cap D\rb{b, r} $?
\end{example*}

(Exercise: directly prove Lemma \ref{lem:1.6} from the definition) Let $ k $ be a field, let $ \abs{\cdot} $ be any norm on $ k $. $ d\rb{x, y} = \abs{x - y} $ is a metric, so $ k $ is a metric space with metric $ d $. We have the usual definitions for sequences $ \rb{x_n} $. Say that $ x_n $ \textbf{converges} to $ x $ if for all $ \epsilon > 0 $, there exists $ N $ such that $ n \ge N $ gives $ \abs{x_n - x} < \epsilon $. Say that $ \rb{x_n} $ is \textbf{Cauchy} if $ \epsilon > 0 $, there exists $ N $ such that $ m, n \ge N $ gives $ \abs{x_n - x_m} < \epsilon $. Say that $ \rb{x_n} $ is \textbf{convergent} if $ x_n $ converges to $ x $ for some $ x \in k $. Write $ x_n \to x $. If $ \rb{x_n} $ is convergent, then $ \rb{x_n} $ is Cauchy. Say that $ k $ is \textbf{complete} with respect to $ \abs{\cdot} $ if all Cauchy sequences converge.

\begin{example*}
Let $ k = \Q $ with the usual archimedean norm $ \abs{\cdot} $. $ 1, 1.4, 1.41, 1.414, \dots \to \sqrt{2} \notin \Q $.
\end{example*}

\begin{example*}
Let $ k = \Q $ with $ \abs{\cdot}_2 $. $ 3, 33, 333, 3333, \dots $ is a Cauchy sequence. In fact, if $ m > n $, $ \abs{x_m - x_n}_2 = 2^{-n} $. So if $ m, n \ge N $, $ \abs{x_m - x_n} \le 2^{-N} $. $ x_n = \rb{10^n - 1} / 3 $ so $ x_n + 1 / 3 = 10^n / 3 = 2^n\rb{5^n / 3} $. $ \abs{x_n + 1 / 3}_2 = 2^{-n} $, so $ x_n + 1 / 3 \to 0 $.
\end{example*}

\begin{example*}
Let $ k = \Q $ with $ \abs{\cdot}_5 $. $ \abs{5^{2^n}}_5 = 5^{-2^n} \to 0 $ as $ n \to \infty $.
\end{example*}

\begin{example*}
Let $ k = \Q $ with $ \abs{\cdot}_2 $. $ \rb{5^{2^n}} $ is $ 5, 25, 625, \dots \to 1 $. Want to show that $ 5^{2^n} - 1 \to 0 $, so that bigger and bigger powers of two divide $ 5^{2^n} - 1 $. Use the lemma that $ t \equiv 1 \mod 2^k $ gives $ t^2 \equiv 1 \mod 2^{k + 1} $.
\end{example*}

\begin{example*}
Let $ k = \Q $ with $ \abs{\cdot}_7 $. $ 5^{2^n} $ is not Cauchy. In fact $ \mod 7 $ it looks like $ 5, 4, 2, 4, 2, \dots $, so $ \abs{x_n - x_{n + 1}}_7 = 1 $.
\end{example*}

\begin{example*}
An example of a Cauchy sequence in $ \Q $ for some $ \abs{\cdot}_p $ which does not converge. Want to find a Cauchy sequence $ \rb{x_n} $ with $ x_n^2 \to 7 $, so write down a sequence $ \rb{x_n} $ such that $ \abs{x_n^2 - 7}_3 \to 0 $. If $ x_1 = 1 $, then $ x_1^2 \equiv 7 \mod 3 $. If $ x_2 = 4 $, then $ x_2^2 \equiv 7 \mod 9 $. Find some $ n \in \N $ such that $ x_3 = 4 + 9n $ and $ x_3^2 \equiv 7 \mod 27 $. $ x_3^2 = \rb{4 + 9n^2} = 16 + 8\rb{9n} + \rb{9n}^2 \equiv 16 + 8\rb{9n} \mod 27 $. So want $ 9 + 8\rb{9n} \equiv 0 \mod 27 $, so $ 1 + 8n \equiv 0 \mod 3 $, such as $ n = 1 $. So $ x_3 = 13 $. Similar $ x_4 = 13 + 27n $ and find some $ n \in \N $.
\end{example*}

\begin{example*}
Show that for all primes $ p $ there is a Cauchy sequence $ \rb{x_n} $ in $ \Q $ for $ \abs{\cdot}_p $ with the property that $ x_n^2 \to t $ for some $ t \in \Z $ which is not a perfect square, such as $ t = 1 - p $ if $ p > 2 $ and $ t = -7 $ if $ p = 2 $. (Exercise)
\end{example*}

There is a general notion of completing a field with respect to a norm. The result is a field $ K $ containing $ k $, for which all Cauchy sequences in $ k $ converge, and such that $ K $ is minimal with this property.

\begin{example*}
Let $ k = \Q $ with the archimedean norm $ \abs{\cdot} $, then $ K = \R $.
\end{example*}

\lecture{4}{Thursday}{11/10/18}

Let $ k $ be a field and $ \abs{\cdot} $ be a norm on $ k $. Let $ R $ be the ring of Cauchy sequences in $ k $, so elements of $ R $ are $ \rb{x_n} $ for $ x_n \in k $, where $ \rb{x_n} + \rb{y_n} = \rb{x_n + y_n} $ and $ \rb{x_n}\rb{y_n} = \rb{x_ny_n} $.

Let $ I = \cb{\rb{x_n} \in R \mid x_n \to 0 \text{ as } n \to \infty} $. Claim that $ I $ is an ideal in $ R $. $ \rb{x_n}, \rb{y_n} \in I $ then $ \rb{x_n + y_n} \in I $ because $ x_n \to 0 $ and $ y_n \to 0 $ gives $ x_n + y_n \to 0 $. If $ \rb{x_n} \in R $, $ \rb{y_n} \in I $ then $ \rb{x_n} $ is bounded, so $ x_ny_n \to 0 $.

Define the completion of $ k $ to be $ \widehat{k} = R / I $. Claim that $ \widehat{k} $ is a field, so $ I $ is a maximal ideal of $ R $. Need to show that if $ \rb{x_n} \in R $ with $ x_n \not\to 0 $, then there exists $ \rb{y_n} \in R $ such that $ \rb{x_ny_n} \in 1 + I $. $ x_n \not\to 0 $ gives $ \epsilon, N $ such that $ n \ge N $ gives $ \abs{x_n} \ge \epsilon > 0 $. Set $ y_n = 0 $ if $ n < N $ and $ y_n = 1 / x_n $ if $ n \ge N $. (Exercise: show that $ \rb{y_n} \in R $) Then $ \rb{x_ny_n} = 0 $ if $ n < N $ and $ \rb{x_ny_n} = 1 $ if $ n \ge N $, so $ x_ny_n \to 1 $ as $ n \to \infty $, as required.

$ k $ is a subfield of $ \widehat{k} $ by $ x \mapsto \rb{x} $.
\begin{itemize}
\item Exercise: check that $ \widehat{k} $ has a natural norm extending $ \abs{\cdot} $ on $ k $, and $ \widehat{k} $ is complete with respect to this norm, so $ \abs{\rb{x_n}} = \lim_{n \to \infty} \abs{x_n} $.
\item Exercise: prove that if $ \abs{\cdot} $ on $ k $ is a nonarchimedean norm, then so is the induced norm on $ \widehat{k} $.
\item Exercise: furthermore, the sequence $ \abs{x_n} $ is eventually constant for any Cauchy sequence $ \abs{x_n} \in R \setminus I $.
\end{itemize}

\begin{note*}
For $ \abs{\cdot}_p $ on $ \Q $ this means that $ \abs{\cdot} $ is taking values in $ p^{\Z} $.
\end{note*}

\begin{definition}
The \textbf{$ p $-adic numbers} $ \Q_p $ is the completion of $ \Q $ with respect to the $ p $-adic norm. The \textbf{$ p $-adic integers} $ \Z_p $ is the closed unit disc in $ \Q_p $, so $ \Z_p = \cb{x \in \Q_p \mid \abs{x}_p = 1} $.
\begin{itemize}
\item Exercise: prove that $ \Z_p $ is a subring of $ \Q_p $, using that $ \abs{\cdot}_p $ is nonarchimedean.
\item Exercise: also $ \Q \cap \Z_p = \cb{r / s \mid \rb{r, s} = 1, \ p \nmid s} $, in particular $ \Z \subset \Z_p $.
\item Exercise: prove that $ \Q $ is dense in $ \Q_p $, indeed $ k $ is dense in $ \widehat{k} $.
\end{itemize}
A less obvious fact is that $ \Z $ is dense in $ \Z_p $.
\end{definition}

\begin{definition}
Let $ k $ be a field with a norm $ \abs{\cdot} $, and $ \rb{x_n} $ be any sequence. Then $ \sum_{n = 1}^\infty = \lim_{N \to \infty} \sum_{n = 1}^N x_n $, if this limit exists.
\end{definition}

\begin{lemma}
\label{lem:1.10}
If $ k $ is nonarchimedean and $ x_1, \dots, x_r $ satisfy $ \abs{x_i} \le R $ for some $ R $, then $ \abs{\sum_{n = 1}^r x_i} \le R $.
\end{lemma}

\begin{proof}
Induction with ultrametric inequality.
\end{proof}

\begin{corollary}
\label{cor:1.11}
If $ k $ is nonarchimedean then $ \rb{x_n} $ is Cauchy iff $ x_n - x_{n + 1} \to 0 $ as $ n \to \infty $.
\end{corollary}

\begin{proof}
Forward direction is obvious. Conversely, $ m > n $ gives $ x_m - x_n = \rb{x_m - x_{m - 1}} + \dots + \rb{x_{n + 1} - x_n} $. If $ \abs{x_{i + 1} - x_i} \le \epsilon $ for all $ i \ge N $ then $ \abs{x_m - x_n} \le \epsilon $ for all $ m, n \ge N $ by Lemma \ref{lem:1.10}.
\end{proof}

\begin{lemma}
\label{lem:1.12}
If $ k $ is complete and nonarchimedean then $ \sum_{n = 1}^\infty x_n $ converges iff $ x_n \to 0 $ as $ n \to \infty $. If $ \abs{x_n}^{n - 1} \le R $ for all $ n $ and $ x_n \to 0 $ then $ \abs{\sum_{n = 1}^\infty x_n} \le R $.
\end{lemma}

\begin{proof}
Apply Lemma \ref{lem:1.10} and Corollary \ref{cor:1.11} to the sequence $ \rb{\sum_{i = 1}^n x_i} $.
\end{proof}

\begin{lemma}
\label{lem:1.13}
If $ a_n \in \Z $ for all $ n \ge 0 $, then $ \sum_{n = 0}^\infty a_np^n $ converges in $ \Q_p $. If $ a_n = 0 $ for $ n < T $ and $ p \nmid a_T $ then $ \abs{\sum_{n = 0}^\infty a_np^n} = p^{-T} $.
\end{lemma}

\begin{proof}
$ \abs{a_np^n}_p \le \abs{p^n}_p = p^{-n} \to 0 $ as $ n \to \infty $, so $ \sum_{n = o}^N a_np^n $ converges by Lemma \ref{lem:1.12}. If $ N \ge T $ then $ \abs{\sum_{n = o}^N a_np^n} = p^{-T} $. (Exercise: use Lemma \ref{lem:1.6})
\end{proof}

\begin{proposition}
\label{prop:1.14}
\hfill
\begin{enumerate}
\item If $ a_n \in \cb{0, \dots p - 1} $, then $ \sum_{n = 0}^\infty a_np^n \in \Z_p $. Further more if $ a_n, b_n \in \cb{0, \dots, p - 1} $ and $ \sum_{n = 0}^\infty a_np^n = \sum_{n = 0}^\infty b_np^n $ then $ a_n = b_n $ for all $ n $.
\item If $ x \in \Z_p $ then there is a unique sequence $ \rb{a_n} $ as in 1 with $ x = \sum_{n = 0}^\infty a_np^n $.
\end{enumerate}
\end{proposition}

\begin{proof}
\hfill
\begin{enumerate}
\item Convergence is Lemma \ref{lem:1.13}. If $ \rb{a_n} \ne \rb{b_n} $ let $ T $ be minimal such that $ a_T \ne b_T $. Then
$$ \abs{\sum_{n = 0}^\infty a_np^n - \sum_{n = 0}^\infty a_np^n} = \abs{\sum_{n = 0}^\infty \rb{a_n - b_n}p^n} = p^{-T} $$
by Lemma \ref{lem:1.13}, so $ \sum_{n = 0}^\infty a_np^n \ne \sum_{n = 0}^\infty a_np^n $.
\item Let $ x \in \Z_p $. Since $ \Q $ is dense in $ \Q_p $ by construction, there exists $ r / s \in \Q $, $ \rb{r, s} = 1 $ with $ \abs{x - r / s}_p < 1 $. Since $ \abs{x}_p \le 1 $, $ \abs{r / s}_p \le 1 $, so $ p \nmid s $. So there exists $ \gamma \in \Z $ with $ \abs{\gamma - r / s}_p < 1 $, so $ s\gamma \equiv r \mod p $. Then $ \abs{\gamma - x}_p < 1 $. Now choose $ a_0 \in \cb{0, \dots, p - 1} $ with $ \gamma \equiv a_0 \mod p $. Then $ \abs{a_0 - x}_p < 1 $. So $ \abs{\rb{a_0 - x} / p}_p \le 1 $, and repeating the process, there exists $ a_1 \in \cb{0, \dots, p - 1} $ with $ \abs{\rb{a_0 - x} / p - a_1}_p < 1 $, and so on. $ \abs{x - a_0 - a_1p}_p \le 1 / p^2 $, etc.
\end{enumerate}
\end{proof}

\lecture{5}{Friday}{12/10/18}

\begin{corollary}
Every element of $ \Q_p $ has a unique expression as $ a = \sum_{n \ge -T} a_np^n $ for $ a_n \in \cb{0, \dots, p - 1} $, $ a_{-T} \ne 0 $, and $ \abs{a}_p = p^T $.
\end{corollary}

\begin{proof}
Given $ a \in \Q_p $, let $ T $ be such that $ \abs{a}_p = p^T $. Apply Proposition \ref{prop:1.14} to $ p^Ta $.
\end{proof}

\begin{corollary}
$ \Z $ is dense in $ \Z_p $.
\end{corollary}

\begin{proof}
If $ a \in \Z_p $, write $ a = \sum_{n = 0}^\infty a_np^n $. Then $ \sum_{n = 0}^N a_np^n $ is a sequence in $ \Z $ which converges to $ a $.
\end{proof}

For each $ n \ge 1 $ there is a ring homomorphism $ \Z_p \to \Z / p^n\Z $ that sends $ \sum_{i = 0}^\infty a_ip^i \mapsto \sum_{i = 0}^n a_ip^i $. The kernel of this map is $ p^n\Z_p $, so $ \Z_p / p^n\Z_p \cong \Z / p^n\Z $.

\begin{note*}
In fact, $ \Z_p \cong \lim_{\leftarrow n} \Z / p^n\Z $.
\end{note*}

$ p\Z_p $ is a maximal ideal of $ \Z_p $ because $ \Z_p / p\Z_p = \Z / p\Z $ is a field. If $ a \in \Z_p $, $ a \notin p\Z_p $, then $ a \in \Z_p^* $. Write $ a = a_0 + pa_1 + \dots $ for $ a_0 \in \cb{1, \dots, p - 1} $.

\begin{example*}
$ a = a_0 + pA $. Show that $ a $ has an inverse using the formula for the sum of a geometric progression. (Exercise) For example $ \rb{1 - p}^{-1} = 1 + p + \dots $.
\end{example*}

Alternatively, $ a \in \Z_p $ is in $ p\Z_p $ iff $ \abs{a}_p < 1 $. So if $ a \in \Z_p \setminus p\Z_p $, then $ \abs{a}_p = 1 $. Then $ \abs{1 / a}_p = 1 / \abs{a}_p = 1 $, so $ 1 / a \in \Z_p $. $ 1 / a $ exists because $ \Q_p $ is a field. So $ p\Z_p $ is the unique maximal ideal of $ \Z_p $, so $ \Z_p $ is a local ring. $ \Z_p^* = \Z_p \setminus p\Z_p $ is the group of units. Terminology: $ a \in \Q_p $ is a \textbf{unit} iff $ a \in \Z_p^* $, so iff $ \abs{a}_p = 1 $.

\begin{lemma}
$ a \in \Z_p^* $ iff $ \abs{a}_p = 1 $.
\end{lemma}

\begin{corollary}
Every element of $ \Q_p^* = \Q_p \setminus \cb{0} $ is uniquely of the form $ p^nu $ for $ n \in \Z $ and $ u \in \Z_p^* $.
\end{corollary}

\begin{proof}
Set $ \abs{a}_p = p^{-n} $, then $ \abs{p^{-n}a}_p = 1 $, so $ u = p^{-n}a $, so $ a = up^n $.
\end{proof}

Aim is to solve the equation $ x^2 - 7 = 0 $ in $ \Z_3 $ starting with a solution to $ x^2 \equiv 7 \mod 3 $. We did this by solving $ x^2 \equiv 7 \mod 3^n $ for $ n = 1, 2, \dots $. Write $ f\rb{x} = x^2 - 7 $. Then $ \abs{f\rb{1}}_3 = 1 / 3 $, $ \abs{f\rb{4}}_3 = 1 / 9 $, $ \abs{f\rb{13}}_3 = 1 / 27 $, etc. So starting with $ x_0 = 1 $, we construct a sequence $ \rb{x_n} $ with $ \abs{x_{n + 1} - x_n} \le 1 / 3^{n + 1} $ and $ \abs{f\rb{x_{n + 1}}}_3 \le 1 / 3^n $. So in particular $ \rb{x_n} $ is Cauchy, and so converges to some $ x \in \Q_3 $ with $ f\rb{x} = 0 $. Newton-Raphson method over $ \R $ works over $ \Q_p $, and this is called \textbf{Hensel's lemma}.

\begin{definition}
If $ R $ is a ring and $ f\rb{X} \in R\sb{X} $ is a polynomial, say $ f\rb{x} = \sum_{n = 0}^m a_nx^n $, then $ f'\rb{x} = \sum_{n = 0}^m na_nx^{n - 1} $.
\end{definition}

In addition, for any $ x, h $ we have $ f\rb{x + h} = f\rb{x} + hf'\rb{x} + \tfrac{1}{2}h^2f''\rb{x} + \dots $. Why is $ \tfrac{1}{2}f''\rb{x} $ defined? If $ f\rb{x} = \sum_{n = 0}^\infty a_nx^n $, $ \tfrac{1}{2}f''\rb{x} = \sum_{n = 2}^\infty \tfrac{1}{2}n\rb{n - 1}a_nx^{n - 2} $, where $ n\rb{n - 1} $ is even.

\begin{theorem}[Hensel's lemma]
Let $ k $ be a field which is complete with respect to a nonarchimedean norm $ \abs{\cdot} $. Let $ R = \cb{x \in k \mid \abs{x} \le 1} $, such as $ k = \Q_p $, $ R = \Z_p $, $ \abs{\cdot} = \abs{\cdot}_p $. Let $ f\rb{x} \in R\sb{x} $, and $ t_0 \in R $ with $ \abs{f\rb{t_0}} > \abs{f'\rb{t_0}}^2 $. Then there is a unique $ t \in R $ with $ f\rb{t} = 0 $ and $ \abs{t - t_0} < \abs{f'\rb{t_0}} $. Moreover, $ \abs{f'\rb{t_0}} = \abs{f'\rb{t_0}} $, and $ \abs{t_- t_0} = \abs{f\rb{t_0}} / \abs{f'\rb{t_0}} < 1 $.
\end{theorem}

\begin{example*}
Let $ k = \Q_3 $ and $ f\rb{x} = x^2 - 7 $. Then $ f'\rb{x} = 2x $. $ t_0 = 1 $, so $ \abs{f\rb{t_0}}_3 = \abs{-6}_3 = 1/ 3 $. $ \abs{f'\rb{t_0}}_3 = \abs{2}_3 = 1 $. So there is a unique $ t \in \Z_3 $ with $ t^2 = 7 $ and $ \abs{t - 1}_3 < 1 $, so $ t \equiv 1 \mod 3 $.
\end{example*}

Harder part of the proof of Hensel's lemma is existence, which is done by constructing a sequence $ t_0, t_1, \dots $ as above. For uniqueness, suppose $ s \ne t $ and $ f\rb{s} = f\rb{t} = 0 $, and $ \abs{s - t_0} < \abs{f'\rb{t_0}} $ and $ \abs{t - t_0} < \abs{f'\rb{t_0}} $, so $ \abs{s - t} \le \max\rb{\abs{s - t_0}, \abs{t - t_0}} < \abs{f'\rb{t_0}} $. Taylor expansion gives $ f\rb{s} = f\rb{t} + \rb{s - t}f'\rb{t} + \dots $. So there exists $ x \in R $ with $ \rb{s - t}f'\rb{t} = \rb{s - t}^2 = x $. So $ f'\rb{t} = \rb{s - t}x $. $ \abs{f'\rb{t}} = \abs{s - t}\abs{x} \le \abs{s - t} < \abs{f'\rb{t_0}} $. So $ \abs{f'\rb{t}} < \abs{f'\rb{t_0}} $. On the other hand it is easy to show that $ \abs{f'\rb{t}} = \abs{f'\rb{t_0}} $, a contradiction. See handout for the full proof of Hensel's lemma and some remarks about in what sense it is best possible.

\lecture{6}{Tuesday}{16/10/18}

What are the squares in $ \Q_p^* $? We know that we can write any element of $ \Q_p^* $ uniquely as $ p^nu $ for $ u $ a unit, that is $ \abs{u}_p = 1 $. Suppose that $ p^nu = \rb{p^rv}^2 $ for $ r \in \Z $ and $ v $ a unit. Then $ p^n = p^{2r} $ and $ u = v^2 $, that is $ n = 2r $ and $ u = v^2 $, that is the squares in $ \Q_p^* $ are exactly the $ p^{2r}u $, where $ u \in \Z_p^* $ is a square. If $ p > 2 $, we claim that $ u $ is a square iff $ u $ is a square mod $ p $, that is $ u $ is a quadratic residue. If $ u = t^2 $, then $ u \equiv t^2 \mod p $. Conversely, suppose that $ u \equiv t_0^2 \mod p $. Now use Hensel's lemma with $ f\rb{X} = X^2 - u $ and $ t_0 $ as above. $ \abs{f\rb{t_0}}_p = \abs{t_0^2 - u}_p \le \abs{p}_p = 1 / p $. $ \abs{f'\rb{t_0}}_p = \abs{2t_0}_p = \abs{t_0}_p = 1 $. If $ p \mid t_0 $ then $ p \mid u $ because $ u \equiv t_0^2 \mod p $. Hensel's lemma gives a unique $ t \in \Z_p $ such that $ t^2 = u $ and $ \abs{t - t_0} < 1 $, that is $ t \equiv t_0 \mod p $. In this case, the equation $ X^2 = u $ has exactly two solutions, namely $ t $ and $ -t $. In any field, a polynomial of degree $ d $ has at most $ d $ roots. For $ p = 2 $, $ u \in \Z_2^* $ is a square iff there exists $ t_0 $ such that $ u \equiv t_0^2 \mod 8 $, that is $ u \equiv 1 \mod 8 $, since $ \rb{2m + 1}^2 = 4m\rb{m + 1} + 1 \equiv 1 \mod 8 $.

(missing)

\pagebreak

\section{Basic algebraic geometry}

We want to consider curves in the plane given by the equations $ f\rb{x, y} = 0 $. We want to understand the points of intersection of two such curves.

\begin{example*}
If $ C $ has degree two and $ g $ has degree one, we expect to get in general two points of intersection.
\end{example*}

More generally, if $ f $ has degree $ a $ and $ g $ has degree $ b $, can hope to get $ ab $ points of intersection. This can fail to be true in several different ways.
\begin{enumerate}
\item $ y = x^2 $ and $ y = 0 $, that is $ f = y - x^2 $, $ g = y $, meet only at $ \rb{0, 0} $. Solution is to count this point with multiplicity two. In this course, we will explain exactly when the multiplicity is one.
\item For $ x^2 - y^2 = 0 $ and $ x^3 - y^3 = 0 $, any point with $ x = y $ is on the intersection. Solution is to demand that $ f $ and $ g $ are coprime, that is they have no common factor, unlike $ x^2 - y^2 = \rb{x - y}\rb{x + y} $ and $ x^3 - y^3 = \rb{x - y}\rb{x^2 + xy + y^2} $.
\item $ y = x^2 $ and $ y = 2 $ intersect at $ \rb{\pm \sqrt{2}, \sqrt{2}} $. But if for example we are working over $ \Q $, these points do not exist. Solution is to only work over fields which are algebraically closed, such as $ \C $.
\item $ y = x + 1 $ and $ y = x $ meet at infinity in projective space.
\end{enumerate}

\begin{definition}
A field $ K $ is \textbf{algebraically closed} if every polynomial over $ K $ of degree at least one factors as a product of linear factors.
\end{definition}

\begin{example*}
$ K = \C $ is algebraically closed. $ K = \Q, \R $ is not algebraically closed. Any finite field is not algebraically closed.
\end{example*}

A basic fact is if $ k $ is an field, there exists an algebraically closed field $ K $ with $ k \subset K $, and in fact there exists a smallest possible choice, the \textbf{algebraic closure} of $ k $.

\lecture{7}{Thursday}{18/10/18}

\begin{definition}
If $ f \in k\sb{x_1, \dots, x_n} $ of degree $ d $, then the \textbf{homogenisation} of $ f $ is the polynomial $ F \in k\sb{X_1, \dots, X_{n + 1}} $ obtained as follows. If
$$ f = \sum_{\underline{i} = \rb{i_1, \dots, i_n} \in \Z_{\ge 0}^n} a_{\underline{i}}x_1^{i_1} \dots x_n^{i_n} \qquad \implies \qquad F = \sum_{\underline{i}} a_{\underline{i}}X_1^{i_1} \dots X_n^{i_n}X_{n + 1}^{d - \rb{i_1 + \dots + i_n}}. $$
Given $ G \in k\sb{X_1, \dots, X_{n + 1}} $ homogeneous of degree $ d $, that is every monomial in $ G $ has degree exactly $ d $, then for each $ 1 \le i \le n + 1 $, the $ i $-th \textbf{dehomogenisation} is given by setting $ X_i = 1 $ and writing $ x_j $ for $ X_j $.
\end{definition}

\begin{example*}
Let $ G = X_1^2X_2^2 + X_3^4 + X_2X_3^2 $, so $ n = 2 $. First dehomogenisation is $ x_2^2 + x_3^4 + x_2x_3^3 $. Second dehomogenisation is $ x_1^2 + x_3^4 + x_3^3 $. Third dehomogenisation is $ x_1^2x_2^2 + 1 + x_2 $.
\end{example*}

If $ n = 2 $, usually write $ x, y $ and $ X, Y, Z $ for our variables.

\begin{example*}
If $ f = y^2 + x^3 + x + 1 $, $ F = Y^2Z + X^3 + XZ^2 + Z^3 $. Dehomogenisations of $ F $ are $ y^2z + 1 + z^2 + z^3, z + x^3 + xz^2 + z^3, y^2 + x^3 + x + 1 $.
\end{example*}

Let $ k $ be a field, and define an equivalence relation on $ k^{n + 1} $ by $ \rb{a_1, \dots, a_{n + 1}} \sim \rb{b_1, \dots, b_{n + 1}} $ iff there exists $ \lambda \in k^* $ such that $ b_i = \lambda a_i $ for $ 1 \le i \le n + 1 $.

\begin{definition}
For each $ n \ge 0 $, $ \P^n\rb{k} $, \textbf{the $ n $-dimensional projective space over $ k $}, is defined to be
$$ \dfrac{k^{n + 1} \setminus \cb{\rb{0, \dots, 0}}}{\sim}. $$
Write $ \sb{a_1 : \dots : a_{n + 1}} $ for the equivalence class of $ \rb{a_1, \dots, a_{n + 1}} $ in $ \P^n\rb{k} $. This makes sense as long as some $ a_i \ne 0 $.
\end{definition}

Let us see why we can think of $ \P^n\rb{k} $ as being $ k^n $ together with some points at infinity.

\begin{example*}
$ \P^1\rb{k} $ should be $ k $ together with a point infinity. Any point of $ \P^1\rb{k} $ is of the form $ \sb{a_1 : a_2} $ with $ a_1, a_2 $ not both zero. If $ a_2 \ne 0 $, then $ \sb{a_1 : a_2} = \sb{a_1 / a_2 : 1} $. On the other hand, if $ \sb{a : 1} = \sb{b : 1} $, then there exists $ \lambda \in k^* $ such that $ \rb{a, 1} = \lambda\rb{b, 1} = \rb{b\lambda, \lambda} $, so $ \lambda = 1 $ and $ a = b $. So there is a bijection between $ k $ and the points $ \sb{a_1 : a_2} $ with $ a_2 \ne 0 $, given by $ k \to \P^1\rb{k} $ by $ a \mapsto \sb{a : 1} $. The remaining points of $ \P^1\rb{k} $ are those with $ a_2 = 0 $. Then $ \sb{a_1 : 0} = \sb{1 : 0} $, so we have exactly one point in $ \P^1\rb{k} \setminus k $, the point at infinity.
\end{example*}

For each $ 1 \le i \le n + 1 $, let $ \phi_i : k^n \to \P^n\rb{k} $ be the map putting one in the $ i $-th coordinate, that is $ \rb{a_1, \dots, a_n} \mapsto \sb{a_1 : \dots : a_{i - 1} : 1 : a_i : \dots : a_n} $.

\begin{example*}
$ \phi_{n + 1}\rb{\rb{a_1, \dots, a_n}} = \sb{a_1 : \dots : a_n : 1} $.
\end{example*}

\begin{lemma}
Each map $ \phi_i $ is injective.
\end{lemma}

\begin{proof}
If $ \sb{a_i : \dots : a_{i - 1} : 1 : a_i : \dots a_n} = \sb{b_1 : \dots : b_{i - 1} : 1 : b_i : \dots : b_n} $ then $ \rb{a_1, \dots, a_{i - 1}, 1, a_i, \dots, a_n} = \lambda\rb{b_1, \dots, b_{i - 1}, 1, b_i, \dots, b_n} $. Looking at $ i $-th coordinate, $ \lambda = 1 $, so $ a_j = b_j $ for all $ j $.
\end{proof}

Usually think of $ k^n $ as lining inside $ \P^n\rb{k} $ via $ \phi_{n + 1} $. Think of the complement as the points at infinity, that is the points $ \sb{x_1 : \dots : x_n : 0} $. Given a polynomial $ F \in k\sb{X_1, \dots, X_{n + 1}} $, it never makes sense to ask what the value of $ F $ at $ \sb{a_1 : \dots : a_{n + 1}} \in \P^n\rb{k} $ is, because $ \sb{a_1 : \dots : a_{n + 1}} = \sb{\lambda a_1 : \dots : \lambda a_{n + 1}} $. On the other hand, if $ F $ is homogeneous of degree $ d $, then $ F\rb{\lambda a_1, \dots, \lambda a_{n + 1}} = \lambda^d F\rb{a_1, \dots, a_{n + 1}} $, so it does makes sense to ask whether or not $ F\rb{\sb{a_1 : \dots : a_{n + 1}}} = 0 $. The \textbf{graph} of $ F\rb{X, Y, Z} $ homogeneous are the points $ P $ of $ \P^2\rb{k} $ where $ F\rb{P} = 0 $. If $ F $ is the homogenisation of $ f $, then the graph of $ F $ is the graph of $ f $ together with points at infinity. (Exercise: $ f\rb{a, b} = 0 $ iff $ F\rb{a, b, 1} = 0 $)

\begin{example*}
$ f $ is a line, say $ f = y - mx - c $, so $ F = Y - mX - cZ $. What are the points at infinity? If $ Z = 0 $ and $ Y = mX $, that is $ \sb{1 : m : 0} $, the line $ y = mx + c $ has one point at infinity, namely $ \sb{1 : m : 0} $. If $ X = 0 $, points at infinity are $ Z = 0 $, $ \sb{0 : Y : 0} = \sb{0 : 1 : 0} $. So two lines meet at infinity iff they are parallel.
\end{example*}

\lecture{8}{Friday}{19/10/18}

Proof of the following theorem is in Fulton's Algebraic Curves.

\begin{theorem}[Bézout's theorem]
\label{thm:2.6}
Let $ F, G $ be homogeneous polynomials in $ k\sb{X, Y, Z} $ of degrees $ a, b $ respectively. If $ F $ and $ G $ have no common factors, and $ k $ is algebraically closed, then the graphs of $ F = 0 $ and $ G = 0 $ meet at $ ab $ points in $ \P^2\rb{k} $, counted with multiplicities.
\end{theorem}

\begin{corollary}
If $ F, G $ are as in Theorem \ref{thm:2.6}, and for some $ k $, $ F = 0 $ and $ G = 0 $ meet at greater than $ ab $ points with multiplicity, then $ F $ and $ G $ have a common factor.
\end{corollary}

\begin{definition}
If $ f \in k\sb{x_1, \dots, x_n} $ is a polynomial, and let $ P = \rb{a_1, \dots, a_n} \in k^n $ with $ f\rb{P} = 0 $. Then $ P $ is a \textbf{smooth} point of the graph of $ f $, or a \textbf{nonsingular} point, if for some $ i $, $ \rb{\partial f / \partial x}\rb{P} \ne 0 $. A singular point is one for which all $ \rb{\partial f / \partial x_i}\rb{P} = 0 $ for $ 1 \le i \le n $.
\end{definition}

\begin{example*}
$ \rb{\partial / \partial x_2}\rb{x_1x_2^5} = 5x_1x_2^4 $.
\end{example*}

\begin{example*}
If $ f = x^2 - y^5 $, $ \partial f / \partial x = 2x $ and $ \partial f / \partial y = -5y^4 $. $ \rb{0, 0} $ is singular and $ \rb{1, 1} $ is nonsingular.
\end{example*}

Motivation is that tangent space to $ f = 0 $ at $ P $ should be $ \sum_{i = 1}^n \rb{\partial f / \partial x_i}\rb{P}\rb{x_i - a_i} = 0 $. Generalising to projective space, consider $ F \in k\sb{X_1, \dots, X_{n + 1}} $ homogeneous. Consider $ P $ with $ F\rb{P} = 0 $. Say that $ P $ is nonsingular if at least one $ \rb{\partial F / \partial X_i}\rb{P} \ne 0 $ for $ 1 \le i \le n + 1 $, singular if all $ \rb{\partial F / \partial X_i}\rb{P} = 0 $.

\begin{fact*}
$ Im\rb{P} \in Im\rb{\phi_i} $ for $ \phi_i : k^n \to \P^n\rb{k} $ insert $ 1 $ at $ i $-th coordinate. Then $ P $ is nonsingular as a point of $ F = 0 $ iff the corresponding point $ \phi_i^{-1}\rb{P} $ is nonsingular for the $ i $-th dehomogenisation of $ F $. A key fact is if $ F $ is homogeneous of degree $ d $, then $ dF = \sum_{i = 0}^{n + 1} X_i\rb{\partial F / \partial X_i} $.
\end{fact*}

\begin{definition}
Say the graph of $ f $ or $ F $ is nonsingular, or smooth, if it is nonsingular at every point of $ k^n $ or $ \P^n\rb{k} $ respectively.
\end{definition}

\begin{example*}
Any polynomial of degree one has a nonsingular graph. If $ f = \sum_{i = 1}^n a_ix_i $, then $ \partial f / \partial x_i = a_i $.
\end{example*}

\begin{example*}
Let $ k = \Q $.
\begin{enumerate}
\item $ y^2 = x^2\rb{x + 1} = x^3 + x^2 $. If $ f = y^2 - x^3 - x^2 $, then $ \partial f / \partial x = -3x^2 - 2x $ and $ \partial f / \partial y = 2y $. So if $ f $ is singular, $ y = 0 $, so $ x = 0 $ or $ x = -1 $, so $ x = 0 $. So $ \rb{0, 0} $ is the only singular point.
\item If $ f = y^2 $, then $ \partial f / \partial x = 0 $ and $ \partial f / \partial y = 2y $. So any point $ \rb{x, 0} $ is singular.
\item $ y^2 = \rb{x^2 - 2}^2 $ has singular points $ \rb{\pm \sqrt{2}, 0} $. Still say that this is singular, even though the singular points are not defined over $ \Q $.
\end{enumerate}
\end{example*}

\begin{example*}
If $ y = x + 1 $, then $ F = Y - X - Z $. $ \partial F / \partial X = -1 $, $ \partial F / \partial Y = 1 $, $ \partial F / \partial Z = -1 $. This is nonsingular.
\end{example*}

\begin{example*}
If $ y = x^3 $, then $ F = YZ^2 - X^3 $. $ \partial F / \partial X = -3X^2 $, $ \partial F / \partial Y = Z^2 $, $ \partial F / \partial Z = 2YZ $. So $ X = Z = 0 $, that is $ \sb{0 : 1 : 0} $ is a singular point. Homogenise with respect to $ Y $ gives $ z^2 - x^3 = 0 $ singular at $ \rb{x, z} = \rb{0, 0} $.
\end{example*}

\begin{theorem}
\label{thm:2.10}
If $ f, g \in k\sb{x, y} $ then a point $ P $ on $ f = 0 $ and $ g = 0 $ has multiplicity one iff
\begin{enumerate}
\item $ P $ is a nonsingular point for both $ f = 0 $ and $ g = 0 $, and
\item the tangent lines to $ f = 0 $ and $ g = 0 $ at $ P $ are distinct.
\end{enumerate}
Tangent line at $ P $ is
$$ \rb{\partial f / \partial x}\rb{P}\rb{x - x\rb{P}} + \rb{\partial f / \partial y}\rb{P}\rb{x - y\rb{P}} = 0. $$
\end{theorem}

\pagebreak

\section{Plane conics}

Let $ k $ be a field. $ k\sb{X} $ is a PID, so a UFD. $ k\sb{X_1, \dots, X_n} $ is a UFD, but not a PID.

\begin{example*}
$ \rb{X_1, X_2} $ in $ k\sb{X_1, X_2} $ is not principal.
\end{example*}

If $ k $ is algebraically closed, then the prime ideals of $ k\sb{X} $ are just $ \rb{0} $ and $ \rb{X - \alpha} $ for $ \alpha \in K $. If $ k $ is not algebraically closed, you have more.

\begin{example*}
$ \rb{X^2 - 2} $ is a prime ideal of $ \Q\sb{X} $.
\end{example*}

Even if $ k $ is algebraically closed, there are many prime ideals in $ k\sb{X, Y} $ than in $ k\sb{X} $.

\begin{example*}
$ X^2 - Y^2 - 1 $ does not factor in $ \C\sb{X, Y} $.
\end{example*}

\lecture{9}{Tuesday}{23/10/18}

A \textbf{plane conic} is just the graph of some $ f \in k\sb{x, y} $ where $ f $ has degree two. Same terminology also covers $ F = 0 $ for $ F \in k\sb{X, Y, Z} $ homogeneous of degree two. The next steps are
\begin{enumerate}
\item to understand when this is singular,
\item to understand how to find all solutions to $ f = 0 $ or $ F = 0 $ if it is singular, and
\item to understand how to find all solutions to $ f = 0 $ in the nonsingular case given one solution.
\end{enumerate}
Then in the next section we will specialise to $ k = \Q $ and see how to find solutions, or prove there are not any.

\begin{example*}
$ x^2 - y^2 $ is singular, because $ \rb{0, 0} $ is a singular point.
\end{example*}

\begin{algorithm}[Checking if $ f = 0 $ is singular]
Suppose $ f \in k\sb{x, y} $ degree two. Then any singular points are given by solving $ \partial f / \partial x = 0 $, $ \partial f / \partial y = 0 $. These are linear, so solve them and substitute back into $ f = 0 $.
\end{algorithm}

\begin{example*}
Let $ f = x^2 - y^2 - 1 $. $ \partial f / \partial x = 2x $ and $ \partial f / \partial y = -2y $ gives $ x = y = 0 $, so $ f\rb{0, 0} \ne 0 $. So $ f $ is nonsingular.
\end{example*}

\begin{example*}
Let $ f = x^2 - 2xy + y^2 = \rb{x - y}^2 $. $ \partial f / \partial x = 2x - 2y $ and $ \partial f / \partial y = 2y - 2x $. If characteristic of $ k \ne 2 $, solutions are $ x = y $, and any such point is a solution. So every point is singular.
\end{example*}

\begin{theorem}
Let $ k $ be algebraically closed, and let $ f \in k\sb{X, Y, Z} $ be homogeneous of degree two. Then $ f $ has a singular point in $ \P^2\rb{k} $ iff $ F $ is a product of two linear factors.
\end{theorem}

\begin{proof}
If $ F $ factors, then any point of intersection of the two lines is a singular point. (Exercise) The lines do meet, by Bézout. Suppose conversely that $ P $ is a singular point of $ F = 0 $. Let $ Q $ be any other point of $ F = 0 $, and let $ L $ be the line joining $ P $ and $ Q $, so $ L $ is given by some equation $ G = 0 $, where $ G $ has degree one. If $ F = GH $ for some $ H $, then we are done. Otherwise, $ F $ and $ G $ are coprime because $ G $ has degree one, so has no proper factors. So by Bézout's theorem, $ F = 0 $ and $ G = 0 $ meet in exactly two points, counted with multiplicity. Since $ P $ is a singular point of $ F = 0 $, it follows from Theorem \ref{thm:2.10} that the multiplicity of $ P $ is at least $ 2 + 1 = 3 $, a contradiction. So $ G $ is a factor of $ F $, as required.
\end{proof}

\begin{algorithm}[Finding all rational points on a singular conic]
\hfill
\begin{enumerate}
\item Find the singular point.
\item Write down another point.
\item Factor the conic as a product of linear factors.
\item Solve the two linear equations.
\end{enumerate}
\end{algorithm}

\begin{example*}
Let $ F = X^2 - 2Y^2 $. Singular point is $ 2X = -4Y = 0 $, so $ X = Y = 0 $. Any other point is $ Y = 1 $ and $ X = \sqrt{2} $. $ X = \sqrt{2}Y $ is the line through these two points. So $ X - 2Y $ is a factor of $ X^2 - 2Y^2 $, and indeed $ X^2 - 2Y^2 = \rb{X + \sqrt{2}Y}\rb{X - \sqrt{2}Y} $. So if $ k = \C $, then the solutions are $ \rb{\pm\sqrt{2}t, t} $ for $ t \in \C $. If $ k = \Q $, the only solution is $ \rb{0, 0} $.
\end{example*}

\begin{example*}
Now for a nonsingular case, let $ x^2 + 2y^2 = 6 $. Start with one solution, such as $ \rb{2, 1} $. Find all solutions by drawing lines through this point.
\end{example*}

\begin{algorithm}
Let $ F $ be a nonsingular conic over some field $ k $, and suppose that we are given a point with coordinates in $ k $ where $ F = 0 $. Then we can find all points by drawing lines with rational slope through this point, and finding the second point of intersection with $ F = 0 $.
\end{algorithm}

\begin{note*}
If $ k $ is algebraically closed, then Bézout tells us that we have exactly two points of intersection, when counted with multiplicity. Since $ f $ is nonsingular, and any line is nonsingular, the only way you can have a multiplicity is if the line you draw is the tangent line at that point. If $ k $ is not algebraically closed, just need to check that the other point of intersection has coordinates in $ k $ iff the slope of the line joining the points is in $ k $. (Exercise)
\end{note*}

\lecture{10}{Thursday}{25/10/18}

Lecture 10 is a problem class.

\pagebreak

\section{The Hasse principle}

\lecture{11}{Friday}{26/10/18}

We have seen that for any field $ k $, and any plane conic, there are algorithms to
\begin{enumerate}
\item check if the conic is singular or not,
\item if it is singular, write it as a product of two linear factors, and so find all the points over $ k $, and
\item if it is nonsingular, or smooth, find all points over $ k $ given one such point, by drawing lines through this point.
\end{enumerate}
What is missing is given a nonsingular curve, determine if it has any points over $ k $, and find such a point if there is one. For $ k = \Q $, we will do this by using the cases $ k = \Q_p $ for all primes $ p $. Over $ \Q_p $, it is not so hard to check if there are points. Use Hensel's lemma to reduce mod $ p $, use quadratic reciprocity. Very easy over $ \R $. Hasse's principle says a nonsingular plane conic over $ \Q $ has points over $ \Q $ iff it does over $ \Q_p $ and $ \R $ for all $ p $. Forward direction is trivial, converse direction is harder. Crucially, this holds for conics, that is polynomials of degree two, but not true for degree three or higher.

\begin{example*}
$ 3X^3 + 4Y^3 + 5Z^3 $.
\end{example*}

\begin{example*}
$ X^2 + Y^2 = 6 $ has no solution in $ \Q_3 $ gives none in $ \Q $. $ X^2 + Y^2 = -1 $ has no solution in $ \R $ gives none in $ \Q $.
\end{example*}

Basic idea is if we have solutions to some equation in every $ \Q_p $, we have solutions mod $ N $ for all $ N \in \Z $. Use Chinese remainder theorem to reduce to $ N = p^k $, Hensel's lemma says that equations have solutions in $ \Z_p $ iff mod $ p^k $ for all $ k $. We would like a technique for upgrading solutions to congruences to solutions to equations in $ \Q $ or in $ \Z $. Aim is to prove that if $ p \equiv 1 \mod 4 $ then $ p = x^2 + y^2 $ for some $ x, y \in \Z $. Input is $ n $ such that $ n^2 \equiv -1 \mod p $.

\begin{lemma}
\label{lem:4.1}
If $ U \subset \R^n $ is measurable, or just open, and its measure, or volume, is bigger than $ m $, then we can find distinct points $ c_0, \dots, c_m $ such that for all $ i $, $ c_0 - c_i \in \Z^n $.
\end{lemma}

\begin{proof}
Let $ C $ be the unit cube, $ C = \cb{\rb{x_1, \dots, x_n} \in \R^n \mid \forall i, \ 0 \le x_i < 1} $. By assumption $ \int_{\R^n} 1_U\rb{x} dx > m $ where
$$ 1_U\rb{x} =
\begin{cases}
1 & x \in U \\
0 & x \notin U
\end{cases}.
$$
$ \int_{\R^n} 1_U\rb{x} dx = \int_C \sum_{t \in \Z^n} 1_U\rb{x + t}dx > m $. The measure of $ C $ is $ 1 $, so there exists $ x \in C $ such that $ \sum_{t \in \Z^n} 1_U\rb{x + t} > m $. So there exist $ t_0, \dots, t_m \in \Z^n $, $ x \in C $ such that $ x + t_i \in U $ for all $ i $. Set $ c_i = x + t_i $.
\end{proof}

\begin{definition}
Say $ U $ is \textbf{convex} if $ x, y \in U $ and $ \lambda \in \sb{0, 1} $ gives $ \lambda x + \rb{1 - \lambda}y \in U $. Say that $ U $ is \textbf{symmetric} if $ x \in U $ gives $ -x \in U $.
\end{definition}

\begin{corollary}[Minkowski]
\label{cor:4.3}
If $ \Lambda \in \Z^n $ is a finite index subgroup, and if $ V $ is convex and symmetric, with measure greater than $ 2^n\sb{\Z^n : \Lambda} $, then $ V \cap \Lambda \ne \cb{0} $.
\end{corollary}

\begin{remark}
$ 0 \in V $, because given $ x \in V $, $ -x \in V $, so $ \tfrac{1}{2}\rb{x + \rb{-x}} \in V $.
\end{remark}

A \textbf{lattice} is a finite index subgroup of $ \Z^n $.

\begin{proof}
Set $ U = \tfrac{1}{2}V = \cb{\tfrac{1}{2}x \mid x \in V} $. So measure of $ U $ is greater than $ \sb{\Z^n : \Lambda} = m $. By Lemma \ref{lem:4.1}, there exist $ c_0, \dots, c_m \in U $ such that $ c_i - c_0 \in \Z^n $ for all $ i $. Consider $ c_0 - c_0, \dots, c_m - c_0 \in \Z^n $. We have $ m + 1 $ differences, so there exists $ i \ne j $ such that $ \rb{c_i - c_0} - \rb{c_j - c_0} \in \Lambda $, that is $ c_i - c_j \in \Lambda $, $ c_i - c_j \ne 0 $. Since $ c_i - c_j = \tfrac{1}{2}\rb{2c_i - 2c_j} $, $ 2c_i, 2c_j \in V $ by definition, and $ -2c_j \in V $ by symmetry, so $ \tfrac{1}{2}\rb{2c_i - 2c_j} \in V $ by convexity.
\end{proof}

How do you use this to prove that $ p = x^2 + y^2 $?

\lecture{12}{Tuesday}{30/10/18}

\begin{theorem}
If $ n \in \Z $ and there exists $ t $ such that $ t^2 \equiv -1 \mod n $, then there exist $ x, y \in \Z $ such that $ x^2 + y^2 = n $.
\end{theorem}

\begin{proof}
$ \Lambda = \cb{\rb{x, y} \mid y \equiv tx \mod n} \subseteq \Z^2 $. $ \Lambda = Ker\rb{\Z^2 \to \Z / n\Z} $ by $ \rb{x, y} \mapsto y - tx $. This is surjective, so by the first isomorphism theorem, $ \Z^2 / \Lambda \cong \Z / n\Z $. So $ \sb{\Z^2 : \Lambda} = \#\rb{\Z / n\Z} = n $. Let $ V = \cb{\rb{x, y} \mid x^2 + y^2 < 2n} $. Measure of $ V $ is $ \pi\rb{\sqrt{2n}}^2 = 2\pi n = 2^2\sb{\Z^2 : \Lambda} = 4n $ because $ \pi > 2 $. So by Corollary \ref{cor:4.3}, there exists $ \rb{x, y} \in \Lambda \cap V $ such that $ \rb{x, y} \ne \rb{0, 0} $. $ \rb{x, y} \in V $ gives $ 0 < x^2 + y^2 < 2n $. $ \rb{x, y} \in \Lambda $ gives $ x^2 + y^2 \equiv x^2\rb{t^2 + 1} \equiv 0 \mod n $. So $ x^2 + y^2 = n $, as required.
\end{proof}

Want to consider plane conics over $ \Q $, that is homogeneous equations of degree two in $ X, Y, Z $ . Completing the square, that is existence of orthogonal basis, can write this in the form $ aX^2 + bY^2 + cZ^2 $. If this is nonsingular, $ abc \ne 0 $. Conversely if $ abc \ne 0 $, this is nonsingular. Rescaling $ X, Y, Z $, we can change $ a, b, c $ by squares of elements of $ \Q^* $. So without loss of generality, $ a, b, c $ are all in $ \Z $. We can then furthermore arrange by rescaling again that each of $ a, b, c $ is squarefree, that is not divisible by the square of any prime. If $ a, b, c $ share a common factor, we can divide by it, so we can assume that $ \rb{a, b, c} = 1 $. If $ p \mid a $, $ p \mid b $, so $ p \mid c $, we can replace $ \rb{a, b, c} $ by $ \rb{a / p, b / p, pc} $ since $ aX^2 + bY^2 + cZ^2 = p\rb{\rb{a / p}X^2 + \rb{b / p}Y^2 + pc\rb{Z / p}^2} $. Repeating this, we can assume that $ \rb{a, b} = 1 $, $ \rb{b, c} = 1 $, $ \rb{c, a} = 1 $. So without loss of generality $ aX^2 + bY^2 + cZ^2 $, $ a, b, c \in \Z $ nonzero squarefree, pairwise coprime. Let $ \Sigma = \cb{p \mid 2abc} $ for $ p $ prime.

\begin{lemma}
Write $ F = aX^2 + bY^2 + cZ^2 $ as above. Then the following are equivalent.
\begin{enumerate}
\item $ F = 0 $ has infinitely many solutions in $ \P^2\rb{\Q} $.
\item $ F = 0 $ has a solution in $ \P^2\rb{\Q} $.
\item $ F = 0 $ has solutions in $ \P^2\rb{\Q_p} $ for all $ p $ and in $ \P^2\rb{\R} $.
\item $ F = 0 $ has solutions in $ \P^2\rb{\Q_p} $ for all $ p \in \Sigma $.
\end{enumerate}
\end{lemma}

\begin{proof}
$ 1 \implies 2 \implies 3 \implies 4 $. $ 2 \implies 1 $ by drawing lines through a point. So we need to do $ 4 \implies 2 $. Use Corollary \ref{cor:4.3} in $ \Z^3 $. Want to find $ V, \Lambda $ with measure of $ V $ greater than $ 2^3\sb{\Z^3 : \Lambda} $, so that $ V \cap \Lambda = \cb{0} $. Want to know that we will have $ ax^2 + by^2 + cz^2 = 0 $ if $ \rb{x, y, z} \in V \cap \Lambda $. Choose $ V = \cb{\abs{a}x^2 + \abs{b}y^2 + \abs{c}z^2 < 4\abs{abc}} $. This has measure $ \tfrac{\pi}{3}\rb{2}^3\rb{4}\abs{abc} $. So to prove the theorem, it is enough to find $ \Lambda $ with $ \sb{\Z^3 : \Lambda} = 4\abs{abc} $, and satisfying $ \rb{x, y, z} \in \Lambda $ gives $ ax^2 + by^2 + cz^2 \equiv 0 \mod 4\abs{abc} $. So if $ \rb{x, y, z} \in V \cap \Lambda $ then $ 0 \le \abs{ax^2 + by^2 + cz^2} < \abs{a}x^2 + \abs{b}y^2 + \abs{c}z^2 < 4\abs{abc} $, so $ ax^2 + by^2 + cz^2 = 0 $.
\end{proof}

\lecture{13}{Thursday}{01/11/18}

\begin{lemma}
If $ p \mid a $ then there is a solution to $ b + cz^2 \equiv 0 \mod p $.
\end{lemma}

\begin{proof}
$ p \mid a $ gives $ p \in \Sigma $ so there exist $ x, y, z \in \Q_p $ not all zero such that $ ax^2 + by^2 + cz^2 = 0 $. By multiplying by an appropriate power of $ p $, we can assume that $ \max\rb{\abs{x}_p, \abs{y}_p, \abs{z}_p} = 1 $. $ p \mid a $ and $ \rb{a, b} = \rb{a, c} = 1 $ give $ p \nmid b $, $ p \nmid c $, that is $ \abs{a}_p < 1 $ and $ \abs{b}_p = \abs{c}_p = 1 $. If $ \abs{y}_p < 1 $ then $ \abs{by^2}_p < 1 $, $ \abs{ax^2}_p \le \abs{a}_p < 1 $, so $ \abs{cz^2}_p = \abs{ax^2 + by^2} < 1 $, so $ \abs{z}_p < 1 $. Then $ \abs{x}_p = 1 $, so $ \abs{ax^2}_p = \abs{a}_p $ but $ \abs{ax^2}_p = \abs{by^2 + cz^2}_p \le \max\rb{\abs{y^2}_p, \abs{z^2}_p} \le 1 / p^2 $. So $ p^2 \mid a $, a contradiction, as $ a $ is squarefree. So $ \abs{y}_p = 1 $. Now divide by $ y $, $ a\rb{x / y}^2 + b + c\rb{z / y}^2 = 0 $ and $ x / y, z / y \in \Z_p $. Reduce mod $ p $ gives $ b + c\rb{z / y}^2 \equiv 0 \mod p $, as claimed.
\end{proof}

If $ p \in \Sigma $ and $ p > 2 $, then by symmetry without loss of generality $ p \mid a $. Then the lemma applies, so there exists $ \alpha $ with $ b + c\alpha^2 \equiv 0 \mod p $. So we impose the condition that $ z \equiv \alpha y \mod p $. Then if $ x, y, z \in \Z $ satisfying $ z \equiv \alpha y \mod p $, then $ ax^2 + by^2 + c \equiv by^2 + cz^2 \equiv by^2 + c\alpha^2y^2 \equiv y^2\rb{b + c\alpha^2} \equiv 0 \mod p $. Now have to deal with $ p = 2 $. Suppose $ 2 \mid a $. Lemma gives a solution to $ b + cz^2 \equiv 0 \mod 2 $, and a solution to $ ax^2 + by^2 + cz^2 = 0 $ in $ \Q_2 $ with $ \abs{y}_2 = 1 $, so $ y^2 \equiv 1 \mod 8 $. $ \rb{2m + 1}^2 = 4m\rb{m + 1} + 1 \equiv 1 \mod 8 $. In fact, squares mod $ 8 $ are $ 0 $, $ 1 $, $ 4 $. Similarly, $ \abs{z}_2 = 1 $ gives $ z^2 \equiv 1 \mod 8 $. Since $ ax^2 + by^2 + cz^2 = 0 $, $ ax^2 + by^2 + cz^2 = 0 $, $ ax^2 + by^2 + cz^2 \equiv 0 \mod 8 $, that is $ ax^2 + b + c \equiv 0 \mod 8 $. If $ \abs{x}_2 = 1 $ then $ x^2 \equiv 1 \mod 8 $, so $ a + b + c \equiv 0 \mod 8 $. Impose
$$ y \equiv z \mod 4, \qquad x \equiv y \mod 2. $$
If $ \rb{x, y, z} \in \Z^3 $ satisfies this condition, then either $ x, y, z $ all odd, and $ ax^2 + by^2 + cz^2 \equiv a + b + c \equiv 0 \mod 8 $, or $ x, y, z $ all even, and $ ax^2 + by^2 + cz^2 \equiv by^2 + cz^2 \mod 8 $. $ y \equiv z \mod 4 $ gives either $ 4 \mid y $, $ 4 \mid z $, which is fine, or $ y \equiv z \equiv 2 \mod 4 $, and $ by^2 + cz^2 \equiv 4\rb{b + c} \equiv -4a \equiv 0 \mod 8 $. If $ \abs{x}_2 < 1 $, then $ ax^2 \equiv 0 \mod 8 $, so $ b + c \equiv 0 \mod 8 $. Impose
$$ y \equiv z \mod 4, \qquad x \equiv 0 \mod 2. $$
Then if these conditions hold, $ ax^2 + by^2 + cz^2 \equiv by^2 + cz^2 \equiv \rb{b + c}y^2 \equiv 0 \mod 8 $. The cases $ 2 \mid b $, $ 2 \mid c $ are dealt with by symmetry. Now suppose that $ 2 \nmid abc $. Again $ 2 \in \Sigma $ gives by assumption $ x, y, z \in \Q_2 $ not all zero with $ ax^2 + by^2 + cz^2 = 0 $. Again, without loss of generality $ \max\rb{\abs{x}_2, \abs{y}_2, \abs{z}_2} = 1 $ or without loss of generality $ \abs{z}_2 = 1 $, so $ \abs{cz^2}_2 = 1 $, so at least one of $ \abs{x}_2, \abs{y}_2 $ is $ 1 $. Without loss of generality $ \abs{y}_2 = 1 $. If $ \abs{x}_2 = 1 $ then $ ax^2 + by^2 + cz^2 \equiv 1 + 1 + 1 \mod 2 $, a contradiction. So $ \abs{x}_2 < 1 $. So $ ax^2 + by^2 + cz^2 \equiv b + c \mod 4 $, that is $ b + c \equiv 0 \mod 4 $. Impose
$$ x \equiv 0 \mod 2, \qquad y \equiv z \mod 2. $$
If these hold, $ ax^2 + by^2 + cz^2 \equiv by^2 + cz^2 \equiv \rb{b + c}y^2 \equiv 0 \mod 4 $. What have we done? If $ p > 2 $ and $ p \in \Sigma $, we found a congruence mod $ p $ which guaranteed that $ ax^2 + by^2 + cz^2 \equiv 0 \mod p $. If $ 2 \mid abc $, found conditions guaranteeing that $ ax^2 + by^2 + cz^2 \equiv 0 \mod 8 $. If $ 2 \nmid abc $, $ ax^2 + by^2 + cz^2 \equiv 0 \mod 4 $. Define $ \Lambda $ as the subgroup of $ \Z^3 $ satisfying all of these congruence conditions. Since $ abc $ is squarefree, if $ \rb{x, y, z} \in \Lambda $ then $ ax^2 + by^2 + cz^2 \equiv 0 \mod 4abc $. It remains to check that $ \sb{\Z^3 : \Lambda} = \abs{4abc} $. To see this, write $ \Lambda $ as the kernel of a map from $ \Z^3 $ to a product of a group of order four or eight and $ \prod_{p > 2, \ p \mid abc} \Z / p\Z $ and use the Chinese remainder theorem.

\lecture{14}{Friday}{02/11/18}

Lecture 14 is a problem class.

\pagebreak

\section{Plane cubics}

\lecture{15}{Tuesday}{06/11/18}

\begin{definition}
A \textbf{plane cubic} is a polynomial $ f\rb{x, y} $ of degree three, or a homogeneous $ F\rb{X, Y, Z} $ of degree three.
\end{definition}

Call $ f $ nonsingular if all its points are nonsingular.

\begin{example*}
$ y^2 - x^3 $ is singular because $ \rb{0, 0} $ is a singular point but it does not factor.
\end{example*}

Main aim of the rest of the course is to try to find all the points where $ f\rb{x, y} = 0 $, particularly in the case that we are working over $ \Q $. We will also think a bit about points over $ \Q_p $.

\begin{example*}
$ 3x^3 + 4y^3 = 5 $ has solutions in $ \Q_p $ for all $ p $ and in $ \R $, but no solution in $ \Q $, so we do not have a version of the Hasse principle. $ x^3 + y^3 = 0 $ has infinitely many rational points, and it is singular. $ x^3 + y^3 = 1 $ has finitely many points over $ \Q $, in fact just $ \rb{1, 0} $ and $ \rb{0, 1} $, and is nonsingular. For $ x^3 + y^3 = 9 $, $ \rb{2, 1} $ and $ \rb{1, 2} $ are rational points. $ x + y = 3 $ is the line through these two points. $ x^3 + \rb{3 - x}^3 = 9 $ is quadratic, so $ x = 1 $ or $ x = 2 $. (Exercise: do this in projective coordinates and find a point at infinity) Instead, try tangent lines. (Exercise: try this with $ x^3 + y^3 = 1 $ and see what happens) Tangent line at $ \rb{1, 2} $ is $ x + 4y = 9 $. Substitute in $ y^3 + \rb{9 - 4y}^3 = 9 $, that is $ -9\rb{y - 2}^2\rb{7y - 20} = 0 $, to give $ \rb{-17 / 7, 20 / 7} $.
\end{example*}

\begin{lemma}
$ x^3 + y^3 = 9 $ is nonsingular and has infinitely many points over $ \Q $.
\end{lemma}

Idea to see that there are infinitely many solutions is to show that the $ z $ coordinate is getting bigger. If $ \sb{x : y : z} \in \P^2\rb{\Q} $, we will say for now that the height of $ \sb{x : y : z} $ is $ \abs{z} $, provided that $ x, y, z \in \Z $, and have no common factor.

\begin{example*}
$ \sb{1 : 2 : 1} $ had height $ 1 $. $ \sb{-17 : 22 : 7} $ had height $ 7 $.
\end{example*}

\begin{proof}
Firstly check for singular points. $ F = X^3 + Y^3 - 9Z^3 $, $ \partial F / \partial X = 3X^2 $ gives $ X = Y = Z = 0 $, which is a contradiction. To prove there are infinitely many points over $ \Q $, we draw tangent lines. Claim that if $ \sb{r : s : t} $ is on $ F = 0 $, then the third point of intersection of the tangent line at $ \sb{r : s : t} $ with $ F = 0 $ is $ \sb{r\rb{r^3 + 2s^3} : -s\rb{2r^3 + s^3} : t\rb{r^3 - s^3}} $. (Exercise: check this in two ways, first way do this as we did for $ \rb{1, 2} $, second way use that I told you the answer) If $ \sb{r : s : t} $ is on $ X^3 + Y^3 = 9Z^3 $, and if $ r, s, t \in \Z $ and have no common factor, then since $ 9 $ is cubefree, in fact have $ \rb{r, s} = \rb{s, t} = \rb{r, t} = 1 $. Suppose that some prime $ p $ divides all three factors of $ \sb{r\rb{r^3 + 2s^3} : -s\rb{2r^3 + s^3} : t\rb{r^3 - s^3}} $. Suppose that some prime $ p $ divides all three factors. If $ p \mid r $ then $ p \mid -s\rb{2r^3 + s^3} $, so $ p \mid -s^4 $, so $ p \mid s $, but $ \rb{r, s} = 1 $. So $ p \nmid r $. Similarly if $ p \mid s $ then $ p \mid r^4 $ gives $ p \mid r $, a contradiction. So $ p \nmid s $. Since $ p \mid r\rb{r^3 + 2s^3} $ and $ p \nmid r $, we have $ p \mid \rb{r^3 + 2s^3} $. Similarly $ p \mid \rb{2r^3 + s^3} $. So $ p \mid \rb{2\rb{2r^3 + s^3} - r^3 + 2s^3} $, so $ p \mid 3r^3 $. Again $ p \nmid r $, so $ p \mid 3 $, so $ p = 3 $. (Exercise: in fact, the same analysis shows that the only power of three which could divide all three terms is three itself) So the height of our new point is at least $ \tfrac{1}{3}\abs{t}\abs{r^3 - s^3} $. So we will be unless $ \abs{r^3 - s^3} \le 3 $. But this inequality is only satisfied if $ r, s \in \cb{-1, 0, 1} $. But then $ r^3 + s^3 = 9t^3 $ has no solutions other than $ \sb{-1 : 1 : 0} $. So as long as we start with any other point, such as $ \sb{1 : 2 : 1} $, we get infinitely many points.
\end{proof}

\begin{algorithm}[Finding all points for a singular cubic]
First find the singular point. Remark that using Galois theory, you can show that if you are working over a field of characteristic zero, then the singular point will have coordinates in this same field. Then draw lines through this point.
\end{algorithm}

\lecture{16}{Thursday}{08/11/18}

\begin{example*}
$ y^2 = x^3 $ has a cusp. $ \rb{0, 0} $ is a singular point. Line through $ \rb{0, 0} $ is $ y = tx $. $ t^2x^2 = x^3 $, so $ x = t^2 $, $ y = t^3 $. So the solutions to $ y^2 = x^3 $ with $ x, y \in \Q $ are given by $ \rb{x, y} = \rb{t^2, t^3} $, $ t \in \Q $.
\end{example*}

\begin{example*}
$ y^2 = x^2\rb{x + 1} $ has a node. For a singular point, $ 2y = 0 $ and $ 3x^2 + 2x = 0 $. So $ \rb{0, 0} $ is the singular point. Again, a line through $ \rb{0, 0} $ is $ y = tx $. $ t^2x^2 = x^2\rb{x + 1} $. So $ x + 1 = t^2 $ is the third point of intersection. $ x = t^2 - 1 $ and $ y = t\rb{t^2 - 1} = t^3 - t $. So all the rational solutions are given by $ \rb{x, y} = \rb{t^2 - 1, t^3 - t} $, $ t \in \Q $.
\end{example*}

Let $ f $ be an irreducible plane cubic, that is it does not factor, over a field $ k $. Let $ G $ be the set of nonsingular points with coordinates in $ k $.

\begin{theorem}
$ G $ has the structure of an abelian group.
\end{theorem}

Fix one nonsingular point $ O $. This will be the zero in the group law. Define $ P + Q $ as follows. Let $ R $ be the third point of intersection of the line through $ P $ and $ Q $ with the curve. Let $ P + Q $ be the third point of intersection of the line through $ O $ and $ R $. Firstly note that if we take two points in $ G $ and draw the line through them, the third point of intersection is also in $ G $ by Bézout. If $ P = Q $, draw the tangent. $ O + P = P $ by definition. $ P + Q = Q + P $ by definition. To define $ -P $, let $ T $ be the third point of intersection of the tangent line to the curve at $ O $, let $ -P $ be the third point of intersection of the line through $ P $ and $ T $. Need to check that $ \rb{P + Q} + R = P + \rb{Q + R} $.

\begin{lemma}
\label{lem:5.7}
Given eight points in general position, there is a unique ninth point such that any cubic passing through the first eight points also goes through the ninth point.
\end{lemma}

What do we mean by in general position? Mean that there is some finite set of polynomial equations, and general position means that none of these equations is satisfied. When trying to prove that polynomial identities hold, it is enough to check them for a set of points in general position.

\begin{example*}
If we are thinking about polynomials in one variable, then the definition of a set of points in general position just means exclude finitely many points. A polynomial equation in one variable holds if it holds at all but finitely many points.
\end{example*}

The point is going to be that you can express $ P + \rb{Q + R} = \rb{P + Q} + R $ as a polynomial identity, and then check it away from some set of points determined by polynomial conditions. This is what we will do, using Lemma \ref{lem:5.7}.

\begin{proof}[Proof of Lemma \ref{lem:5.7}]
Any cubic in $ X, Y, Z $ homogeneous is a linear combination of
$$ X^3, Y^3, Z^3, X^2Y, XY^2, X^2Z, XZ^2, Y^2Z, YZ^2, XYZ. $$
So we have ten coefficients to choose. The condition of passing through the first eight points imposes eight conditions on the coefficients. Overall, we find that for eight points in general position, we have a two-dimensional space of cubics passing through them, that is there exist $ F, G $ cubics such that the general cubic of this form is given by $ \lambda F + \mu G $. By Bézout, $ F, G $ have nine points of intersection, and this gives us the ninth point we want.
\end{proof}

\lecture{17}{Friday}{09/11/18}

\begin{center}
\begin{tikzpicture}
\node(t) at (0, 0) {$ T $};
\node(pq) at (2, 0) {$ P + Q $};
\node(o) at (4, 0) {$ O $};
\node(q) at (0, 1) {$ Q $};
\node(r) at (2, 1) {$ R $};
\node(v) at (4, 1) {$ V $};
\node(p) at (0, 2) {$ P $};
\node(pqr) at (2, 2) {$ ? $};
\node(qr) at (4, 2) {$ Q + R $};
\draw (p) -- (pqr) -- (qr);
\draw (q) -- (r) -- (v);
\draw (t) -- (pq) -- (o);
\draw (p) -- (q) -- (t);
\draw (pqr) -- (r) -- (pq);
\draw (qr) -- (v) -- (o);
\end{tikzpicture}
\end{center}

Need to show that $ ? $ lies on the cubic, that is if we find the unique point of intersection of the line joining $ P + Q $ and $ R $, with the line joining $ P $ and $ Q + R $, this lies on the cubic we are considering. Apply Lemma \ref{lem:5.7} to the eight points $ O, P, Q, R, T, V, P + Q, Q + R $. Consider the cubic given by the union of the three horizontal lines, and the cubic given by the union of the three vertical lines. So the ninth point in Lemma \ref{lem:5.7} is $ ? $. Since the eight points all lie on our cubic, Lemma \ref{lem:5.7} gives $ ? $ lies on the cubic, as required.

\begin{definition}
An \textbf{elliptic curve} over a field $ k $ is a nonsingular plane cubic $ E $ together with a fixed choice of $ O $ defined over $ k $.
\end{definition}

\begin{definition}
A \textbf{point of inflexion} on a nonsingular plane cubic is a point where the tangent line does not meet the curve again, that is it meets the curve with multiplicity three.
\end{definition}

\begin{lemma}
\label{lem:5.10}
If $ E $ is an elliptic curve and $ O $ is chosen to be a point of inflexion, then $ P + Q + R = O $ iff $ P, Q, R $ are collinear.
\end{lemma}

\begin{proof}
Exercise.
\end{proof}

\begin{lemma}
Consider the cubic $ y^2 = f\rb{x} $, where $ f\rb{x} $ is monic of degree three.
\begin{enumerate}
\item This curve has a unique point at infinity, which is nonsingular.
\item The point at infinity is a point of inflexion.
\item The cubic is irreducible, and if $ char\rb{k} \ne 2 $, then the cubic is nonsingular iff $ f\rb{x} $ has distinct roots.
\end{enumerate}
\end{lemma}

\begin{proof}
\hfill
\begin{enumerate}
\item $ Y^2Z - X^3 - aX^2Z - bXZ^2 - cZ^3 $. Points at infinity have $ Z = 0 $, so $ X^3 = 0 $, so $ \sb{0 : 1 : 0} $ is the only point at infinity. $ \partial / \partial Z = Y^2 \ne 0 $ at $ \sb{0 : 1 : 0} $.
\item Since $ \partial / \partial X = \partial / \partial Y = 0 $ at $ \sb{0 : 1 : 0} $, the tangent line through $ \sb{0 : 1 : 0} $ is just $ Z = 0 $. So the tangent line meets the curve at points where $ Z = 0 $, that is at $ \sb{0 : 1 : 0} $.
\item (Exercise: irreducibility) Nonsingular points at infinity. For $ y^2 - f\rb{x} $, $ \partial / \partial y = 2y $, so any singular point would have $ y = 0 $ using $ char\rb{k} \ne 2 $. $ \partial / \partial x = -f'\rb{x} $, so a singular point would need $ f\rb{x} = f'\rb{x} = 0 $, that is you need a repeated root of $ f $.
\end{enumerate}
\end{proof}

\begin{fact*}
Any nonsingular plane cubic with a given point $ O $ can be put into the form $ y^2 + \gamma xy + \delta y = g\rb{x} $, with $ g\rb{x} $ a cubic, and the given point $ O $ being sent to infinity, by the Riemann-Roch theorem. If $ char\rb{k} \ne 2 $, can complete the square to get rid of the $ xy, y $ terms. If $ char\rb{k} \ne 3 $, can complete the cube, so we can assume that the elliptic curve is $ y^2 = x^3 + ax + b $. Lines through $ O = \sb{0 : 1 : 0} $ are lines $ x = k $ for some $ k $. If $ T = \rb{x_0, y_0} $, then $ P + Q = \rb{x_0, -y_0} $. In fact, since $ O $ is a point of inflexion, we can compute $ P $ as follows. Since $ P + \rb{-P} = O $ so $ O + P + \rb{-P} = O $, so by Lemma \ref{lem:5.10}, $ O, P, -P $ are collinear. So if $ P = \rb{x_1, y_1} $ then $ -P = \rb{x_1, -y_1} $. In particular, $ 2P = O $ iff $ P = O $ or $ P = \rb{x, 0} $ for some $ x $, that is the points with $ 2P = O $ are $ O $ and $ \rb{\alpha, 0}, \rb{\beta, 0}, \rb{\gamma, 0} $ where $ \alpha, \beta, \gamma $ are the roots of $ X^3 + aX + b $. The discriminant $ \Delta $ of $ y^2 = x^3 + ax + b $ is by definition the discriminant of $ x^3 + ax + b $, which is $ 4a^3 + 27b^2 $. So this is an elliptic curve iff $ x^3 + ax + b $ has three distinct roots, iff $ 4a^3 + 27b^2 \ne 0 $. (Exercise: the only changes of variable that take the equation $ y^2 = x^3 + ax + b $ to an equation of the same form are those of the form $ x' = u^2x $ and $ y' = u^3y $, such a change of variables scales $ a $ by $ u^4 $ and $ b $ by $ u^6 $, so $ \Delta $ changes by $ u^{12} $)
\end{fact*}

\lecture{18}{Tuesday}{13/11/18}

\begin{definition}
Let $ E $ be the elliptic curve $ y^2 = x^3 + ax + b $. Then $ j\rb{E} = -1728\rb{4a}^3 / \Delta $. This is called the \textbf{$ j $-invariant} of $ E $.
\end{definition}

\begin{proposition}
If $ k $ is algebraically closed, then two elliptic curves $ E $ and $ E' $ are isomorphic iff $ j\rb{E} = j\rb{E'} $.
\end{proposition}

\begin{proof}
Assume $ char\rb{k} \ne 2, 3 $, so that $ y^2 = x^3 + ax + b $ is $ E $ and $ y^2 = x^3 + Ax + B $ is $ E' $. If the two curves are isomorphic, then we can find $ u $ such that $ A = u^4a $ and $ B = u^6B $ and then $ j\rb{E} = j\rb{E'} $. Suppose conversely that $ j\rb{E} = j\rb{E'} $, that is $ A^3 / \rb{4A^3 + 27B^2} = a^3 / \rb{4a^3 + 27b^2} $, that is $ B^2 / A^3 = b^2 / a^3 $, that is $ \rb{B / b}^2 = \rb{A / a}^3 $, assuming that $ a, b, A, B \ne 0 $. Choose $ u $ such that $ u^{12} = \rb{B / b}^2 = \rb{A / a}^3 $. Modifying $ u $ by an appropriate twelfth root of unity, we can arrange that $ A / a = u^4 $, $ B / b = u^6 $. Then $ y' = u^3y $, $ x' = u^2x $ gives the required change of variables. It remains to consider the cases that some of $ a, b, A, B $ are zero. Writing the equation as $ A^3b^2 = a^3B^2 $, if $ A = 0 $ then $ B \ne 0 $ as $ 4A^3 + 27B^2 \ne 0 $, so $ a^3 = 0 $ and $ a = 0 $. Similarly $ B = 0 $ iff $ b = 0 $. In the first case $ A = a = 0 $, choose $ u $ such that $ B / b = u^6 $. In the second case $ B = b = 0 $, choose $ u $ such that $ A / a = u^4 $.
\end{proof}

Return to the case of irreducible singular cubics. Write as $ y^2 = f\rb{x} $, $ f\rb{x} $ a monic cubic, then this is singular iff $ f\rb{x} $ has a repeated root. There are two cases.
\begin{enumerate}
\item All three roots are equal, so $ y^2 = x^3 $. The only singular point is $ \rb{0, 0} $. Consider the group law with $ 0 = \sb{0 : 1 : 0} $. Because $ \sb{0 : 1 : 0} $ is a point of inflexion, Lemma \ref{lem:5.10} tells us that three points add to $ O $ iff they are collinear. Consider a line of the form $ ax + by = 1 $. This is the general form of a line not going through $ \rb{0, 0} $. $ x^3 = y^2 = y^2\rb{1} = y^2\rb{ax + by} $. Since $ \rb{x, y} \ne 0 $ we actually have $ y \ne 0 $, so we can divide by $ y $ and get $ \rb{x / y}^3 = a\rb{x / y} + b $. This is a cubic in $ x / y $, with no quadratic term. Conversely, any cubic in $ x / y $ with no quadratic term is of this form for some $ a, b $. So we see that $ \rb{x_1, y_1} + \rb{x_2, y_3} + \rb{x_3, y_3} = O $ iff $ x_1 / y_1 + x_2 / y_2 + x_3 / y_3 = 0 $. So in this case by mapping $ \rb{x, y} \mapsto x / y $ we get that the group law is given by the usual additive group.
\item Two roots are equal and the third is distinct, so $ y^2 = x^2\rb{x + 1} $. Again $ \rb{0, 0} $ is singular. $ x^3 = y^2 - x^2 = \rb{y + x}\rb{y - x} $. Let $ U = y + x $ and $ V = y - x $, so $ \rb{U - V}^3 = 8UV $. Again $ \rb{0, 0} $ is the only singular point. Consider lines $ aU + bV = 1 $. $ \rb{U - V}^3 = 8UV\rb{1} = 8UV\rb{aU + bV} $. Write $ t = U / V $, get $ \rb{t - 1}^3 = 8t\rb{at + b} $, so $ t^3 - \rb{8a + 3}t^2 + \rb{8b + 3}t - 1 = 0 $. So we have a general monic cubic with constant term $ -1 $, that is with roots having product equal to one. So sending $ \rb{x, y} \mapsto U / V = \rb{y + x} / \rb{y - x} $ gives an isomorphism between the group law on $ y^2 = x^2\rb{x + 1} $ and the group $ k^* $ under multiplication.
\end{enumerate}

\lecture{19}{Thursday}{15/11/18}

\begin{example*}
Good way to practice explicit computations is to work over small fields. Let $ k = \F_5 = \Z / 5\Z $ be the field of five elements and $ y^2 = x^3 + 1 $. $ \Delta = 4 \ne 0 $ in $ k $. So this is an elliptic curve. Squares mod five are $ 0, 1, 4 $.
$$
\begin{array}{c|ccccc}
x & 0 & 1 & 2 & 3 & 4 \\
\hline
y & \pm 1 & \times & \pm 2 & \times & 0
\end{array}.
$$
Also $ O = \sb{0 : 1 : 0} $. So six points in total. Since $ \Z / 6\Z \cong \Z / \Z / 2\Z \times \Z / 3\Z $ by the Chinese remainder theorem, the points form an abelian group of order six. $ \rb{4, 0} $ is the only point of order two. The line $ Y = Z $ meets the curve at $ \rb{0, 1} $ in projective coordinates, and is actually the tangent line at this point, so this point has order three. If we are identifying our group of points with $ \Z / 6\Z $, then we have to identify $ \rb{4, 0} $ with $ 3 + 6\Z $, $ \rb{0, 1} $ with $ \pm 2 + 6\Z $, and $ \rb{0, -1} = -\rb{0, 1} $ with $ \mp 2 + 6\Z $. So $ \rb{2, \pm 2} $ must be the points of order six. More explicitly, the line through $ \rb{4, 0} $ and $ \rb{0, 1} $ is $ y = x + 1 $, which also passes through $ \rb{2, -2} $. So $ \rb{4, 0} + \rb{0, 1} + \rb{2, -2} = O $, so $ \rb{4, 0} + \rb{0, 1} = -\rb{2, -2} = \rb{2, 2} $. (Exercise: figure out $ n\rb{2, 2} $ for $ 1 \le n \le 5 $)
\end{example*}

For $ E $ an elliptic curve, $ E\rb{k} $ are the points of $ E $ with coordinates in $ k $, including $ O = \sb{0 : 1 : 0} $. If $ k \subset K $, $ E\rb{K} $ are the points with coordinates in $ K $.

\begin{example*}
If $ k $ is a finite field, then $ E\rb{k} $ is either cyclic or a product of two cyclic groups, by the Hasse bound or Hasse-Weil bound on number of points.
\end{example*}

\begin{example*}
If $ k = \C $, $ E\rb{\C} \cong \C / \Lambda $ as abelian groups or Riemann surfaces, where $ \Lambda \subset \C $ is a lattice, that is a subgroup isomorphic to $ \Z^2 $. Over $ \C $, we have exactly $ n^2 $ points $ P $ with the property that $ nP = O $. These points are just
$$ \dfrac{\tfrac{1}{n}\Lambda}{\Lambda} \cong \dfrac{\tfrac{1}{n}\Z^2}{\Z^2} \cong \dfrac{\Z^2}{n\Z^2} \cong \rb{\dfrac{\Z}{n\Z}}^2. $$
\end{example*}

\begin{definition}
We say that $ P $ is an \textbf{$ n $-torsion point} if $ nP = O $. Say that $ P $ is \textbf{torsion} if it is $ n $-torsion for some $ n \ge 1 $. The set of $ n $-torsion points is a group, because if $ nP = O $, $ nQ = O $, then $ n\rb{-P} = -\rb{nP} = -O = O $, and $ n\rb{P + Q} = nP + nQ = O + O = O $. Write $ E\rb{k}\sb{n} $ for the $ n $-torsion points of $ E\rb{k} $.
\end{definition}

\begin{example*}
We just saw that $ E\rb{C}\sb{n} \cong \rb{\Z / n\Z}^2 $.
\end{example*}

$ k = \Q_p $ is the next few lectures. $ E\rb{\Q_p} $ is locally isomorphic to $ \Z_p $. Approximation of formal groups in the next few lectures. $ k = \Q $ is the Mordell-Weil theorem. $ E\rb{\Q} $ is a finitely generated abelian group, that is there is a surjection $ \Z^N \twoheadrightarrow E\rb{\Q} $ for some $ N $. Structure theorem for finitely generated abelian groups states that any finitely generated abelian group is isomorphic to a group of the form $ \Z^r \oplus X $, where $ X $ is finite. Furthermore, $ r $ is uniquely determined, as is the isomorphism classes of $ X $. Call $ r $ the \textbf{rank} of the group. In fact if $ G $ is a finitely generated abelian group. We can write $ G \cong G^{free} \times G_{tor} $, where $ G^{free} \cong \Z^r $ for some $ r $, and $ G_{tor} $ are the elements of $ G $ of finite order, which are the elements which are $ n $-torsion for some $ n $. $ G_{tor} $ is a finite group.

\lecture{20}{Friday}{16/11/18}

So $ E\rb{\Q} = T \times F \cong T \times \Z^r $, where $ T $ is a finite torsion subgroup and $ r \ge 0 $ is the rank of $ E $ over $ \Q $. It is a hard theorem of Mazur in 1970s that there are only finitely many possibilities for $ T $. It is unknown whether or not $ r $ can be arbitrarily large. Same results hold if $ \Q $ is replaced by a number field. Main goal of the remaining lectures is to prove that $ E\rb{\Q} $ is finitely generated, and to see how to compute its torsion subgroup, and its rank. $ \Q / \Z $ is an infinite group, all of whose elements have finite order, not finitely generated.

\pagebreak

\section{Torsion in $ E\rb{\Q} $}

$ y^2 = x^3 + ax + b $. By choosing $ u $ appropriately in $ x \mapsto u^2x $ and $ y \mapsto u^3y $, we can assume that our equation is of the form $ y^2 = x^3 + Ax + B $ for $ A, B \in \Z $.

\begin{theorem}[Lutz-Nagell]
The subgroup of $ E\rb{\Q} $ consisting of torsion points is finite. Furthermore, if $ \rb{x, y} $ is a torsion point, then $ x, y \in \Z $. Either $ y = 0 $, or $ y^2 \mid \Delta = 4A^3 + 27B^2 $.
\end{theorem}

\begin{note*}
Converse is not true, that is there can be points $ \rb{x, y} $ of $ E\rb{\Q} $ with $ y^2 \mid \Delta $, but $ \rb{x, y} $ is not torsion.
\end{note*}

\begin{example*}
$ y^2 = x^3 + 17 $. $ \rb{4, 9} $ has coordinates in $ \Z $, but is not a torsion point, because $ 9^2 \nmid \rb{27}\rb{17}^2 $. So the rank of this curve is at least one.
\end{example*}

To prove $ x, y \in \Z $, it suffices to prove that $ x, y \in \Z_p $ for all $ p $, so it suffices to prove that any torsion point in $ E\rb{\Q_p} $ has coordinates in $ \Z_p $. We will come back to this. Suppose from now on that we have proved that all torsion points coordinates in $ \Z $. Doubling formula is if $ P = \rb{x_0, y_0} $, then $ 2P = \rb{x_d, y_d} $, where $ x_d = m^2 - 2x_0 $ for $ m = \rb{3x_0^2 + A} / \rb{2y_0} $.

\begin{example*}
For $ y^2 = x^3 + 17 $, $ 2\rb{4, 9} = \rb{-8 / 9, 109 / 27} $.
\end{example*}

If $ \rb{x, y} \in E\rb{\Q} $ with $ x, y \in \Z $, and $ \rb{x, y} $ is torsion, then $ 2\rb{x, y} $ is torsion, and thus has coordinates in $ \Z $. If $ y = 0 $, there is nothing to prove, because $ 2\rb{x, y} = 0 $. Otherwise, the doubling formula tells us that $ \rb{3x^2 + A} / \rb{2y} \in \Z $. In particular, $ y^2 \mid \rb{3x^2 + A}^2 $. We also have $ y^2 = x^3 + Ax + B $.
$$ \rb{3x^2 + A}^2\rb{3x^2 + 4A} - \rb{x^3 + Ax + B}\rb{27x^3 + 27Ax - 27B} = 4A^3 + 27B^2 = \Delta, $$
by theory of resultants. So $ y^2 \mid \Delta $, as required.

\begin{example*}
Let $ y^2 = x^3 + 1 $. $ y = 0 $ or $ y^2 \mid 27 $, so $ y = \pm 1, \pm 3 $, so $ \rb{-1, 0} $, $ \rb{0, \pm 1} $, $ \rb{2, \pm 3} $. So there are at most six torsion points. $ \rb{-1, 0} $ has order two. $ y = 1 $ meets the curve exactly at $ \rb{0, 1} $, so $ \rb{0, 1} $ is a point of order three. So the torsion subgroup has order a multiple of six, so exactly six, and it is therefore cyclic of order six.
\end{example*}

\begin{example*}
Let $ y^2 = x^3 + 17 $. $ y = 0 $ has no solutions, so $ y^2 \mid \rb{27}\rb{17}^2 $ gives $ y \mid \rb{3}\rb{17} $. If $ 17 \mid y $, then $ x^3 = 17 - y^2 $ is divisible by $ 17 $ but not $ 17^2 $, a contradiction. So $ y \mid 3 $. Only possibilities are $ \rb{x, y} = \rb{-2, \pm 3} $. Applying the doubling formula, get $ \rb{8, \pm 9} $. Since we already showed that the only possibilities for torsion points are $ \rb{-2, \pm 3} $ and $ O $, $ \rb{8, \pm 9} $ is not torsion, so $ \rb{-2, \pm 3} $ is also not torsion. So the only torsion point is $ O $.
\end{example*}

\pagebreak

\section{Elliptic curves over $ \Q_p $}

\lecture{21}{Tuesday}{20/11/18}

Let $ y^2 = x^3 + ax + b $ for $ a, b \in \Q_p $ and $ \Delta = 4a^3 + 27b^2 \ne 0 $. Recall that there is a ring homomorphism $ \Z_p \to \F_p = \Z / p\Z $. In fact $ \Z_p / p\Z_p \cong \Z / p\Z $ by $ a_0 + a_1p + \dots \mapsto a_0 $. There is no ring homomorphism $ \Q_p \to \F_p $. There is a well-defined map $ \P^n\rb{\Q_p} \to \P^n\rb{\F_p} $. $ \sb{a_0 : \dots : a_n} = \sb{b_0 : \dots : b_n} $ for $ b_i \in \Z_p $, at least one $ b_i \in \Z_p^* $. Take $ b_i = p^sa_i $ for appropriate $ s $. Then $ \abs{a_0 : \dots : a_n} \mapsto \sb{\overline{b_0} : \dots : \overline{b_n}} $, where $ \overline{b_i} $ is the image of $ b_i $ in $ \F_p $. This is a well-defined map $ \P^n\rb{\Q_p} \to \P^n\rb{\F_p} $.

\begin{example*}
$ \rb{x, y} = \rb{1 / p, 1 / p} $ is $ \sb{x : y : 1} = \sb{1 / p : 1 / p : 1} = \sb{1 : 1 : p} \mapsto \sb{1 : 1 : 0} $.
\end{example*}

By replacing $ x $ with $ p^{2n}x $, $ y $ by $ p^{3n}y $, can assume $ a, b \in \Z_p $. Reduction mod $ p $ of $ y^2 = x^3 + ax + b $ is $ y^2 = x^3 + \overline{a}x + \overline{b} $ for $ \overline{a}, \overline{b} \in \F_p $. This would be singular if either $ p = 2 $, or $ p \mid \Delta = 4a^3 + 27b^2 $. Even if it is singular, we still have a group law on the nonsingular points.

\begin{example*}
Let $ y^2 = x^3 - 18 $. Modulo $ 3 $, we get $ y^2 = x^3 $, and $ \rb{0, 0} $ is singular. $ \rb{3, 3} $ is a point over $ \Q_3 $, which reduces to the singular point $ \rb{0, 0} $. Claim there is a point over $ \Q_3 $ with $ x = 1 / 9 $. Need to solve $ y^2 = \rb{1 / 9}^3 - 18 = \rb{1 - 2\rb{3}^8} / 3^6 $, that is $ \rb{3^3y}^2 = 1 - 2\rb{3}^8 $, which has two solutions in $ \Q_3 $ by Hensel's lemma. This point $ \rb{1 / 3^2, 1 / 3^3 + \epsilon} $ reduces to $ \sb{0 : 1 : 0} \mod 3 $, that is to the point $ O $ at infinity.
\end{example*}

\begin{lemma}
\label{lem:7.1}
\hfill
\begin{enumerate}
\item Let $ y^2 = g\rb{x} $ for $ g\rb{x} \in \Z_p\sb{X} $ monic cubic. If $ \rb{x_0, y_0} $ is a point of this cubic, then either $ x_0, y_0 \in \Z_p $ or there exists $ n \ge 1 $ such that $ \abs{x_0}_p = p^{2n} $, $ \abs{y_0}_p = p^{3n} $. Say that such a point has \textbf{level} $ n $.
\item If $ g\rb{x} $ reduces to $ x^3 \mod p $ and $ \abs{x_0}_p, \abs{y_0} \le 1 $, then either $ \abs{x_0}_p, \abs{y_0}_p < 1 $, or $ \abs{x_0}_p = \abs{y_0}_p = 1 $.
\end{enumerate}
\end{lemma}

\begin{proof}
\hfill
\begin{enumerate}
\item Let $ y^2 = x^3 + \alpha x^2 + \beta x + \gamma $ for $ \alpha, \beta, \gamma \in \Z_p $. If $ \abs{x}_p > 1 $, then $ \abs{x^3}_p > \abs{\alpha x^2}_p, \abs{\beta x}_p, \abs{\gamma}_p $. So by the ultrametric inequality, $ \abs{y}_p^2 = \abs{y^2}_p = \abs{x^3 + \alpha x^2 + \beta x + \gamma}_p = \abs{x^3}_p = \abs{x}_p^3 $.
\item Need $ p \mid x_0 $ iff $ p \mid y_0 $, but $ x_0^3 \equiv y_0^2 \mod p $, so this is clear.
\end{enumerate}
\end{proof}

Write $ E $ for $ y^2 = g\rb{x} $ for $ g\rb{x} \in \Z_p\sb{x} $ monic cubic. Let $ E\rb{\Q_p}^{\rb{0}} $ be $ P \in E\rb{\Q_p} $ such that $ P $ maps to a nonsingular point mod $ p $, and let $ E\rb{\Q_p}^{\rb{1}} \subseteq E\rb{\Q_p}^{\rb{0}} $ be $ P \in E\rb{\Q_p} $ such that $ P $ maps to $ O \mod p $. What are the points $ \rb{x, y} \in E\rb{\Q_p}^{\rb{1}} $? They are the points with $ \abs{x}_p, \abs{y}_p > 1 $ by Lemma \ref{lem:7.1}.

\begin{lemma}
\label{lem:7.2}
$ E\rb{\Q_p}^{\rb{0}} $ is a subgroup of $ E\rb{\Q_p} $. Reduction mod $ p $ is a group homomorphism from $ E\rb{\Q_p}^{\rb{0}} $ to the group of nonsingular points mod $ p $. The kernel of this homomorphism is $ E\rb{\Q_p}^{\rb{1}} $.
\end{lemma}

\begin{proof}
The group law is determined by three points add up to $ O $ iff they are collinear. Also, by Bézout, if two points on a line are nonsingular, so is the third. To see that $ E\rb{\Q_p}^{\rb{0}} $ is a subgroup of $ E\rb{\Q_p} $, suppose that $ P + Q + R = O $ and $ P, Q \in E\rb{\Q_p}^{\rb{0}} $. Then $ P, Q, R $ are collinear. Then their images mod $ p $ are also collinear, and the images of $ P, Q $ are nonsingular. So the image of $ R \mod p $ is also nonsingular, so $ R \in E\rb{\Q_p}^{\rb{0}} $. This also shows that reduction mod $ p $ is a group homomorphism. The kernel is $ E\rb{\Q_p}^{\rb{1}} $ by definition.
\end{proof}

In fact, the homomorphism $ E\rb{\Q_p}^{\rb{0}} $ to the nonsingular points over $ \F_p $ is surjective, in example sheet 4 by Hensel's lemma.

\lecture{22}{Thursday}{22/11/18}

For $ n \ge 1 $, let $ E\rb{\Q_p}^{\rb{n}} $ be the points of $ E\rb{\Q_p} $ with level at least $ n $.

\begin{corollary}
$ E\rb{\Q_p}^{\rb{n}} $ is a subgroup of $ E\rb{\Q_p} $.
\end{corollary}

\begin{proof}
$ n = 0 $, $ n = 1 $ already done. If $ n = 2 $, then let $ E' $ be the elliptic curve obtained from $ E $ by the change of variables $ Y^3 = p^3y $, $ X = p^2x $. Then the map $ E \to E' $ given by $ \rb{x, y} \mapsto \rb{p^2x, p^3y} $ identifies $ E\rb{\Q_p}^{\rb{n}} $ with $ E'\rb{\Q_p}^{\rb{n - 1}} $. So the result follows by induction on $ n $.
\end{proof}

\begin{corollary}
For $ n \ge 1 $, there is a natural injection from $ E\rb{\Q_p}^{\rb{n}} / E\rb{\Q_p}^{\rb{n + 1}} \to \Z / p\Z $.
\end{corollary}

\begin{proof}
Let $ E_n $ be the curve with equation $ y^2 = x^3 + p^{4n}ax + p^{6n}b $. Then we have a map $ E \to E_n $ by $ \rb{x, y} \mapsto \rb{p^{2n}x, p^{3n}y} $. This map identifies $ E\rb{\Q_p}^{\rb{n}} $ with $ E_n\rb{\Q_p}^{\rb{0}} $, and $ E\rb{\Q_p}^{\rb{n + 1}} $ with $ E_n\rb{\Q_p}^{\rb{1}} $. So $ E\rb{\Q_p}^{\rb{n}} / E\rb{\Q_p}^{\rb{n + 1}} \cong E\rb{\Q_p}^{\rb{0}} / E\rb{\Q_p}^{\rb{1}} $. By Lemma \ref{lem:7.2}, this injects into the group of smooth points of $ E_n $ over $ \F_p $. But $ E_n $ is given by $ y^2 = x^3 \mod p $. We saw that over any field, the nonsingular points are isomorphic to the additive group of the field, via $ \rb{x, y} \mapsto x / y $.
\end{proof}

Define $ u : E\rb{\Q_p}^{\rb{1}} \to p\Z_p $ by $ O \mapsto O $ and $ \rb{x, y} \mapsto x / y $. If $ \rb{x, y} $ has level exactly $ n $, then $ \abs{x} = p^{2n} $, $ \abs{y} = p^{3n} $, so $ \abs{u\rb{x, y}} = p^{-n} $, so $ u\rb{x, y} \in p^n\Z_p^* $. $ \abs{u\rb{P}} = \abs{u\rb{-P}} $ because if $ P = \rb{x, y} $, then $ -P = \rb{x, -y} $, so $ \abs{u\rb{P}} = \abs{x / y} = \abs{-x / y} = \abs{u\rb{-P}} $.

\begin{lemma}
\label{lem:7.5}
If $ P, Q \in E\rb{\Q_p}^{\rb{1}} $, then $ \abs{u\rb{P + Q} - u\rb{P} - u\rb{Q}} \le \max\rb{\abs{u\rb{P}}^5, \abs{u\rb{Q}}^5} $.
\end{lemma}

\begin{proof}
If any of $ P, Q, P + Q $ equal $ O $, then left hand side is zero and we are done. Assume by symmetry that $ P $ has level exactly $ n $, and $ Q $ has level at least $ n $. Then $ \abs{u\rb{P}} = \abs{u\rb{Q}} $ and $ \abs{u\rb{P}} = p^{-n} $, so right hand side is $ p^{-5n} $. Since $ P, Q \in E\rb{\Q_p}^{\rb{n}} $, and $ E\rb{\Q_p}^{\rb{n}} $ is a group, we have $ P + Q \in E\rb{\Q_p}^{\rb{n}} $. Let $ P_n, Q_n $ be the points on $ E_n $ corresponding to $ P, Q $. Let $ R $ be the third point of intersection of $ E_n $ and the line through $ P_n, Q_n $. $ P_n $ has level zero, and $ Q_n $ has level at least zero, so both are nonsingular mod $ p $, so $ R $ is nonsingular mod $ p $. Since $ P_n $ has level exactly zero, the reduction of $ P_n $ is a nonsingular point of $ E_n\rb{\F_p} $, which is not $ O $. This implies that the line through $ P_n, Q_n $ has an equation of the form $ lx + my = 1 $, with $ l, m \in \Z_p $. A priori we have a line $ lx + my = 1 $ with $ l, m \in \Q_p $, but if either $ l, m \notin \Z_p $, when you reduce mod $ p $, the line would go through $ \rb{0, 0} $. Since $ P_n, Q_n, R $ are all nonsingular mod $ p $, this would be a contradiction. The equation of $ E_n $ is $ y^2 = x^3 + p^{4n}ax + p^{6n}b $. To compute the intersection, we can solve $ y^2\rb{lx + my} = x^3 + p^{4n}ax\rb{lx + my}^2 + p^{6n}b\rb{lx + my}^3 $. Write $ t = x / y $. Dividing by $ y^3 $, we get $ lt + m = t^3 + p^{4n}at\rb{lt + m}^2 + p^{6n}b\rb{lt + m}^3 $. Write as $ c_0t^3 + c_1t^2 + c_2t + c_3 = 0 $, with $ c_0 = 1 + p^{4n}al^2 + p^{6n}bl^3 $ and $ c_1 = p^{4n}\rb{2alm + 3p^{2n}bl^2m} $. Therefore the sum of the roots of this polynomial has norm at most $ p^{-4n} $. By definition, the roots of this polynomial are $ p^{-n}u\rb{P} $, $ p^{-n}u\rb{Q} $, $ -p^{-n}u\rb{P + Q} $. So we are done.
\end{proof}

\lecture{23}{Friday}{23/11/18}

If $ \abs{x + y} < \abs{x} $, then $ \abs{x} = \abs{y} $. Write $ x = \rb{x + y} + \rb{-y} $. So $ \abs{x + y} < \abs{x} = \abs{\rb{x + y} + \rb{-y}} = \max\rb{\abs{x + y}, \abs{y}} = \abs{y} $.

\begin{corollary}
\hfill
\begin{enumerate}
\item For all $ n \ge 1 $, $ \abs{u\rb{nP} - nu\rb{P}} \le \abs{u\rb{P}}^5 $ and $ \abs{u\rb{nP}} \le \abs{u\rb{P}} $.
\item If $ p \nmid n $, then $ \abs{u\rb{nP}} = \abs{u\rb{P}} $.
\item $ \abs{u\rb{pP}} = \abs{p}\abs{u\rb{P}} = p^{-1}\abs{u\rb{P}} $.
\item $ \abs{u\rb{nP}} = \abs{n}\abs{u\rb{P}} $ for all $ n \ge 1 $.
\end{enumerate}
\end{corollary}

\begin{proof}
\hfill
\begin{enumerate}
\item Induction on $ n $. $ n = 1 $ is trivial. If $ \abs{u\rb{nP} - nu\rb{P}} \le \abs{u\rb{P}}^5 $ then we claim that $ \abs{u\rb{nP}} \le \abs{u\rb{P}} $, otherwise $ \abs{u\rb{nP}} = \abs{u\rb{P}} $ gives $ \abs{u\rb{nP} - nu\rb{P}} = \abs{u\rb{nP}} > \abs{u\rb{P}} > \abs{u\rb{P}}^5 $, a contradiction. So assume both parts of (i) hold for $ n $. It is then enough to check that the first part holds for $ n + 1 $.
\begin{align*}
\abs{u\rb{\rb{n + 1}P} - \rb{n + 1}u\rb{P}}
& = \abs{u\rb{\rb{n + 1}P} - u\rb{nP} - u\rb{P} + u\rb{nP} - nu\rb{P}} \\
& \le \max\rb{\abs{u\rb{\rb{n + 1}P} - u\rb{nP} - u\rb{P}}, \abs{u\rb{nP} - nu\rb{P}}} \\
& \le \max\rb{\abs{u\rb{nP}}^5, \abs{u\rb{P}}^5, \abs{u\rb{nP} - u\rb{P}}} \\
& \le \max\rb{\abs{u\rb{P}}^5, \abs{u\rb{P}}^5} \\
& = \abs{u\rb{P}}^5,
\end{align*}
by Lemma \ref{lem:7.5}.
\item If $ p \nmid n $ then $ \abs{u\rb{nP} = \abs{u\rb{P}}} $. We have $ \abs{nu\rb{P}} = \abs{n}\abs{u\rb{P}} = \abs{u\rb{P}} $.
$$ \abs{u\rb{nP} - nu\rb{P}} \le \abs{u\rb{P}}^5 = \abs{nu\rb{P}}^5 < \abs{nu\rb{P}}. $$
So $ \abs{u\rb{nP}} = \abs{nu\rb{P}} = \abs{u\rb{P}} $ by $ x = u\rb{nP} $ and $ y = nu\rb{P} $ above.
\item $ \abs{u\rb{pP} - pu\rb{P}} \le \abs{u\rb{P}}^5 < \abs{pu\rb{P}} $, since $ \rb{\abs{u\rb{P}}^4 \le \abs{p}^4 < \abs{p}} $. So again $ \abs{u\rb{pP}} = \abs{pu\rb{P}} $.
\item Induction on $ n $. If $ p \nmid n $ then use (ii). Otherwise use (iii).
$$ \abs{u\rb{nP}} = \abs{u\rb{p\rb{\dfrac{n}{p}P}}} = \abs{p}\abs{u\rb{\dfrac{n}{p}P}} = \abs{p}\abs{\dfrac{n}{p}}\abs{u\rb{P}} = \abs{n}\abs{u\rb{P}}, $$
by (iii) and induction.
\end{enumerate}
\end{proof}

\begin{corollary}
\label{cor:7.7}
$ E\rb{\Q_p}^{\rb{1}} $ has no torsion points other than $ O $.
\end{corollary}

\begin{proof}
If $ nP = O $ for some $ n \ge 1 $, then $ \abs{n}\abs{u\rb{P}} = u\rb{O} = 0 $. So $ u\rb{P} = 0 $, and $ P = O $.
\end{proof}

By Lemma \ref{lem:7.1}, $ E\rb{\Q_p}^{\rb{1}} $ is the points in $ E\rb{\Q_p} $ which are not of the form $ \rb{x, y} $ with $ x, y \in \Z_p $, so this completes the proof of Lutz-Nagell.

\begin{corollary}
\label{cor:7.8}
If $ E $ is of the form $ y^2 = x^3 + Ax + B $ for $ A, B \in \Z $, and $ p \nmid 2\Delta $, then the subgroup of torsion points of $ E\rb{\Q} $ maps injectively to the points $ E\rb{\F_p} $.
\end{corollary}

\begin{proof}
Since $ p \nmid \Delta $ and $ p \ne 2 $, $ y^2 = x^3 + Ax + B $ is nonsingular over $ \F_p $, so by definition, $ E\rb{\Q_p} = E\rb{\Q_p}^{\rb{0}} $. The kernel of $ E\rb{\Q_p} \to E\rb{\F_p} $ is by definition $ E\rb{\Q_p}^{\rb{1}} $, which does not contain any torsion points, by Corollary \ref{cor:7.7}.
\end{proof}

\begin{example*}
Let $ E : y^2 = x^3 + 105^{10^{10}}x + 3 $. Use Corollary \ref{cor:7.8} to study torsion points. $ 5, 7 \nmid \Delta = 4\rb{105}^{10^{10}} + 7\rb{3}^2 $. Modulo $ 5 $ or $ 7 $, we have $ y = x^3 + 3 $. Modulo $ 5 $, squares are $ 0, 1, 4 $.
$$
\begin{array}{c|ccccc}
x & 0 & 1 & 2 & 3 & 4 \\
\hline
x^3 + 3 & 3 & 4 & 1 & 0 & 2 \\
\#y & 0 & 2 & 2 & 1 & 0
\end{array}.
$$
Thus $ \#E\rb{\F_5} = 6 $. Note that $ y = 0 $ gives $ x^3 + 105^{10^{10}}x + 3 = 0 $, so $ x \mid 3 $ gives no solutions in $ \Q $. So already $ E\rb{\Q}_{tor} $ is either trivial, or has order three. Modulo $ 7 $, squares are $ 0, 1, 2, 4 $.
$$
\begin{array}{c|ccccccc}
x & 0 & 1 & 2 & 3 & 4 & 5 & 6 \\
\hline
x^3 + 3 & 3 & 4 & 4 & 2 & 4 & 2 & 2 \\
\#y & 0 & 2 & 2 & 2 & 2 & 2 & 2
\end{array}.
$$
Thus $ \#E\rb{\F_7} = 13 $. $ \rb{6, 13} = 1 $ so $ E\rb{\Q}_{tors} = \cb{1} $.
\end{example*}

\lecture{24}{Tuesday}{27/11/18}

Lecture 24 is a problem class.

\pagebreak

\section{$ E\rb{\Q} / 2E\rb{\Q} $}

\lecture{25}{Thursday}{29/11/18}

Goal is if $ E\rb{\Q} $ is an elliptic curve then $ E\rb{\Q} / 2E\rb{\Q} $ is finite. Assume from now on that we can write $ E $ as $ y^2 = \rb{x - e_1}\rb{x - e_2}\rb{x - e_3} $ for $ e_1, e_2, e_3 \in \Q $. This is equivalent to all of the $ 2 $-torsion points being defined over $ \Q $, by Silverman's arithmetic of elliptic curves.

\begin{example*}
Let $ y^2 = x^3 - x = \rb{x - 1}x\rb{x + 1} $. Suppose that we have a point $ \rb{x, y} $ with $ x, y \in \Q $, $ y \ne 0 $. Write $ x - 1 = au^2 $, $ x = bv^2 $, $ x + 1 = cw^2 $, where $ u, v, w \in \Q^* $ and $ a, b, c \in Z $ are squarefree, such as $ -3 / 25 = \rb{-3}\rb{1 / 5}^2 $. $ y^2 = \rb{x - 1}x\rb{x + 1} = \rb{abc}\rb{uvw}^2 $ gives $ abc = \rb{y / uvw}^2 $ is a square. Claim that $ a, b, c \in \cb{\pm 1, \pm 2} $. Let $ p $ be an odd prime. We need to show that $ \abs{x}_p, \abs{x + 1}_p, \abs{x - 1}_p $ are all squares. If $ \abs{x}_p > 1 $, then by the ultrametric inequality, $ \abs{x}_p = \abs{x + 1}_p = \abs{x - 1}_p $, and since $ \abs{y}_p^2 = \abs{x}_p\abs{x + 1}_p\abs{x - 1}_p $, so each of $ \abs{x}_p, \abs{x + 1}_p, \abs{x - 1}_p $ is a square. If $ \abs{x}_p \le 1 $, then since $ a, b, c $ are squarefree and their product is a square, we either have $ \abs{x}_p = \abs{x + 1}_p = \abs{x - 1}_p = 1 $, or at least two of $ x, x + 1, x - 1 $ are divisible by $ p $. Then $ p $ would have to divide the difference of these two quantities, which is one or two. But $ p $ is odd, a contradiction. Since $ abc $ is a square, $ a $ and $ b $ determine $ c $, such as $ a = -1 $, $ b = -2 $, then $ c = 2 $. Since $ x + 1 > x > x - 1 $, and $ \rb{x + 1}x\rb{x - 1} = y^2 > 0 $, we either have $ x + 1 > x > x - 1 > 0 $, or $ x + 1 > 0 > x > x - 1 $. So either $ a, b < 0 $ and $ c > 0 $ or $ a, b, c > 0 $. So the only possibilities for $ a, b, c $ are
$$ \rb{1, 1, 1}, \rb{-1, -1, 1}, \rb{1, 2, 2}, \rb{-1, -2, 2}, \rb{2, 1, 2}, \rb{-2, -1, 2}, \rb{2, 2, 1}, \rb{-2, -2, 1}. $$
Consider the point $ \rb{x, y} = \rb{0, 0} $. Then $ x - 1 = -1 = -1\rb{1}^2 $, $ x = 0 $, $ x + 1 = 1 = 1\rb{1}^2 $, so $ a = -1 $, $ b = -1 $, $ c = 1 $. $ \rb{-1, 0} $ gives $ \rb{-2, -1, 2} $ and $ \rb{1, 0} $ gives $ \rb{2, 1, 2} $. Claim that there are no points $ \rb{x, y} $ of $ E\rb{\Q} $ giving $ \rb{a, b, c} = \rb{1, 2, 2} $, so $ x - 1 = u^2 $, $ x = 2v^2 $, $ x + 1 = 2w^2 $. Eliminating $ x $, we must solve $ 2v^2 - 1 = u^2 $, $ 2v^2 + 1 = 2w^2 $. Write $ u = U / Z $, $ v = V / Z $, $ w = W / Z $, for $ U, V, W, Z \in \Z $. $ 2V^2 - Z^2 = U^2 $, $ 2V^2 + Z^2 = 2W^2 $ so $ 2 \mid Z $ and $ 2 \mid U $. Then $ 2V^2 = U^2 + Z^2 \equiv 0 \mod 4 $ gives $ 2 \mid V $. Then $ 2 \mid W $. So $ 2 \mid U, V, W, Z $. Dividing by two and repeating, $ U = V = W = Z = 0 $, a contradiction. We will soon prove that the map
$$ \function[\delta]{E\rb{\Q}}{\Q^* / \rb{\Q^*}^2 \times \Q^* / \rb{\Q^*}^2 \times \Q^* / \rb{\Q^*}^2}{\rb{x, y}}{\rb{a, b, c}} $$
is a group homomorphism. In particular, the image of $ \delta $ is a subgroup of the right hand side. We have seen that it contains $ \rb{1, 1, 1}, \rb{-1, -1, 1}, \rb{2, 1, 2}, \rb{-2, -1, 2} $. It does not contain $ \rb{1, 2, 2} $. So it also cannot contain $ \rb{-1, -2, 2}, \rb{2, 2, 1}, \rb{-2, -2, 1} $. So the image of $ \delta $ has order four. We will also show that $ Ker\rb{\delta} = 2E\rb{\Q} $. So by the first isomorphism theorem, we have $ E\rb{\Q} / 2E\rb{\Q} \cong \rb{\Z / 2\Z}^2 $. In particular, $ E\rb{\Q} / 2E\rb{\Q} $ is finite, for this choice of $ E $.
\end{example*}

\lecture{26}{Friday}{30/11/18}

\begin{theorem}
\label{thm:8.2}
Let $ K $ be a field with $ char\rb{K} \ne 2 $, such as $ K = \Q $. Let $ E : y^2 = \rb{x - e_1}\rb{x - e_2}\rb{x - e_3} $ for $ e_i \in K $ distinct. Define
\begin{align*}
\delta : E\rb{K} & \to \rb{K^* / \rb{K^*}^2}^3 \\
O & \mapsto \rb{1, 1, 1} \\
\rb{e_1, 0} & \mapsto \rb{\rb{e_1 - e_2}\rb{e_1 - e_3}, e_1 - e_2, e_1 - e_3} \\
\rb{e_2, 0} & \mapsto \rb{e_2 - e_1, \rb{e_2 - e_1}\rb{e_2 - e_3}, e_2 - e_3} \\
\rb{e_3, 0} & \mapsto \rb{e_3 - e_1, e_3 - e_2, \rb{e_3 - e_1}\rb{e_3 - e_2}} \\
\rb{x, y} & \mapsto \rb{x - e_1, x - e_2, x - e_3} & y \ne 0.
\end{align*}
Then
\begin{enumerate}
\item $ \delta $ is a group homomorphism.
\item $ Ker\rb{\delta} = 2E\rb{K} $.
\end{enumerate}
\end{theorem}

\begin{lemma}
\label{lem:8.3}
Take $ E, K $ as above. If $ x - e_1, x - e_2, x - e_3 $ are all squares for some $ P = \rb{x, y} \in E\rb{K} $, then there exists $ Q \in E\rb{K} $ such that $ 2Q = P $.
\end{lemma}

\begin{proof}
By assumption there exist $ t_1, t_2, t_3 \in K $ such that $ x - e_i = t_i^2 $ for $ i = 1, 2, 3 $. Then
$$ y^2 = \rb{x - e_1}\rb{x - e_2}\rb{x - e_3} = t_1^2t_2^2t_3^2 = \rb{t_1t_2t_3}^2. $$
Changing the signs of the $ t_i $ if necessary, assume that $ y = t_1t_2t_3 $. Set $ x_0 = x + t_1t_2 + t_2t_3 + t_3t_1 $ and $ y_0 = \rb{t_1 + t_2}\rb{t_2 + t_3}\rb{t_3 + t_1} $. Set $ Q = \rb{x_0, y_0} $. Claim that $ Q \in E\rb{K} $, and $ 2Q = P $.
$$ x_0 - e_1 = \rb{x - e_1} + t_1t_2 + t_2t_3 + t_3t_1 = t_1^2 + t_1t_2 + t_2t_3 + t_3t_1 = \rb{t_1 + t_2}\rb{t_1 + t_3}, $$
$$ x_0 - e_2 = \rb{t_2 + t_1}\rb{t_2 + t_3}, \qquad x_0 - e_3 = \rb{t_3 + t_1}\rb{t_3 + t_2}. $$
$ \rb{x_0 - e_1}\rb{x_0 - e_2}\rb{x_0 - e_3} = \rb{t_1 + t_2}^2\rb{t_2 + t_3}^2\rb{t_3 + t_1}^2 = y_0^2 $. So $ Q \in E\rb{K} $. Let $ y = mx + c $ be the tangent at $ Q $. $ y^2 = f\rb{x} = \rb{x - e_1}\rb{x - e_2}\rb{x - e_3} $ gives
\begin{align*}
m
& = \dfrac{f'\rb{x_0}}{2y_0} \\
& = \dfrac{\rb{x_0 - e_2}\rb{x_0 - e_3} + \rb{x_0 - e_1}\rb{x_0 - e_3} + \rb{x_0 - e_1}\rb{x_0 - e_2}}{2y_0} \\
& = \dfrac{\rb{t_1 + t_2}\rb{t_2 + t_3}\rb{t_3 + t_1}\rb{\rb{t_2 + t_3} + \rb{t_3 + t_1} + \rb{t_1 + t_2}}}{2\rb{t_1 + t_2}\rb{t_2 + t_3}\rb{t_3 + t_1}} \\
& = t_1 + t_2 + t_3.
\end{align*}
Want $ P = 2Q $, that is $ -P + 2Q = O $, that is $ \rb{x, -y} $ needs to lie on the tangent line at $ Q $, that is $ -y = mx + c $. We know that $ y_0 = mx_0 + c $. So we need $ y + y_0 = \rb{x_0 - x} $. But
$$ m\rb{x_0 - x} = \rb{t_1 + t_2 + t_3}\rb{t_1t_2 + t_2t_3 + t_3t_1} = \rb{t_1 + t_2}\rb{t_2 + t_3}\rb{t_3 + t_1} + t_1t_2t_3 = y_0 + y, $$
as required.
\end{proof}

\begin{proof}[Proof of Theorem \ref{thm:8.2}]
\hfill
\begin{enumerate}
\item $ \delta\rb{O} = \rb{1, 1, 1} $. Sufficient to prove that if $ P_1 + P_2 + P_3 = O $ then $ \delta\rb{P_1}\delta\rb{P_2}\delta\rb{P_3} = 1 $. By definition, $ \delta\rb{P} = \delta\rb{-P} $ since $ -\rb{x, y} = \rb{x, -y} $.
\begin{enumerate}
\item If all $ P_i = O $, trivial.
\item If $ P_1 = O $, then $ P_2 + P_3 = O $ so $ P_3 = -P_2 $. Then $ \delta\rb{P_3} = \delta\rb{-P_2} = \delta\rb{P_2} $. So
$$ \delta\rb{P_1}\delta\rb{P_2}\delta\rb{P_3} = \delta\rb{P_2}^2 = 1. $$
\item By symmetry $ P_2 = O $ or $ P_3 = O $ also trivial.
\end{enumerate}
So we may assume that no $ P_i $ is $ O $. So all the $ P_i $ lie on some line $ Y = mX + c $. Write $ P_i = \rb{x_i, y_i} $. Assume firstly that each $ y_i \ne 0 $. We need to prove that for each $ i $, $ \rb{x_1 - e_i}\rb{x_2 - e_i}\rb{x_3 - e_i} $ is a square. $ Y^2 = \rb{X - e_1}\rb{X - e_2}\rb{X - e_3} $ and $ Y = mX + c $, so
$$ \rb{X - e_1}\rb{X - e_2}\rb{X - e_3} - \rb{mX + c}^2 = \rb{X - x_1}\rb{X - x_2}\rb{X - x_3}. $$
So for each $ i $,
$$ -\rb{me_i + c}^2 = \rb{e_i - x_1}\rb{e_i - x_2}\rb{e_i - x_3}, $$
that is $ \rb{x_1 - e_i}\rb{x_2 - e_i}\rb{x_3 - e_i} = \rb{me_i + c}^2 $ is a square, as required. Suppose now that exactly one of $ P_1, P_2, P_3 $ has $ y_i = 0 $. Say $ P_1 = \rb{e_1, 0} $. The calculation we just did shows that the second and third coordinates of $ \delta\rb{P_1}, \delta\rb{P_2}, \delta\rb{P_3} $ are still squares. By the definition of $ \delta $, for any element of $ \rb{K^* / \rb{K^*}^2}^3 $ in $ Im\rb{\delta} $, the product of the three coordinates is always a square, so the first coordinate of $ \delta\rb{P_1}\delta\rb{P_2}\delta\rb{P_2} $ is also a square, as required. Now assume that at least two of the $ P_i $ have $ y_i = 0 $, that is satisfy $ 2P_i = O $. Since $ P_1 + P_2 + P_3 = O $, we have $ 2P_1 + 2P_2 + 2P_3 = O $. So in fact $ 2P_i = O $ for all $ i $. If for example also $ P_1 = P_2 $, then $ O = P_1 + P_2 + P_3 = 2P_1 + P_3 = P_3 $, but we are assuming that each $ P_i \ne O $. So without loss of generality $ P_1 = \rb{e_1, 0} $, $ P_2 = \rb{e_2, 0} $, $ P_3 = \rb{e_3, 0} $. Now follows from the definition. So $ \delta $ is a homomorphism.
\item If $ P \in 2E\rb{K} $ gives $ \delta\rb{P} = 1 $ trivially as $ \delta\rb{2Q} = \delta\rb{Q}^2 = 1 $. Conversely, $ \delta\rb{P} = 1 $ gives $ P \in 2E\rb{K} $ by Lemma \ref{lem:8.3}.
\end{enumerate}
\end{proof}

\lecture{27}{Tuesday}{04/12/18}

$$ \delta : \dfrac{E\rb{K}}{2E\rb{K}} \cong Im\rb{\delta} = \cb{\rb{a, b, c} \in \rb{\dfrac{K^*}{\rb{K^*}^2}}^3 \ \Bigg| \ abc = 1} \subseteq \rb{\dfrac{K^*}{\rb{K^*}^2}}^3. $$
Next is prove that if $ K = \Q $, $ Im\rb{\delta} $ is finite.

\begin{proposition}
\label{prop:8.5}
If $ K = \Q $ then $ Im\rb{\delta} $ is finite. More precisely, suppose that $ e_i \in \Z $ for $ i = 1, 2, 3 $. Then
$$ S = \cb{p \mid \rb{e_1 - e_2}\rb{e_2 - e_3}\rb{e_3 - e_1}} = \cb{p \mid \Delta}. $$
Define
$$ \Q\rb{S, 2} = \cb{a\rb{\Q^*}^2 \in \dfrac{\Q^*}{\rb{\Q^*}^2} \ \Bigg| \ a \in \Z \ \text{is squarefree, only divisible by primes in} \ S}. $$
Then $ Im\rb{\delta} \subset \Q\rb{S, 2}^3 $.
\end{proposition}

\begin{example*}
$ Y^2 = X^3 - X = X\rb{X + 1}\rb{X - 1} $ so $ S = \cb{2} $.
\end{example*}

\begin{proof}
Let $ p $ be any prime, and let $ P = \rb{x, y} \in E\rb{\Q} $. $ \rb{x - e_1}\rb{x - e_2}\rb{x - e_3} = y^2 $. So
$$ \abs{x - e_1}_p\abs{x - e_2}_p\abs{x - e_3}_p = \abs{y}_p^2 = p^{2\Z}. $$
So either
\begin{enumerate}
\item $ \abs{x - e_i}_p \in p^{2\Z} $ for $ i = 1, 2, 3 $, or
\item Precisely one of the $ \abs{x - e_i}_p $ is in $ p^{2\Z} $.
\end{enumerate}
If $ 1 $ holds, then each $ x - e_i $ is of the form $ p^{2k} \in \rb{\Q^*}^2 $ times a $ p $-adic unit. We need to show that if $ 2 $ holds, then $ p \in S $. If we had $ \abs{x}_p > 1 $, then since $ e_i \in \Z $, $ \abs{e_i}_p \le 1 $, and $ \abs{x}_p = \abs{x - e_i}_p $ for $ i = 1, 2, 3 $, so we cannot be in case $ 2 $. If $ \abs{x}_p \le 1 $, then $ \abs{x - e_i}_p \le 1 $, and at least two of these have $ \abs{x - e_i}_p < 1 $. If $ \abs{x - e_i}_p < 1 $ and $ \abs{x - e_j}_p < 1 $, then $ \abs{e_i - e_j}_p \le \max\rb{\abs{x - e_i}_p, \abs{x - e_j}_p} < 1 $, that is $ p \mid \rb{e_i - e_j} $, so $ p \in S $.
\end{proof}

\begin{theorem}[Weak Mordell-Weil theorem]
Let $ E $ be an elliptic curve over $ \Q $ all of whose $ 2 $-torsion points are defined over $ \Q $. Then $ E\rb{\Q} / 2E\rb{\Q} $ is finite.
\end{theorem}

\begin{proof}
Write $ E $ as $ y^2 = f\rb{x} $. The condition on the $ 2 $-torsion points says that the roots of $ f\rb{x} $ are rational. After rescaling $ x, y $, we can assume that
$$ y^2 = \rb{x - e_1}\rb{x - e_2}\rb{x - e_3}, \qquad e_i \in \Z. $$
Then by Theorem \ref{thm:8.2} and Proposition \ref{prop:8.5}, we have an injection
$$ \delta : \dfrac{E\rb{\Q}}{2E\rb{\Q}} \hookrightarrow \Q\rb{S, 2} \times \Q\rb{S, 2} \times \Q\rb{S, 2}, $$
which is finite.
\end{proof}

Question is how big can $ Im\rb{\delta} $ be in terms of $ S $? Let $ s = \#S $. Then $ \#\Q\rb{S, 2} = 2^{s + 1} $. If $ S = 2 $, then $ \Q\rb{S, 2} = \cb{\pm 1, \pm 2} $. So $ \#E\rb{\Q} / 2E\rb{\Q} \le \rb{\#\Q\rb{S, 2}}^2 = 8^{s + 1} $. Since $ Im\rb{\delta} \subset \cb{\rb{a, b, c} \mid abc = 1} $, actually $ \rb{\#\Q\rb{S, 2}}^2 = 4^{s + 1} $. In fact either two of $ x - e_i $ are less than zero and one is greater than zero or all three are greater than zero, so $ \#E\rb{\Q} / 2E\rb{\Q} \le 2^{2s + 1} $. Assume that $ E\rb{\Q} $ is a finitely generated abelian group, so $ E\rb{\Q} \cong \Z^r \times T $, where the torsion subgroup $ T $ is finite. Then
$$ \dfrac{E\rb{\Q}}{2E\rb{\Q}} \cong \dfrac{\Z^r}{2\Z^r} \times \dfrac{T}{2T} \cong \rb{\dfrac{\Z}{2\Z}}^r \times \rb{\dfrac{T}{2T}}. $$
So $ \#\rb{E\rb{\Q} / 2E\rb{\Q}} = 2^r \times \#\rb{T / 2T} $. Fact is if $ T $ is any finite abelian group, then $ \#\rb{T / 2T} = \#T\sb{2} = \cb{x \in T \mid 2x = 0} $. In our situation, $ T $ is the torsion subgroup of $ E\rb{\Q} $. So
$$ T\sb{2} = E\rb{\Q}\sb{2} = \cb{P \in E\rb{\Q} \mid 2P = O}. $$
Since we assumed that $ E $ is $ y^2 = \rb{x - e_1}\rb{x - e_2}\rb{x - e_3} $, for $ e_i \in \Z $, $ E\sb{2} = \cb{O, \rb{e_1, 0}, \rb{e_2, 0}, \rb{e_3, 0}} $. So $ \#\rb{T / 2T} = 4 $, and $ \#Im\rb{\delta} = \#E\rb{\Q} / 2E\rb{\Q} = 2^{r + 2} $.

\begin{example*}
Let $ Y^2 = X^3 - X $. We saw that $ Im\rb{\delta} $ had at least four elements, but strictly less than eight elements. So $ Im\rb{\delta} $ has exactly four elements, so $ r = 0 $, that is $ E\rb{\Q} $ is finite.
\end{example*}

Consider the group homomorphism $ T \mapsto T $ by $ x \mapsto 2x $. The image is $ 2T $. The kernel is $ T\sb{2} $. So by the first isomorphism theorem,
$$ \dfrac{T}{T\sb{2}} \cong 2T \qquad \implies \qquad \dfrac{\#T}{\#T\sb{2}} = \#\rb{\dfrac{T}{T\sb{2}}} = \#2T \qquad \implies \qquad \#\rb{\dfrac{T}{2T}} = \dfrac{\#T}{\#2T} = \#T\sb{2}. $$
That is, the following sequence is exact.
$$
\begin{tikzcd}
0 \arrow{r} & T\sb{2} \arrow{r} & T \arrow{r}{\times 2} & T \arrow{r} & \dfrac{T}{2T} \arrow{r} & 0
\end{tikzcd}.
$$

\lecture{28}{Thursday}{06/12/18}

Write $ E $ as $ y^2 = x^3 + Ax + B $ for $ A, B \in \Z $. Remember that we showed that $ x^3 + y^3 = 9 $ has infinitely many solutions in $ \Q $, by drawing tangent lines, that is multiplication by two, and showing that the height of the coordinates increased.

\begin{definition}
If $ l / m \in \Q $ with $ \rb{l, m} = 1 $, the the height $ H\rb{l / m} = \max\rb{\abs{l}, \abs{m}} $. If $ P \in E\rb{\Q} $, write $ P = \rb{x_P, y_P} $, and set $ H\rb{P} = H\rb{x_P} $. $ h\rb{P} = \log\rb{H\rb{P}} $.
\end{definition}

\begin{proposition}
Let $ A $ be an abelian group. Let $ h : A \to \R $ such that
\begin{enumerate}
\item for all $ Q \in A $, there exists $ C\rb{\Q} \in \R $ such that for all $ P \in A $, $ h\rb{P + Q} \le 2h\rb{P} + C\rb{Q} $,
\item there exists $ D \in R $, such that for all $ P \in A $, $ h\rb{2P} > 4h\rb{P} - D $,
\item for all $ K \in \R $, $ \cb{P \in A \mid h\rb{P} \le K} $ is finite, and
\item $ A / 2A $ is finite.
\end{enumerate}
Then $ A $ is finitely generated.
\end{proposition}

We will apply this to $ A = E\rb{\Q} $. We will check $ 1 $ to $ 3 $, and $ 4 $ is weak Mordell-Weil.

\begin{proof}
Let $ Q_1, \dots, Q_r \in A $ be any coset representatives for $ 2A $, that is $ A / 2A = \cb{\overline{Q}_1, \dots, \overline{Q}_r} $, by $ 4 $. Set $ C_1 = \max\rb{C\rb{Q_i}} $, $ C\rb{Q_i} $ as in $ 1 $. Let $ C_2 = \rb{C_1 + D} / 2 $, $ D $ as in $ 2 $. Let $ S = \cb{a \in A \mid h\rb{a} \le C_2} \cup \cb{Q_1, \dots, Q_r} $. By $ 3 $, $ S $ is finite. Let $ B $ be the subgroup of $ A $ generated by $ S $. Take any $ P \in A $. Define a sequence $ P = P_0, P_1, P_2, \dots $ as follows. Write $ P_j = 2P_{j + 1} - Q_{i_j} $, using the definition of the $ Q_i $.
$$ h\rb{P_j} \le \dfrac{h\rb{2P_j} + D}{4} = \dfrac{h\rb{P_{j - 1} + Q_{i_{j - 1}}} + D}{4} \le \dfrac{2h\rb{P_{j - 1}} + C_1 + D}{4} = \dfrac{h\rb{P_{j - 1}} + C_2}{2}. $$
In particular if $ h\rb{P_{j - 1}} > C_2 $, then $ h\rb{P_{j - 1}} > h\rb{P_j} $. Consequently, if $ h\rb{P_j} > C_2 $ for all $ j $, then $ h\rb{P_0} > h\rb{P_1} > \dots $. This contradicts $ 3 $. So there exists $ j $ such that $ h\rb{P_j} \le C_2 $. So $ P_j \in S $, and in particular, $ P_j \in B $. If for some $ j $, $ P_j \in B $, then $ P_{j - 1} = 2P_j - Q_{i_{j - 1}} \in B $. So
$$ P_j \in B \qquad \implies \qquad \dots \qquad \implies \qquad P_0 \in B, $$
that is $ P \in B $, that is $ A = B $ is finitely generated.
\end{proof}

Now $ A = E\rb{\Q} $. Start with axiom $ 3 $. For all $ K \in \R $ there exist finitely many $ P \in E\rb{\Q} $ such that $ h\rb{P} \le K $. Equivalently, $ H\rb{x_P} = H\rb{P} \le e^K $ for $ P = \rb{x_P, y_P} $. Any $ x_P $ corresponds to at most two points $ P $, so it suffices to prove there exist finitely many $ l / m $ such that $ \max\rb{\abs{l}, \abs{m}} \le e^K $. This is obvious.

\begin{lemma}
\hfill
\begin{enumerate}
\item For any $ Q \in E\rb{\Q} $ there exists $ C = C\rb{\Q} $ depending only on $ E, Q $ such that for all $ P \in E\rb{\Q} $, $ h\rb{P + Q} \le 2h\rb{P} + C $.
\end{enumerate}
\end{lemma}

\begin{proof}
It suffices to prove this for all but finitely many $ P $, so in particular without loss of generality $ P = \pm Q $, $ P \ne O $. The problem is trivial if $ Q = O $, just take $ C = O $, so assume $ Q \ne O $. Write $ P = \rb{x_P, y_P} $, $ Q = \rb{x_Q, y_Q} $. We can write $ \rb{x_P, y_P} = \rb{a / d^2, b / d^3} $, for some $ a, b, d \in \Z $, $ \rb{a, d} = \rb{b, d} = 1 $. This is because for all primes $ p $, either $ \abs{x}_p, \abs{y}_p \le 1 $ or $ \abs{x}_p = p^{2n} $, $ \abs{y}_p = p^{3n} $ for some $ p $. Similarly $ \rb{x_Q, y_Q} = \rb{e / g^2, f / g^3} $, $ \rb{e, g} = \rb{f, g} = 1 $. In particular, $ H\rb{P} = \max\rb{\abs{a}, \abs{d}^2} $. The $ x $-coordinate of $ P + Q $ is, by the addition formula, $ N / D $ where $ N = \rb{Bd^4 + Aad^2}g^4 + \rb{Aed^4 + ea^2}g^2 - 2bfdg + e^2ad^2 $ and $ D = a^2g^4 - 2ead^2g^2 + e^2d^4 $. So $ H\rb{P + Q} \le \max\rb{\abs{N}, \abs{D}} $. $ Q $ is fixed, so $ e, f, g $ are fixed, as are $ \abs{A}, \abs{B} $. $ H\rb{P}^2 = \max\rb{a^2, d^4} $. So just need to show that every term in $ N, D $ is bounded by a constant times $ \max\rb{a^2, d^4} $. Only difficulty is $ 2bfdg $ term in $ N $. But $ b^2 = a^3 + Aad^4 + Bd^6 $, so $ \rb{bd}^2 = a^3d^2 + Aad^6 + Bd^8 $, which is at most a constant times $ \max\rb{a, d^2}^4 $.
\end{proof}

\lecture{29}{Friday}{07/12/18}

Let $ y^2 = \rb{x - 5}\rb{x - 6}\rb{x + 11} $ be $ E $. $ S $ is the primes dividing $ \rb{5 - 6}\rb{6 - \rb{-11}}\rb{5 - \rb{-11}} $, that is $ S = \cb{2, 17} $. What is the torsion subgroup of $ E\rb{\Q} $? Reduce mod three, $ y^2 = x\rb{x + 1}\rb{x - 1} = x^3 - x $. Points mod three are $ O $, $ \rb{0, 0} $, $ \rb{1, 0} $, $ \rb{-1, 0} $. So the group of torsion points of $ E\rb{\Q} $ injects into $ \rb{\Z / 2\Z}^2 \cong E\rb{\F_3} $. We have $ O $, $ \rb{5, 0} $, $ \rb{6, 0} $, $ \rb{-11, 0} $ as torsion points over $ \Q $. So the torsion subgroup is exactly these four points, so it is isomorphic to $ \rb{\Z / 2\Z}^2 $. To compute the rank, we use the homomorphism $ \delta : E\rb{\Q} / 2E\rb{\Q} \hookrightarrow \rb{\Q^* / \rb{\Q^*}^2}^3 $. We know that $ Im\rb{\delta} = \cb{\rb{\delta_1, \delta_2, \delta_3} \mid \delta_1\delta_2\delta_3 = 1} $. The image is contained in the subgroup whose entries are $ \cb{\pm 1, \pm 2, \pm 17, \pm 34} $. $ \delta : \rb{x, y} \mapsto \rb{x - 5, x - 6, x + 11} $. Since $ \delta_3 = \rb{\delta_1\delta_2}^{-1} = \delta_1\delta_2 $, we will ignore the third coordinate, and write
\begin{align*}
\delta : \rb{x, y} & \mapsto \rb{x - 5, x - 6} \\
O & \mapsto \rb{1, 1} \\
\rb{5, 0} & \mapsto \rb{-1, -1} \\
\rb{6, 0} & \mapsto \rb{1, 17} \\
\rb{-11, 0} & \mapsto \rb{-1, -17}.
\end{align*}
$ x - 5 = b_1u^2 $, $ x - 6 = b_2v^2 $, $ x + 11 = b_3w^2 $, for $ b_1, b_2, b_3 \in \cb{\pm 1, \pm 2, \pm 17, \pm 34} $. Solve $ b_1u^2 - b_2v^2 = 1 $ and $ \rb{b_1b_2}w^2 - b_1u^2 = 16 $ for $ u, v, w \in \Q $ not all zero.
\begin{enumerate}
\item If $ \rb{b_1, b_2} = \rb{2, 1} $, $ 2u^2 - v^2 = 1 $ and $ 2w^2 - 2u^2 = 16 $, that is $ w^2 - u^2 = 8 $. $ u = v = 1 $ and $ w = 3 $ is a solution. $ x = 7 $ and $ y^2 = \rb{x - 5}\rb{x - 6}\rb{x + 11} = 36 $, so $ \rb{7, \pm 6} $.
\item If $ \rb{b_1, b_2} = \rb{1, 2} $, $ u^2 - 2v^2 = 1 $ and $ 2w^2 - u^2 = 16 $. $ u = U / Z $, $ v = V / Z $, $ w = W / Z $. $ U^2 - 2V^2 = Z^2 $ and $ 2W^2 - U^2 = 16Z^2 $. $ 2 \mid U $, so $ 2 \mid W $. $ 2 \mid Z $, so $ 2 \mid V $. So $ 2 \mid U, V, W, Z $ gives $ U = V = W = Z = 0 $, a contradiction.
\item The argument for $ \rb{1, 2} $ only used $ 2 \nmid b_1 $, $ 2 \mid b_2 $. $ b_1U^2 - b_2V^2 = Z^2 $ and $ \rb{b_1b_2}W^2 - b_1U^2 = 16Z^2 $.
\item If $ \rb{b_1, b_2} = \rb{17, 1} $, $ 17u^2 - v^2 = 1 $ and $ 17w^2 - 17u^2 = 16 $. $ 17U^2 - V^2 = Z^2 $ and $ 17W^2 - 17U^2 = 16Z^2 $. $ 17 \mid Z $, $ 17 \mid V $ gives $ 17 \mid U $, and $ 17 \mid W $, a contradiction.
\end{enumerate}
Thus
$$
\begin{array}{c|cccc|cccc}
& 1 & 2 & 17 & 34 & -1 & -2 & -17 & -34 \\
\hline
1 & \checkmark & \times & \checkmark & \times & \times & \times & \times & \times \\
2 & \checkmark & \times & \checkmark & \times & \times & \times & \times & \times \\
17 & \times & \times & \times & \times & \times & \times & \times & \times \\
34 & \times & \times & \times & \times & \times & \times & \times & \times \\
\hline
-1 & \times & \times & \times & \times & \checkmark & \times & \checkmark & \times \\
-2 & \times & \times & \times & \times & \checkmark & \times & \checkmark & \times \\
-17 & \times & \times & \times & \times & \times & \times & \times & \times \\
-34 & \times & \times & \times & \times & \times & \times & \times & \times
\end{array}.
$$
$ \#Im\rb{\delta} = 8 $. $ E\rb{\Q} \cong \Z^r \times \rb{\Z / 2\Z}^2 $. $ E\rb{\Q} / 2E\rb{\Q} \cong Im\rb{\delta} \cong \rb{\Z / 2\Z}^3 $ gives $ r = 1 $. So $ E\rb{\Q} \cong \Z \times \rb{\Z / 2\Z}^2 $.

\lecture{30}{Tuesday}{11/12/18}

Lecture 30 is a problem class.

\end{document}