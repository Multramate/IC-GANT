\def\module{M4P57 Complex Manifolds}
\def\lecturer{Prof Paolo Cascini}
\def\term{Spring 2020}
\def\cover{}
\def\syllabus{}
\def\thm{section}

\documentclass{article}

% Packages

\usepackage{amssymb}
\usepackage{amsthm}
\usepackage[UKenglish]{babel}
\usepackage{commath}
\usepackage{enumitem}
\usepackage{etoolbox}
\usepackage{fancyhdr}
\usepackage[margin=1in]{geometry}
\usepackage{graphicx}
\usepackage[hidelinks]{hyperref}
\usepackage[utf8]{inputenc}
\usepackage{listings}
\usepackage{mathtools}
\usepackage{stmaryrd}
\usepackage{tikz-cd}
\usepackage{csquotes}

% Formatting

\addto\captionsUKenglish{\renewcommand{\abstractname}{Syllabus}}
\delimitershortfall5pt
\ifx\thm\undefined\newtheorem{n}{}\else\newtheorem{n}{}[\thm]\fi
\newcommand\newoperator[1]{\ifcsdef{#1}{\cslet{#1}{\relax}}{}\csdef{#1}{\operatorname{#1}}}
\setlength{\parindent}{0cm}

% Environments

\theoremstyle{plain}
\newtheorem{algorithm}[n]{Algorithm}
\newtheorem*{algorithm*}{Algorithm}
\newtheorem{algorithm**}{Algorithm}
\newtheorem{conjecture}[n]{Conjecture}
\newtheorem*{conjecture*}{Conjecture}
\newtheorem{conjecture**}{Conjecture}
\newtheorem{corollary}[n]{Corollary}
\newtheorem*{corollary*}{Corollary}
\newtheorem{corollary**}{Corollary}
\newtheorem{lemma}[n]{Lemma}
\newtheorem*{lemma*}{Lemma}
\newtheorem{lemma**}{Lemma}
\newtheorem{proposition}[n]{Proposition}
\newtheorem*{proposition*}{Proposition}
\newtheorem{proposition**}{Proposition}
\newtheorem{theorem}[n]{Theorem}
\newtheorem*{theorem*}{Theorem}
\newtheorem{theorem**}{Theorem}

\theoremstyle{definition}
\newtheorem{aim}[n]{Aim}
\newtheorem*{aim*}{Aim}
\newtheorem{aim**}{Aim}
\newtheorem{axiom}[n]{Axiom}
\newtheorem*{axiom*}{Axiom}
\newtheorem{axiom**}{Axiom}
\newtheorem{condition}[n]{Condition}
\newtheorem*{condition*}{Condition}
\newtheorem{condition**}{Condition}
\newtheorem{definition}[n]{Definition}
\newtheorem*{definition*}{Definition}
\newtheorem{definition**}{Definition}
\newtheorem{example}[n]{Example}
\newtheorem*{example*}{Example}
\newtheorem{example**}{Example}
\newtheorem{exercise}[n]{Exercise}
\newtheorem*{exercise*}{Exercise}
\newtheorem{exercise**}{Exercise}
\newtheorem{fact}[n]{Fact}
\newtheorem*{fact*}{Fact}
\newtheorem{fact**}{Fact}
\newtheorem{goal}[n]{Goal}
\newtheorem*{goal*}{Goal}
\newtheorem{goal**}{Goal}
\newtheorem{law}[n]{Law}
\newtheorem*{law*}{Law}
\newtheorem{law**}{Law}
\newtheorem{plan}[n]{Plan}
\newtheorem*{plan*}{Plan}
\newtheorem{plan**}{Plan}
\newtheorem{problem}[n]{Problem}
\newtheorem*{problem*}{Problem}
\newtheorem{problem**}{Problem}
\newtheorem{question}[n]{Question}
\newtheorem*{question*}{Question}
\newtheorem{question**}{Question}
\newtheorem{warning}[n]{Warning}
\newtheorem*{warning*}{Warning}
\newtheorem{warning**}{Warning}
\newtheorem{acknowledgements}[n]{Acknowledgements}
\newtheorem*{acknowledgements*}{Acknowledgements}
\newtheorem{acknowledgements**}{Acknowledgements}
\newtheorem{annotations}[n]{Annotations}
\newtheorem*{annotations*}{Annotations}
\newtheorem{annotations**}{Annotations}
\newtheorem{assumption}[n]{Assumption}
\newtheorem*{assumption*}{Assumption}
\newtheorem{assumption**}{Assumption}
\newtheorem{conclusion}[n]{Conclusion}
\newtheorem*{conclusion*}{Conclusion}
\newtheorem{conclusion**}{Conclusion}
\newtheorem{claim}[n]{Claim}
\newtheorem*{claim*}{Claim}
\newtheorem{claim**}{Claim}
\newtheorem{notation}[n]{Notation}
\newtheorem*{notation*}{Notation}
\newtheorem{notation**}{Notation}
\newtheorem{note}[n]{Note}
\newtheorem*{note*}{Note}
\newtheorem{note**}{Note}
\newtheorem{remark}[n]{Remark}
\newtheorem*{remark*}{Remark}
\newtheorem{remark**}{Remark}

% Lectures

\newcommand{\lecture}[3]{ % Lecture
  \marginpar{
    Lecture #1 \\
    #2 \\
    #3
  }
}

% Blackboard

\renewcommand{\AA}{\mathbb{A}} % Blackboard A
\newcommand{\BB}{\mathbb{B}}   % Blackboard B
\newcommand{\CC}{\mathbb{C}}   % Blackboard C
\newcommand{\DD}{\mathbb{D}}   % Blackboard D
\newcommand{\EE}{\mathbb{E}}   % Blackboard E
\newcommand{\FF}{\mathbb{F}}   % Blackboard F
\newcommand{\GG}{\mathbb{G}}   % Blackboard G
\newcommand{\HH}{\mathbb{H}}   % Blackboard H
\newcommand{\II}{\mathbb{I}}   % Blackboard I
\newcommand{\JJ}{\mathbb{J}}   % Blackboard J
\newcommand{\KK}{\mathbb{K}}   % Blackboard K
\newcommand{\LL}{\mathbb{L}}   % Blackboard L
\newcommand{\MM}{\mathbb{M}}   % Blackboard M
\newcommand{\NN}{\mathbb{N}}   % Blackboard N
\newcommand{\OO}{\mathbb{O}}   % Blackboard O
\newcommand{\PP}{\mathbb{P}}   % Blackboard P
\newcommand{\QQ}{\mathbb{Q}}   % Blackboard Q
\newcommand{\RR}{\mathbb{R}}   % Blackboard R
\renewcommand{\SS}{\mathbb{S}} % Blackboard S
\newcommand{\TT}{\mathbb{T}}   % Blackboard T
\newcommand{\UU}{\mathbb{U}}   % Blackboard U
\newcommand{\VV}{\mathbb{V}}   % Blackboard V
\newcommand{\WW}{\mathbb{W}}   % Blackboard W
\newcommand{\XX}{\mathbb{X}}   % Blackboard X
\newcommand{\YY}{\mathbb{Y}}   % Blackboard Y
\newcommand{\ZZ}{\mathbb{Z}}   % Blackboard Z

% Brackets

\renewcommand{\eval}[1]{\left. #1 \right|}          % Evaluation
\newcommand{\br}{\del}                              % Brackets
\newcommand{\abr}[1]{\left\langle #1 \right\rangle} % Angle brackets
\newcommand{\fbr}[1]{\left\lfloor #1 \right\rfloor} % Floor brackets
\newcommand{\lbr}[1]{\left\lfloor #1 \right\rfloor} % Ceiling brackets
\newcommand{\st}{\ \middle| \ }                     % Such that

% Calligraphic

\newcommand{\AAA}{\mathcal{A}} % Calligraphic A
\newcommand{\BBB}{\mathcal{B}} % Calligraphic B
\newcommand{\CCC}{\mathcal{C}} % Calligraphic C
\newcommand{\DDD}{\mathcal{D}} % Calligraphic D
\newcommand{\EEE}{\mathcal{E}} % Calligraphic E
\newcommand{\FFF}{\mathcal{F}} % Calligraphic F
\newcommand{\GGG}{\mathcal{G}} % Calligraphic G
\newcommand{\HHH}{\mathcal{H}} % Calligraphic H
\newcommand{\III}{\mathcal{I}} % Calligraphic I
\newcommand{\JJJ}{\mathcal{J}} % Calligraphic J
\newcommand{\KKK}{\mathcal{K}} % Calligraphic K
\newcommand{\LLL}{\mathcal{L}} % Calligraphic L
\newcommand{\MMM}{\mathcal{M}} % Calligraphic M
\newcommand{\NNN}{\mathcal{N}} % Calligraphic N
\newcommand{\OOO}{\mathcal{O}} % Calligraphic O
\newcommand{\PPP}{\mathcal{P}} % Calligraphic P
\newcommand{\QQQ}{\mathcal{Q}} % Calligraphic Q
\newcommand{\RRR}{\mathcal{R}} % Calligraphic R
\newcommand{\SSS}{\mathcal{S}} % Calligraphic S
\newcommand{\TTT}{\mathcal{T}} % Calligraphic T
\newcommand{\UUU}{\mathcal{U}} % Calligraphic U
\newcommand{\VVV}{\mathcal{V}} % Calligraphic V
\newcommand{\WWW}{\mathcal{W}} % Calligraphic W
\newcommand{\XXX}{\mathcal{X}} % Calligraphic X
\newcommand{\YYY}{\mathcal{Y}} % Calligraphic Y
\newcommand{\ZZZ}{\mathcal{Z}} % Calligraphic Z

% Fraktur

\newcommand{\aaa}{\mathfrak{a}}   % Fraktur a
\newcommand{\bbb}{\mathfrak{b}}   % Fraktur b
\newcommand{\ccc}{\mathfrak{c}}   % Fraktur c
\newcommand{\ddd}{\mathfrak{d}}   % Fraktur d
\newcommand{\eee}{\mathfrak{e}}   % Fraktur e
\newcommand{\fff}{\mathfrak{f}}   % Fraktur f
\renewcommand{\ggg}{\mathfrak{g}} % Fraktur g
\newcommand{\hhh}{\mathfrak{h}}   % Fraktur h
\newcommand{\iii}{\mathfrak{i}}   % Fraktur i
\newcommand{\jjj}{\mathfrak{j}}   % Fraktur j
\newcommand{\kkk}{\mathfrak{k}}   % Fraktur k
\renewcommand{\lll}{\mathfrak{l}} % Fraktur l
\newcommand{\mmm}{\mathfrak{m}}   % Fraktur m
\newcommand{\nnn}{\mathfrak{n}}   % Fraktur n
\newcommand{\ooo}{\mathfrak{o}}   % Fraktur o
\newcommand{\ppp}{\mathfrak{p}}   % Fraktur p
\newcommand{\qqq}{\mathfrak{q}}   % Fraktur q
\newcommand{\rrr}{\mathfrak{r}}   % Fraktur r
\newcommand{\sss}{\mathfrak{s}}   % Fraktur s
\newcommand{\ttt}{\mathfrak{t}}   % Fraktur t
\newcommand{\uuu}{\mathfrak{u}}   % Fraktur u
\newcommand{\vvv}{\mathfrak{v}}   % Fraktur v
\newcommand{\www}{\mathfrak{w}}   % Fraktur w
\newcommand{\xxx}{\mathfrak{x}}   % Fraktur x
\newcommand{\yyy}{\mathfrak{y}}   % Fraktur y
\newcommand{\zzz}{\mathfrak{z}}   % Fraktur z

% Geometry

\newcommand{\CP}{\mathbb{CP}}                                              % Complex projective space
\newcommand{\iintd}[4]{\iint_{#1} \, #2 \, \dif #3 \, \dif #4}             % Double integral
\newcommand{\RP}{\mathbb{RP}}                                              % Real projective space
\newcommand{\intd}[4]{\int_{#1}^{#2} \, #3 \, \dif #4}                     % Single integral
\newcommand{\iiintd}[5]{\iint_{#1} \, #2 \, \dif #3 \, \dif #4 \, \dif #5} % Triple integral

% Logic

\newcommand{\iffb}[2]{\br{#1 \leftrightarrow #2}} % Biconditional
\newcommand{\andb}[2]{\br{#1 \land #2}}           % Conjunction
\newcommand{\orb}[2]{\br{#1 \lor #2}}             % Disjunction
\newcommand{\nib}[2]{\br{#1 \notin #2}}           % Element of
\newcommand{\eqb}[2]{\br{#1 = #2}}                % Equal to
\newcommand{\teb}[1]{\br{\exists #1}}             % Existential quantifier
\newcommand{\impb}[2]{\br{#1 \rightarrow #2}}     % Implication
\newcommand{\ltb}[2]{\br{#1 < #2}}                % Less than
\newcommand{\leb}[2]{\br{#1 \le #2}}              % Less than or equal to
\newcommand{\notb}[1]{\br{\neg #1}}               % Negation
\newcommand{\inb}[2]{\br{#1 \in #2}}              % Not element of
\newcommand{\neb}[2]{\br{#1 \ne #2}}              % Not equal to
\newcommand{\subb}[2]{\br{#1 \subseteq #2}}       % Subset
\newcommand{\fab}[1]{\br{\forall #1}}             % Universal quantifier

% Maps

\newcommand{\bijection}[7][]{    % Bijection
  \ifx &#1&
    \begin{array}{rcl}
      #2 & \longleftrightarrow & #3 \\
      #4 & \longmapsto         & #5 \\
      #6 & \longmapsfrom       & #7
    \end{array}
  \else
    \begin{array}{ccrcl}
      #1 & : & #2 & \longrightarrow & #3 \\
         &   & #4 & \longmapsto     & #5 \\
         &   & #6 & \longmapsfrom   & #7
    \end{array}
  \fi
}
\newcommand{\correspondence}[2]{ % Correspondence
  \cbr{
    \begin{array}{c}
      #1
    \end{array}
  }
  \qquad
  \leftrightsquigarrow
  \qquad
  \cbr{
    \begin{array}{c}
      #2
    \end{array}
  }
}
\newcommand{\function}[5][]{     % Function
  \ifx &#1&
    \begin{array}{rcl}
      #2 & \longrightarrow & #3 \\
      #4 & \longmapsto     & #5
    \end{array}
  \else
    \begin{array}{ccrcl}
      #1 & : & #2 & \longrightarrow & #3 \\
         &   & #4 & \longmapsto     & #5
    \end{array}
  \fi
}
\newcommand{\functions}[7][]{    % Functions
  \ifx &#1&
    \begin{array}{rcl}
      #2 & \longrightarrow & #3 \\
      #4 & \longmapsto     & #5 \\
      #6 & \longmapsto     & #7
    \end{array}
  \else
    \begin{array}{ccrcl}
      #1 & : & #2 & \longrightarrow & #3 \\
         &   & #4 & \longmapsto     & #5 \\
         &   & #6 & \longmapsto     & #7
    \end{array}
  \fi
}

% Matrices

\newcommand{\onebytwo}[2]{      % One by two matrix
  \begin{pmatrix}
    #1 & #2
  \end{pmatrix}
}
\newcommand{\onebythree}[3]{    % One by three matrix
  \begin{pmatrix}
    #1 & #2 & #3
  \end{pmatrix}
}
\newcommand{\twobyone}[2]{      % Two by one matrix
  \begin{pmatrix}
    #1 \\
    #2
  \end{pmatrix}
}
\newcommand{\twobytwo}[4]{      % Two by two matrix
  \begin{pmatrix}
    #1 & #2 \\
    #3 & #4
  \end{pmatrix}
}
\newcommand{\threebyone}[3]{    % Three by one matrix
  \begin{pmatrix}
    #1 \\
    #2 \\
    #3
  \end{pmatrix}
}
\newcommand{\threebythree}[9]{  % Three by three matrix
  \begin{pmatrix}
    #1 & #2 & #3 \\
    #4 & #5 & #6 \\
    #7 & #8 & #9
  \end{pmatrix}
}
\newcommand{\twobytwosmall}[4]{ % Two by two small matrix
  \begin{psmallmatrix}
    #1 & #2 \\
    #3 & #4
  \end{psmallmatrix}
}

% Number theory

\renewcommand{\symbol}[2]{\br{\tfrac{#1}{#2}}} % Power residue symbol
\newcommand{\unit}[1]{\br{\ZZ / #1\ZZ}^\times} % Unit group

% Operators

\newoperator{ab}    % Abelian
\newoperator{AG}    % Affine geometry
\newoperator{alg}   % Algebraic
\newoperator{Ann}   % Annihilator
\newoperator{area}  % Area
\newoperator{Aut}   % Automorphism
\newoperator{card}  % Cardinality
\newoperator{ch}    % Characteristic
\newoperator{Cl}    % Class
\newoperator{col}   % Column
\newoperator{Corr}  % Correspondence
\newoperator{diam}  % Diameter
\newoperator{Disc}  % Discriminant
\newoperator{dom}   % Domain
\newoperator{Em}    % Embedding
\newoperator{End}   % Endomorphism
\newoperator{fin}   % Finite
\newoperator{Fix}   % Fixed
\newoperator{Frac}  % Fraction
\newoperator{Frob}  % Frobenius
\newoperator{Fun}   % Function
\newoperator{Gal}   % Galois
\newoperator{GL}    % General linear
\newoperator{Ham}   % Hamming
\newoperator{Homeo} % Homeomorphism
\newoperator{Hom}   % Homomorphism
\newoperator{id}    % Identity
\newoperator{Im}    % Image
\newoperator{Ind}   % Index
\newoperator{Ker}   % Kernel
\newoperator{lcm}   % Least common multiple
\newoperator{Mat}   % Matrix
\newoperator{mult}  % Multiplicity
\newoperator{new}   % New
\newoperator{Nm}    % Norm
\newoperator{old}   % Old
\newoperator{ord}   % Order
\newoperator{Pay}   % Payley
\newoperator{PG}    % Projective geometry
\newoperator{PGL}   % Projective general linear
\newoperator{PSL}   % Projective special linear
\newoperator{rad}   % Radical
\newoperator{ran}   % Range
\newoperator{Res}   % Residue
\newoperator{rk}    % Rank
\newoperator{Re}    % Real
\newoperator{row}   % Row
\newoperator{sgn}   % Sign
\newoperator{Sing}  % Singular
\newoperator{sp}    % Span
\newoperator{SL}    % Special linear
\newoperator{SO}    % Special orthogonal
\newoperator{Spec}  % Spectrum
\newoperator{Stab}  % Stabiliser
\newoperator{star}  % Star
\newoperator{srg}   % Strongly regular graph
\newoperator{Sym}   % Symmetric
\newoperator{tors}  % Torsion
\newoperator{Tr}    % Trace
\newoperator{vol}   % Volume
\newoperator{wt}    % Weight

% Roman

\newcommand{\A}{\mathrm{A}}   % Roman A
\newcommand{\B}{\mathrm{B}}   % Roman B
\newcommand{\C}{\mathrm{C}}   % Roman C
\newcommand{\D}{\mathrm{D}}   % Roman D
\newcommand{\E}{\mathrm{E}}   % Roman E
\newcommand{\F}{\mathrm{F}}   % Roman F
\newcommand{\G}{\mathrm{G}}   % Roman G
\renewcommand{\H}{\mathrm{H}} % Roman H
\newcommand{\I}{\mathrm{I}}   % Roman I
\newcommand{\J}{\mathrm{J}}   % Roman J
\newcommand{\K}{\mathrm{K}}   % Roman K
\renewcommand{\L}{\mathrm{L}} % Roman L
\newcommand{\M}{\mathrm{M}}   % Roman M
\newcommand{\N}{\mathrm{N}}   % Roman N
\renewcommand{\O}{\mathrm{O}} % Roman O
\renewcommand{\P}{\mathrm{P}} % Roman P
\newcommand{\Q}{\mathrm{Q}}   % Roman Q
\newcommand{\R}{\mathrm{R}}   % Roman R
\renewcommand{\S}{\mathrm{S}} % Roman S
\newcommand{\T}{\mathrm{T}}   % Roman T
\newcommand{\U}{\mathrm{U}}   % Roman U
\newcommand{\V}{\mathrm{V}}   % Roman V
\newcommand{\W}{\mathrm{W}}   % Roman W
\newcommand{\X}{\mathrm{X}}   % Roman X
\newcommand{\Y}{\mathrm{Y}}   % Roman Y
\newcommand{\Z}{\mathrm{Z}}   % Roman Z

\renewcommand{\a}{\mathrm{a}} % Roman a
\renewcommand{\b}{\mathrm{b}} % Roman b
\renewcommand{\c}{\mathrm{c}} % Roman c
\renewcommand{\d}{\mathrm{d}} % Roman d
\newcommand{\e}{\mathrm{e}}   % Roman e
\newcommand{\f}{\mathrm{f}}   % Roman f
\newcommand{\g}{\mathrm{g}}   % Roman g
\newcommand{\h}{\mathrm{h}}   % Roman h
\renewcommand{\i}{\mathrm{i}} % Roman i
\renewcommand{\j}{\mathrm{j}} % Roman j
\renewcommand{\k}{\mathrm{k}} % Roman k
\renewcommand{\l}{\mathrm{l}} % Roman l
\newcommand{\m}{\mathrm{m}}   % Roman m
\renewcommand{\n}{\mathrm{n}} % Roman n
\renewcommand{\o}{\mathrm{o}} % Roman o
\newcommand{\p}{\mathrm{p}}   % Roman p
\newcommand{\q}{\mathrm{q}}   % Roman q
\renewcommand{\r}{\mathrm{r}} % Roman r
\newcommand{\s}{\mathrm{s}}   % Roman s
\renewcommand{\t}{\mathrm{t}} % Roman t
\renewcommand{\u}{\mathrm{u}} % Roman u
\renewcommand{\v}{\mathrm{v}} % Roman v
\newcommand{\w}{\mathrm{w}}   % Roman w
\newcommand{\x}{\mathrm{x}}   % Roman x
\newcommand{\y}{\mathrm{y}}   % Roman y
\newcommand{\z}{\mathrm{z}}   % Roman z

% Tikz

\tikzset{
  arrow symbol/.style={"#1" description, allow upside down, auto=false, draw=none, sloped},
  subset/.style={arrow symbol={\subset}},
  cong/.style={arrow symbol={\cong}}
}

% Fancy header

\pagestyle{fancy}
\lhead{\module}
\rhead{\nouppercase{\leftmark}}

% Make title

\title{\module}
\author{Lectured by \lecturer \\ Typed by David Kurniadi Angdinata}
\date{\term}

\begin{document}

% Title page
\maketitle
\cover
\vfill
\begin{abstract}
\noindent\syllabus
\end{abstract}

\pagebreak

% Contents page
\tableofcontents

\pagebreak

% Document page
\setcounter{section}{-1}

\setcounter{section}{0}

\section{Introduction}

\lecture{1}{Thursday}{09/01/20}

The following are references.
\begin{itemize}
\item O Biquard and A H\"oring, K\"ahler geometry and Hodge theory, 2008.
\item J P Demailly, Complex analytic and differential geometry, 2012.
\item C Voisin, Hodge theory and complex algebraic geometry, 2002.
\item R O Wells, Differential analysis on complex manifolds, 1973.
\item A Gathmann, Algebraic geometry, 2002
\item P Griffiths and J Harris, Principles of algebraic geometry, 1978.
\end{itemize}

Complex manifolds are manifolds over $ \CC^n $.

\begin{example}
$ \CC^1 $ is a complex manifold. Any open $ U \subset \CC^n $ is a complex manifold.
\end{example}

\begin{example}
The sphere $ \S^2 \subset \RR^3 $ is a complex manifold by
$$ \S^2 \cong \CC \cup \cbr{\infty} = \PP_\CC^1 = \CP^1. $$
More in general $ \PP_\CC^n $ is a complex manifold for all $ n $.
\end{example}

\begin{example}
The torus
$$ \S^1 \times \S^1 = \RR^2 / \ZZ^2 = \CC / \ZZ^2 $$
is a complex manifold. More in general a $ 2n $-dimensional torus $ \CC^n / \Lambda $ for a lattice $ \Lambda \cong \ZZ^{2n} $ is a complex manifold.
\end{example}

\begin{example}
Compact Riemannian surfaces of genus $ g > 1 $, called \textbf{hyperbolics}, are all complex manifolds.
\end{example}

\begin{example}
\label{eg:1.5}
Let $ f : \CC \to \CC $ be holomorphic. The graph of $ f $,
$$ \Gamma_f = \cbr{\br{x, f\br{x}} \st x \in \CC} \subset \CC \times \CC, $$
is a complex manifold. From $ \Gamma_f $ we can recover $ f $, by
$$ f\br{x} = q\br{p^{-1}\br{x} \cap \Gamma_f}, $$
where $ p, q : \CC^2 \to \CC $ are the projections to the first and second factors. This allows us to define $ f^{-1} $. Assume $ f $ is bijective. Define
$$ \function[\tau]{\CC^2}{\CC^2}{\br{x, y}}{\br{y, x}}. $$
Define
$$ \Gamma_{f^{-1}} = \tau\br{\Gamma_f}. $$
Then $ f^{-1} $ is the function induced by $ \Gamma_{f^{-1}} $. This makes sense even if $ f $ is not bijective. Then we get a multivalued function, such as $ \log z $ as the inverse of $ \exp z $.
\end{example}

\begin{example}
Generalising Example \ref{eg:1.5}, we can consider two complex manifolds $ M $ and $ N $ and we can consider holomorphisms $ f : M \to N $. Given $ M $,
$$ \Aut M = \cbr{f : M \to M \ \text{holomorphic bijective and} \ f^{-1} \ \text{holomorphic}}. $$
If $ M = \CC $, there are lots of $ \C^\infty $-functions $ \CC \to \CC $ but the automorphisms of $ \CC $ are just affine linear maps. If $ M = \CC / \ZZ^2 $, then $ \Aut M $ is interesting.
\end{example}

\pagebreak

\begin{example}
Algebraic geometry is the zeroes of polynomials. That is, fix $ m $, and take polynomials $ f_1, \dots, f_k $ in $ m $ variables. Define
$$ M = \cbr{\br{x_1, \dots, x_m} \in \CC^m \st f_1\br{x_1, \dots, x_m} = \dots = f_k\br{x_1, \dots, x_m} = 0}. $$
Then $ M $ is called an \textbf{algebraic variety}. If $ M $ is smooth then $ M $ is a complex manifold. Fix $ m $, take homogeneous polynomials $ F_1, \dots, F_k $ in $ m + 1 $ variables, where $ F $ is \textbf{homogeneous} if it is the sum of monomials of the same degree. Consider
$$ N = \cbr{\br{x_0, \dots, x_m} \in \PP_\CC^m \st F_1\br{x_0, \dots, x_m} = \dots = F_k\br{x_0, \dots, x_m} = 0}. $$
Then $ N $ is called a \textbf{projective variety}. If $ N $ is smooth then $ N $ is a complex manifold.
\end{example}

The idea is if $ M $ is a differentiable manifold, then $ M $ contains lots of submanifolds $ N $. This is not true for complex manifolds. There exist complex manifolds without any proper complex submanifolds, which is not true for projective varieties. The following are questions.
\begin{itemize}
\item What can we say about the topology of complex manifolds? For example, what is $ \pi_1\br{M} $? What is the cohomology of $ M $?
\item Assume that $ M $ and $ N $ are complex manifolds which are diffeomorphic. Are they also isomorphic, so there exists a biholomorphism $ M \to N $?
\end{itemize}
What is next?
\begin{itemize}
\item Hodge decomposition theorem. Understand the cohomology of $ M $ by using the complex structure.
\item Kodaira embedding theorem. Understand when a compact complex manifold is projective.
\end{itemize}

\begin{note*}
If $ M \subset \PP_\CC^m $ is a compact complex manifold then $ M $ is projective.
\end{note*}

\begin{example*}
Let $ M = \Gamma_{\exp} $ for $ \exp : \CC \to \CC $. Then $ M \subset \CC^2 $ but it is not algebraic.
\end{example*}

\pagebreak

\section{Local theory}

\subsection{Holomorphic functions in several variables}

\lecture{2}{Thursday}{09/01/20}

\begin{notation}
Given $ z_0 \in \CC $ and $ r > 0 $, the \textbf{disc} is
$$ \D\br{z_0, r} = \cbr{z \in \CC \st \abs{z - z_0} < r}, $$
and $ \partial\D\br{z_0, r} $ is the boundary of $ \D\br{z_0, r} $.
\end{notation}

\begin{definition}
Let $ U \subset \CC $, and let $ f : U \to \CC $ be a function. Then $ f $ is \textbf{holomorphic at $ z_0 \in U $} if
$$ \lim_{z \to z_0} \dfrac{f\br{z} - f\br{z_0}}{z - z_0} $$
exists.
\end{definition}

\begin{theorem}[Cauchy]
\label{thm:2.3}
Let $ U \subset \CC $ be open, let $ f $ be holomorphic on $ U $, and let $ z_0 \in U $. Assume that if $ D = \D\br{z_0, r} \subset U $ then $ \overline{D} \subset U $. Then
$$ f\br{z_0} = \dfrac{1}{2\pi i}\intd{\partial D}{}{\dfrac{f\br{z}}{z - z_0}}{z}. $$
\end{theorem}

\begin{notation}
Fix $ z_0 = \br{z_{01}, \dots, z_{0n}} \in \CC^n $ and $ R = \br{r_1, \dots, r_n} \in \RR_{> 0}^n $. Then the \textbf{polydisc} is
$$ \D\br{z_0, R} = \cbr{z = \br{z_1, \dots, z_n} \in \CC^n \st \abs{z_i - z_{0i}} < r_i \ \text{for each} \ i}, $$
where $ R $ is the \textbf{polyradius}.
\end{notation}

\begin{definition}
Let $ U \subset \CC^n $ be open, let $ f : U \to \CC $ be a continuous function, and let $ z = \br{z_1, \dots, z_n} \in \CC^n $. Then $ f $ is \textbf{holomorphic} at $ z $, if assuming that $ D = \D\br{z, R} \subset U $ for some $ R = \br{r_1, \dots, r_n} $ then
$$ f\br{z_1, \dots, z_{i - 1}, \cdot, z_{i + 1}, \dots, z_n} : \D\br{z_i, r_i} \to \CC $$
is holomorphic for all $ i $.
\end{definition}

\begin{example}
Any convergent power series in $ n $-variables is holomorphic.
\end{example}

The opposite is also true.

\begin{theorem}[Cauchy]
\label{thm:2.7}
Let $ U \subset \CC^n $ be an open set, let $ f : U \to \CC $ be holomorphic, and let $ z = \br{z_1, \dots, z_n} \in U $. Assume that if $ D = \D\br{z_0, R} $ for some $ R = \br{r_1, \dots, r_n} $ then $ \overline{D} \subset U $. If $ z' = \br{z_1', \dots, z_n'} \in D $ then
$$ f\br{z'} = \dfrac{1}{\br{2\pi i}^n}\intd{\partial\D\br{z_1, r_1}}{}{\dots \intd{\partial\D\br{z_n, r_n}}{}{\dfrac{f\br{z}}{\br{z - z_1'} \dots \br{z - z_n'}}}{z_n \dots}}{z_1}. $$
\end{theorem}

\begin{proof}
Use induction on $ n $ and Cauchy theorem at each step.
\end{proof}

\begin{corollary}
Let $ U \subset \CC^n $ be open, let $ f : U \to \CC $ be holomorphic, and let $ z = \br{z_1, \dots, z_n} \in U $. Then there exists $ D = \D\br{z, R} \subset U $ for some $ R = \br{r_1, \dots, r_n} $ and there exists
$$ p\br{w} = \sum_{m_1, \dots, m_n \ge 0} a_{m_1, \dots, m_n} \br{w_1 - z_1}^{m_1} \dots \br{w_n - z_n}^{m_n}, $$
such that $ p $ is convergent on $ D $ and $ f\br{w} = p\br{w} $ inside $ D $.
\end{corollary}

\begin{proof}
The idea is to use Theorem \ref{thm:2.7} and $ 1 / \br{1 - w} = \sum_{k \ge 0} w^k $.
\end{proof}

\begin{definition}
Let $ U \subset \CC^n $ be open. Then $ f : U \to \CC^m $ is \textbf{holomorphic} if $ f_i = p_i \circ f $ is holomorphic for any $ i = 1, \dots, m $ where $ p_i : \CC^m \to \CC $ is the $ i $-th projection, so $ f = \br{f_1, \dots, f_m} $.
\end{definition}

\begin{fact*}
If $ f : U \to \CC^m $ is holomorphic and $ g : V \to \CC^p $ is holomorphic where $ V \supset f\br{U} $ then $ g \circ f $ is holomorphic.
\end{fact*}

\begin{definition}
Let $ U \subset \CC^n $ be open. A holomorphic function $ f : U \to \CC^m $ is \textbf{biholomorphic at $ z_0 \in U $} if there exists an open neighbourhood $ V \subset U $ of $ z_0 $ such that $ f : V \to f\br{V} $ is bijective and $ f^{-1} : f\br{V} \to V $ is holomorphic. Then $ f $ is \textbf{biholomorphic} if $ f $ is bijective and $ f $ is biholomorphic at any point.
\end{definition}

\begin{note*}
$ f\br{V} $ is automatically open in $ \CC^m $ if $ m = n $.
\end{note*}

\pagebreak

\begin{example}
Let $ \Phi : \CC^n \to \CC^n $ be linear such that $ \det \Phi \ne 0 $. Then $ \Phi $ is a biholomorphism.
\end{example}

\begin{example}
Let $ U = \CC \setminus \cbr{0} $ and
$$ \function[f]{U}{U}{z}{z^2}. $$
Check that $ f $ is biholomorphic at any point of $ U $ but $ f $ is not biholomorphic.
\end{example}

\begin{remark*}
$ \CC^n \cong \RR^{2n} $ and $ \CC^m \cong \RR^{2m} $. Then a holomorphic $ f : U \subset \CC^n \to \CC^m $ is also a diffeomorphism $ U \subset \RR^{2n} \to \RR^{2m} $.
\end{remark*}

\begin{theorem}[Hartogs]
Let $ n \ge 2 $, let $ \epsilon = \br{\epsilon_1, \dots, \epsilon_n} $ and $ \delta = \br{\delta_1, \dots, \delta_n} $ such that $ \epsilon_i > \delta_i > 0 $, let $ U = \D\br{0, \epsilon} \setminus \overline{\D\br{0, \delta}} $, and let $ f : U \to \CC^m $ be holomorphic. Then there exists a holomorphic $ \overline{f} : \D\br{0, \epsilon} \to \CC^m $ such that $ \eval{\overline{f}}_U = f $.
\end{theorem}

\begin{example*}
Hartogs theorem is false for $ n = 1 $. If $ f\br{z} = 1 / z $, for all $ \epsilon > \delta > 0 $, then $ f $ cannot be extended.
\end{example*}

\subsection{Cauchy formula in one variable}

\lecture{3}{Tuesday}{14/01/20}

Let $ \omega = x + iy \in \CC $ for $ x, y \in \RR $, and let $ f : U \to \CC $ be $ \C^\infty $ for some $ U \subset \CC $. Recall that
$$ \dpd{f}{\omega} = \dfrac{1}{2}\br{\dpd{}{x} - i\dpd{}{y}}f, \qquad \dpd{f}{\overline{\omega}} = \dfrac{1}{2}\br{\dpd{}{x} + i\dpd{}{y}}f. $$
Recall that $ f $ is holomorphic if and only if $ \tpd{f}{\overline{\omega}} = 0 $ on $ U $. More in general, let $ U \subset \CC^n $ be open, let $ z_i = x_i + iy_i $, and let $ f : U \to \CC $ be a $ \C^\infty $-function. Then $ f $ is holomorphic if and only if $ \tpd{f}{\overline{z_i}} = 0 $ for all $ i = 1, \dots, n $. Let $ \omega \in \CC $. Since $ \d x \wedge \d y = -\d y \wedge \d x $, let
$$ \d A = \dfrac{i}{2} \d\omega \wedge \d\overline{\omega} = \dfrac{i}{2} \br{\d x + i \d y} \wedge \br{\d x - i \d y} = \d x \wedge \d y, $$
which is the Lebesgue measure on $ \RR^2 \cong \CC $.

\begin{proposition}
Let $ f : U \to \CC $ for $ U \subset \CC $ be a $ \C^\infty $-function, and let $ D = \D\br{z, r} $ such that $ \overline{D} \subset U $. Then
$$ f\br{z} = \dfrac{1}{2\pi i}\intd{\partial D}{}{\dfrac{f}{\omega - z}}{\omega} - \dfrac{1}{\pi}\intd{D}{}{\dfrac{1}{\omega - z}\dpd{f}{\overline{\omega}}}{A}. $$
\end{proposition}

\begin{proof}
Assume $ z = 0 $. Recall that $ f\br{\omega} = 1 / \omega $ is locally integrable around zero, so
$$ \intd{D}{}{\dfrac{1}{\omega}\dpd{f}{\overline{\omega}}}{A} = \lim_{\epsilon \to 0} \intd{D \setminus \D\br{0, \epsilon}}{}{\dfrac{1}{\omega}\dpd{f}{\overline{\omega}}}{A}. $$
Away from zero
\begin{align*}
\d\br{\dfrac{f}{\omega} \d\omega}
& = \dfrac{1}{\omega} \d f \wedge \d\omega + f \d\br{\dfrac{1}{\omega}} \wedge \d\omega
= \dfrac{1}{\omega} \br{\dpd{f}{\omega} \d\omega + \dpd{f}{\overline{\omega}} \d\overline{\omega}} \wedge \d\omega + f\dpd{}{\omega}\br{\dfrac{1}{\omega}} \d\omega \wedge \d\omega \\
& = \dfrac{1}{\omega}\dpd{f}{\overline{\omega}} \d\overline{\omega} \wedge \d\omega
= \dfrac{2i}{\omega}\dpd{f}{\overline{\omega}} \d A.
\end{align*}
Then
\begin{align*}
\dfrac{1}{\pi}\intd{D}{}{\dfrac{1}{\omega}\dpd{f}{\overline{\omega}}}{A}
& = \dfrac{1}{\pi}\lim_{\epsilon \to 0} \intd{D \setminus \D\br{0, \epsilon}}{}{\dfrac{1}{\omega}\dpd{f}{\overline{\omega}}}{A} \\
& = \dfrac{1}{2\pi i}\lim_{\epsilon \to 0} \intd{D \setminus \D\br{0, \epsilon}}{}{}{\br{\dfrac{f}{\omega} \d\omega}} & \dfrac{1}{\omega}\dpd{f}{\overline{\omega}} \d A = \dfrac{1}{2i} \d\br{\dfrac{f}{\omega} \d\omega} \\
& = \dfrac{1}{2\pi i}\lim_{\epsilon \to 0} \br{\intd{\partial D}{}{\dfrac{f}{\omega}}{\omega} - \intd{\partial\D\br{0, \epsilon}}{}{\dfrac{f}{\omega}}{\omega}} & \text{Stokes' theorem} \\
& = \dfrac{1}{2\pi i}\br{\intd{\partial D}{}{\dfrac{f}{\omega}}{\omega} - 2\pi if\br{0}} & \lim_{\epsilon \to 0} \intd{\partial\D\br{0, \epsilon}}{}{\dfrac{1}{\omega}}{\omega} = 2\pi i.
\end{align*}
\end{proof}

If $ f $ is holomorphic, then $ \tpd{f}{\overline{\omega}} = 0 $, which implies Theorem \ref{thm:2.3}.

\pagebreak

\subsection{Rank theorem}

Let $ U \subset \CC^n $ be open, and let $ f : U \to \CC^m $ be holomorphic. Then the \textbf{Jacobian} is
$$ \J_f = \br{\dpd{f_j}{z_i}\br{z}}, $$
where $ f_j = p_j \circ f $ and $ p_j : \CC^m \to \CC $ is the $ j $-th projection.

\begin{exercise*}
Show that the real Jacobian, which is $ 2n \times 2n $, has non-negative determinants.
\end{exercise*}

\begin{theorem}[Rank theorem]
Let $ z \in U $ such that $ r = \rk \J_f\br{z'} $ is constant around $ z $. Then there exist open $ z \in V \subset U \subset \CC^n $ and $ f\br{z} \in W \subset f\br{U} \subset \CC^m $ such that $ \phi : \D\br{0, 1}^n \to V $ and $ \psi : \D\br{0, 1}^m \to W $ are biholomorphisms such that
$$ \function[\eta = \psi^{-1} \circ f \circ \phi]{\D\br{0, 1}^n}{\D\br{0, 1}^m}{\br{z_1, \dots, z_n}}{\br{z_1, \dots, z_r, 0, \dots, 0}}, $$
so
$$
\begin{tikzcd}
\CC^n \supset U & V \arrow[subset]{l} \ni z \arrow{r}{f} & f\br{z} \in W \arrow[subset]{r} & f\br{U} \subset \CC^m \\
& \D\br{0, 1}^n \arrow{u}{\phi} \arrow[swap]{r}{\eta} & \D\br{0, 1}^m \arrow[swap]{u}{\psi} &
\end{tikzcd}.
$$
\end{theorem}

\begin{theorem}[Inverse function theorem]
Let $ f : U \to \CC^n $ be holomorphic for $ U \subset \CC^n $, and let $ z \in U $ such that $ \det \J_f\br{z} \ne 0 $. Then $ f $ is a biholomorphism at $ z $.
\end{theorem}

\begin{proof}
$ \det \J_f\br{z} \ne 0 $ if and only if $ \rk \J_f\br{z} = n $, so $ \rk \J_f\br{z'} = n $ around $ z $, since $ \det \J_f\br{z} $ is a continuous function. Let $ \phi $ and $ \psi $ as in the theorem. Then $ \eta = \psi^{-1} \circ f \circ \phi = \id $, so on $ V $, $ f = \psi \circ \phi^{-1} $ is a composition of biholomorphisms, which is a biholomorphism.
\end{proof}

\begin{remark}
Let $ f : U \to \CC^n $ for $ U \subset \CC^n $. Then $ \det \J_f\br{z} $ is a holomorphism, so
$$ Z = \cbr{z \in U \st \det \J_f\br{z} = 0} $$
is closed.
\end{remark}

\subsection{Holomorphic differential forms}

Let $ U \subset \CC^n $ be open.

\begin{definition}
A \textbf{holomorphic vector field} on $ U $ is the expression
$$ X = \sum_i a_i\dpd{}{z_i}, $$
where $ a_i $ are holomorphic functions on $ U $.
\end{definition}

For all $ x \in U $, the \textbf{tangent space} is
$$ \T_xU = \abr{\dpd{}{x_1}, \dots, \dpd{}{x_n}} \cong \CC^n. $$
If $ x \in U $, then $ X\br{x} \in \T_xU $.

\begin{notation}
$$ \H^0\br{U, \OOO_U} = \cbr{\text{holomorphic functions} \ f : U \to \CC}, \qquad \H^0\br{U, \T_U} = \cbr{\text{holomorphic vector fields on} \ U}. $$
\end{notation}

\begin{remark*}
$ R = \H^0\br{U, \OOO_U} $ is a ring and $ M = \H^0\br{U, \T_U} $ is a module over $ R $. That is, if $ X \in \H^0\br{U, \T_U} $ and $ f \in \H^0\br{U, \OOO_U} $, then $ fX \in \H^0\br{U, \T_U} $.
\end{remark*}

\pagebreak

\begin{definition}
Let $ R $ be a ring and $ M $ be an $ R $-module for $ p \ge 1 $. The \textbf{$ p $-th exterior power} $ \Lambda^pM $ of $ M $ is the $ R $-module $ M^{\otimes p} $ with the relations
$$ m_1 \otimes \dots \otimes m_p - \epsilon\br{\sigma}m_{\sigma\br{1}} \otimes \dots \otimes m_{\sigma\br{p}}, \qquad m_1, \dots, m_p \in M, \qquad \sigma \in \SSS_p, $$
where $ \epsilon\br{\sigma} = \br{-1}^m $ is the signature of $ \sigma $ and $ m $ is the number of transpositions defining $ \sigma $. Then $ M^* = \Hom_R\br{M, R} $ is the \textbf{dual} of $ M $ as an $ R $-module.
\end{definition}

\lecture{4}{Thursday}{16/01/20}

Let $ R = \H^0\br{U, \OOO_U} $ and $ M = \H^0\br{U, \T_U} $.

\begin{definition}
Let $ p > 0 $. We define a \textbf{holomorphic $ p $-form}, as an element of
$$ \H^0\br{U, \Omega_U^p} = \Lambda^pM^*. $$
If $ p = 0 $, by convention a \textbf{holomorphic $ 0 $-form} is just an element in $ R $.
\end{definition}

Let $ z_1, \dots, z_n $ be coordinates for $ U $. Recall $ \eta \in M $ is given by $ \eta = \sum_i a_i\tpd{}{z_i} $ for holomorphic functions $ a_i \in R $. Then $ \omega \in M^* $ is given by the expression
$$ \sum_i b_i \d z_i, \qquad b_i \in R, \qquad \d z_i\br{\dpd{}{z_j}} = \delta_{ij}. $$
More in general $ \omega \in \H^0\br{U, \Omega_U^p} $ is given by
$$ \omega = \sum_{\abs{I} = p} f_I \d z_{i_1} \wedge \dots \wedge \d z_{i_p}, \qquad f_I \in R, \qquad I = \br{i_1, \dots, i_p}, \qquad i_1 < \dots < i_p, $$
where $ \d z_{i_1}, \dots, \d z_{i_p} $ is an $ R $-basis of $ \H^0\br{U, \Omega_U^p} $.

\begin{example*}
$$ \H^0\br{U, \Omega_U^p} \cong \Lambda^p\H^0\br{U, \Omega_U^1} $$
is an isomorphism as $ R $-modules. This is not true for complex manifolds in general.
\end{example*}

The \textbf{exterior product} is
$$ \function{\H^0\br{U, \Omega_U^p} \otimes \H^0\br{U, \Omega_U^q}}{\H^0\br{U, \Omega_U^{p + q}}}{\omega_1 \otimes \omega_2}{\omega_1 \wedge \omega_2}, $$
where we just need to define
$$ \omega_1 \wedge \omega_2 = f \d z_{i_1} \wedge \d z_{i_p} \otimes g \d z_{j_1} \wedge \d z_{j_q} = fg \d z_{i_1} \wedge \dots \wedge \d z_{i_p} \wedge \d z_{j_1} \wedge \dots \wedge \d z_{j_q}, $$
by linearity. Then $ \omega_1 \wedge \omega_2 = 0 $ if $ \cbr{i_1, \dots, i_p} \cap \cbr{j_1, \dots, j_q} \ne \emptyset $, since $ \d z_i \wedge \d z_i = 0 $.

\begin{exercise*}
Check that this definition coincides with the definition in M4P54.
\end{exercise*}

The \textbf{exterior derivative} is
$$ \function[\d]{\H^0\br{U, \Omega_U^p}}{\H^0\br{U, \Omega_U^{p + 1}}}{f \d z_{i_1} \wedge \dots \wedge \d z_{i_p}}{\sum_{j = 1}^n \dpd{f}{z_j} \d z_j \wedge \d z_{i_1} \wedge \dots \wedge \d z_{i_p}}. $$
By definition $ \d $ is $ \CC $-linear, but not $ R $-linear. That is,
$$ \d\br{a\omega_1 + b\omega_2} = a \d\omega_1 + b \d\omega_2, \qquad \omega_1, \omega_2 \in \H^0\br{U, \Omega_U^p}, \qquad a, b \in \CC. $$

\begin{theorem}
Let $ U \subset \CC^n $ be open. Then
\begin{itemize}
\item the Leibnitz rule
$$ \d\br{\omega_1 \wedge \omega_2} = \d\omega_1 \wedge \omega_2 + \br{-1}^p \omega_1 \wedge \d\omega_2, \qquad \omega_1 \in \H^0\br{U, \Omega_U^p}, \qquad \omega_2 \in \H^0\br{U, \Omega_U^q}, $$
\item $ \d^2 = 0 $, that is
$$ \d\br{\d\omega} = 0, \qquad \omega \in \H^0\br{U, \Omega_U^p}. $$
\end{itemize}
\end{theorem}

\pagebreak

\begin{definition}
Let $ f : U \subset \CC^n \to \CC^m $ be holomorphic, let $ f_i = p_i \circ f : V \to \CC $ where $ p_i : \CC^m \to \CC $ is the $ i $-th projection, and let $ f\br{U} \subset V \subset \CC^m $ be open. Then if
$$ \omega = h \d z_{i_1} \wedge \dots \wedge \d z_{i_p} \in \H^0\br{V, \Omega_V^p}, \qquad h \in \H^0\br{U, \OOO_U}, $$
then we can define the \textbf{pull-back} of $ \omega $,
$$ f^*\br{\omega} = h \circ f \d f_{i_1} \wedge \dots \wedge \d f_{i_p} \in \H^0\br{U, \Omega_U^p}, $$
since $ f_i \in \H^0\br{V, \OOO_V} = \H^0\br{V, \Omega_V^0} $ implies that $ \d f_i \in \H^0\br{V, \Omega_V^1} $, so
$$
\begin{tikzcd}
U \arrow{r}{f} \arrow[swap]{dr}{h \circ f \in \H^0\br{U, \OOO_U}} & f\br{U} \subset V \arrow{d}{h} \\
& \CC
\end{tikzcd}.
$$
\end{definition}

This is linear over $ \CC $ and over $ \H^0\br{U, \OOO_U} $.

\begin{proposition}
Let $ U \subset \CC^n $, $ V \subset \CC^m $, and $ W \subset \CC^{m'} $ be open, let $ f : U \to \CC^m $ and $ g : V \to \CC^{m'} $ be holomorphic such that $ V \supset f\br{U} $ and $ W \supset g\br{V} $, and let $ \omega \in \H^0\br{V, \Omega_V^p} $ and $ \eta \in \H^0\br{V, \Omega_V^q} $. Then
\begin{itemize}
\item $ f^*\br{\omega + \eta} = f^*\br{\omega} + f^*\br{\eta} $ if $ p = q $,
\item $ f^*\br{\omega \wedge \eta} = f^*\br{\omega} \wedge f^*\br{\eta} $,
\item $ \d f^*\br{\omega} = f^*\br{\d\omega} $, and
\item $ f^*\br{g^*\br{\omega}} = \br{g \circ f}^*\br{\omega} $.
\end{itemize}
\end{proposition}

Let $ U \subset \CC^n \cong \RR^{2n} $, and let $ z_i = x_i + iy_i $ for $ i = 1, \dots, n $ and $ x_i, y_i \in \RR $. Then
$$ \d z_i = \d x_i + i \d y_i, $$
so any holomorphic form is a differentiable form on $ \RR^{2n} $. A \textbf{$ \br{p, q} $-form} is a differentiable $ \br{p + q} $-form of the expression
$$ \omega = \sum_{\abs{I} = p, \ \abs{J} = q} f_{I, J} \d z_{i_1} \wedge \dots \wedge \d z_{i_p} \wedge \d\overline{z_{j_1}} \wedge \dots \wedge \d\overline{z_{j_q}}, \qquad \d\overline{z_j} = \d x_j - i \d y_j, \qquad f_{I, J} : U \to \CC \cong \RR^2 \in \C^\infty. $$
We denote
$$ \C^\infty\br{U, \Omega_U^{p, q}} = \cbr{\text{differentiable $ \br{p + q} $-forms on} \ U}. $$
If $ \omega $ is a $ \br{p, q} $-form, then the \textbf{conjugate} $ \overline{\omega} $ of $ \omega $ is the $ \br{q, p} $-form defined by
$$ \overline{\omega} = \sum_{\abs{I} = p, \ \abs{J} = q} \overline{f_{I, J}} \d\overline{z_{i_1}} \wedge \dots \wedge \d\overline{z_{i_p}} \wedge \d z_{j_1} \wedge \dots \wedge \d z_{j_q}. $$

\pagebreak

\section{Complex manifolds}

\subsection{Objects}

\begin{definition}
A \textbf{complex manifold} of dimension $ n $ is a connected Hausdorff topological space $ X $, with a countable open cover $ \cbr{U_\alpha} $ of $ X $ such that for all $ \alpha $, there exists $ \phi_\alpha : U_\alpha \to \CC^n $ such that $ \phi_\alpha : U_\alpha \to \phi_\alpha\br{U_\alpha} $ is a homeomorphism and
$$ \phi_\alpha \circ \phi_\beta^{-1} : \phi_\beta\br{U_\alpha \cap U_\beta} \to \phi_\alpha\br{U_\alpha \cap U_\beta} $$
is a biholomorphism for each $ \alpha $ and $ \beta $, so
$$
\begin{tikzcd}
& U_\alpha \cap U_\beta \arrow[swap]{dl}{\phi_\alpha} \arrow{dr}{\phi_\beta} & \\
\CC^n \supset \phi_\alpha\br{U_\alpha \cap U_\beta} \arrow[swap]{rr}{\phi_\alpha \circ \phi_\beta^{-1}} & & \phi_\beta\br{U_\alpha \cap U_\beta} \subset \CC^n
\end{tikzcd}.
$$
The pair $ \br{U_\alpha, \phi_\alpha} $ is called a \textbf{holomorphic chart}. The set $ \cbr{\br{U_\alpha, \phi_\alpha}} $ is called a \textbf{holomorphic atlas} or a \textbf{complex structure}.
\end{definition}

Recall $ X $ is Hausdorff if for all $ x, y \in X $ there exist $ U $ and $ V $ open in $ X $ such that $ U \cap V = \emptyset $ and $ x \in U $ and $ y \in V $.

\lecture{5}{Thursday}{16/01/20}

\begin{example}
If $ U \subset \CC^n $ is an open set then $ U $ is a complex manifold. More in general if $ X $ is a complex manifold and $ U \subset X $ is open then $ U $ is a complex manifold. Let $ \cbr{\br{U_\alpha, \phi_\alpha}} $ be a complex structure on $ M $. Then
$$ \cbr{\br{\overline{U_\alpha}, \overline{\phi_\alpha}}} = \cbr{\br{U_\alpha \cap U, \eval{\phi_\alpha}_{\overline{U_\alpha}}}} $$
is a complex structure of $ M $.
\end{example}

\begin{example}
If $ X $ and $ Y $ are complex manifolds, then $ X \times Y $ is a complex manifold.
\end{example}

\begin{example}
The projective space $ \PP_\CC^n $ or $ \CP^n $. Let
$$ V^* = \CC^{n + 1} \setminus \cbr{0}, $$
with coordinates $ \br{z_0, \dots, z_n} $. Define an equivalence on $ V^* $ as $ v_1 \sim v_2 $ for $ v_1, v_2 \in V^* $ if there exists $ \lambda \in \CC $ such that $ v_1 = \lambda v_2 $. Check that $ \sim $ is an equivalence. Consider the Euclidean topology on $ V^* $. Then there exists an induced topology on
$$ X = V^* / \sim = \cbr{\sbr{v} \st v \in V^*}. $$
with quotient map
$$ \function[q]{V^*}{X}{v}{\sbr{v}}. $$
Given $ v = \br{z_0, \dots, z_n} \in V^* $ we denote $ \sbr{v} = \sbr{z_0, \dots, z_n} $ such that $ z_i \ne 0 $ for some $ i $. Two elements $ \sbr{x_0, \dots, x_n} $ and $ \sbr{y_0, \dots, y_n} $ of $ X $ define the same point if and only if there exists $ \lambda $ such that $ x_i = \lambda y_i $ for all $ i $. Let
$$ V_i = \cbr{\br{z_0, \dots, z_n} \in V^* \st z_i \ne 0}, $$
which is open in $ V^* $, and let
$$ U_i = q\br{V_i}, $$
which is open in $ X $, such that $ \cbr{U_i} $ is a cover of $ X $, that is $ \bigcup_i U_i = X $. Let
$$ H_i = \cbr{\br{z_0, \dots, z_n} \in V^* \st z_i = 1}. $$
Then there exists a homeomorphism
$$ \function[r_i]{H_i}{\CC^n}{\br{z_0, \dots, z_n}}{\sbr{z_0, \dots, z_{i - 1}, z_{i + 1}, \dots, z_n}}, $$
and let
$$ \function[q_i = \eval{q}_{H_i}]{H_i \subset V^*}{U_i \subset X}{\br{z_0, \dots, z_n}}{\sbr{z_0, \dots, z_n}} $$
be also a homeomorphism.

\pagebreak

\begin{itemize}
\item $ q_i $ is surjective. Take $ \sbr{x_0, \dots, x_n} \in U_i $. Then $ x_i \ne 0 $, so choose $ \lambda = 1 / x_i $. Then
$$ \sbr{x_0, \dots, x_n} = \sbr{\dfrac{x_0}{x_i}, \dots, \dfrac{x_n}{x_i}} = q\br{z_0, \dots, z_n}, \qquad z_j = \dfrac{x_j}{x_i}, $$
and in particular $ z_i = 1 $, so there exists $ \br{z_0, \dots, z_n} \in H_i $ such that $ q_i\br{z_0, \dots, z_n} = \sbr{x_0, \dots, x_n} $.
\item $ q_i $ is injective. \footnote{Exercise}
\end{itemize}
For all $ i $, $ q_i^{-1} : U_i \to H_i $ and $ r_i : H_i \to \CC^n $ are homeomorphisms, so $ \phi_i = r_i \circ q_i^{-1} : U_i \to \CC^n $ is also a homeomorphism. We want to show that $ \br{U_i, \phi_i} $ define a holomorphic atlas, so
$$ \phi_i \circ \phi_j^{-1} : \phi_j\br{U_i \cap U_j} \to \phi_i\br{U_i \cap U_j} $$
is a biholomorphism. Consider the case $ j = 0 $ and $ i = 1 $. Then $ \phi_0\br{U_0 \cap U_1} = \cbr{\br{x_1, \dots, x_n} \st x_1 \ne 0} $, so
$$ \function[\phi_1 \circ \phi_0^{-1}]{\phi_0\br{U_0 \cap U_1}}{\phi_1\br{U_0 \cap U_1}}{\br{x_1, \dots, x_n}}{\br{1, \dfrac{x_2}{x_1}, \dots, \dfrac{x_n}{x_1}}} $$
is a biholomorphism. Thus $ X $ is a compact complex manifold. If $ n = 1 $, then $ \PP_\CC^1 \cong \S^2 $.
\end{example}

\begin{example}
The complex torus. Let
$$ \function{\Lambda = \ZZ^{2n}}{\CC^n}{\br{a_1, \dots, a_n, b_1, \dots, b_n}}{\br{a_1 + ib_1, \dots, a_n + ib_n}}. $$
Define an equivalence on $ \CC^n $ by $ v_1 \sim v_2 $ for $ v_1, v_2 \in \CC^n $ if $ v_1 - v_2 \in \Lambda $. Then
$$ X = \CC^n / \sim, $$
with quotient map $ q : \CC^n \to X $ is Hausdorff and compact. Topologically $ X \cong \sbr{0, 1}^{2n} / \sim $. For each $ x \in \CC^n $, we can find an open set $ x \in U \subset \CC^n $ such that $ \eval{q}_U : U \to X $ is a homeomorphism. The idea is if $ x \in \br{0, 1}^{2n} $ then we can take $ U = \br{0, 1}^{2n} $. If not, there exists a translation of $ \CC^n \to \CC^n $ such that the property holds. We define
$$ \phi_V = \eval{q}_U^{-1} : V \subset \CC^n / \Lambda \to U \subset \CC^n, \qquad V = q\br{U}. $$
Show that $ \br{V, \phi_V} $ define a complex structure on $ X $. \footnote{Exercise} This is also a compact complex manifold. More in general $ \CC^n / \Lambda $ where $ \Lambda \cong \ZZ^{2n} $ is a lattice is a compact complex manifold.
\end{example}

\subsection{Morphisms}

\lecture{6}{Tuesday}{21/01/20}

\begin{definition}
Let $ f : X \to Y $ be a continuous morphism between complex manifolds. Then $ f $ is \textbf{holomorphic} if there exists a complex structure $ \cbr{\br{U_\alpha, \phi_\alpha}} $ on $ Y $ and for all $ y \in Y $ there exists a holomorphic chart $ \br{V_\alpha, \psi_\alpha} $ such that $ x \in V_\alpha $ and $ f\br{V_\alpha} \subset U_\alpha $ around any point $ x $ of $ f^{-1}\br{y} $ and $ \phi_\alpha \circ f \circ \psi_\alpha^{-1} $ is holomorphic, so
$$
\begin{tikzcd}
X \supset V_\alpha \arrow{r}{f} \arrow[swap]{d}{\psi_\alpha} & U_\alpha \subset Y \arrow{d}{\phi_\alpha} \\
\psi_\alpha\br{V_\alpha} \arrow[swap]{r}{\widetilde{f}} & \phi_\alpha\br{U_\alpha}
\end{tikzcd}.
$$
Then $ \J_f = \J_{\widetilde{f}} $, and a \textbf{holomorphic function on $ X $} is a holomorphic function $ f : X \to \CC $.
\end{definition}

\begin{exercise*}
If $ X $ is a compact complex manifold then any holomorphic function $ f : X \to \CC $ is constant.
\end{exercise*}

\pagebreak

\begin{definition}
Let $ f : X \to Y $ be a holomorphic function between complex manifolds. Then $ f $ is
\begin{itemize}
\item a \textbf{submersion} if $ \dim Y \ge \dim Y = r $ and $ \rk \J_f = r $ at any point,
\item an \textbf{immersion} if $ r = \dim X \le \dim Y $ and $ \rk \J_f = r $ at any point, and
\item an \textbf{embedding} if it is an immersion and $ f : X \to f\br{X} $ is a homeomorphism.
\end{itemize}
\end{definition}

\begin{example}
Let $ f_2, \dots, f_n : \CC \to \CC $ be holomorphic, and let
$$ \function[f]{\CC}{\CC^n}{z}{\br{z, f_2\br{z}, \dots, f_n\br{z}}}. $$
Then $ f $ is an immersion.
\end{example}

\begin{definition}
Let $ i : X \to Y $ be an embedding of complex manifolds. If $ i\br{X} \subset Y $ is closed then $ i\br{X} $ is called a \textbf{complex submanifold} of $ Y $. The \textbf{codimension} of $ X $ in $ Y $ is $ \dim Y - \dim X $.
\end{definition}

\begin{example}
Let $ X = \CC^2 / \Lambda $ for $ \Lambda = \ZZ^4 \subset \CC^2 $, and let $ q : \CC^2 \to X $. Fix $ \lambda \in \CC $. Let
$$ \function[f]{\CC}{\CC^2}{z}{\br{z, \lambda z}}. $$
Then $ \widetilde{f} = q \circ f : \CC \to X $ is an embedding.
\begin{itemize}
\item If $ \lambda = 0 $ or $ \lambda = \tfrac{1}{2} $, then $ \widetilde{f}\br{\CC} $ is a closed submanifold.
\item If $ \lambda $ is general then $ \widetilde{f}\br{\CC} $ is dense inside $ X $, so it is not closed. Thus it is not a complex submanifold of $ X $.
\end{itemize}
\end{example}

\begin{theorem}
\hfill
\begin{enumerate}
\item Let $ i : X \to Y $ be a submanifold of codimension $ k $, and let $ n = \dim X $. Then for all $ x \in X $, there exists an open neighbourhood $ x \in U \subset Y $ and a submersion $ f : U \to \D\br{0, 1}^k \subset \CC^k $ such that $ X \cap U = f^{-1}\br{0} $.
\item If $ X \subset Y $ is a closed subset such that for all $ x \in X $ there exists $ U \ni x $ open in $ Y $ and a submersion $ f : U \to \D\br{0, 1}^k $ such that $ X \cap U = f^{-1}\br{0} $, then $ X $ is a complex submanifold.
\end{enumerate}
\end{theorem}

\begin{proof}
\hfill
\begin{enumerate}
\item We can assume that if there exists a holomorphic chart $ \br{U, \psi} $ on $ Y $ such that $ x \in U $ and if $ V = i^{-1}\br{U} $ then there exists $ \phi : V \to \CC^n $ such that $ \br{V, \phi} $ is a holomorphic chart on $ X $. After possibly shrinking $ U $ smaller, by the rank theorem, there exist biholomorphic $ a : \psi\br{U} \to \D\br{0, 1}^{n + k} $ and $ b : \phi\br{U} \to \D\br{0, 1}^n $ such that the induced morphism is given by
$$ \function{\D\br{0, 1}^n}{\D\br{0, 1}^{n + k}}{\br{z_1, \dots, z_n}}{\br{z_1, \dots, z_n, 0, \dots, 0}}. $$
Let
$$ \function[c]{\D\br{0, 1}^{n + k}}{\D\br{0, 1}^k}{\br{z_1, \dots, z_{n + k}}}{\br{z_{n + 1}, \dots, z_{n + k}}}, $$
so
$$
\begin{tikzcd}
Y & U \arrow[subset]{l} \arrow{r}{\phi} & \phi\br{U} \arrow{r}{b} & \D\br{0, 1}^n \subset \CC^n \arrow{d} \\
X \arrow[hookrightarrow]{u}{i} & V \arrow[hookrightarrow]{u}{i} \arrow[subset]{l} \arrow[swap]{r}{\psi} & \psi\br{U} \arrow[swap]{r}{a} & \D\br{0, 1}^{n + k} \subset \CC^{n + k} \arrow[bend right=90, dashed, swap]{u}{c}
\end{tikzcd}.
$$
Then $ f $ is the composition $ c \circ a \circ \psi : U \to \D\br{0, 1}^n $.

\pagebreak

\item Let $ \cbr{\br{U_\alpha, \phi_\alpha}} $ be a complex structure on $ Y $, and let $ V_\alpha = X \cap U_\alpha $ and $ \psi_\alpha = \eval{\phi_\alpha}_{V_\alpha} $. The goal is to show that $ \cbr{\br{V_\alpha, \psi_\alpha}} $ defines a complex structure on $ X $. By assumption,
$$ \phi_\alpha \circ \phi_\beta^{-1} : \phi_\beta\br{U_\alpha \cap U_\beta} \subset \CC^{n + k} \to \phi_\alpha\br{U_\alpha \cap U_\beta} \subset \CC^{n + k} $$
is biholomorphic. Let $ U' = \phi_\beta\br{U} $, let $ X' = \phi_\beta\br{X \cap U} $, and let $ f' = f \circ \phi_\beta^{-1} $, so
$$
\begin{tikzcd}
& & & \phi_\alpha\br{U} \arrow[subset]{r} & \phi_\alpha\br{U_\alpha \cap U_\beta} \subset \CC^{n + k} \\
Y & U_\alpha \cap U_\beta \arrow[subset]{l} & U \arrow{ur}{\phi_\alpha} \arrow[subset]{l} \arrow{r}{\phi_\beta} \arrow{drr}{f} & U' \arrow[subset]{r} \arrow{dr}{f'} & \phi_\beta\br{U_\alpha \cap U_\beta} \arrow[swap]{u}{\phi_\alpha \circ \phi_\beta^{-1}} \subset \CC^{n + k} \\
X \arrow[hookrightarrow]{u}{i} & X \cap U_\alpha \cap U_\beta \arrow[subset]{u} \arrow[subset]{l} & X \cap U \arrow[subset]{u} \arrow[subset]{l} & X' \arrow[near start, subset]{u} & \D\br{0, 1}^k \subset \CC^k
\end{tikzcd}.
$$
Then $ f'^{-1}\br{0} = \phi_\beta\br{X \cap U_\alpha \cap U_\beta} $ and $ f' $ is also a submersion. By the rank theorem, we may assume that $ U' = \D\br{0, 1}^{n + k} $ and $ f'\br{z_1, \dots, z_{n + k}} = \br{z_1, \dots, z_k} $, so $ \phi_\beta\br{X' \cap U_\alpha \cap U_\beta} = f'^{-1}\br{0} $. Thus
$$ \br{\psi_\alpha \circ \psi_\beta^{-1}}\br{z_1, \dots, z_n} = \br{\phi_\alpha \circ \phi_\beta^{-1}}\br{z_1, \dots, z_n, 0, \dots, 0} $$
is also a biholomorphism.
\end{enumerate}
\end{proof}

\end{document}