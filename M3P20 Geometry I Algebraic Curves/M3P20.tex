\documentclass{article}

\usepackage{amssymb}
\usepackage{amsthm}
\usepackage[UKenglish]{babel}
\usepackage{enumitem}
\usepackage{fancyhdr}
\usepackage[margin=1in]{geometry}
\usepackage{graphicx}
\usepackage[utf8]{inputenc}
\usepackage{listings}
\usepackage{mathtools}
\usepackage{tikz-cd}
\usepackage{csquotes}

\newcommand{\F}{\mathbb{F}}
\newcommand{\N}{\mathbb{N}}
\newcommand{\Z}{\mathbb{Z}}
\newcommand{\Q}{\mathbb{Q}}
\newcommand{\R}{\mathbb{R}}
\newcommand{\C}{\mathbb{C}}
\newcommand{\A}{\mathbb{A}}
\renewcommand{\P}{\mathbb{P}}

\newcommand{\val}[1]{\left. #1 \right\rvert}
\newcommand{\rb}[1]{\left( #1 \right)}
\renewcommand{\sb}[1]{\left[ #1 \right]}
\newcommand{\cb}[1]{\left\{ #1 \right\}}
\newcommand{\ab}[1]{\left\langle #1 \right\rangle}
\newcommand{\abs}[1]{\left\lvert #1 \right\rvert}
\newcommand{\two}[2]{\begin{pmatrix} #1 \\ #2 \end{pmatrix}}
\newcommand{\three}[3]{\begin{pmatrix} #1 & #2 & #3 \end{pmatrix}}

\theoremstyle{definition}\newtheorem{definition}{Definition}[section]
\theoremstyle{definition}\newtheorem{notation}[definition]{Notation}
\theoremstyle{definition}\newtheorem{remark}[definition]{Remark}
\theoremstyle{definition}\newtheorem{example}[definition]{Example}
\theoremstyle{definition}\newtheorem{fact}{Fact}
\theoremstyle{definition}\newtheorem{exercise}{Exercise}
\newtheorem{proposition}[definition]{Proposition}
\newtheorem{lemma}[definition]{Lemma}
\newtheorem{theorem}[definition]{Theorem}
\newtheorem{corollary}[definition]{Corollary}

\pagestyle{fancy}
\lhead{M3P20 Geometry I: Algebraic Curves}
\rhead{Autumn 2018}

\title{M3P20 Geometry I: Algebraic Curves}
\author{Lectured by Dr Mattia Talpo \\ Typeset by David Kurniadi Angdinata}
\date{Autumn 2018}

\begin{document}

\maketitle

\vfill

\tableofcontents

\pagebreak

\marginpar{Lecture 1 \\ Monday \\ 08/10/18}

\section{Introduction}

This course is intended as a first course in algebraic geometry, the area of mathematics that studies spaces defined by polynomial equations using algebra. It will focus on one-dimensional algebraic varieties. The reference books for the course are the following.
\begin{enumerate}
\item F Kirwan, Complex algebraic curves, 1992
\item W Fulton, Algebraic curves: an introduction to algebraic geometry, 1969
\end{enumerate}

Geometry is the study of shapes in suitable spaces, such as sets of points on the real line $ \R $, lines and circles in $ \R^2 $, spheres in higher dimensional Euclidean spaces $ \R^n $, etc. One way to think about shapes is to see them as the locus of zeroes defined by
$$ \cb{\rb{x_1, \dots, x_n} \in \R^n \mid f\rb{x_1, \dots, x_n} = 0} \qquad \iff \qquad \cb{f\rb{x_1, \dots, x_n} = 0} \subset \R^n $$
for some suitable function $ f $.

\begin{example}
\hfill
\begin{enumerate}
\item Circles $ \cb{f_1\rb{x, y} = x^2 + y^2 - R^2 = 0} $ in $ \R^2 $ for some $ R \in \R $.
\item The unit square with vertices at $ \cb{\rb{\pm 1, 0}, \rb{0, \pm 1}} $ in $ \R^2 $ defined by $ \cb{f_2\rb{x, y} = \abs{x} + \abs{y} - 1 = 0} $.
\item Spheres $ \cb{f_3\rb{x_1, \dots, x_n} = x_1^2 + \dots + x_n^2 - R^2} $ in $ \R^n $ for some $ R \in \R $.
\end{enumerate}
\end{example}

\begin{remark}
Every subset $ S \subseteq \R^n $ is the zero-set of some function. We can define $ \chi_S\rb{x} = 1 $ if $ x \notin S $, $ \chi_S\rb{x} = 0 $ if $ x \in S $.
\end{remark}

The class of functions used to defined our shapes has great consequences on their geometry. For the circle, $ f_1 $ is a polynomial so that it is differentiable and also $ C^\infty $. For the square, $ f $ is continuous but not differentiable at $ \cb{\rb{0, \pm 1}, \rb{\pm 1, 0}} $, the vertices of the square. The function $ \chi_S $ is not even continuous, unless $ S $ is empty, or the whole $ \R^n $. As these examples illustrate, an underlying principle is the following equivalence.

\begin{fact}
Regularity properties of $ f $ are regularity properties of $ \cb{f = 0} $.
\end{fact}

Such shapes are called \textbf{algebraic varieties}. Their geometric properties are intimately related to the algebraic properties of the defining polynomial equations.

\begin{example}
\hfill
\begin{enumerate}
\item Let $ f\rb{x} $ be a polynomial. Then the zero set of $ f\rb{x} $, $ \cb{f\rb{x} = 0} \subseteq \R $ is a finite set of points in $ \R $, and every finite set of points arises in this manner.
\item The circle is an algebraic variety.
\item Spheres in higher dimensions are algebraic varieties.
\end{enumerate}
\end{example}

\begin{exercise}
\hfill
\begin{enumerate}
\item Is $ \Z \subseteq \R $ an algebraic variety? No.
\item Is the unit square an algebraic variety? No.
\end{enumerate}
\end{exercise}

\begin{definition}
Let $ K $ be a field, such as $ K = \Q, \R, \C $. For $ \alpha = \rb{\alpha_1, \dots, \alpha_n} \in \N^n $ a multi-index, denote by $ \abs{\alpha} = \sum_{i = 1}^n \alpha_i $ and by $ x^\alpha = x_1^{\alpha_1} \dots x_n^{\alpha_n} $, a \textbf{monomial}. A \textbf{polynomial of degree $ d $} in $ n $ variables with coefficients in $ K $ is a finite sum.
$$ P\rb{x_1, \dots, x_n} = \sum_{\alpha \in \N^n} a_\alpha x^\alpha, $$
where $ a_\alpha \in K $, $ a_\alpha = 0 $ for all $ \abs{\alpha} > d $ and $ a_\alpha \ne 0 $ for some $ \alpha $ with $ \abs{\alpha} = d $. The set of polynomials of arbitrary degree in $ n $ variables with coefficients in $ K $ is denoted $ K\sb{x_1, \dots, x_n} $.
\end{definition}

\begin{example}
Let $ n = 3 $. $ P\rb{x_1, x_2, x_3} = 3 + x_1^2x_2 + x_3^{10} $ for $ \alpha = \rb{2, 1, 0} $ and $ \alpha = \rb{0, 0, 10} $ has degree ten.
\end{example}

\begin{exercise}
\hfill
\begin{enumerate}
\item Show that $ K\sb{x_1, \dots, x_n} $ is a ring, and that if $ P, Q $ are polynomials of degrees $ p, q $ respectively, then the degree of $ \lambda P + \mu Q $ for $ \lambda, \mu \in K $ is at most $ \max\cb{p, q} $. Give an example of polynomials $ P, Q \in K\sb{x} $ such that $ \deg\rb{P + Q} < \max\cb{\deg\rb{P}, \deg\rb{Q}} $.
\item Show that $ \rb{P \cdot Q}\rb{x_1, \dots, x_n} = P\rb{x_1, \dots, x_n}Q\rb{x_1, \dots, x_n} $ is a polynomial $ P \cdot Q \in K\sb{x_1, \dots, x_n} $ with $ \deg\rb{P \cdot Q} = \deg\rb{P} + \deg\rb{Q} $. What if $ P = 0 $? What is $ \deg\rb{0} $?
\end{enumerate}
\end{exercise}

\begin{definition}
An \textbf{affine plane curve} defined over $ K $ is
$$ C = \cb{\rb{x, y} \in K^2 \mid P\rb{x, y} = 0} \subset K^2, $$
where $ P \in K\sb{x, y} $. More generally, an \textbf{algebraic variety} $ V \subset K^n $ is a subset of $ K^n $ defined as the locus
$$ \cb{f_1 = \dots = f_k = 0} \subset K^n, $$
where $ f_1, \dots, f_k \in K\sb{x_1, \dots, x_n} $ are polynomials in $ n $ variables with coefficients in $ K $.
\end{definition}

\begin{example}
\hfill
\begin{enumerate}
\item Let $ a, b, c \in \R $ with $ \rb{a, b} \ne \rb{0, 0} $, and let
$$ f\rb{x, y} = ax + by + c. $$
Then $ \cb{\rb{x, y} \in \R^2 \mid f\rb{x, y} = 0} $ is a line.
\item Let $ a, b \in \R^* = \R \setminus \cb{0} $ and
$$ f\rb{x, y} = \dfrac{x^2}{a^2} + \dfrac{y^2}{b^2} - 1. $$
The curve $ \cb{\rb{x, y} \in \R^2 \mid f\rb{x, y} = 0} $ is an ellipse.
\item Let $ a, b \in \R^* $ and
$$ g\rb{x, y} = \dfrac{x^2}{a^2} - \dfrac{y^2}{b^2} - 1. $$
The curve $ \cb{\rb{x, y} \in \R^2 \mid g\rb{x, y} = 0} $ is a hyperbola.
\item Spheres, quadrics such as ellipsoids, paraboloids, and hyperboloids in $ \R^3 $ are all defined via a single polynomial equation of degree two. A line in $ \R^3 $ can be defined by two equations in degree one.
\end{enumerate}
\end{example}

The first property of algebraic curves is the following.

\begin{lemma}
\label{lem:1.7}
The union of two affine plane curves is again an affine plane curve.
\end{lemma}

\begin{proof}
Let $ f_1, f_2 \in K\sb{x, y} $ and let $ C_1 = \cb{f_1 = 0} $ and $ C_2 = \cb{f_2 = 0} $. Then $ f_1 \cdot f_2 \in K\sb{x, y} $ is a polynomial and
$$ C_1 \cup C_2 = \cb{f_1 \cdot f_2 = 0}, $$
so that $ C_1 \cup C_2 $ is an affine plane curve.
\end{proof}

\begin{exercise}
Write down an equation for the plane curve that is the union of the lines through any two vertices of the unit square.
\end{exercise}

Recall the following.

\begin{definition}
A polynomial $ P \in K\sb{x_1, \dots, x_n} $ is \textbf{reducible} over $ K $ if there are non-constant polynomials $ Q, R \in K\sb{x_1, \dots, x_n} $, so $ \deg\rb{Q}, \deg\rb{R} > 0 $, such that $ P = Q \cdot R $. A polynomial $ P $ is \textbf{irreducible} if it is not reducible.
\end{definition}

\begin{example}
$ x_1x_2 $ is reducible, $ x_1 + x_2 $ is irreducible.
\end{example}

\begin{fact}
Recall also that every polynomial $ P \in K\sb{x_1, \dots, x_n} $ can be written as a product of irreducible factors $ P = f_1 \dots f_k $ in an essentially unique way up to multiplication by constants. We have
$$ \cb{P = 0} = \cb{f_1 = 0} \cup \dots \cup \cb{f_k = 0} \subseteq K^n, $$ so in particular every algebraic curve is a union of algebraic curves defined by irreducible polynomials.
\end{fact}

In the course, we will consider questions such as the following.
\begin{enumerate}
\item When do polynomials $ f, g \in K\sb{x, y} $ define the same affine plane curve?
\item What can be said about the intersection $ \cb{f = 0} \cap \cb{g = 0} \subset K^2 $?
\end{enumerate}
Very different questions can be approached through algebraic curves. For example, we can study integer solutions to some Diophantine equations.

\begin{example}
The unit circle is the curve
$$ C = \cb{x^2 + y^2 = 1} \subset \R^2. $$
Several parametrisations are known, such as
$$ t \in [0, 2\pi) \mapsto \rb{\cos t, \sin t} \in \R^2. $$
We can write down another parametrisation of $ C $ by considering lines through the point $ P = \rb{-1, 0} $ using a stereographic projection. A line through $ P $ with slope $ t \in \R $ has equation
$$ L_t = \cb{y = t\rb{x + 1}} \subset \R^2 $$
and meets $ C $ in two points, $ P $ and $ P_t = \rb{x\rb{t}, y\rb{t}} $. We can determine the coordinate of $ P_t $ by solving the system
$$ L_t \cap C = \begin{cases} y = t\rb{x + 1} \\ x^2 + y^2 = 1 \end{cases}. $$
Replacing the value of $ y $ given by the first equation into the second yields two solutions for $ x\rb{t} $. The first one is $ x = -1 $ and corresponds to the point $ P = \rb{-1, 0} $. The second is $ \rb{x\rb{t}, y\rb{t}} $, where
$$ x\rb{t} = \dfrac{1 - t^2}{1 + t^2}, \qquad y\rb{t} = \dfrac{2t}{1 + t^2}. $$
Note that when $ t \to \infty $, $ \rb{x\rb{t}, y\rb{t}} \to \rb{-1, 0} $, so that $ t \mapsto \rb{x\rb{t}, y\rb{t}} $ is a parametrisation of $ C $ that identifies it with $ \R \cup \cb{\infty} $. The advantage of this parametrisation is that it is given by rational functions, that is $ x\rb{t} $ and $ y\rb{t} $ are of the form
$$ t \mapsto \dfrac{p\rb{t}}{q\rb{t}}, $$
where $ p, q $ are polynomials. One can use this parametrisation to get the general solution of the equation
\begin{equation}
\label{eq:1}
x^2 + y^2 = z^2
\end{equation}
for $ x, y, z \in \Z $ coprime. If $ t = p / q \in \Q $, where $ p, q \in \Z $ are coprime, then $ x\rb{t}, y\rb{t} \in \Q $ becomes
$$ x\rb{t} = \dfrac{p^2 - q^2}{p^2 + q^2}, \qquad y\rb{t} = \dfrac{2pq}{p + q^2}. $$
If $ x = p^2 - q^2 $, $ y = 2pq $, and $ z = p^2 + q^2 $, $ x, y, z \in \Z $ satisfy $ \rb{\ref{eq:1}} $. They are coprime precisely when $ p, q $ are coprime and not both odd. When $ p, q $ are coprime and both odd, then
$$ x = \dfrac{p^2 - q^2}{2}, \qquad y = pq, \qquad \dfrac{p^2 + q^2}{2} $$
satisfy $ \rb{\ref{eq:1}} $. Conversely, this is the general form of solutions in $ \rb{\ref{eq:1}} $. Indeed, given $ x, y, z \in \Z $ coprime that satisfy $ \rb{\ref{eq:1}} $, $ z \ne 0 $ and
$$ \dfrac{x^2}{z^2} + \dfrac{y^2}{z^2} = 1, $$
so that $ \rb{x / z, y / z} \in \C $ and if $ \rb{x, y, z} \ne \rb{-1, 0, 1} $, there is $ t \in \R $ such that $ \rb{x / z, y / z} = \rb{x\rb{t}, y\rb{t}} $. But then since $ x / z, y / z \in \Q $, we can take $ t \in \Q $ and $ x, y, z $ have the form above.
\end{example}

\marginpar{Lecture 2 \\ Thursday \\ 11/10/18}

\begin{definition}
Let $ f \in \R\sb{x, y} $ and let $ C = \cb{f = 0} $. A \textbf{rational point} of $ C $ is a point $ \rb{x, y} \in C $, that is $ f\rb{x, y} = 0 $, such that $ x, y \in \Q $.
\end{definition}

\begin{example}
There are infinitely many rational points on the circle $ \cb{x^2 + y^2 = 1} \subseteq \R^2 $, which can be described explicitly, and can be used to solve $ a^2 + b^2 = c^2 $ for $ a, b, c \in \Z $, a problem in number theory. Now take $ n \ge 3 $ and consider
$$ C = \cb{x^n + y^n - 1 = 0}. $$
What are the rational points of $ C $? Write
$$ x = \dfrac{a}{c}, \qquad y = \dfrac{b}{c}, \qquad a, b, c \in \Z, \qquad c \ne 0. $$
Then
$$ \rb{x, y} \in C \qquad \iff \qquad a^n + b^n = c^n. $$
Fermat's Last Theorem by Wiles then states that there exists no solution with $ a, b \ne 0 $.
\end{example}

\section{Complex plane curves}

Let $ P \in \R\sb{x, y} $ be a polynomial with coefficients in $ \R $. A priori, it is natural to study the real plane curve $ C_\R = \cb{\rb{x, y} \in \R^2 \mid P\rb{x, y} = 0} $. However, $ P $ can also been seen as a polynomial with coefficients in $ \C $, and it will often be simpler to study the complex plane curve $ C_\C = \cb{\rb{x, y} \in \C^2 \mid P\rb{x, y} = 0} $. We first explain some of the properties of algebraic curves that we would like to hold and explain why these properties do not necessarily hold for real plane curves and some unpleasant things happen.

\begin{fact}
Many real curves are so degenerate that they do not even have points, that is $ C_\R = \emptyset $. If $ C_\R \ne \emptyset $, the dimension of $ C_\R $ is difficult to determine.
\end{fact}

\begin{example}
\label{eg:2.1}
Let $ t \in \R $ and consider $ f_t\rb{x, y} = x^2 + y^2 - t $ and the real plane curve $ C_t = \cb{f_t\rb{x, y} = 0} \subseteq \R^2 $. If $ t > 0 $, $ C_t $ is a circle with radius $ \sqrt{t} $, if $ t = 0 $, has $ C_0 = \cb{\rb{0, 0}} $, and if $ t < 0 $, $ C_t = \emptyset $.
\end{example}

\begin{fact}
In general, it is not clear when two polynomials $ f, g \in \R\sb{x, y} $ define the same real plane curve, that is when
$$ \cb{\rb{x, y} \in \R^2 \mid f\rb{x, y} = 0} = \cb{\rb{x, y} \in \R^2 \mid g\rb{x, y} = 0}. $$
\end{fact}

\begin{example}
Let $ f, g $ denote the polynomials
$$ f\rb{x, y} = x^2y + y^2 + x^3 + x, \qquad g\rb{x, y} = x^2 + 2xy + y^2. $$
Then, since $ f\rb{x, y} = \rb{x + 1} \cdot \rb{x^2 + 1} $ and $ g\rb{x, y} = \rb{x + y}^2 $,
$$ \cb{\rb{x, y} \in \R^2 \mid f\rb{x, y} = 0} = \cb{\rb{x, y} \in \R^2 \mid g\rb{x, y} = 0}. $$
\end{example}

\begin{fact}
In general, it is hard to predict when a curve intersects a fixed line, or more generally when two real curves intersect.
\end{fact}

\begin{example}
In the notation of Example \ref{eg:2.1}, let $ C = \cb{\rb{x, y} \in \R^2 \mid x^2 + y^2 - 1} \subset \R^2 $ be the unit circle. Consider the line $ \cb{ax + by + c = 0} $ for $ \rb{a, b} \ne \rb{0, 0} $. Then, depending on $ \rb{a, b, c} \in \R^3 $, $ L \cap C $ consists of two points, one point, or is empty.
\end{example}

Most of these difficulties disappear when working with curves $ C_\C \subset \C^2 $, essentially because $ \C $ is algebraically closed, in other words the following theorem holds.

\begin{theorem}[Fundamental theorem of algebra]
\label{thm:2.4}
Let $ P \in \C\sb{x} $ be a non-constant polynomial. Then $ P $ has at least one complex root, that is there exists $ \alpha \in \C $ such that $ P\rb{\alpha} = 0 $.
\end{theorem}

A consequence of the fundamental theorem of algebra is that if $ P \in \C\sb{x, y} $ is non-constant, then $ C = \cb{P = 0} $ has infinitely many points. Assume without loss of generality that the polynomials in one variable $ P\rb{\cdot, y} $ and $ P\rb{x, \cdot} $ are not constant. This means that if
$$ P\rb{x, y} = \sum_{\rb{r, s} \in \N^2} c_{r, s}x^ry^s, $$
there exist $ \rb{r, s} $ and $ \rb{r', s'} $ in $ \N^2 $, which may be equal, such that $ r \ne 0 $ and $ s' \ne 0 $, and $ c_{r, s} \ne 0 $ and $ c_{r', s'} \ne 0 $. When $ y_0 \ne 0 $ the polynomial $ P\rb{x, y_0} \in \C\sb{x} $ is non-constant and by Theorem \ref{thm:2.4}, there exists $ x_0 \in \C $ such that $ P\rb{x_0, y_0} = 0 $. In fact, for most choices of $ y_0 \ne y_0' $, the polynomials $ P\rb{\cdot, y_0} $ and $ P\rb{\cdot, y_0'} $ are different, and have some distinct roots. It follows that $ \cb{P = 0} $ contains infinitely many points.

\begin{example}
Let $ a, b, c \in \C $ with $ \rb{a, b} \ne \rb{0, 0} $, and let $ f\rb{x, y} = ax + by + c $. If $ a \ne 0 $, for each $ y \in \C $, there is precisely one solution of $ f\rb{x, y} = 0 $, namely
$$ x = -\dfrac{b}{a}y - \dfrac{c}{a}. $$
Thus $ \C^2 \supset \cb{f = 0} = C \to \C \cong \R^2 $ is an isomorphism. We will call $ C $ a \textbf{complex line}.
\end{example}

\begin{remark}
It is difficult to draw complex curves. Our intuition is for real vector spaces, and this makes complex curves hard to visualise. They are objects of real dimension two in $ \C^2 \cong \R^4 $, a four-dimensional real vector space.
\end{remark}

\begin{example}
Let $ f\rb{x, y} = x^2 + y^2 $. Then $ f\rb{x, y} = \rb{x + iy} \cdot \rb{x - iy} $ and, as in Lemma \ref{lem:1.7}, $ C = \cb{f = 0} \subset \C^2 $ is the union of the two complex lines $ \cb{x + iy = 0} $ and $ \cb{x - iy = 0} $. When seen as $ \R $-vector spaces, these two planes meet at exactly one point corresponding to $ \rb{0, 0} \in \R^2 \subseteq \C^2 $, the only real point of $ C $. It is difficult to imagine two planes meeting in one point, because our intuition relies on three-dimensional space $ \R^3 $, while $ \C^2 = \R^4 $.
\end{example}

Describing intersections is also easier.

\begin{example}
Consider $ C = \cb{x^2 + y^2 - 1 = 0} \subseteq \C^2 $ and $ L = \cb{ax + by + c = 0} \subseteq \C^2 $. If $ b \ne 0 $, we determine the intersection $ C \cap L $ by solving the equation
$$ x^2 + y^2 = 1, \qquad y = -\dfrac{a}{b}x - \dfrac{c}{b}. $$
If $ a^2 = -b^2 $ or $ c = 0 $, there are one or two solutions. Again, it is hard to imagine a two-dimensional real surface which meets a real plane in two points.
\end{example}

We now turn to the question of recognising when two polynomials define the same plane curve. Here again, working in $ \C $ is a simplification.

\begin{theorem}[Consequence of Hilbert's Nullstellensatz]
\label{thm:2.9}
Let $ f, g \in \C\sb{x, y} $ be two polynomials. Then
$$ \cb{f = 0} = \cb{g = 0} $$
if and only if there exist $ P_1, \dots, P_k \in \C\sb{x, y} $, $ a_1, \dots, a_k, b_1, \dots, b_k \in \Z_{> 0} $ and $ \lambda_1, \lambda_2 \in \C^* $ such that
\begin{equation}
\label{eq:2}
f\rb{x, y} = \lambda_1 P_1^{a_1} \dots P_k^{a_k}, \qquad g\rb{x, y} = \lambda_2 P_1^{b_1} \dots P_k^{b_k}.
\end{equation}
\end{theorem}

\marginpar{Lecture 3 \\ Friday \\ 12/10/18}

\begin{proof}
Assume that $ \rb{\ref{eq:2}} $ holds. Then by the proof of Lemma \ref{lem:1.7},
$$ \cb{f = 0} = \cb{P_1^{a_1} = 0} \cup \dots \cup \cb{P_k^{a_k} = 0} = \cb{P_1 = 0} \cup \dots \cup \cb{P_k = 0}, $$
because if $ \alpha \in \C $ is such that $ \alpha^n = 0 $, then $ \alpha = 0 $. The same holds for $ \cb{g = 0} $. Therefore $ \cb{f = 0} = \cb{g = 0} $. The second half of the proof needs tools of commutative algebra and is omitted.
\end{proof}

\begin{remark}
Theorem \ref{thm:2.9} fails over $ \R $. Let $ f\rb{x, y} = x^2 + 1 $ and $ g\rb{x, y} = 1 $. Then $ \cb{f = 0} = \cb{g = 0} = \emptyset $ but the conclusion in $ \rb{\ref{eq:2}} $ is not true.
\end{remark}

Thus, the relation between the geometric shape $ C = \cb{f = 0} $ in $ \C^2 $ and the polynomial $ f \in \C\sb{x, y} $ is more transparent than in $ \R $. We will always work in $ \C $. Let us introduce some important notions for the study of polynomials.

\begin{definition}
A polynomial $ f \in K\sb{x, y} $ has \textbf{no repeated factors} over $ K $ if it cannot be written as a product of the form
$$ f\rb{x, y} = g\rb{x, y}^2 \cdot h\rb{x, y}, $$
where $ g, h \in K\sb{x, y} $ and $ g $ is non-constant.
\end{definition}

\begin{exercise}
Show that this is equivalent to
$$ f = P_1 \dots P_k, $$
where $ P_1, \dots, P_k $ are distinct irreducible polynomials.
\end{exercise}

\begin{corollary}
Let $ f, g \in \C\sb{x, y} $ be polynomials with no repeated factors. Then $ f, g $ define the same complex plane curve
$$ \cb{f = 0} = \cb{g = 0} $$
if and only if there is a non-zero constant $ \lambda \in \C^* $ such that $ f = \lambda g $.
\end{corollary}

\begin{proof}
Follows immediately from Theorem \ref{thm:2.9}.
\end{proof}

\begin{remark}
If $ f = P_1^{a_1} \dots P_k^{a_k} $ with $ P_i $ irreducible for all $ i $ and $ a_i \in \N $, then $ \cb{f = 0} = \cb{g = 0} $ where $ g = P_1 \dots P_k $. We do not lose anything by only looking at $ f $ with no repeated factors.
\end{remark}

Let $ C \subseteq \C^2 $ be a complex plane curve. We have proved that, up to multiplication by $ \lambda \in \C^* $, there is a unique non-constant polynomial $ f \in \C $ with no repeated factors such that
$$ C = \cb{f = 0}. $$
It makes sense to define the following.

\begin{definition}
The \textbf{degree} of an affine curve $ C \subseteq \C^2 $ is the degree of any polynomial with no repeated factors $ f $ such that $ C = \cb{f = 0} $, that is
$$ \deg\rb{C} = \deg\rb{f}. $$
\end{definition}

\begin{example}
Lines have degree one, since they are defined by a linear polynomial. Conics have degree two. $ \cb{x^2y + y^2 + x + 1 = 0} $ has degree 3, assuming it has no repeated factors.
\end{example}

Unless mentioned otherwise, in the first few weeks, we will assume that polynomials have no repeated factors.

\begin{definition}
Let $ f_1, f_2 \in \C\sb{x, y} $ be polynomials with no repeated factors and let $ C_1 = \cb{f = 0} $ and $ C_2 = \cb{g = 0} $ be the associated complex curves. The curves $ C_1 $ and $ C_2 $ have \textbf{no common component} if there is no non-constant polynomial $ P $ that divides both $ f $ and $ g $.
\end{definition}

This is equivalent to saying that if $ f = P_1^{a_1} \dots P_k^{a_k} $ and $ g = Q_1^{b_1} \dots Q_k^{b_k} $ with $ P_i, Q_i $ irreducibles, $ P_i $ distinct distinct, and $ Q_i $ distinct, then $ \lambda P_i \ne Q_j $ for all $ i, j, \lambda \in \C^* $.

\begin{exercise}
Show that if $ C_1 $ and $ C_2 $ have no common component, then $ \deg\rb{C_1 \cup C_2} = \deg\rb{C_1} + \deg\rb{C_2} $.
\end{exercise}

\begin{exercise}
Let $ L, L' $ be the lines
$$ L = \cb{ax + by + c = 0} \subset \C^2, \qquad L' = \cb{a'x + b'y + c' = 0} \subset \C^2. $$
\begin{enumerate}
\item Show that $ L $ and $ L' $ meet at exactly one point if and only if $ ab' - a'b \ne 0 $.
\item Show that $ L = L' $ if and only if there exists $ \lambda \in \C $ such that $ \lambda \ne 0 $ and
$$ a' = \lambda a, \qquad b' = \lambda b, \qquad c' = \lambda c. $$
\end{enumerate}
\end{exercise}

\begin{remark}[First aid topology]
\label{rem:2.19}
\hfill
\begin{enumerate}
\item A \textbf{topological space} is a set $ X $ with a collection of open subsets $ \cb{U_i \subset X} $ such that
\begin{enumerate}
\item $ \emptyset $ and $ X $ are open,
\item any union $ \cup_{i \in I} U_i $ of open sets $ U_i $ is open, and
\item any finite intersection $ \cap_{i = 1}^k U_i $ of open sets $ U_i $ is open.
\end{enumerate}
\item A \textbf{metric space} $ X $, such as $ \rb{\C^n, \abs{\abs{.}}} $, is a topological space. The open sets are given by arbitrary unions and finite intersections of the familiar open balls $ B\rb{x, \epsilon} = \cb{z \in X \mid \abs{\abs{z - x}}} < \epsilon $.
\item A subset $ X \subset Y $ of a topological space $ Y $ inherits a topology from $ Y $. The open sets of $ X $ are the sets $ X \cap U $, where $ U \subset Y $ is an open set of $ Y $.
\item $ X $ is \textbf{compact} if for all open covering $ X = \sum_{i \in I} $ where $ U_i $ are open, there exists a finite subcovering $ \Cup_{i_1, \dots, i_k} U_{i_j} $ for $ \cb{i_1, \dots, i_k} \subseteq I $.
\item The \textbf{Heine-Borel theorem} states that a subset $ X $ of $ \R^n $ or of $ \C^m $ is compact if and only if $ X $ is closed, that is its complement is open, and bounded for the usual norm.
\item A closed subset of a compact space is compact.
\item A map $ f : X \to Y $ between topological spaces is \textbf{continuous} if and only if $ f^{-1}\rb{U} $ is open in $ X $ whenever $ U \subset Y $ is open. It follows that $ f^{-1}\rb{F} $ is closed whenever $ F \subset Y $ is closed. In particular, if $ f \in \C\sb{x_1, \dots, x_n} $ is a polynomial, $ f $ defines a map $ f : \C^n \to \C $ that is continuous, and
$$ f^{-1}\rb{\cb{0}} = \cb{f = 0} \subset \C^n $$
is closed because $ \cb{0} $ is a closed subset of $ \C $.
\end{enumerate}
\end{remark}

In particular, $ \C^2 $ is a topological space with the Euclidean distance in $ \R^4 $ and the affine plane curve $ C = \cb{f = 0} \subseteq \C^2 \cong \R^4 $ inherits a topology. Open sets of $ C $ are $ U \cup C $ where $ U \subseteq \C^2 $ is open. So algebraic curves have a natural topology.

\begin{lemma}
Let $ C \subset \C^2 $ be an affine plane curve, then $ C $ is not compact.
\end{lemma}

\begin{proof}
Since $ f $ is a continuous function $ \C^2 \to \C $, $ C = \cb{f = 0} = f^{-1}\rb{\cb{0}} $, and $ \cb{0} $ is closed in $ \C $, $ C $ is closed in $ \C^2 $. We check that $ C \subset \C^2 $ is not bounded. Assume that it is, then there is a constant $ M > 0 $ such that $ C \subset B\rb{0, M} $, where the open ball is
$$ B\rb{0, M} = \cb{\abs{x}^2 + \abs{y}^2 < M}. $$
Want to show that some points in $ \C $ are outside this open ball. Let $ y_0 \in \C $ be such that $ \abs{y_0} > M $ and assume we can arrange for $ g = f\rb{\cdot, y_0} $ to be a non-constant polynomial of $ x $. By the fundamental theorem of algebra, $ g $ has a root $ x_0 \in \C $ and the point $ \rb{x_0, y_0} \in C $. This is a contradiction, as $ \rb{x_0, y_0} \notin B\rb{0, M} $.
\end{proof}

(TODO Exercise: what if $ f\rb{x, y} $ happens to be a polynomial of $ y $ alone, so that this cannot be arranged?)

\marginpar{Lecture 4 \\ Monday \\ 15/10/18}

\section{Projective space}

Recall that it is difficult to determine when two affine plane curves $ C, C' \in \C^2 $ intersect, and some curves do not in fact intersect even over $ \C $. We want to fix that, and the key is adding points at infinity.

\begin{example}
\hfill
\begin{enumerate}
\item Consider two distinct lines
$$ L_1 = \cb{ax + by + c = 0}, \qquad L_2 = \cb{a'x + b'y + c' = 0}. $$
Then $ L_1 $ and $ L_2 $ meet at exactly one point if and only if
$$ \det\two{a & b}{a' & b'} \ne 0. $$
But we can pretend that parallel lines meet at a point at infinity corresponding to the direction vector.
\item Consider the asymptotic curve and line
$$ C = \cb{xy - 1 = 0}, \qquad L = \cb{x = 0}. $$
Then $ C $ and $ L $ do not meet, but again we can pretend that they meet at a point at infinity.
\end{enumerate}
\end{example}

A heuristic trick is to introduce a variable $ z $.
\begin{enumerate}
\item Replace $ x $ by $ x / z $ and $ y $ by $ y / z $.
\item Solve for $ z = 0 $.
\end{enumerate}

\begin{example}
\hfill
\begin{enumerate}
\item Consider the lines
$$ L_1 = \cb{x + y + 1 = 0}, \qquad L_2 = \cb{x + y - 1 = 0}. $$
Clearly $ L_1 $ and $ L_2 $ do not meet. Let us apply the trick. By $ 1 $ and $ 2 $ we get
$$ \begin{cases} x / z + y / z + 1 = 0 \\ x / z + y / z - 1 = 0 \end{cases} \qquad \implies \qquad \begin{cases} x + y + z = 0 \\ x + y - z = 0 \end{cases} \qquad \implies \qquad \begin{cases} x + y = 0 \\ x + y = 0 \end{cases}. $$
We get that the point $ \rb{1, -1, 0} $ is a common solution. This will be called the point at infinity.
\item Consider the asymptotic curve and line
$$ C = \cb{xy - 1 = 0}, \qquad L = \cb{x = 0}. $$
Apply $ 1 $ and $ 2 $ to get
$$ \begin{cases} xy - z^2 = 0 \\ x / z = 0 \end{cases} \qquad \implies \qquad \begin{cases} xy = 0 \\ x = 0 \end{cases}. $$
We get that $ \rb{0, 1, 0} $ is a common solution. Again, this will be called the point at infinity.
\end{enumerate}
\end{example}

To make this formal, we introduce the projective plane $ \P^2 $. We will add points at infinity to $ \C^2 $, in such a way that asymptotic curves meet at infinity. We will then compactify an affine plane curve $ C $ so that the two compactifications are compatible, that is
$$ \rb{C \subseteq \C^2} \hookrightarrow \rb{\overline{C} \subseteq \P^2}. $$

\begin{notation}
Fix $ n \ge 0 $ and $ \C^{n + 1} $. Let $ \underline{0} = \rb{0, \dots, 0} \in \C^{n + 1} $ be the origin of the $ \rb{n + 1} $-dimensional complex Euclidean space. We will denote
$$ W = \C^{n + 1} \setminus \cb{\underline{0}}, $$
that is a point $ x \in W $ is given by $ x = \rb{x_0, \dots, x_n} $ where $ x_0, \dots, x_n \in \C $ are not all zero. We define the equivalence relation on $ W $, for any $ x, y \in W $ by
$$ x \sim y \qquad \iff \qquad \exists \lambda \in \C^* = \C \setminus \cb{0}, \ x = \lambda y. $$
\end{notation}

\begin{exercise}
Show that $ \sim $ is an equivalence relation on $ W $.
\end{exercise}

\begin{notation}
Given $ x \in W $, we denote
$$ \sb{x} = \cb{y \in W \mid x \sim y}. $$
For simplicity, if $ x = \rb{x_0, \dots, x_n} $ we will denote $ \sb{x} = \sb{x_0, \dots, x_n} $ instead of $ x = \sb{\rb{x_0, \dots, x_n}} $.
\end{notation}

\begin{definition}
The \textbf{$ n $-dimensional projective space} $ \P^n_\C $ or $ \P^n\rb{\C} $ or simply $ \P^n $ is defined as the quotient of $ W $ by $ \sim $, that is
$$ \P^n = \dfrac{W}{\sim} = \cb{\sb{x} \mid x \in W = \C^{n + 1} \setminus \cb{\underline{0}}}. $$
The coordinates of $ \P^n $ are $ \sb{x} \in \P^n $ except $ \sb{0, \dots, 0} $ and $ \sb{\lambda x_0, \dots, \lambda x_n} = \sb{x_0, \dots, x_n} $ for $ \lambda \in \C^* $. In other words, in $ \P^n $, two points $ \sb{x_0, \dots, x_n} $ and $ \sb{y_0, \dots, y_n} $ are the same point if and only if there exists a non-zero constant $ \lambda $ such that
$$ x_0 = \lambda y_0, \qquad \dots, \qquad x_n = \lambda y_n. $$
\end{definition}

\begin{exercise}
Show that $ \sb{x} = \sb{y} $ if and only if $ x \sim y $. Show that if $ y \notin \sb{x} $ then $ \sb{x} \cap \sb{y} = \emptyset $.
\end{exercise}

\begin{example}
The point $ \sb{1, 2, i} $ is the same as the point $ \sb{i, 2i, -1} $.
\end{example}

\begin{exercise}
Show that there exists a bijection between $ \P^n $ and the set of all the one-dimensional subspaces of $ \C^{n + 1} $. In fact, if $ V $ is a finite-dimensional vector space over $ \C $ without the choice of a basis, we can defined the associated projective space $ \P\rb{V} $ as the set of one-dimensional linear subspaces of $ V $.
\end{exercise}

\begin{example}
For any non-zero $ x \in \C $ we have $ \sb{x} = \sb{1} $. So $ \P^0 = \C^1 \setminus \cb{0} / \sim $ is a point $ \C^0 = \cb{\sb{1}} $.
\end{example}

\begin{notation}
For any $ i = 0, \dots, n $, denote the \textbf{affine chart}
$$ U_i = \cb{\sb{x} = \sb{x_1, \dots, x_n} \in \P^n \mid x_i \ne 0} \subseteq \P^n. $$
\end{notation}

\begin{lemma}
$ \P^n = U_0 \cup \dots \cup U_n $.
\end{lemma}

\begin{proof}
Take $ \sb{x} = \sb{x_0, \dots, x_n} \in \P^n $ in $ \P^n $ then $ x \in W $ and in particular $ x = \rb{x_0, \dots, x_n} $ where at least one of the coefficients is non-zero, say $ x_i \ne 0 $. Then $ \sb{x} \in U_i $. Thus any $ \sb{x} \in \P^n $ is contained in the union of $ U_0, \dots, U_n $.
\end{proof}

\begin{lemma}
Pick $ i = 0, \dots, n $. Define $ \phi_i : \C^n \to U_i $ by
$$ \phi_i\rb{y_1, \dots, y_n} = \sb{y_1, \dots, y_i, 1, y_{i + 1}, \dots, y_n}. $$
Then $ \phi_i $ is a bijection and its inverse $ \rho_i : U_i \to \C^n $ is given by
$$ \rho_i\sb{x_0, \dots, x_n} = \rb{\dfrac{x_0}{x_i}, \dots, \dfrac{x_{i - 1}}{x_i}, \dfrac{x_{i + 1}}{x_i}, \dots, \dfrac{x_n}{x_i}}. $$
\end{lemma}

\begin{proof}
First note that both $ \phi_i $ and $ \rho_i $ is well-defined, indeed, if
$$ \rb{y_1, \dots, y_n} \in \C^n $$
then
$$ \rb{y_1, \dots, y_i, 1, y_{i + 1}, \dots, y_n} \in W $$
and therefore
$ \sb{y_1, \dots, y_i, 1, y_{i + 1}, \dots, y_n} \in \P^n $. Similarly, if $ \sb{x_0, \dots, x_n} = \sb{x_0', \dots, x_n'} $ then it follows that $ \rho_i\sb{x_0, \dots, x_n} = \rho_i\sb{x_0', \dots, x_n'} $. Thus, it is enough to show that both $ \phi_i \circ \rho_i $ and $ \rho_i \circ \phi_i $ coincide with the identity. We have
$$ \rho_i\rb{\phi_i\rb{y_1, \dots, y_n}} = \rho_i\sb{y_1, \dots, y_i, 1, y_{i + 1}, \dots, y_n} = \rb{\dfrac{y_1}{1}, \dots, \dfrac{y_{i - 1}}{1}, \dfrac{y_i}{1}, \dots, \dfrac{y_n}{1}} = \rb{y_1, \dots, y_n}. $$
Similarly,
$$ \phi_i\rb{\rho_i\sb{x_0, \dots, x_n}} = \phi_i\rb{\dfrac{x_0}{x_i}, \dots, \dfrac{x_{i - 1}}{x_i}, \dfrac{x_{i + 1}}{x_i}, \dots, \dfrac{x_n}{x_i}} = \sb{\dfrac{x_0}{x_i}, \dots, \dfrac{x_{i - 1}}{x_i}, 1, \dfrac{x_{i + 1}}{x_i}, \dots, \dfrac{x_n}{x_i}} = \sb{x_0, \dots, x_n}. $$
Thus they are inverses.
\end{proof}

\begin{example}
Let $ n = 2 $ and $ \sb{x_0, x_1, x_2} \in \P^2 $. Let $ \phi_1 : \C^2 \to U_1 $ be defined by $ \rb{0, 0} \mapsto \sb{0, 1, 0} $ and $ \phi_2 : \C^2 \to U_2 $ be defined by $ \rb{0, 0} \mapsto \sb{0, 0, 1} $. Check that
$$ \rho_i\rb{\phi_i\rb{y_1, \dots, y_n}} = \rho_i\rb{\sb{y_1, \dots, 1, \dots, y_n}} = \rb{y_1, \dots, y_n}. $$
\end{example}

The previous two lemmas can be used to define a topology on $ \P^n $. Let $ U \subseteq \P^n $, then $ U $ is open if and only if $ \phi_i^{-1}\rb{U \cap U_i} \subseteq \C^n $ is open in $ U_i \cong \C^n $ for any $ i = 0, \dots, n $.

\begin{exercise}
Show that $ U_i \subseteq \P^n $ is open in $ \P^n $ for all $ i = 0, \dots, n $.
\end{exercise}

\begin{exercise}
\label{ex:11}
We can define another topology on $ \P^n $ as the quotient topology induced by the map $ \pi : W \to \P^n $ defined by $ \rb{x_0, \dots, x_n} \mapsto \sb{x_0, \dots, x_n} $. A subset $ U \subseteq \P^n $ is open if and only if its preimage $ \pi^{-1}\rb{U} \subseteq W $ is open in $ W $. Show that this indeed defines a topology, and that this topology coincides with the one defined above using the maps $ \phi_i $. Check that with this topology on $ \P^n $, $ \pi $ is continuous.
\end{exercise}

\begin{exercise}
\label{ex:12}
Prove that $ \P^n $ is compact. Restrict the projection $ \pi $ of Exercise \ref{ex:11} to the $ \rb{n + 1} $-dimensional sphere $ S^{n + 1} \subseteq W $, and check that this restriction is surjective and continuous. Since $ S^{n + 1} $ is compact, it follows that $ \P^n $ is compact as well.
\end{exercise}

\begin{example}
\hfill
\begin{enumerate}
\item $ \P^1 = \cb{\sb{x_0, x_1} \mid \rb{x_0, x_1} \in \C^2 \setminus \cb{\underline{0}}} $ is the union of two copies $ U_0 $ and $ U_1 $ of $ \C^1 $. The intersection of $ U_0 $ and $ U_1 $ is $ U_0 \cap U_1 = \cb{\sb{x_0, x_1} \mid x_0 \ne 0, x_1 \ne 0} $ can be identified with $ \C^* = \C \setminus \cb{0} $ via the map $ \sb{x_0, x_1} \mapsto x_1 / x_0 $. Using this identification and the maps $ \rho_i $, the inclusion $ U_0 \cap U_1 \subseteq U_0 $ is the map $ \C^* \to \C^1 $ sending $ z $ to itself, and the inclusion $ U_0 \cap U_1 \subseteq U_1 $ is the map $ \C^* \to \C^1 $ given by $ z \mapsto 1 / z $. $ \P^1 $ is glued together from two copies of $ \C $ along $ U_0 \cap U_1 $ by these inclusions, so $ \P^1 = \C^1 \cup \cb{\infty} $. Over the real numbers, $ \P^1\rb{\R} $ is built up in the same way from two copies of $ \R^1 $, and can be identified with the circle $ S^1 $.
\item $ \P^2 $ is the union of three copies of $ U_0 \cong U_1 \cong U_2 \cong \C^2 $, and the intersection can be described similarly. At infinity we will have a line. More generally, $ \P^n $ can be described similarly.
\end{enumerate}
\end{example}

In practice, we will view the affine complex plane $ \C^2 $ as being embedded in $ \P^2 $ as one of the open sets $ U_i $. For example, we identify
$$ \rb{x, y} \in \C^2 \qquad \iff \qquad \sb{x, y, 1} \in \P^2. $$
What is the complement of this embedding, that is the points at infinity?

\begin{notation}
For any $ i = 0, \dots, n $, denote
$$ \mathcal{P}_i = \cb{\sb{x_0, \dots, x_n} \in \P^n \mid x_i = 0} \subseteq \P^n. $$
\end{notation}

\begin{lemma}
\label{lem:3.13}
For any $ i = 0, \dots, n $, we have
$$ \P^n = U_i \sqcup \mathcal{P}_i. $$
Moreover if we define $ f_i : \P^{n - 1} \to \mathcal{P}_i $ by
$$ f_i\sb{z_0, \dots, z_{n - 1}} \mapsto \sb{z_0, \dots, z_{i - 1}, 0, z_i, \dots, z_{n - 1}} $$
then $ f_i $ is a bijection.
\end{lemma}

\begin{proof}
TODO Exercise: both statements are easy to check.
\end{proof}

In conclusion, we have that
$$ \P^n = U_i \sqcup \mathcal{P}_i \cong \C^n \sqcup \P^{n - 1} = \dots = \C^n \cup \dots \cup \C^0. $$
$ \mathcal{P}_i $ is called the \textbf{hyperplane at infinity}.

\begin{example}
\hfill
\begin{enumerate}
\item We have already seen that $ \P^0 $ is a point.
\item $ \P^1 \cong \C^1 \cup \P^0 $. In other words, $ \P^1 $ is obtained by adding a point at infinity to the complex line $ \C $. It is a way to compactify the real plane. If $ \C^1 $ above is $ U_0 $, this point at infinity is the origin of the other open subset $ U_1 \cong \C^1 $. One can show that there is an bijection which is also a homeomorphism between $ \P^1 $ and the sphere $ S^2 $ in $ \R^3 $ by the identification $ r \mapsto 1 / r $. $ \P^1 $ is called the projective line.
\item $ \P^2 \cong \C^2 \cup \P^1 $. Thus $ \P^2 $ is obtained by adding a projective line at infinity to the complex plane $ \C^2 $. $ \P^2 $ is called the projective plane.
\end{enumerate}
\end{example}

Note that the point at infinity or the line at infinity is not unique, it depends on the choice of the coordinate and in our settings, it depends on $ i $. In the future, we will often fix $ i $, and this will give us a unique choice of the point or line at infinity.

\marginpar{Lecture 5 \\ Thursday \\ 18/10/18}

\section{Projective curves}

Recall that an algebraic curve $ C \subseteq \C^2 $ is given by $ C = \cb{f\rb{x, y} = 0} $ for $ f \in \C\sb{x, y} $. In this section, we want to define projective curves similarly in $ \P^2 $ and then compactify these affine curves. If we try to define a plane projective curve in the same way as an affine curve, that is as $ \cb{f = 0} \subseteq \P^2 $, where $ f $ is a polynomial in $ \C\sb{x_0, x_1, x_2} $, the first hurdle we encounter is that $ f $ does not define a function on $ \P^2 $.

\begin{example}
\hfill
\begin{enumerate}
\item Let $ f\rb{x, y, z} = x^2 + y^2 - z^2 $. Then $ f\rb{1, 1, 1} = 1 + 1 - 1 = 1 \ne 4 = 4 + 4 - 4 = f\rb{2, 2, 2} $, so that $ f $ does not define a function on $ \P^2 $. It makes no sense to talk about $ f\rb{1, 1, 1} $, because in $ \P^2 $, $ \sb{1, 1, 1} = \sb{2, 2, 2} $, so the value would not be well-defined. However, the locus where $ f $ is vanishes in this case is well-defined. If $ f\rb{x, y, z} = 0 $, then $ f\rb{\lambda x, \lambda y, \lambda z} = 0 $ for all $ \lambda \in \C^* $, so that the subset $ \cb{\sb{x, y, z} \in \P^2 \mid f\rb{x, y, z} = 0} $ is well-defined.
\item Let $ g\rb{x_0, x_1, x_2} = x_0^2 + x_1 $, then $ g\rb{i, 1, 0} = 0 $, but $ g\rb{2i, 2, 0} \ne 0 $, so that in this case, the vanishing locus $ \cb{\sb{x_0, x_1, x_2} \in \P^2 \mid g\rb{x_0, x_1, x_2} = 0} $ is not even well-defined.
\end{enumerate}
\end{example}

The example shows that if we want to define projective curves as
$$ \cb{\sb{x_0, x_1, x_2} \in \P^2 \mid f\rb{x_0, x_1, x_2} = 0}, $$
then $ f \in \C\sb{x_0, x_1, x_2} $ has to satisfy some additional properties.

\begin{definition}
A polynomial $ f \in \C\sb{x_0, \dots, x_n} $ is \textbf{homogeneous} if all its monomials have the same degree $ d \in \N $, that is
$$ f\rb{x_0, \dots, x_n} = \sum_{\abs{\alpha} = \sum_{i = 0}^n \alpha_i = d} a_\alpha x^\alpha, \qquad \alpha \in \N^{n + 1}. $$
\end{definition}

\begin{example}
\hfill
\begin{enumerate}
\item The polynomial $ f\rb{x, y, z} = x^2 + y^2 - z^2 $ is homogeneous of degree two. The polynomial $ g\rb{x_0, x_1, x_2} = x_0^2 + x_1 $ is not homogeneous.
\item $ f\rb{x, y} = Ax^3 + Bx^2y + Cxy^2 + Dy^3 $ is homogeneous of degree three, and all the homogeneous polynomials of degree three in $ x, y $ can be written in this form.
\item $ g\rb{x, y, z} = x^4 - 2x^2yz + yz^3 $ is homogeneous of degree four.
\end{enumerate}
\end{example}

\begin{lemma}
\label{lem:4.4}
Let $ P \in \C\sb{x_0, \dots, x_n} $ be a polynomial. If $ P $ is homogeneous of degree $ d $, then
$$ P\rb{\lambda x_0, \dots, \lambda x_n} = \lambda^dP\rb{x_0, \dots, x_n}, $$ for all $ \lambda \in \C $, and all $ \rb{x_0, \dots, x_n} \in \C^{n + 1} $.
\end{lemma}

\begin{proof}
Let $ P $ be a homogeneous polynomial of degree $ d $. Then $ P = M_1 + \dots + M_k $, where each $ M_i $ is a monomial of degree $ d $, that is a product $ a_i \cdot x^{\alpha_i} $ where $ \alpha_i \in \N^{n + 1} $ and $ a_i \in \C $. For each monomial $ M = a_\alpha x^\alpha = a_\alpha x_0^{\alpha_0} \dots x_n^{\alpha_n} $ of degree $ d $, for $ a_\alpha \in \C $ and $ \alpha \in \N^{n + 1} $, we have
$$ M\rb{\lambda x_0, \dots, \lambda x_n} = a_\alpha\rb{\lambda^{\alpha_0}x_0^{\alpha_0}} \dots \rb{\lambda^{\alpha_n}x_n^{\alpha_n}} = \lambda^{\sum_{i = 0}^n \alpha_i}a_\alpha x^\alpha = \lambda^da_\alpha x^\alpha = \lambda^dM\rb{x_0, \dots, x_n}, $$
because $ \sum_{i = 0}^n \alpha_i = \abs{\alpha} = d $. For the arbitrary homogeneous polynomial $ P $, write $ P = \sum_{i = 1}^k M_i $. Thus
$$ P\rb{\lambda x_0, \dots, \lambda x_n} = \sum_{i = 1}^k M_i\rb{\lambda x_0, \dots, \lambda_n x_n} = \lambda^d\sum_{i = 1}^k M_i\rb{x_0, \dots, x_n} = \lambda^dP\rb{x_0, \dots, x_n}. $$
\end{proof}

\begin{exercise}
Prove the converse implication for Lemma \ref{lem:4.4}, that is if $ P \in \C\sb{x_0, \dots, x_n} $ and
$$ P\rb{\lambda x_0, \dots, \lambda x_n} = \lambda^dP\rb{x_0, \dots, x_n}, $$
for all $ \lambda \in \C $, and all $ \rb{x_0, \dots, x_n} \in \C^{n + 1} $, then $ P $ is homogeneous of degree $ d $.
\end{exercise}

\begin{proposition}
Let $ P $ be a homogeneous polynomial. Then
$$ \cb{\sb{x_0, \dots, x_n} \in \P^n \mid P\rb{x_0, \dots, x_n} = 0} $$
is well-defined.
\end{proposition}

\begin{proof}
We have to check that if $ \rb{x_0, \dots, x_n} \sim \rb{y_0, \dots, y_n} $, $ P\rb{x_0, \dots, x_n} = 0 $ if and only if $ P\rb{y_0, \dots, y_n} = 0 $. This follows immediately from Lemma \ref{lem:4.4}, because by definition of $ \sim $, $ \rb{y_0, \dots, y_n} = \rb{\lambda x_0, \dots, \lambda x_n} $ for some $ \lambda \in \C^* $, so $ P\rb{x_0, \dots, x_n} = 0 $ if and only if $ P\rb{\lambda x_0, \dots, \lambda x_n} = \lambda^dP\rb{x_0, \dots, x_n} = 0 $.
\end{proof}

\begin{notation}
Unless mentioned otherwise, arbitrary polynomials will be denoted $ f, g, h, \dots $ while homogeneous polynomials will be denoted $ P, Q, R, \dots $.
\end{notation}

Let $ P_1, \dots, P_k \in \C\sb{x_0, \dots, x_n} $ be homogeneous polynomials. Then the vanishing locus
$$ \cb{P_1 = \dots = P_k = 0} \subseteq \P^n $$
is a \textbf{projective variety}. These are the main object of study of algebraic geometry. In this class, we will focus on plane projective curves.

\begin{definition}
A \textbf{complex projective plane algebraic curve} is
$$ C = \cb{\sb{x_0, x_1, x_2} \in \P^2 \mid P\rb{x_0, x_1, x_2} = 0} \subseteq \P^2, $$
where $ P $ is a non-constant homogeneous polynomial $ P \in \C\sb{x_0, x_1, x_2} $.
\end{definition}

From now on $ n = 2 $, so consider $ \P^2 $.

\begin{example}
The hyperplanes at infinity $ \mathcal{P}_i = \cb{x_i = 0} \subseteq \P^2 $ for $ i = 0, 1, 2 $ are projective curves of degree one, where $ x_0 $ is a homogeneous polynomial of degree one.
\end{example}

\begin{definition}
A \textbf{projective line} is a projective curve defined by a homogeneous polynomial $ ax_0 + bx_1 + cx_2 = 0 $ of degree one.
\end{definition}

We have seen that in the case of affine plane curves $ \cb{f = 0} \subseteq \C^2 $, the irreducible factors of the polynomial $ f $ were important when studying curves. The next lemma ensures that the same type of results hold for projective curves.

\begin{lemma}
\label{lem:4.10}
Let $ P \in K\sb{x_0, \dots, x_n} $ be a non-zero homogeneous polynomial. Assume that $ P = Q \cdot R $, where $ Q, R \in K\sb{x_0, \dots, x_n} $. Then the polynomials $ Q $ and $ R $ are homogeneous.
\end{lemma}

\begin{proof}
TODO Exercise.
\end{proof}

\begin{remark}
As in the case of affine plane curves, we can therefore make sense of irreducible projective plane curves $ C = \cb{P = 0} \subset \P^2 $ with $ P $ irreducible. When $ P $ is reducible $ P = P_1^{a_1} \dots P_k^{a_k} $ with $ P_i $ distinct irreducible polynomials and $ a_i \in \N_{> 0} $, the projective curves $ C_i = \cb{P_i = 0} $ in $ \cb{P = 0} = \cb{P_1 = 0} \cup \dots \cup \cb{P_k = 0} $ are called the \textbf{irreducible components} of $ C = \cb{P = 0} $.
\end{remark}

\begin{remark}
\label{rem:4.12}
Hilbert's Nullstellensatz, and Theorem \ref{thm:2.9}, still holds. In particular, if $ P, Q \in \C\sb{x, y, z} $ are homogeneous polynomials with no repeated factors, $ \cb{P = 0} = \cb{Q = 0} \subseteq \P^2 $ if and only if $ P = \lambda Q $ for $ \lambda \in \C^* $.
\end{remark}

\begin{definition}
Let $ C \subseteq \P^2 $ be a projective plane curve and $ P $ any homogeneous polynomial with no repeated factor such that $ C = \cb{P = 0} \subseteq \P^2 $. The degree of $ C $ is the degree of $ P $.
\end{definition}

\begin{exercise}
Show that the union of two projective curves $ C_1, C_2 \subseteq \P^2 $ is a projective curve. More generally, show that if $ X_1, X_2 \subseteq \P^2 $ are projective varieties, then $ X_1 \cup X_2 \subseteq \P^n $ is again a projective variety.
\end{exercise}

\begin{exercise}
Show that if $ X = \cb{P = 0} \subseteq \P^n $ for a non-zero homogeneous polynomial $ P \in \C\sb{x_0, \dots, x_n} $, then
$$ X = \cb{x_0P = \dots = x_nP = 0}. $$
\end{exercise}

\begin{lemma}
Let $ C \subseteq \P^2 $ be a projective curve. Then $ C $ is compact.
\end{lemma}

Recall that affine curves are never compact.

\begin{proof}
Recall that a closed subset of a compact is compact itself. Since $ \P^2 $ is compact by Exercise \ref{ex:12}, we only need to prove that $ C $ is closed, or equivalently, that $ \pi^{-1}\rb{C} \subseteq W = \C^3 \setminus \cb{\underline{0}} $ is closed, where $ \pi : W \to W / \sim = \P^2 $ by $ \rb{x_0, x_1, x_2} \mapsto \rb{x_0, x_1, x_2} $, by Remark \ref{rem:2.19}. Assume that $ C = \cb{P = 0} $ for a homogeneous polynomial $ P $. Then, $ P : W \to \C $ is a continuous map, so that $ \pi^{-1}\rb{C} = P^{-1}\rb{\cb{0}} = \cb{P\rb{x_0, x_1, x_2} = 0} \cap W $ is a closed subset of $ W $.
\end{proof}

\begin{remark}
You may have heard of the Zariski topology for algebraic varieties. The Zariski topology is defined on affine space $ \C^n $ or on projective space $ \P^n $ as the topology whose closed sets are of the form $ V\rb{S} = \cb{\underline{x} \in \C^n \mid f\rb{\underline{x}} = 0} $ where $ S $ is a set of polynomials or homogeneous polynomials $ f \in \C\sb{x_0, \dots, x_n} $. In this course, we do not work in the Zariski topology, but in the classical metric or Euclidean topology on $ \C^{n + 1} $ and in the topology this induces on $ \P^n $ as outlined above. One reason for that is that the Zariski topology is not Hausdorff, or separated.
\end{remark}

\marginpar{Lecture 6 \\ Friday \\ 19/10/18}

Lecture 6 is a problem class.

\section{Affine vs projective plane curves}

We now show how to projectivise affine curves, and study the relationship between affine and projective curves.

\begin{notation}
We will typically use coordinates $ x, y $ on $ \C^2 $ and $ x_0, x_1, x_2 $ on $ \P^2 $.
\end{notation}

Starting with an affine curve $ C = \cb{f = 0} \subseteq \C^2 $ for $ f \in \C\sb{x, y} $ non-constant, want to construct a projective curve $ \overline{C} \subseteq \P^2 $, a projectivisation of $ C $. Recall that we have identified $ \C^2 $ with the open subset $ U_2 = \cb{x_2 \ne 0} \subseteq \P^2 $, via the function $ \phi : U_2 \to \C^2 $ defined by $ \sb{x_0, x_1, x_2} \mapsto \rb{x_0 / x_2, x_1 / x_2} $ and its inverse $ \rb{x, y} \mapsto \sb{x, y, 1} $. We now want to identify $ U_2 \cap \overline{C} \subseteq \C^2 $ with the original curve $ C $ for a suitable projective plane curve $ \overline{C} = \cb{P = 0} \subseteq \P^2 $ for $ P $ homogeneous, picking $ \C^2 \cong U_2 \subseteq \P^2 $ to be the points not at infinity.

\begin{example}
Idea is to let $ x = x_0 / x_2 $ and $ y = x_1 / x_2 $. $ f\rb{x, y} = x^3 + y + 1 = \rb{x_0 / x_2}^3 + \rb{x_1 / x_2} + 1 $ and clear denominators to get $ P\rb{x_0, x_1, x_2} = x_0^3 + x_1x_2^2 + x_2^3 $, homogeneous of degree three. Check that restricting $ P $ to $ U_2 \cong \C^2 $ I get back $ f $. Restricting $ P $ to $ U_2 \cong \C^2 $ gives $ P\rb{x, y, 1} = x^3 + y + 1 $. $ \cb{P = 0} = \overline{C} $ is the projectivisation of $ \cb{f = 0} = C $.
\end{example}

\begin{theorem}
\label{thm:5.2}
There is a bijective correspondence between
\begin{enumerate}
\item projective plane curves $ \overline{C} = \cb{P = 0} \subseteq \P^2 $ that do not contain the line at infinity $ \mathcal{P}_2 = \cb{x_2 = 0} $, and
\item affine curves $ C = \cb{f = 0} \subseteq \C^2 $.
\end{enumerate}
The bijection $ C \leftrightarrow \overline{C} $ is obtained by
$$ P \mapsto \sb{f : \rb{x, y} \mapsto P\rb{x, y, 1}}, \qquad f \mapsto \sb{P : \rb{x_0, x_1, x_2} \mapsto x_2^d \cdot f\rb{x_0 / x_2, x_1 / x_2}}, $$
for $ d = \deg\rb{f} $. The affine curve $ \cb{f = 0} \subseteq \C^2 $ is $ \phi\rb{U \cap \overline{C}} $, where $ \overline{C} = \cb{P = 0} $.
\end{theorem}

\begin{notation}
In general, $ P $ is called the \textbf{homogenisation} of the polynomial $ f $. $ \overline{C} $ is the \textbf{projectivisation} of $ C $.
\end{notation}

\begin{proof}
Let $ \overline{C} = \cb{P = 0} \subseteq \P^2 $ be a projective curve that does not contain $ \cb{x_2 = 0} $ as in the statement. $ \cb{x_2 = 0} $ is not contained in $ \overline{C} $ if and only if $ P $ contains at least one monomial without $ x_2 $. Then $ P $ contains at least one monomial without $ x_2 $, so that $ f : \rb{x, y} \to P\rb{x, y, 1} $ is a polynomial in $ x, y $. If not, $ P = x_2 \cdot Q $ and $ \cb{P = 0} = \cb{x_2 = 0} \cup \cb{Q = 0} $. So $ f\rb{x, y} $ has degree equal to $ d = \deg\rb{P} $. Under the identification $ \phi : U \to \C^2 $ defined in the previous section,
$$ \phi\rb{U \cap \overline{C}} = \cb{\rb{x, y} \in \C^2 \mid f\rb{x, y} = 0}. $$
Conversely, if $ C = \cb{f = 0} $, then $ C $ is the image by $ \phi $ of the intersection of $ U $ and $ \overline{C} $, where $ \overline{C} = \cb{P = 0} $, where the polynomial $ P $ is defined by $ P\rb{x_0, x_1, x_2} = x_2^d \cdot f\rb{x_0 / x_2, x_1 / x_2} $. Check that $ P $ is a well-defined homogeneous polynomial of degree $ d $.
\end{proof}

Look at an example where $ \overline{C} $ does contain $ \cb{x_2 = 0} $.

\begin{example}
Let $ P = x_0x_2^2 + x_1^2x_2 = x_2\rb{x_0x_2 + x_1^2} $. Then $ f\rb{x, y} = P\rb{x, y, 1} = x + y^2 $, so
$$ P\rb{x_0, x_1, x_2} = \rb{x_0 / x_2 + \rb{x_1 / x_2}^2}x_2^2 = x_0x_2 + x_1^2. $$
\end{example}

\begin{remark}
To intersect $ \cb{P = 0} $ and $ \cb{Q = 0} $ in $ \P^2 $, solve $ P = 0 $ and $ Q = 0 $ in $ x_0, x_1, x_2 $ to get homogeneous coordinates of points of intersection, where $ \underline{0} $ is not a valid solution and $ \rb{\lambda x_0, \lambda x_1, \lambda x_2} $ is the same as $ \rb{x_0, x_1, x_2} $.
\end{remark}

\begin{example}
\hfill
\begin{enumerate}
\item Let $ \overline{C} $ be the projective curve $ \overline{C} = \cb{P\rb{x_0, x_1, x_2} = x_0^2 + x_1^2 + x_2^2 = 0} $. Then $ C = \phi\rb{\overline{C} \cap U_2} $ is $ \cb{f = 0} \subseteq \C^2 $, where $ f\rb{x, y} = x^2 + y^2 + 1 $.
\item Let $ C $ be the affine curve $ x^2y + y - 1 = 0 $. Then $ \overline{C} $ is defined by $ x_2^3\rb{\rb{x_0 / x_2}^2\rb{x_1 / x_2} + x_1 / x_2 - 1} = x_0^2x_1 + x_1x_2^2 - x_2^3 $.
\end{enumerate}
\end{example}

\begin{exercise}
Let $ i = 1 $ or $ 2 $, recall that $ \phi_i $ denotes the homeomorphism $ \phi : U_i \to \C^2 $. Show that if $ \overline{C} \subseteq \P^2 $ is a projective curve that does not contain $ \mathcal{P}_i = \cb{x_i = 0} $, then
$$ C_i = \phi_i\rb{\overline{C} \cap U_i} \subseteq \C^2 $$
is an affine curve.
\end{exercise}

\marginpar{Lecture 7 \\ Monday \\ 22/10/18}

Intersect projectivisations to find points of intersection at infinity.

\begin{example}
\hfill
\begin{enumerate}
\item Consider the parallel lines
$$ L_1 = \cb{x + y + 1 = 0}, \qquad L_2 = \cb{x + y - 1 = 0}. $$
The corresponding projective lines are given by
$$ \overline{L_1} = \cb{x_0 + x_1 + x_2 = 0}, \qquad \overline{L_2} = \cb{x_0 + x_1 - x_2 = 0}, $$
then solve to give
$$ \begin{cases} x_0 + x_1 = -x_2 \\ x_0 + x_1 = x_2 \end{cases} \qquad \implies \qquad \begin{cases} x_0 + x_1 = 0 \\ x_2 = 0 \end{cases}. $$
Inside $ \P^2 $, the intersection $ L_1 \cap L_2 $ consists of exactly one point
$$ p = \sb{1, -1, 0}. $$
Thus, the two projective lines meet at one point $ p \in \mathcal{P}_0 $, that is $ p $ is a point at infinity.
\item Similarly, let
$$ C = \cb{xy - 1 = 0}, \qquad L = \cb{x = 0}, $$
then the projectivisations are
$$ \overline{C} = \cb{x_0 \cdot x_1 - x_2^2 = 0}, \qquad \overline{L} = \cb{x_0 = 0}, $$
then intersect to give
$$ \begin{cases} x_0 = 0 \\ x_0x_1 - x_2^2 = 0 \end{cases} \qquad \implies \qquad \begin{cases} x_0 = 0 \\ x_2 = 0 \end{cases}. $$
As above, in $ \P^2 $, $ \overline{C} \cap \overline{L} $ consists of a single point
$$ \cb{\sb{0, 1, 0}} $$
lying on $ \mathcal{P}_2 = \cb{x_2 = 0} $, the hyperplane at infinity.
\end{enumerate}
\end{example}

\begin{example}
Projective conics over $ \R $ are defined by degree two equations. In $ \R^2 $, smooth conics are three kinds, ellipses, hyperbolas, and parabolas. Passing to $ \P^2\rb{\R} $, these three kinds of curves become the same. Ellipses have no new points at infinity, hyperbolas have two new points at infinity, and parabolas have one new point at infinity.
\begin{enumerate}
\item Let $ f\rb{x, y} = x^2 - y^2 + 1 $. Projectivising, $ P\rb{x_0, x_1, x_2} = x_0^2 - x_1^2 + x_2^2 $. Restrict to $ U_1 = \cb{x_1 \ne 0} \subseteq \R^2 $. $ \overline{C} \cap U_1 $, the equation of the unit circle $ g\rb{x, y} = x^2 - 1 + y^2 = x^2 + y^2 - 1 $ is obtained by setting $ x_1 = 1 $.
\item Consider the parabola $ C $ with equation $ y = x^2 $ in $ \R^2 $. Projectivising, this becomes the curve $ \overline{C} = \cb{x_1x_2 = x_0^2} \subseteq \P^2\rb{\R} $. The intersection with the line at infinity $ x_2 = 0 $ gives $ x_0^2 = 0 $, so the only point is $ \sb{0, 1, 0} $. The square suggests some kind of tangency, and indeed $ \overline{C} $ is tangent to $ x_2 = 0 $. In the chart $ x_0 \ne 0 $, the equation becomes $ xy = 1 $, and we see a hyperbola. In the chart $ x_1 \ne 0 $ we have again a parabola, $ y = x^2 $.
\item Consider the unit circle $ C = \cb{x^2 + y^2 - 1 = 0} \subseteq \R^2 $. Projectivising we obtain $ \overline{C} = \cb{x_0^2 + x_1^2 = x_2^2} \subseteq \P^2\rb{\R} $. The intersection with the line at infinity $ x_2 = 0 $ is empty as we could expect. In the chart $ x_0 \ne 0 $ we obtain the curve $ 1 + x^2 - y^2 = 0 $, which is a hyperbola, and similarly in $ x_1 \ne 0 $.
\end{enumerate}
\end{example}

\begin{exercise}
Let $ a, b, c, d, e, f \in \C $, with $ \rb{a, b, c} \ne \rb{0, 0, 0} $, and define
$$ C = \cb{ax^2 + bxy + cy^2 + dx + ey + f = 0}. $$
Define the projectivisation $ \overline{C} $ of $ C $ and determine its points on the line at infinity.
\end{exercise}

\begin{exercise}
\label{ex:18}
\hfill
\begin{enumerate}
\item Show that there exists a unique projective line $ L \subseteq \P^2 $ through two distinct points $ P, Q \in \P^2 $.
\item Show that two distinct projective lines in $ \P^2 $ meet in exactly one point.
\end{enumerate}
\end{exercise}

\begin{exercise}
\label{ex:19}
Let $ P_1, P_2, P_3 $ be three points of $ \P^2 $, and denote
$$ P_i = \sb{b_{i, 1}, b_{i, 2}, b_{i, 3}}, \qquad i = 1, 2, 3 $$
for their coordinates. Let $ B = \rb{b_{i, j}} $ be the associated $ 3 \times 3 $ matrix. Show that $ P_1, P_2, P_3 $ lie in the same projective line if and only if $ \det\rb{B} = 0 $.
\end{exercise}

\section{Points at infinity}

Given an affine curve $ C \subseteq \C^2 $ and its projectivisation $ \overline{C} \subseteq \P^2 $, we study the points at infinity of $ C $, that is the points of $ \overline{C} $ that do not lie on $ C $. Let us first recall the notation in use. Denote by $ U_2 $ the open set of $ \P^2 $ defined by $ \C^2 \cong U_2 = \cb{x_2 \ne 0} \subseteq \P^2 $ and let $ \mathcal{P}_2 = \P^2 \setminus U_2 = \cb{x_2 = 0} $ be the corresponding hyperplane or line at infinity. The homeomorphism $ \phi : \C^2 \to U_2 $ is defined by $ \rb{x, y} \mapsto \sb{x, y, 1} $, and its inverse $ \psi : U_2 \to \C^2 $ is the map $ \psi : \sb{x_0, x_1, x_2} \mapsto \sb{x_0 / x_2, x_1 / x_2} $. Let $ C $ be the affine curve
$$ C = \cb{\rb{x, y} \in \C^2 \mid f\rb{x, y} = 0}, $$
where $ f \in \C\sb{x, y} $ is a polynomial of degree $ d $ with no repeated factors. The projectivisation of $ C $ is then the curve $ \overline{C} $ such that $ \overline{C} \cap \cb{x_2 \ne 0} = \phi\rb{C} $, or equivalently such that $ \psi\rb{\overline{C} \cap U_2} = C $. The curve $ \overline{C} $ is
$$ \overline{C} = \cb{\sb{x_0, x_1, x_2} \in \P^2 \mid P\rb{x_0, x_1, x_2} = 0}, $$
where $ P $ is the homogeneous polynomial
$$ P\rb{x_0, x_1, x_2} = x_2^df\rb{x_0 / x_2, x_1 / x_2}. $$
Let $ f_i \in \C\sb{x, y} $ be the homogeneous polynomials of degree $ 0 \le i \le d $ with
$$ f\rb{x, y} = f_0\rb{x, y} + \dots + f_d\rb{x, y}, $$
and note that
$$ P\rb{x_0, x_1, x_2} = f_0\rb{x_0, x_1}x_2^d + \dots + f_d\rb{x_0, x_1}. $$

\begin{definition}
The \textbf{points at infinity} of $ C $ are the points on $ \overline{C} \setminus \phi\rb{C} $. We will often just write $ \overline{C} \setminus C $.
\end{definition}

How many points at infinity can there be? If $ \cb{x_2 = 0} \subseteq \overline{C} $, if and only if $ x_2 $ divides $ P\rb{x_0, x_1, x_2} $, then there are infinitely many. Assume $ \cb{x_2 = 0} \not\subseteq \overline{C} $. Look at $ P\rb{x_0, x_1, 0} $, and find solutions of $ P\rb{x_0, x_1, 0} = 0 $ in $ \P^1 \cong \mathcal{P}_2 $. The points at infinity of $ C $ are thus
$$ \overline{C} \setminus C = \overline{C} \cap \mathcal{P}_2 = \cb{\sb{x_0, x_1, 0} \in \P^2 \mid P\rb{x_0, x_1, 0} = 0}. $$
From the above equation, and under the identification $ \mathcal{P}_2 \cong \P^1 $, this becomes
$$ \overline{C} \setminus C = \cb{\sb{x_0, x_1} \in \P^1 \mid f_d\rb{x_0, x_1} = 0}, $$
that is a projective variety in $ \mathcal{P}_2 \cong \P^1 $. It is natural to expect that such a projective variety consists of a finite number of points. This is what we will prove next.

\begin{lemma}
\label{lem:6.2}
Let $ Q \in \C\sb{x_0, x_1} $ be a non-zero homogeneous polynomial of degree $ d \ge 1 $. Then there exist $ \alpha_i, \beta_i \in \C $ with $ \rb{\alpha_i, \beta_i} \ne \rb{0, 0} $ for $ i = 1, \dots, d $ such that
$$ Q\rb{x_0, x_1} = \prod_{i = 1}^d \rb{\alpha_ix_0 + \beta_ix_1}. $$
Therefore, $ \cb{\sb{x_0, x_1} \in \P^1 \mid Q\rb{x_0, x_1} = 0} = \cb{P_i = \sb{-\beta_i, \alpha_i}} \subseteq \P^1 $. Note that the $ P_i $ need not be distinct.
\end{lemma}

\begin{proof}
Let us pass to $ U_1 \cong \C^1 \subseteq \P^1 $, one of the two affine charts. Write $ Q\rb{x_0, x_1} = \sum_{r = 0}^d a_rx_0^rx_1^{d - r} $ and let $ e = \max\cb{r \mid a_r \ne 0} $ so
$$ Q\rb{x_0, x_1} = x_1^d \cdot \sum_{r = 0}^e a_r\rb{\dfrac{x_0}{x_1}}^r. $$
Define $ f \in \C\sb{x} $ as $ f\rb{x} = \sum_{r = 0}^e a_rx^r $. By the fundamental theorem of algebra, there are $ \lambda_1, \dots, \lambda_e \in \C $ such that
$$ f\rb{x} = a_e \cdot \prod_{i = 1}^e \rb{x - \lambda_i}, $$
where $ \lambda_i $ are the roots of $ f $, and hence
$$ Q\rb{x_0, x_1} = x_1^d \cdot a_e \cdot \prod_{i = 1}^e \rb{\dfrac{x_0}{x_1} - \lambda_i} = a_e \cdot x_1^{d - e} \cdot \prod_{i = 1}^e \rb{x_0 - \lambda_ix_1}. $$
This proves Lemma \ref{lem:6.2}, if we set
$$ \rb{\alpha_i, \beta_i} = \begin{cases} \rb{1, -\lambda_i} & i \le e \\ \rb{0, a_e} & i = e + 1 \\ \rb{0, 1} & i > e + 1 \end{cases}. $$
If $ e = d $, then set
$$ \rb{\alpha_i, \beta_i} = \begin{cases} \rb{1, -\lambda_i} & i < e \\ \rb{a_e, -a_e\lambda_e} & i = e \end{cases}. $$
The description that $ \cb{Q = 0} \subseteq \P^1 $ is clear, once we note that $ \rb{\alpha_i, \beta_i} \ne \rb{0, 0} $.
\end{proof}

We have proved the following.

\begin{theorem}
Let $ \overline{C} \subseteq \P^2 $ be a projective curve defined by a polynomial of degree $ d $ that does not contain $ \mathcal{P}_2 $. Then $ \overline{C} \cap \mathcal{P}_2 $ is a non-empty set of at most $ d = \deg\rb{Q} $ points.
\end{theorem}

\begin{definition}
Let $ C = \cb{f = 0} \subseteq \C^2 $ for $ f \in \C\sb{x, y} $ be an affine curve of degree $ d $ and let $ \overline{C} $ be its projectivisation, where, as usual, $ C = \psi\rb{\overline{C} \cap U} $. Denote
$$ f = f_d + \dots + f_0, \qquad \deg\rb{f_i} = i, \qquad f_d \ne 0, $$
the decomposition of $ f $ into its homogeneous polynomial parts, so that
$$ P\rb{x_0, x_1, x_2} = f_d\rb{x_0, x_1} + \dots + f_0\rb{x_0, x_1}x_2^d = \sum_{i = 0}^d x_2^{d - i}f_i\rb{x_0, x_1}. $$
Then, if $ \rb{\alpha_i, \beta_i} \in \C^2 \setminus \cb{\underline{0}} $ for $ i = 1, \dots, d $ are such that
$$ f_d\rb{x_0, x_1} = \prod_{i = 1}^d \rb{\alpha_ix_0 + \beta_ix_1}, $$
the affine lines $ \cb{\alpha_ix + \beta_iy = 0} \subseteq \C^2 $ are the \textbf{asymptotes} of $ C $. The points at infinity at $ x_2 = 0 $ of $ C $ depend on the equation $ f_d\rb{x_0, x_1} = 0 $ and are
$$ \overline{C} \setminus C = \cb{\sb{-\beta_i, \alpha_i}} \subseteq \mathcal{P}. $$
\end{definition}

Each asymptote meets $ \overline{C} $ at the point $ \sb{-\beta_i, \alpha_i} \in \P^1 $, the line at infinity. (TODO Exercise: check this)

\begin{remark}
It is clear from Lemma \ref{lem:6.2} that asymptotes are well-defined. Note that the asymptotes are lines in $ \C^2 $ because $ \rb{\alpha_i, \beta_i} \ne \rb{0, 0} $.
\end{remark}

\begin{example}
Consider the affine curve
$$ C = \cb{\rb{x, y} \in \C^2 \mid x^2 + y^2 - 1 = 0}. $$
The projectivisation of $ C $ is
$$ \overline{C} = \cb{\sb{x_0, x_1, x_2} \in \P^2 \mid x_0^2 + x_1^2 - x_2^2}. $$
so that the points at infinity of $ C $ are given by
$$ \overline{C} \setminus C = \cb{\sb{x_0, x_1, 0} \in \P^2 \mid x_0^2 + x_1^2 = 0}. $$
This shows that $ C $ has two asymptotes, namely $ L_\pm = \cb{x \pm iy = 0} $, and two points at infinity $ \cb{\sb{\pm i, 1, 0}} $.
\end{example}

\begin{exercise}
Find the asymptotes of the affine curve given by the equation $ x^2y + y^3 + xy + x + 1 = 0 $.
\end{exercise}

\begin{exercise}
Write an example of an affine curve whose asymptotes include the lines $ x + 2y = 0 $, $ 4x - 3y = 0 $, and $ 7x + 9y = 0 $.
\end{exercise}

\marginpar{Lecture 8 \\ Thursday \\ 25/10/18}

\section{Smoothness and singularities}

Let $ C \subseteq \C^2, \P^2 $ be an affine or projective algebraic curve. Topologically, $ C $ is a two-dimensional object.

\begin{example}
A line $ C \subseteq \C^2 $ is a real plane $ \C \cong \R^2 $. $ \P^1 \cong S^2 $ is a two-dimensional sphere.
\end{example}

Smooth or non-singular points of $ C $ are points where you do not have angles, pinches, etc.

\begin{example}
The union of two projective lines $ \P^1 $ in $ \P^2 $ has a non-smooth point. A pinched torus has a non-smooth point.
\end{example}

Idea is that can make sense of the tangent line at smooth points.

\begin{notation}
Given a function $ f\rb{x_1, \dots, x_n} $, we will denote by $ f_{x_i} $ or $ \partial_{x_i}f $ the partial derivative of $ f $ with respect to $ x_i $.
\end{notation}

Let $ C = \cb{f = 0} \subseteq \C^2 $ be an affine algebraic curve for $ f \in \C\sb{x, y} $ and let $ \rb{a, b} \in C $. Then the \textbf{tangent line} of $ C $ at the point $ \rb{a, b} \in C $ is the line defined by the equation
$$ f_x\rb{a, b}\rb{x - a} + f_y\rb{a, b}\rb{y - b} = 0. $$
Note that this equation defines a complex line if and only if either $ f_x\rb{a, b} $ or $ f_y\rb{a, b} $ or both is not zero.

\begin{definition}
The affine curve $ C $ is \textbf{smooth} at a point $ \rb{a, b} \in C $ if at least one of $ f_x\rb{a, b} $ and $ f_y\rb{a, b} $ is not zero, so that the above equation defines a line in $ \C^2 $. On the other hand if $ f_x\rb{a, b} = f_y\rb{a, b} = 0 $, then $ \rb{a, b} $ is called a \textbf{singular} point of $ C $. We will simply say that $ C $ is smooth if $ C $ is smooth at every point.
\end{definition}

\begin{example}
\hfill
\begin{enumerate}
\item Any line in $ \C^2 $ is smooth.
\item $ x^2 + y^2 + 1 = 0 $ is smooth.
\item $ y^2 - x^2\rb{x + 1} = 0 $ is singular at one point, a node.
\item $ y^2 - x^3 = 0 $ is singular at one point, a cusp.
\end{enumerate}
\end{example}

What about projective curves?

\begin{theorem}[Euler relation]
\label{thm:7.3}
Let $ P \in \C\sb{x_0, x_1, x_2} $ be a homogeneous polynomial of degree $ d $. Then the following relation holds at each point $ \sb{x_0, x_1, x_2} \in \P^2 $.
$$ x_0 \cdot P_{x_0} + x_1 \cdot P_{x_1} + x_2 \cdot P_{x_2} = d \cdot P. $$
\end{theorem}

\begin{example}
Let $ P\rb{x_0, x_1, x_2} = x_0^2 + x_1^2 + x_2^2 $.
$$ P_{x_0} = 2x_0, \qquad P_{x_1} = 2x_1, \qquad P_{x_2} = 2x_2 \qquad \implies \qquad x_0 \cdot P_{x_0} + x_1 \cdot P_{x_1} + x_2 \cdot P_{x_2} = 2\rb{x_0^2 + x_1^2 + x_2^2} = d \cdot P. $$
\end{example}

\begin{proof}
By Lemma \ref{lem:4.4}, for any $ \lambda \in \C $ we have
$$ P\rb{\lambda x_0, \lambda x_1, \lambda x_2} = \lambda^d \cdot P\rb{x_0, x_1, x_2}. $$
We now compute the derivative with respect to $ \lambda $ of both sides of this equation,
$$ \partial_\lambda\rb{\lambda^d \cdot P\rb{x_0, x_1, x_2}} = d\lambda^{d - 1}P\rb{x_0, x_1, x_2}, \qquad \partial_\lambda\rb{P\rb{\lambda x_0, \lambda x_1, \lambda x_2}} = \sum_{i = 0}^2 x_iP_{x_i}\rb{\lambda x_0, \lambda x_1, \lambda x_2}. $$
By the equality above we have
$$ \sum_{i = 0}^2 x_iP_{x_i}\rb{\lambda x_0, \lambda x_1, \lambda x_2} = d\lambda^{d - 1}P\rb{x_0, x_1, x_2}, $$
for all $ \lambda \in \C $. Thus, if we plug in $ \lambda = 1 $, we get the claim.
\end{proof}

\begin{definition}
If a point $ p = \sb{z_0, z_1, z_2} $ of a projective curve $ C = \cb{P = 0} \subseteq \P^2 $ for $ P \in \C\sb{x_0, x_1, x_2} $ homogeneous satisfies
$$ P_{x_i}\rb{z_0, z_1, z_2} = 0, \qquad i = 0, 1, 2, $$
then we will say that $ C $ is singular at $ p $. $ C $ is smooth at $ p $ if it is not singular. $ C $ is said to be smooth if it does not admit any singular point, so it is smooth at every point.
\end{definition}

\begin{lemma}
Let $ C = \cb{P = 0} $ be a projective curve for $ \C^2 \cong U_2 \subseteq \P^2 $ which does not contain the line $ \cb{x_2 = 0} $ and let $ C_0 = \cb{f = 0} \subseteq \C^2 $ be the associated affine curve, where $ f\rb{x, y} = P\rb{x, y, 1} $. Then $ \rb{a, b} $ is a singular point of $ C_0 $ if and only if $ \sb{a, b, 1} $ is a singular point of $ C $.
\end{lemma}

\begin{proof}
If $ \rb{a, b} \in C_0 $ is a singular point of $ C_0 $, it is a singular point for $ C $. Note that $ P\rb{x, y, 1} = f\rb{x, y} $ by construction, so $ P_{x_0}\rb{x, y, 1} = f_x\rb{x, y} $, $ P_{x_1}\rb{x, y, 1} = f_y\rb{x, y} $. Then, since $ f\rb{a, b} = 0 $ we have
$$ P\rb{a, b, 1} = 0. $$
Since $ f_x\rb{a, b} = f_y\rb{a, b} = 0 $, we have
$$ P_{x_0}\rb{a, b, 1} = P_{x_1}\rb{a, b, 1} = 0. $$
Finally by Theorem \ref{thm:7.3}, we have
$$ P_{x_2}\rb{a, b, 1} = x_0 \cdot P_{x_0}\rb{a, b, 1} + x_1 \cdot P_{x_1}\rb{a, b, 1} + x_2 \cdot P_{x_2}\rb{a, b, 1} = dP\rb{a, b, 1} = 0, $$
since everything is zero except $ P_{x_2}\rb{a, b, 1} $. Thus, we have
$$ P_{x_i}\rb{a, b, 1} = 0, \qquad i = 0, 1, 2. $$
The converse implication is similar.
\end{proof}

\begin{exercise}
Show that
\begin{enumerate}
\item $ C = \cb{x_0^2 + x_1^2 + x_2^2 = 0} $ is smooth,
\item $ C = \cb{x_1^2x_0^2 - x_2^3 = 0} $ is singular at the point $ \sb{1, 0, 0} $, and
\item any projective line is smooth.
\end{enumerate}
\end{exercise}

\begin{definition}
Let $ p = \sb{z_0, z_1, z_2} $ be a smooth point of the projective curve $ C = \cb{P = 0} $. The \textbf{projective tangent line} of the curve $ C = \cb{P = 0} $ at the point $ p $ is given by the equation
$$ \sum_{i = 0}^2 P_{x_i}\rb{z_1, z_2, z_3} \cdot x_i = 0. $$
\end{definition}

Note that the line is well-defined since the curve $ C $ is smooth at the point $ p $.

\begin{proposition}
Let $ C \subseteq \P^2 $ be a projective algebraic curve not containing the line $ \cb{x_2 = 0} $. Then the affine line in $ U_2 $ associated to the projective tangent line at $ \sb{a, b, 1} \in C $ is the tangent line at $ \rb{a, b} $ of the affine curve $ C_0 $ associated to $ C $ in $ U_2 $, for $ \rb{a, b} $ a smooth point in $ C_0 $.
\end{proposition}

Equivalently, the projectivisation of the tangent line to $ C_0 \subseteq \C^2 $ at $ \rb{a, b} \in C_0 $ is the projective tangent line to the projectivisation $ C $ of $ C_0 $ at $ \sb{a, b, 1} $.

\begin{proof}
Assume $ C = \cb{P = 0} \subseteq \P^2 $ does not contain the line $ \cb{x_2 = 0} $. Let $ f\rb{x, y} = P\rb{x, y, 1} $ and let $ C_0 = \cb{f = 0} $. The affine line associated to the projective tangent line of $ C $ at the point $ \rb{a, b} \in C_0 $ is given by the equation
$$ P_{x_0}\rb{a, b, 1} \cdot x_0 + P_{x_1}\rb{a, b, 1} \cdot x_1 + P_{x_2}\rb{a, b, 1} \cdot x_2 = 0. $$
Since $ P\rb{a, b, 1} = 0 $, by Theorem \ref{thm:7.3} we have
$$ P_{x_2}\rb{a, b, 1} = -a \cdot P_{x_0}\rb{a, b, 1} - b \cdot P_{x_1}\rb{a, b, 1}. $$
Moreover, as above we have $ f_x\rb{a, b} = P_{x_0}\rb{a, b, 1} $ and $ f_y\rb{a, b} = P_{x_1}\rb{a, b, 1} $. Combining everything together we get
$$ f_x\rb{a, b}\rb{x - a} + f_y\rb{a, b}\rb{y - b} = 0, $$
which is the equation of the tangent line of $ C_0 $ at $ \rb{a, b} $.
\end{proof}

\begin{theorem}
\label{thm:7.9}
Let $ C $ be a smooth projective curve. Then $ C $ is irreducible.
\end{theorem}

Assume that any two projective curves in $ \P^2 $ intersect in at least one point, so there exists $ p \in C_1 \cap C_2 $. Later on we will prove this.

\begin{proof}
Suppose not. Then $ \cb{P = 0} = C = C_1 \cup C_2 $ where $ C_i = \cb{P_i = 0} $ is a projective curve for $ i = 1, 2 $. In particular, $ C = \cb{P = Q \cdot R = 0} $, where $ Q $ and $ R $ defines $ C_1 $ and $ C_2 $ respectively. We want to show that $ A = \cb{a, b, c} \in C_1 \cap C_2 \ne 0 $ is a singular point for $ C $, which contradicts the assumption on $ C $. We have
$$ P_{x_i}\rb{A} = \rb{Q \cdot R}_{x_i}\rb{A} = Q\rb{A} \cdot R_{x_i}\rb{A} + R\rb{A} \cdot Q_{x_i}\rb{A} = 0, \qquad i = 1, 2. $$
The last equality follows from the fact that $ Q\rb{A} = R\rb{A} = 0 $. Thus $ C $ is singular at $ A $ as claimed, a contradiction.
\end{proof}

\marginpar{Lecture 9 \\ Friday \\ 26/10/18}

\begin{example}
Example of a reducible curve, where the two pieces have the same tangent line where they intersect. $ C_1 = \cb{y - x^2} $ and $ C_2 = \cb{y + x^2} $ has $ \rb{0, 0} $ as the only point of intersection. Tangent line at $ \rb{0, 0} $ to both $ C_1 $ and $ C_2 $ is $ y = 0 $. The definition gives that $ C = C_1 \cup C_2 = \cb{\rb{y - x^2}\rb{y + x^2} = 0} = \cb{y^2 - x^4 = 0} $ is not smooth at $ \rb{0, 0} $. Note that the tangent line at $ \rb{0, 0} \in C $ for some affine curve $ C $ is given by the linear part of $ f = 0 $ where $ \cb{f = 0} = C $. Equation $ f_x\rb{0, 0}\rb{x - 0} + f_y\rb{0, 0}\rb{y - 0} = 0 $ is a linear approximation of $ C $ around $ \rb{0, 0} $. Now $ y^2 - x^4 = 0 $ does not have a linear part. The best possible approximation is a double line $ y^2 = 0 $, the term of lowest degree of $ f $. Thus $ \rb{0, 0} \in \cb{y^2 - x^4 = 0} $ should not be a smooth point.
\end{example}

\begin{remark}
When talking about smoothness, take $ f $ for $ C = \cb{f = 0} $ without repeated factors. Otherwise if $ f = g^2 \cdot h $ for $ g $ non-constant, then $ \cb{f = 0} $ would be singular along $ \cb{g = 0} $. We do not want this.
\end{remark}

\begin{theorem}
\label{thm:7.11}
Let $ C $ be an irreducible projective curve of degree $ d $. Then the number of singular points of $ C $ is at most
$$ d \cdot \rb{d - 1}. $$
In particular it is finite.
\end{theorem}

We will prove this later on, after seeing some results on intersections of curves.

\begin{exercise}
For which values of $ \lambda \in \C $ does the algebraic curve
$$ y^2 - x\rb{x - 1}\rb{x - \lambda} = 0 $$
admit at least one singular point? For each of those values of $ \lambda $, what are the singular points?
\end{exercise}

\section{Projective transformations}

How do we move around points or change coordinates in $ \P^n $? In linear algebra, that is in $ \C^n $, we do that with linear maps $ L : \C^n \to \C^n $, that is functions such that $ L\rb{\lambda v + \mu w} = \lambda L\rb{v} + \mu L\rb{w} $ for all $ v, w \in \C^n $ and $ \lambda, \mu \in \C $. These correspond to $ n \times n $ matrices $ A = \rb{a_{ij}} $ with coefficients in $ \C $. Given such a matrix, we can define a linear function $ L : \C^n \to \C^n $, by
$$ L\rb{x_1, \dots, x_n} = A \cdot \begin{pmatrix} x_1 \\ \vdots \\ x_n \end{pmatrix} = \begin{pmatrix} \sum_{j = 1}^n a_{ij}x_j \\ \vdots \\ \sum_{j = 1}^n a_{nj}x_j \end{pmatrix}, $$
and a linear function $ L $ gives a matrix by setting $ a_{ij} $ as the $ i $-th coefficient of the vector $ L\rb{e_j} \in \C^n $. How about in $ \P^n = \C^{n + 1} \setminus \cb{\underline{0}} / \sim $? Use matrices or linear transformations in $ \C^{n + 1} $.

\begin{lemma}
\label{lem:8.1}
Assume that $ A $ is an $ \rb{n + 1} \times \rb{n + 1} $ matrix with coefficients in $ \C $ such that
$$ \det\rb{A} \ne 0, $$
if and only if it is invertible. Then the function $ \Phi : \P^n \to \P^n $, defined by
$$ \Phi\rb{\sb{x_1, \dots, x_n}} = \sb{A \cdot \begin{pmatrix} x_0 \\ \vdots \\ x_n \end{pmatrix}}, $$
is well-defined and a bijection.
\end{lemma}

\begin{proof}
Since $ \det\rb{A} \ne 0 $, we know from linear algebra that if
$$ A \cdot \begin{pmatrix} x_0 \\ \vdots \\ x_n \end{pmatrix} = 0, $$
then $ x = \rb{x_0, \dots, x_n} = \underline{0} $. If $ A $ is not invertible there is no associated $ \Phi $. Thus if $ x \in W = \C^{n + 1} \setminus \cb{\underline{0}} $, then $ A \cdot x \in W $. Moreover, if $ x, y \in W $ are such that $ \sb{x} = \sb{y} $, then there exists a non-zero $ \lambda \in \C $ such that $ x = \lambda y $ and
$$ A \cdot x = \lambda A \cdot y, $$
which implies that $ \sb{A \cdot x} = \sb{A \cdot y} $. It follows that $ \Phi $ is well-defined. It is a bijection, with inverse given by the same kind of transformation obtained from the inverse matrix $ A^{-1} $.
\end{proof}

\begin{definition}
Any such function like $ \Phi $ in Lemma \ref{lem:8.1} is called a \textbf{projective transformation}.
\end{definition}

\begin{remark}
Note that an arbitrary $ \rb{n + 1} \times \rb{n + 1} $ matrix $ A $ does not define a well-defined function as above, because there might be non-zero vectors $ v \in \C^{n + 1} $ such that $ A \cdot v = \underline{0} $, and thus $ \sb{A \cdot v} $ would not define a point of projective space.
\end{remark}

\begin{example}
You probably have encountered Möbius transformations in complex analysis. They are rational functions $ f : \P^1 \to \P^1 $ of one complex variable $ z $
$$ f : z \mapsto \dfrac{az + b}{cz + d}, \qquad a, b, c, d \in \C, \qquad ad - bc \ne 0. $$
In terms of geometry, these are the projective transformations of $ \P^1 $ to itself. Indeed, if $ \sb{x_0, x_1} $ are the homogeneous coordinates on $ \P^1 $, a projective transformation $ \phi $ associated to a $ 2 \times 2 $ invertible matrix $ T $ is given by
$$ \Phi\rb{\sb{x_0, x_1}} = \sb{A \cdot \rb{x_0, x_1}} = \sb{ax_0 + bx_1, cx_0 + dx_1}, \qquad T = \two{a & b}{c & d}, \qquad ad - bc \ne 0. $$
If, as in Lemma \ref{lem:3.13}, we write $ \P^1 = \C \cup \cb{\infty} $, where $ \cb{\infty} = \cb{x_1 = 0} = \cb{\sb{1, 0}} $, then if $ x_1 \ne 0 $, $ \sb{x_0, x_1} = \sb{x_0 / x_1, 1} = \sb{z, 1} $ for $ z = x_0 / x_1 $ and
$$ \Phi\rb{\sb{z, 1}} = \sb{ax_0 + bx_1, cx_0 + dx_1} = \sb{az + b, cz + d}, $$
and if $ cz + d \ne 0 $, that is $ \sb{z, 1} \ne \sb{-d, c} $, this is
$$ \Phi\rb{\sb{z, 1}} = \sb{\dfrac{az + b}{cz + d}, 1} = \sb{f\rb{z}, 1}. $$
Here, unlike in complex analysis, the point $ \cb{\infty} = \sb{1, 0} $ plays no special role, and we see right away that $ \Phi\rb{\sb{1, 0}} = \sb{a, c} $ and $ \Phi\rb{\sb{-d, c}} = \sb{1, 0} $.
\end{example}

\begin{theorem}
\label{thm:8.5}
Assume that $ P_1, P_2, P_3 $ are three non-collinear points in $ \P^2 $, that is not on the same line. Then there exists a projective transformation
$$ \Phi : \P^2 \to \P^2 $$
such that
$$ \Phi\rb{P_1} = \sb{1, 0, 0}, \qquad \Phi\rb{P_2} = \sb{0, 1, 0}, \qquad \Phi\rb{P_3} = \sb{0, 0, 1}. $$
\end{theorem}

\begin{remark}
Every projective transformation $ \Phi = \sb{A} $ is invertible. Just take $ \Phi^{-1} = \sb{A^{-1}} $.
\end{remark}

\begin{proof}
Let $ P_i = \rb{a_{1i}, a_{2i}, a_{3i}} \in \C^3 $ and $ A = \rb{a_{ij}} $. Then, since $ P_1, P_2, P_3 $ are not collinear, it follows that $ \det\rb{A} \ne 0 $ and $ A $ is invertible. Let $ B $ be the inverse matrix of $ A $ and let $ \Phi = \sb{B} : \P^2 \to \P^2 $ be the projective transformation associated to the matrix $ B $. Then
$$ BP_i = e_i \qquad \implies \qquad \Phi\rb{P_i} = \sb{e_i}, $$
as desired.
\end{proof}

\begin{exercise}
When do two linear transformations $ L, L' : \C^{n + 1} \to \C^{n + 1} $ give rise to the same projective transformation $ \P^n \to \P^n $?
\end{exercise}

Can do better.

\begin{exercise}
Show that if $ P_1, P_2, P_3 \in \P^1_\C $ are distinct points, there exists a unique projective transformation $ \Psi : \P_1 \to \P_1 $ such that
$$ \Psi\rb{P_1} = \sb{1, 0}, \qquad \Psi\rb{P_2} = \sb{0, 1}, \qquad \Psi\rb{P_3} = \sb{1, 1}. $$
\end{exercise}

\begin{exercise}
Assume that $ P_1, P_2, P_3, P_4 \in \P^2_\C $ are four points such that no three of them lie in the same line. Show that there exists a unique projective transformation $ \Psi : \P^2 \to \P^2 $ such that
$$ \Psi\rb{P_1} = \sb{1, 0, 0}, \qquad \Psi\rb{P_2} = \sb{0, 1, 0}, \qquad \Psi\rb{P_3} = \sb{0, 0, 1}, \qquad \Psi\rb{P_4} = \sb{1, 1, 1}. $$
Take a $ \Phi $ as in Theorem \ref{thm:8.5}. I can scale the representatives of the $ P_i $, so $ \sb{v_i} = P_i $. Choose $ v_4 $ such that $ P_4 = \sb{v_4} $ and $ v_4 = \sum_{i = 1}^3 \lambda_i v_i = \lambda_1 \cdot v_1 + \lambda_2 \cdot v_2 + \lambda_3 \cdot v_3 $ for $ \lambda_i \in \C^* $. We are in $ \C^2 $, and the $ v_1, v_2, v_3 $ are linearly independent, so they are a basis. If $ \lambda_i = 0 $ for some $ i $, the remaining vectors are in the same two-dimensional subspace of $ \C $, contradicting non-collinearity in $ \P^2 $. Take $ \lambda_i \cdot v_i $ as representatives for $ P_i $ for $ i = 1, 2, 3 $ and $ v_4 $ for $ P_4 $. Get matrix $ A $ as before, so $ A \cdot e_i = P_i $ and $ A \cdot \three{1}{1}{1} = \sum_{i = 1}^3 \lambda_iv_i = v_4 $, so $ B = A^{-1} $, $ \Phi = \sb{B} $ and
$$ \Phi\rb{P_i} = \sb{e_i}, \qquad i = 1, 2, 3, \qquad \Phi\rb{P_4} = \sb{1, 1, 1}. $$
\end{exercise}

There is a version of the previous statement in higher dimensions.

\begin{definition}
A \textbf{hyperplane} in $ \P^n $ is the image of a subspace of dimension $ n $ in $ \C^{n + 1} $ via the map $ \Pi : \C^{n + 1} \setminus \cb{\underline{0}} \to \P^n $. Equivalently, it is the locus of points $ \sb{x_0, \dots, x_n} $ of $ \P^n $ satisfying some linear equation $ \sum_{i = 1}^n a_ix_i = 0 $, where $ \rb{a_0, \dots, a_n} \ne \underline{0} $.
\end{definition}

\begin{example}
Projective lines are precisely the hyperplanes of $ \P^2 $.
\end{example}

\begin{exercise}
Let $ \mathcal{S} = \cb{p_0, \dots, p_n, q} \subseteq \P^n $ be a collection of $ n + 2 $ distinct points such that no $ n + 1 $ points of $ \mathcal{S} $ lie in a hyperplane. Denote by $ e_i $, for $ i = 0, \dots, n $, the vectors of the standard basis of $ \C^{n + 1} $. Then there is a unique projective transformation $ f : \P^n \to \P^n $ such that
$$ f\rb{p_i} = \sb{e_i} = \sb{0, \dots, 0, 1, 0, \dots, 0}, \qquad i = 0, \dots, n, \qquad f\rb{q} = \sb{1, \dots, 1}. $$
Sometimes such a set of points is called a \textbf{projective basis}, in analogy with bases of vector spaces.
\end{exercise}

\begin{exercise}
Let $ L $ be a line in $ \P^2 $. Then given $ i \in \cb{0, 1, 2} $, there exists a projective transformation $ f : \P^2 \to \P^2 $ such that $ f\rb{L} = \mathcal{P}_i = \cb{x_i = 0} $.
\end{exercise}

\begin{exercise}
\label{ex:29}
Let $ C = \cb{P = 0} \subseteq \P^2 $ be a projective curve and let $ \Phi = \sb{A} : \P^2 \to \P^2 $ be a projective transformation, a $ 3 \times 3 $ matrix $ A $. Show that
\begin{enumerate}
\item $ \Phi\rb{C} $ is again a projective curve of degree $ \deg\rb{\Phi\rb{C}} = \deg\rb{C} $, in particular $ C $ is a line if and only if $ \Phi\rb{C} $ is a line,
\item $ \Phi\rb{C} $ is irreducible if and only if $ C $ is, and there is a natural bijection between the sets of components of the two curves,
\item $ C $ is smooth if and only if $ \Phi\rb{C} $ is smooth, and
\item if $ p \in C $ is a smooth point and $ l $ is the tangent line of $ C $ at $ p $ then $ \Phi\rb{C} $ is smooth at the point $ \Phi\rb{p} $ and $ \Phi\rb{l} $ is the tangent line of $ \Phi\rb{C} $ at $ \Phi\rb{p} $.
\end{enumerate}
What is the equation of $ \Phi\rb{C} $?
\end{exercise}

\begin{remark}
As Exercise \ref{ex:29} suggests, one should think of curves only differing from each other via a projective transformation to be the same curve, since projective transformations preserve their properties. These transformations should be thought of as a change of projective coordinate system.
\end{remark}

\marginpar{Lecture 10 \\ Monday \\ 29/10/18}

\section{Resultants and weak Bézout}

Now we start studying how projective curves intersect in the projective plane, heading towards Bézout's theorem. Let $ C = \cb{P = 0} $ and $ D = \cb{Q = 0} $ be two projective curves of degree $ n $ and $ m $ defined by homogeneous polynomials $ P, Q \in \C\sb{x_0, x_1, x_2} $.

\begin{theorem}[Bézout's theorem]
$ C \cap D $ has $ n \cdot m $ points, unless $ C $ and $ D $ have a common component, and count points with multiplicity.
\end{theorem}

\begin{example}
If you do not use multiplicity, there might be less than $ n \cdot m $ points.
\begin{enumerate}
\item Tangencies. Let $ C = \cb{y = x^2} $ and $ D = \cb{y = 0} $. In $ \P^2 $, they intersect in $ 1 \ne \rb{2}\rb{1} = \deg\rb{C}\deg\rb{D} $ point in this case.
\item Singularities of $ C $ and $ D $. Let $ C $ be two lines and $ D $ be a line passing through the intersection of $ C $. $ 1 \ne \rb{2}\rb{1} = \deg\rb{C}\deg\rb{D} $ point of intersection.
\end{enumerate}
Think of perturbing the equations.
\end{example}

We first have to develop some algebraic tools, the resultant. We want to study $ C \cap D = \cb{P = 0} \cap \cb{Q = 0} $, where $ P, Q $ no repeated factors, that is points $ \sb{x_0, x_1, x_2} $ that are solutions up to scaling of the equations for $ C $ and $ D $
$$ P\rb{x_0, x_1, x_2} = 0, \qquad Q\rb{x_0, x_1, x_2} = 0, $$
where $ \rb{x_0, x_1, x_2} \ne \rb{0, 0, 0} $. We first take a step back to the case of polynomials in one variable and study solutions of $ p\rb{x} = q\rb{x} = 0 $ for polynomials $ p, q \in \C\sb{x} $. Let $ p, q $ be
$$ p\rb{x} = a_0 + \dots + a_nx^n = \sum_{i = 0}^n a_ix^i, \qquad q\rb{x} = b_0 + \dots + b_mx^m = \sum_{i = 0}^m b_ix^i, \qquad a_n, b_m \ne 0, $$
for $ a_i, b_i \in \C $, and $ \deg\rb{P} = n $ and $ \deg\rb{Q} = m $ . When does $ p\rb{x} = q\rb{x} = 0 $ have solutions? Transform this question into linear algebra. The polynomials $ p $ and $ q $ have a common root, that is there is $ \lambda \in \C $ such that $ p\rb{\lambda} = q\rb{\lambda} = 0 $, precisely if there is a non-constant polynomial $ l \in \C\sb{x} $ such that
$$ p\rb{x} = l\rb{x}r\rb{x}, \qquad q\rb{x} = l\rb{x}s\rb{x}, $$
where $ r $ and $ s $ are non-zero polynomials, which thus have $ \deg\rb{r} \le n - 1 $ and $ \deg\rb{s} \le m - 1 $.

\begin{lemma}
\label{lem:9.1}
The polynomials $ p $ and $ q $ have a common root if and only if there are non-zero polynomials $ r, s $ with $ \deg\rb{r} \le n - 1 $ and $ \deg\rb{s} \le m - 1 $ with
\begin{equation}
\label{eq:3}
p\rb{x}s\rb{x} - q\rb{x}r\rb{x} = 0,
\end{equation}
the zero polynomial
\end{lemma}

\begin{proof}
By taking $ r $ and $ s $ in the previous discussion, it is clear that if $ p $ and $ q $ have a common root, then $ \rb{\ref{eq:3}} $ holds. Conversely assume that $ \rb{\ref{eq:3}} $ holds. There is a unique decomposition $ p\rb{x} = cp_1\rb{x}^{a_1} \dots p_k\rb{x}^{a_k} $, where $ c \in \C^* $, $ p_1, \dots, p_k $ are monic non-constant irreducible polynomials, and $ a_i \in \N^* $. Since $ p_1\rb{x}^{a_1} \dots p_k\rb{x}^{a_k} \mid q\rb{x}r\rb{x} $ and $ \deg\rb{r} \le n - 1 < \sum_{i = 1}^k \deg\rb{p_i} $, at least one of the irreducible polynomials $ p_i $ divides $ q\rb{x} $, by irreducibility. It follows that $ p $ and $ q $ have a common factor and, by the fundamental theorem of algebra, that $ p $ and $ q $ have a common root.
\end{proof}

We now show that $ \rb{\ref{eq:3}} $ is a system of $ m + n $ equations in $ m + n $ variables. Now write $ r\rb{x} $ and $ s\rb{x} $ in terms of unknown coefficients. Let $ \underline{v} = \rb{v_0, \dots, v_{n + m - 1}} \in \C^{n + m} $ be defined by
$$ s\rb{x} = v_0 + \dots + v_{m - 1}x^{m - 1} = \sum_{i = 0}^{m - 1} v_ix^i, \qquad -r\rb{x} = v_m + \dots + v_{n + m - 1}x^{n - 1} = \sum_{i = 0}^{n - 1} v_{m + i}x^i. $$
Substitute these into $ p\rb{x} \cdot s\rb{x} - q\rb{x} \cdot \rb{x} = 0 $. Get a big system of equations. Then $ \rb{\ref{eq:3}} $ is
$$ \rb{v_0a_0 + v_mb_0} + \dots + \rb{\sum_{k = 0}^i \rb{v_ka_{i - k} + v_{m + k}b_{i - k}}}x^i + \dots + \rb{v_{m - 1}a_n + v_{n + m - 1}b_m}x^{n + m - 1} = 0. $$
Since this is a polynomial of degree $ n + m - 1 $, we thus have a system of $ n + m $ equations in $ n + m $ variables of the form
\begin{equation}
\label{eq:4}
A \cdot v = 0,
\end{equation}
where $ A $ is an $ \rb{n + m} \times \rb{n + m} $ matrix
$$ A = \begin{pmatrix} a_0 & & 0 & b_0 & & 0 \\ \vdots & \ddots & & \vdots & \ddots & \\ a_n & & a_0 & b_m & & b_0 \\ & \ddots & \vdots & & \ddots & \vdots \\ 0 & & a_n & 0 & & b_m \end{pmatrix}, $$
where there are $ m $ columns of $ a_i $ and $ n $ columns of $ b_i $. From linear algebra, we know that $ \rb{\ref{eq:4}} $ has a non-trivial solution if and only if $ \det\rb{A} = 0 $. This motivates the following definition.

\begin{definition}
Let $ p, q \in \C\sb{x} $ be as above. Then the \textbf{resultant} of $ p $ and $ q $ is the determinant of the $ \rb{n + m} \times \rb{n + m} $ matrix $ A^T $,
$$ R_{f, g} = \det\rb{A^T} = \det\begin{pmatrix} a_0 & \dots & a_n & & 0 \\ & \ddots & & \ddots & \\ 0 & & a_0 & \dots & a_n \\ b_0 & \dots & b_m & & 0 \\ & \ddots & & \ddots & \\ 0 & & b_0 & \dots & b_m \end{pmatrix}. $$
\end{definition}

Thus, Lemma \ref{lem:9.1} can be reformulated as the following.

\begin{theorem}
\label{thm:9.3}
Let $ p, q \in \C\sb{x} $ be two non-zero polynomials. Then, $ p $ and $ q $ have a common root if and only if $ R_{p, q} = 0 $.
\end{theorem}

\begin{example}
\hfill
\begin{enumerate}
\item Let $ p\rb{x} = x^2 - 1 $ and $ q\rb{x} = x^2 + 2x + 1 $, so
$$ a_0 = 1, \qquad a_1 = 0, \qquad a_2 = -1, \qquad b_0 = 1, \qquad b_1 = 2, \qquad b_2 = 1. $$
Then the resultant of $ f $ and $ g $ is the $ 4 \times 4 $ matrix
$$ R_{f, g} = \det\begin{pmatrix} -1 & 0 & 1 & 0 \\ 0 & -1 & 0 & 1 \\ 1 & 2 & 1 & 0 \\ 0 & 1 & 2 & 1 \end{pmatrix}. $$
(TODO Exercise: check that the determinant is $ R_{f, g} = 0 $) Thus, $ f $ and $ g $ have a common solution $ x = -1 $.
\item Let $ n = 1 $ and $ m = 4 $, so $ p\rb{x} = a_0 + a_1x $ and $ q\rb{x} = b_0 + b_1x + b_2x^2 + b_3x^3 + b_4x^4 $. Then $ A $ is a $ 5 \times 5 $ matrix
$$ A = \begin{pmatrix} a_0 & 0 & 0 & 0 & b_0 \\ a_1 & a_0 & 0 & 0 & b_1 \\ 0 & a_1 & a_0 & 0 & b_2 \\ 0 & 0 & a_1 & a_0 & b_3 \\ 0 & 0 & 0 & a_1 & b_4 \end{pmatrix}. $$
\item Recall that $ p \in \C\sb{x} $ has a repeated root if and only if $ p $ and $ p' $ have a common root, that is by what precedes if and only if $ R_{p, p'} = 0 $. If $ p\rb{x} = ax^2 + bx + c $, where $ a \ne 0 $, then
$$ R_{p, p'} = \det\begin{pmatrix} c & b & a \\ b & 2a & 0 \\ 0 & b & 2a \end{pmatrix} = a\rb{4ac - b^2}. $$
We see that $ -R_{p, p'} = a \cdot \Delta $, where $ \Delta $ is the discriminant of the quadratic equation $ p\rb{x} = 0 $.
\item Let $ n = 2 $ and $ m = 3 $, so $ p\rb{x} = a_0 + a_1x + a_2x^2 $ and $ q\rb{x} = b_0 + b_1x + b_2x^2 + b_3x^3 $. Then $ A $ is a $ 5 \times 5 $ matrix
$$ A = \begin{pmatrix} a_0 & 0 & 0 & b_0 & 0 \\ a_1 & a_0 & 0 & b_1 & b_0 \\ a_2 & a_1 & a_0 & b_2 & b_1 \\ 0 & a_2 & a_1 & b_3 & b_2 \\ 0 & 0 & a_2 & 0 & b_3 \end{pmatrix}. $$
\end{enumerate}
\end{example}

\marginpar{Lecture 11 \\ Thursday \\ 01/11/18}

In general, the resultant is useful because it tends to be easier to compute the determinant of a matrix than to factorise polynomials in order to determine whether they have a common root. Assume now that $ P, Q \in \C\sb{x_0, x_1, x_2} $ are homogeneous polynomials of degrees $ n $ and $ m $ respectively. We write $ P $ and $ Q $ as polynomials in $ x_0 $, with coefficients in $ \C\sb{x_1, x_2} $.
$$ P\rb{x_0, x_1, x_2} = \sum_{i = 0}^n a_i\rb{x_1, x_2}x_0^i, \qquad Q\rb{x_0, x_1, x_2} = \sum_{j = 0}^m b_j\rb{x_1, x_2}x_0^j, $$
where $ a_i \in \C\sb{x_1, x_2} $ is a homogeneous polynomial of degree $ n - i $, and $ b_j \in \C\sb{x_1, x_2} $ is homogeneous of degree $ m - j $. In particular, $ a_n = P\rb{1, 0, 0} $ and $ b_m = Q\rb{1, 0, 0} $ are constants.

\begin{definition}
\label{def:9.5}
Assume that $ P\rb{1, 0, 0} \ne 0 $ and $ Q\rb{1, 0, 0} \ne 0 $. The resultant of $ P $ and $ Q $ is the determinant of the $ \rb{n + m} \times \rb{n + m} $ matrix
$$ R_{P, Q}\rb{x_1, x_2} = \det\begin{pmatrix} a_0\rb{x_1, x_2} & \dots & a_n\rb{x_1, x_2} & & 0 \\ & \ddots & & \ddots & \\ 0 & & a_0\rb{x_1, x_2} & \dots & a_n\rb{x_1, x_2} \\ b_0\rb{x_1, x_2} & \dots & b_m\rb{x_1, x_2} & & 0 \\ & \ddots & & \ddots & \\ 0 & & b_0\rb{x_1, x_2} & \dots & b_m\rb{x_1, x_2} \end{pmatrix}. $$
\end{definition}

The entries of the matrix above are homogeneous polynomials in $ \C\sb{x_1, x_2} $. If you specify $ x_1 = a $, $ x_2 = b $ for $ a, b \in \C $, then $ R_{P, Q} $ is the resultant of $ P\rb{x_0, a, b} $ and $ Q\rb{x_0, a, b} $. We admit the following theorem, which follows from commutative algebra.

\begin{theorem}
\label{thm:9.6}
Let $ P, Q \in \C\sb{x_0, x_1, x_2} $ be homogeneous polynomials of degrees $ n $ and $ m $, and assume that $ a_n, b_m \ne 0 $. Then $ P $ and $ Q $ have a common factor if and only if $ R_{P, Q} \equiv 0 $, that is if
$$ R_{P, Q}\rb{x_1, x_2} = 0, \qquad x_1, x_2 \in \C. $$
\end{theorem}

In fact, slightly more is true.

\begin{theorem}
\label{thm:9.7}
Let $ P, Q \in \C\sb{x_0, x_1, x_2} $ be homogeneous polynomials such that $ P\rb{1, 0, 0}, Q\rb{1, 0, 0} \ne 0 $. Then $ R_{P, Q} \in \C\sb{x_1, x_2} $ is a homogeneous polynomial of degree $ nm $.
\end{theorem}

\begin{proof}
Let $ A $ be the $ \rb{n + m} \times \rb{n + m} $ matrix above, that is such that $ R_{P, Q} = \det\rb{A} $. Recall from linear algebra that
$$ \det\rb{A} = \sum_{\sigma \in S_{n + m}} sign\rb{\sigma}\prod_{k = 1}^{n + m}A_{k, \sigma\rb{k}}. $$
The entries of $ A $ are either zero or are homogeneous polynomials, we thus only need to check that for each $ \sigma \in S_{n + m} $ such that $ A_{k, \sigma\rb{k}} \ne 0 $ for all $ k $, $ \prod_{k = 1}^{n + m} A_{k, \sigma\rb{k}} $ is of degree $ nm $. By definition of $ A $, the non-zero entries of $ A $ are
$$ A_{k, l} = \begin{cases} a_{l - k}\rb{x_1, x_2} & k \le m, \ 0 \le l - k \le n \\ b_{l - k + m}\rb{x_1, x_2} & k \ge m + 1, \ 0 \le l - k + m \le m \end{cases}. $$
$ \deg\rb{a_{l - k}} = n - \rb{l - k} $ and $ \deg\rb{b_{l - k + m}} = m - \rb{l - k + m} = k - l $, so that when non-zero, $ \prod_{k = 1}^{n + m} A_{k, \sigma\rb{k}} $ has degree
$$ \deg\rb{\prod_{k = 1}^{n + m} A_{k, \sigma\rb{k}}} = \sum_{k = 1}^m \rb{n - \rb{\sigma\rb{k} - k}} + \sum_{k = m + 1}^{m + n} \rb{k - \sigma\rb{k}} = nm + \sum_{k = 1}^{m + n} k - \sum_{k = 1}^{m + n} \sigma\rb{k}. $$
The last two terms are equal because $ \sigma $ is a permutation, so this is equal to $ nm $ for all permutations $ \sigma \in S_{m + n} $, and this proves that $ R_{P, Q} $ is a homogeneous polynomial of degree $ nm $.
\end{proof}

Recall that homogeneous polynomials in two variables always factor as a product of linear homogeneous polynomials. We can now prove a weak form of Bézout's theorem. We will refine the statement later on.

\begin{theorem}[Weak Bézout theorem]
\label{thm:9.8}
Let $ C, D \subseteq \P^2 $ be projective curves of degrees $ \deg\rb{C} = n $ and $ \deg\rb{D} = m $ respectively. Then
\begin{enumerate}
\item $ C $ and $ D $ intersect in at least one point, that is $ C \cap D $ is not empty, and
\item $ C $ and $ D $ have at most $ \#\cb{C \cap D} \le nm $ distinct points of intersection, if they do not have a common component.
\end{enumerate}
\end{theorem}

\begin{proof}
\hfill
\begin{enumerate}
\item Let $ P, Q \in \C\sb{x_0, x_1, x_2} $ be homogeneous polynomials of degrees $ n $ and $ m $ with no repeated factors that define $ C $ and $ D $, that is
$$ C = \cb{P = 0}, \qquad D = \cb{Q = 0}. $$
$ P\rb{1, 0, 0} \ne 0 $ and $ Q\rb{1, 0, 0} \ne 0 $ if and only if $ p = \sb{1, 0, 0} $ is not in $ C $ and not in $ D $. Use a projective transformation to ensure this. Let $ q \in \P^2 $ be a point that lies neither on $ C $ nor on $ D $, so not on $ C \cup D $. (TODO Exercise: such a point always exists) There is a projective transformation $ \Phi : \P^2 \to \P^2 $ that sends $ p $ to $ \sb{1, 0, 0} $, that is $ \Phi\rb{p} = \sb{1, 0, 0} $. Transform $ C $ and $ D $ using $ \Phi $, so replace them with $ \Phi\rb{C} $ and $ \Phi\rb{D} $. We have seen that $ \Phi\rb{C}, \Phi\rb{D} $ are projective curves of the same degrees $ \deg\rb{\Phi\rb{C}} = \deg\rb{C} $ and $ \deg\rb{\Phi\rb{D}} = \deg\rb{D} $, and that $ \#\cb{C \cap D} = \#\cb{\Phi\rb{C} \cap \Phi\rb{D}} $. Thus, after replacing $ C $ with $ \Phi\rb{C} $ and $ D $ with $ \Phi\rb{D} $ we may assume that $ \sb{1, 0, 0} \notin C \cup D $, so $ P\rb{1, 0, 0} \ne 0 $ and $ Q\rb{1, 0, 0} \ne 0 $. We assume that $ C $ and $ D $ have no common component, as the result is trivial if they do. Denote by $ R $ the resultant polynomial of $ P $ and $ Q $, $ R\rb{x_1, x_2} = R_{P, Q}\rb{x_1, x_2} $ for all $ x_1, x_2 \in \C $, and factor it. By Theorems \ref{thm:9.6} and \ref{thm:9.7}, $ R $ is a homogeneous polynomial of degree $ nm $ that is not identically zero, since $ C $ and $ D $ have no common component. By Lemma \ref{lem:6.2}, there are $ \rb{\alpha_i, \beta_i} \in \C^2 \setminus \cb{\rb{0, 0}} $ for $ i = 1, \dots, nm $ such that
$$ R\rb{x_1, x_2} = \prod_{i = 1}^{nm} \rb{\alpha_ix_1 + \beta_ix_2}. $$
Fix $ \rb{a, b} \in \C^2 \setminus \cb{\rb{0, 0}} $, and let $ p, q \in \C\sb{x} $ be the polynomials defined by
$$ p\rb{x} = P\rb{x, a, b}, \qquad q\rb{x} = Q\rb{x, a, b}. $$
Then, $ R\rb{a, b} = R_{p, q} $. In particular, if $ \rb{a, b} = \rb{\beta_1, \alpha_1} $, $ R\rb{-\beta_1, \alpha_1} = 0 $, and by Theorem \ref{thm:9.3}, there is $ \lambda \in \C $ such that
$$ P\rb{\lambda, -\beta_1, \alpha_1} = p\rb{\lambda} = q\rb{\lambda} = Q\rb{\lambda, -\beta_1, \alpha_1} = 0, $$
so $ p\rb{x_0}, q\rb{x_0} $ have a common root $ \lambda $. Since $ \rb{-\beta_1, \alpha_1} \ne \rb{0, 0} $, $ \rb{\lambda, \beta_1, \alpha_1} \ne \rb{0, 0, 0} $, so that $ p = \sb{\lambda, -\beta_1, \alpha_1} $ is a well-defined point of $ \P^2 $ and is on both $ C $ and $ D $, so $ p \in C \cap D $.

\marginpar{Lecture 12 \\ Friday \\ 02/11/18}

\item We will show that if $ C \cap D $ contains at least $ nm + 1 $ distinct points, $ C $ and $ D $ have a common factor. Let $ \mathcal{S} = \cb{p_1, \dots, p_{nm + 1}} $ be any set of $ nm + 1 $ distinct points and denote by $ L_{i, j} $ the unique line containing $ p_i $ and $ p_j $ for $ 1 \le i, j \le nm + 1 $. For each $ i $, denote by $ \sb{x_0^i, x_1^i, x_2^i} $ the coordinates of $ p_i $. The point $ x = \sb{x_0, x_1, x_2} \in \P^2 $ belongs to $ L_{i, j} $ if and only if $ \det\rb{A} = 0 $, where
\begin{equation}
\label{eq:5}
A_{i, j} = \begin{pmatrix} x_0 & x_0^i & x_0^j \\ x_1 & x_1^i & x_1^j \\ x_2 & x_2^i & x_2^j \end{pmatrix}.
\end{equation}
Let $ p \in \P^2 $ be any point that lies neither on $ C, D $ nor on any $ L_{i, j} $, that is $ p \notin C \cup D \cup \rb{\cup_{i, j} L_{i, j}} $. Can apply a projective transformation $ \Phi : \P^2 \to \P^2 $ that sends $ p $ to $ \sb{1, 0, 0} $. Taking images along $ \Phi $, the image of $ \mathcal{S} $ is a set of $ nm + 1 $ distinct points and the image of $ L_{i, j} $ by $ \Phi $ is the unique line through $ \Phi\rb{p_i} $ and $ \Phi\rb{p_j} $. As in the proof of $ 1 $, after replacing $ C $, $ D $, and $ \mathcal{S} $ by their images by $ \Phi $, we may assume that $ \sb{1, 0, 0} \notin C \cup D $ and that $ \sb{1, 0, 0} $ lies on no line through two points $ p_i, p_j \notin \mathcal{S} $. In particular, by $ \rb{\ref{eq:5}} $,
$$ \det\begin{pmatrix} 1 & x_0^i & x_0^j \\ 0 & x_1^i & x_1^j \\ 0 & x_2^i & x_2^j \end{pmatrix} \ne 0, $$
that is $ \rb{x_1^i, x_2^i} \ne \rb{0, 0} $ for all $ i $, otherwise determinant is zero, so that $ \sb{x_1^i, x_2^i} $ is a well-defined point of $ \P^1 $, and $ \sb{x_1^i, x_2^i} \ne \sb{x_1^j, x_2^j} $ for all $ i \ne j $. Let $ P, Q \in \C\sb{x_0, x_1, x_2} $ be homogeneous polynomials of degrees $ n $ and $ m $ that define $ C $ and $ D $, that is
$$ C = \cb{P = 0}, \qquad D = \cb{Q = 0}. $$
Since $ \sb{1, 0, 0} \notin C \cup D $, $ P\rb{1, 0, 0} \ne 0 $ and $ Q\rb{1, 0 , 0} \ne 0 $. Denote by $ R $ the resultant $ R_{P, Q} $ of $ P, Q $. This is homogeneous of degree $ nm $ and factors into linear factors $ R\rb{x_1, x_2} = \prod_{i = 1}^{nm}\rb{\alpha_ix_1 + \beta_ix_2} $ for $ \rb{\alpha_i, \beta_i} \ne \rb{0, 0} $, so $ \cb{R = 0} \subseteq \P^1 $ has at most $ nm $ distinct elements. Now assume $ C \cap D $ contains at least $ nm + 1 $ points and let $ \mathcal{S} \subseteq C \cap D $ be a set of $ nm + 1 $ distinct points. Set $ \rb{\alpha_i, \beta_i} = \rb{-x_2^i, x_1^i} $ for $ i = 1, \dots, nm + 1 $. By construction, for all $ i $, $ p\rb{x} = P\rb{x_0, x_1^i, x_2^i} $ and $ q\rb{x} = Q\rb{x_0, x_1^i, x_2^i} $ have a common root so that $ R_{p, q} = 0 $. But we have
$$ R_{p, q} = R\rb{x_1^i, x_2^i}, $$
so that $ \rb{\alpha_ix_1 + \beta_ix_2} $ is a factor of the polynomial $ R\rb{x_1, x_2} $. Since $ \sb{x_1^i, x_2^i} \ne \sb{x_1^j, x_2^j} $ these factors are distinct, so $ \cb{\sb{x_1^i, x_2^i}} \subseteq \P^1 $ has at least $ nm + 1 $ distinct points and
\begin{equation}
\label{eq:6}
R\rb{x_1, x_2} = \prod_{i = 1}^{nm + 1} \rb{\alpha_ix_1 + \beta_ix_2}S\rb{x_1, x_2},
\end{equation}
for some possibly constant homogeneous polynomial $ S $. If $ R\rb{x_1, x_2} $ is not identically zero, by Theorem \ref{thm:9.7}, it has degree $ nm $. Thus, $ \rb{\ref{eq:6}} $ shows that $ R $ is identically zero, because the degree of the polynomial on the right hand side of $ \rb{\ref{eq:6}} $ is at least $ nm + 1 $ if this polynomial is not identically zero. Thus, by Theorem \ref{thm:9.6}, $ P $ and $ Q $ have a common factor and $ C $ and $ D $ have a common component.
\end{enumerate}
\end{proof}

This finishes the proof of Theorem \ref{thm:7.9}, and now we can also prove Theorem \ref{thm:7.11}, stated before.

\begin{proof}[Proof of Theorem \ref{thm:7.11}]
Use Bézout's theorem. Let $ C = \cb{P = 0} $ for $ P $ a polynomial of degree $ d $ defining $ C $. Singular points are solutions $ \sb{x_1, x_2, x_3} $ at
$$ P = 0, \qquad P_{x_0} = 0, \qquad P_{x_1} = 0, \qquad P_{x_2} = 0. $$
Without loss of generality we may assume that there exists $ i = 0, 1, 2 $ such that $ Q = P_{x_i} $ is not zero, otherwise $ P $ is constant. Thus $ Q $ is a homogeneous polynomial of degree $ d - 1 $. If $ p $ is a singular point of $ C $ then $ P\rb{p} = Q\rb{p} = 0 $. Clearly the opposite is not true. Let $ D = \cb{Q = 0} $. Then $ D $ is another projective curve of degree $ d - 1 $ and the intersection $ C \cap D $ contains all the singular points of $ C $. Moreover, since $ C $ is irreducible, it follows that $ C $ and $ D $ do not have any common component. By Theorem \ref{thm:9.8}, it follows that the number of intersection points of $ C $ and $ D $ is at most $ \deg\rb{C}\deg\rb{D} = d \cdot \rb{d - 1} $ and the claim follows.
\end{proof}

\begin{proposition}[Pascal's mystic hexagon]
Let $ C \subseteq \P^2 $ be an irreducible conic and let $ p_1, \dots, p_6 \in C $ be six distinct points that form a hexagon, that is no three of the points $ p_1, \dots, p_6 $ lie on a projective line. Then the points of intersection of lines passing through opposite sides are collinear.
\end{proposition}

\begin{proof}
Assume that $ p_1, \dots, p_6 $ are ordered such that the lines $ L_1, \dots, L_6 $, with $ L_i $ the unique line through $ p_i $ and $ p_{i + 1} $, form the sides of a hexagon. Denote by $ D_1 $ the cubic curve $ L_1 \cup L_3 \cup L_5 $ and by $ D_2 $ the cubic curve $ L_2 \cup L_4 \cup L_6 $. Both cubic curves are reducible by definition, and both contain the points $ p_1, \dots, p_6 $. Let $ q_1 = L_1 \cap L_4 $, $ q_2 = L_3 \cap L_6 $, and $ q_3 = L_5 \cap L_2 $ be the not necessarily distinct points of intersection of opposite sides. We want to prove that there is a projective line $ L $ through $ q_1, q_2, q_3 $. First note that if $ q_1, q_2, q_3 $ are not distinct, there is a projective line through $ q_1, q_2, q_3 $ by Exercise \ref{ex:18}. We therefore assume that $ q_1, q_2, q_3 $ are three distinct points. The points $ q_1, q_2, q_3 $ are not in the set $ \cb{p_1, \dots, p_6} $ because otherwise, one of the lines $ L_1, \dots, L_6 $ would have to contain at least three of the points $ p_1, \dots, p_6 $. Since $ C $ is irreducible, it contains none of the lines $ L_1, \dots, L_6 $, in particular, $ C $ has no common component with $ D_1 $ or with $ D_2 $. Since $ C \cap D_1 \supset \cb{p_1, \dots, p_6} $ and $ C \cap D_2 \supset \cb{p_1, \dots, p_6} $, and since $ C $ and $ D_1 $ and $ C $ and $ D_2 $ have no common component, their intersections consist of at most $ \rb{2}\rb{3} = 6 $ points and therefore $ C \cap D_1 = C \cap D_2 = \cb{p_1, \dots, p_6} $. This shows that the points $ q_1, q_2, q_3 $ do not lie on $ C $. I will construct another cubic, having seven points of intersection with $ C $. Let $ p_0 \in C $ be a point that does not belong to the set $ \cb{p_1, \dots, p_6} $. In particular, the point $ p_0 $ does not lie on $ D_1 $ or on $ D_2 $. Let $ P, Q $ be homogeneous polynomials that define the cubic curves $ D_1 $ and $ D_2 $, products of the equations of the lines, that is
$$ D_1 = \cb{P = 0}, \qquad D_2 = \cb{Q = 0}. $$
Since $ p_0 \notin D_1 \cup D_2 $, $ P\rb{p_0} \ne 0 $ and $ Q\rb{p_0} \ne 0 $, so that there exists $ \sb{\lambda, \mu} \in \P^1 $ such that $ \lambda P\rb{p_0} + \mu Q\rb{p_0} = 0 $, such as $ \sb{-Q\rb{p_0}, P\rb{p_0}} $. Consider the cubic curve
$$ D = \cb{\lambda P + \mu Q = 0} \subseteq \P^2, $$
where $ \lambda P + \mu Q $ is a homogeneous polynomial of degree three. The points $ p_1, \dots, p_6 $ lie on $ D_1 \cap D_2 $, hence they satisfy $ P\rb{p_i} = Q\rb{p_i} = 0 $ for all $ i $, so $ \rb{\lambda P + \mu Q}\rb{p_i} = 0 $ and $ p_1, \dots, p_6 \in D $. Since $ p_0 \in D $, the intersection $ C \cap D $ thus contains at least seven distinct points $ p_0, \dots, p_6 $. By Bézout's theorem, this implies that $ C $ and $ D $ have a common component, and since $ C $ is irreducible, that has to be $ C $ and $ C \subseteq D $. If $ Q $ is a homogeneous polynomial of degree two that defines $ C $, there is a homogeneous polynomial $ S $ of degree one such that
$$ D = \cb{Q \cdot S = 0} = C \cup \cb{S = 0}, $$
of which $ \cb{S = 0} $ has to be a line. Let $ L $ be the projective line $ L = \cb{S = 0} $ that I was looking for. Since the points $ q_1, q_2, q_3 \in D_1 \cap D_2 $, they also lie on $ D $. We have seen that $ q_1, q_2, q_3 \notin C $, thus, they lie on $ L $.
\end{proof}

\marginpar{Lecture 13 \\ Monday \\ 05/11/18}

Lecture 13 is a problem class.

\marginpar{Lecture 14 \\ Thursday \\ 08/11/18}

\section{Conics}

In $ \R^2 $ there are three types of conics, which are projective curves of degree two,
\begin{enumerate}
\item ellipse, which is compact and therefore it does not admit any point at infinity,
\item parabola, which admits a unique point at infinity, and
\item hyperbola, which admits two points at infinity.
\end{enumerate}
We have seen how looking at them in projective space eliminates the difference, but we still have examples with only one point, or without any points, such as $ x^2 + y^2 = 0 $ and $ x^2 + y^2 = -1 $. In $ \C^2 $, case 1 cannot happen because no affine curve is compact or, in other words, any affine curve has a point at infinity, and we know that in any case there are infinitely many points. Let $ C = \cb{P = 0} $ where $ P $ has degree two for $ P \in \C\sb{x_0, x_1, x_2} $. By Theorem \ref{thm:9.8}, the number of points at infinity in the line at infinity $ \cb{x_2 = 0} $ of any conic $ C $ is either one or two.

\begin{example}
\hfill
\begin{enumerate}
\item $ \sb{0, 1, 0} $ is the only point at infinity of $ \cb{x_2x_1 - x_0^2 = 0} $ since $ x_2 = 0 $ gives $ x_0^2 = 0 $.
\item $ \sb{1, \pm i, 0} $ are the two points at infinity of $ \cb{x_0^2 + x_1^2 + x_2^2 = 0} $.
\end{enumerate}
\end{example}

\begin{theorem}
\label{thm:10.2}
Let $ C $ be an irreducible conic, if and only if $ P $ is irreducible. Then there exists a projective transformation $ \Psi : \P^2 \to \P^2 $ such that $ \Psi\rb{C} $ has equation
$$ x_2^2 + x_1x_0 = 0. $$
\end{theorem}

\begin{proof}
By Theorem \ref{thm:7.11} there exists $ p \in C $ smooth point. After taking a projective transformation we may assume that $ p = \sb{0, 1, 0} $ and that the tangent line of $ C $ at the point $ p $ has equation
$$ x_0 = 0, $$
by Exercise \ref{ex:29}. Let $ C = \cb{P = 0} $ where $ P \in \C\sb{x_0, x_1, x_2} $ is a homogeneous polynomial of degree two. We may write
$$ P\rb{x_0, x_1, x_2} = ax_0^2 + bx_1^2 + cx_2^2 + dx_0x_1 + ex_0x_2 + fx_1x_2, $$
for some $ a, b, c, d, e, f \in \C $. The tangent line of $ C $ at $ \sb{0, 1, 0} $ is given by
$$ \sum_{i = 0}^2 x_i \cdot P_{x_i}\rb{0, 1, 0} = x_0 \cdot P_{x_0}\rb{0, 1, 0} + x_1 \cdot P_{x_1}\rb{0, 1, 0} + x_2 \cdot P_{x_2}\rb{0, 1, 0} = 0. $$
But we know that this line is $ x_0 = 0 $ and therefore
$$ P_{x_1}\rb{0, 1, 0} = P_{x_2}\rb{0, 1, 0} = 0. $$
Compute these, and see that they are $ b $ and $ f $, so it follows that $ b = f = 0 $ and
$$ P\rb{x_0, x_1, x_2} = ax_0^2 + cx_2^2 + dx_0x_1 + ex_0x_2 = cx_2^2 + x_0\rb{ax_0 + dx_1 + ex_2}. $$
Want to change coordinates to get $ x_2^2 + x_1x_0 = 0 $. Let
$$ A = \begin{pmatrix} 1 & 0 & 0 \\ a & d & e \\ 0 & 0 & \sqrt{c} \end{pmatrix}, $$
where $ \sqrt{c} $ is any of the two square roots. Check that $ A $ is invertible. Assume that $ \det\rb{A} = 0 $. Then either $ c = 0 $ or $ d = 0 $. If $ c = 0 $ then $ P = x_0\rb{ax_0 + dx_1 + ex_2} $ is not irreducible. If $ d = 0 $ then $ P = cx_2^2 + ax_0^2 + ex_0x_2 $, which can be factored, being a homogeneous polynomial in two variables. In both cases, $ P $ is not irreducible, which contradicts the assumption. Then $ \det\rb{A} \ne 0 $. Let $ \Psi : \P^2 \to \P^2 $ be the projective transformation associated to $ A $ by $ \Psi = \sb{A} $. Then if $ \sb{z_0, z_1, z_2} = \Psi\rb{\sb{x_0, x_1, x_2}} $ we have
$$ z_0 = x_0, \qquad z_1 = ax_0 + dx_1 + ex_2, \qquad z_2 = \sqrt{c}x_2, $$
and the equation of $ \Psi\rb{C} $ becomes
$$ z_2^2 + z_1z_0 = 0, $$
as claimed. (TODO Exercise: convince yourself that this is the same that you get by doing $ PA^{-1}\rb{z_0, z_1, z_2} $)
\end{proof}

\begin{corollary}
\label{cor:10.4}
Let $ C $ be a conic. Then $ C $ is smooth if and only if it is irreducible.
\end{corollary}

\begin{proof}
By Theorem \ref{thm:7.9}, if $ C $ is reducible then it is not smooth, so if $ C $ is smooth then it is irreducible. Assume now that $ C $ is irreducible. Then by Theorem \ref{thm:10.2}, after taking a projective transformation we may assume that $ C $ is given by the equation
$$ x_2^2 + x_0x_1 = 0, $$
which defines a smooth conic, by Exercise \ref{ex:29}.
\end{proof}

\begin{remark}
In general it is not true that if $ C $ is irreducible then $ C $ is smooth for higher degree curves.
\end{remark}

\begin{example}
Let $ C = \cb{x_2^3 - x_0x_1^2 = 0} \subseteq \P^2 $. Then $ C $ is singular at the point $ \sb{1, 0, 0} $ but $ C $ is irreducible otherwise $ C $ would contain a line. (TODO Exercise: check that $ C $ does not contain any line)
\end{example}

\begin{exercise}
\label{ex:30}
\hfill
\begin{enumerate}
\item Show that for every reducible conic $ C $ there exists a projective transformation $ \Psi : \P^2 \to \P^2 $ such that $ \Psi\rb{C} = \cb{x_0^2 + x_1^2 = 0} $.
\item Show that for any linear homogeneous polynomial $ L\rb{x_0, x_1, x_2} $, there exists a projective transformation $ \Psi : \P^2 \to \P^2 $ such that $ \Psi\rb{C} $ has equation $ x_0^2 = 0 $, where $ C = \cb{L^2 = 0} $.
\end{enumerate}
\end{exercise}

The outcome of thinking of the reducible case is the following. Up to projective transformation, $ C = \cb{f = 0} $ where $ f $ has degree two is one of
\begin{enumerate}
\item $ x_0^2 + x_1^2 + x_2^2 = 0 $, which is irreducible and smooth,
\item $ x_0^2 + x_1^2 = 0 $, which is a pair of distinct lines, or
\item $ x_0^2 = 0 $, which is a double line.
\end{enumerate}

\begin{remark}
There exists a natural bijection between $ \P^1 $ and the conic
$$ C = \cb{x_2^2 + x_1x_0 = 0} \subseteq \P^2, $$
defined by $ f : \P^1 \to C $, so that
$$ f\rb{\sb{z_0, z_1}} = \sb{z_0^2, -z_1^2, z_0z_1} \in \P^2, $$
which satisfies the equation of $ C $. (TODO Exercise: check that this is a bijection) Why? Points of $ C $ are in bijection with lines through $ p $, by looking at the other point of intersection of $ C $ with a line through $ p $. Lines through $ p $ are in bijection with $ \P^1 $, since lines through $ p $ have an equation as above, so lines through $ p $ are a line in $ \P^2 $. Thus, it follows by Theorem \ref{thm:10.2} that any irreducible conic admits a natural bijection with $ \P^1 $ and therefore with the sphere
$$ \cb{\rb{x, y, z} \in \R^3 \mid x^2 + y^2 + z^2 = 1} \subseteq \R^3. $$
Thus, the affine curve defined by the equation
$$ x^2 + y^2 = 1 $$
admits a bijection with the sphere minus two distinct points.
\end{remark}

\begin{exercise}
\label{ex:31}
Let $ C \subseteq \P^2 $ be a conic. Show that there exists a symmetric $ 3 \times 3 $ matrix $ A = \rb{a_{ij}} $ such that
$$ P\rb{z_0, z_1, z_2} = \sum_{i, j = 0}^2 a_{ij}z_iz_j $$
is the homogeneous polynomial which defines $ C $. Show that $ C $ is irreducible if and only if $ \det\rb{A} \ne 0 $.
\end{exercise}

\section{Multiplicities and strong Bézout}

In order to refine the statement of Bézout's theorem, we need to introduce the intersection multiplicity of curves $ C, D \subseteq \P^2 $ at a point $ p \in \P^2 $. This will encode both how singular $ p $ is as a point of $ C $ and $ D $ and the relative position or tangency of $ C $ and $ D $ at $ p $.

\begin{notation}
Let $ f \in \C\sb{x_0, \dots, x_n} $ and let $ \alpha \in \N^{n + 1} $ be a multi-index. Recall that $ \abs{\alpha} = \alpha_0 + \dots + \alpha_n $. Denote
$$ \partial^\alpha f = f_{x_0^{\alpha_0} \dots x_n^{\alpha_n}} = \dfrac{\partial^{\abs{\alpha}} f}{\partial x_0^{\alpha_0} \dots \partial x_n^{\alpha_n}}, $$
and recall that for all $ \alpha $, $ \partial^\alpha f $ is a polynomial, and that $ \partial^\alpha f \equiv 0 $ whenever all its coefficients are zero. Set $ \partial^{\rb{0, \dots, 0}} f = f $.
\end{notation}

\begin{example}
Let $ n = 1 $. $ \partial^{\rb{1, 1}} f = f_{x_1, x_0} = f_{x_0, x_1} $.
\end{example}

\begin{example}
Let $ f = x_0^2x_1 + x_0x_1x_2 $.
\begin{enumerate}
\item $ \alpha = \rb{0, 0, 0} $ gives $ \partial^\alpha f = f = x_0^2x_1 + x_0x_1x_2 $.
\item $ \alpha = \rb{1, 0, 0} $ gives $ \partial^\alpha f = f_{x_0} = 2x_0x_1 + x_1x_2 $.
\item $ \alpha = \rb{0, 1, 1} $ gives $ \partial^\alpha f = f_{x_1, x_2} = x_0 $.
\end{enumerate}
\end{example}

Let $ f\rb{x} \in \C\sb{x} $ and $ \alpha \in \C $. Multiplicity of $ f $ in $ \alpha $ is $ \max\cb{d \mid \rb{x - \alpha}^d \mid f\rb{x}} $. Let
$$ f\rb{x} = a \cdot \rb{x - \lambda_1}^{a_1} \dots \rb{x - \lambda_k}^{a_k}. $$
If $ \alpha = \lambda_i $, the multiplicity is $ a_i $. Multiplicity is zero if $ \alpha $ is not a root, that is $ f\rb{\alpha} \ne 0 $. Note that multiplicity of $ f $ in $ \alpha $ is $ k $ if and only if $ f\rb{\alpha} = \dots = f^{\rb{k - 1}}\rb{\alpha} = 0 $ and $ f^{\rb{k}}\rb{\alpha} \ne 0 $.

\begin{example}
If multiplicity is one, then $ f\rb{\alpha} = 0 $ and $ f'\rb{\alpha} \ne 0 $.
\end{example}

\begin{definition}
Let $ f \in \C\sb{x_0, \dots, x_n} $ be a polynomial and $ p = \rb{a_0, \dots, a_n} \in \C^{n + 1} $ be a point. The \textbf{multiplicity of $ f $ at $ p $} is
$$ mult_p\rb{f} = \min\cb{\abs{\alpha} \mid \partial^\alpha f\rb{p} = \partial^\alpha f\rb{a_0, \dots, a_n} \ne 0}. $$
Equivalently,
$$ mult_p\rb{f} = \max\cb{k \in \N \mid \forall \abs{\alpha} < k, \ \partial^\alpha f\rb{a_0, \dots, a_n} = 0}. $$
\end{definition}

\marginpar{Lecture 15 \\ Friday \\ 09/11/18}

In other words, this is the order of vanishing of $ f $ at $ p $.

\begin{example}
Multiplicity of $ f = x_0^2x_1 + x_0x_1x_2 $ at $ p = \rb{0, 0, 0} $.
\begin{enumerate}
\item $ \abs{\alpha} = 0 $ gives $ \partial^{\rb{0, 0, 0}} f\rb{p} = f\rb{p} = 0 $.
\item $ \abs{\alpha} = 1 $ gives
\begin{enumerate}
\item $ \partial^{\rb{1, 0, 0}} f\rb{p} = f_{x_0}\rb{p} = \rb{2x_0x_1 + x_1x_2}\rb{0, 0, 0} = 0 $,
\item $ \partial^{\rb{0, 1, 0}} f\rb{p} = f_{x_1}\rb{p} = \rb{x_0^2 + x_0x_2}\rb{0, 0, 0} = 0 $, and
\item $ \partial^{\rb{0, 0, 1}} f\rb{p} = f_{x_2}\rb{p} = \rb{x_0x_1}\rb{0, 0, 0} = 0 $.
\end{enumerate}
\item $ \abs{\alpha} = 2 $ gives $ \partial^\alpha = 0 $ for all $ \alpha $ with $ \abs{\alpha} = 2 $.
\item But $ \partial^{\rb{2, 1, 0}} f\rb{p} = f_{x_0^2x_1}\rb{p} = 2 $.
\end{enumerate}
So multiplicity of $ f $ at $ p $ is three.
\end{example}

Let $ C = \cb{P = 0} \subseteq \P^2 $ be a projective curve defined by a homogeneous polynomial $ P \in \C\sb{x_0, x_1, x_2} $ with no repeated factors.

\begin{definition}
The multiplicity of $ C $ at $ p = \sb{a, b, c} \in \P^2 $ is
$$ mult_p\rb{C} = mult_{\rb{a, b, c}}\rb{P}. $$
\end{definition}

Want to check $ mult_{\rb{\lambda a, \lambda b, \lambda c}}\rb{P} = mult_{\rb{a, b, c}}\rb{P} $. This is true as
$$ \partial^\alpha P\rb{\lambda a, \lambda b, \lambda c} = \lambda^{\deg\rb{P} - \abs{\alpha}}\partial^\alpha P\rb{a, b, c}. $$
Left hand side is zero if and only if $ \partial^\alpha P\rb{a, b, c} $ is zero.
\begin{enumerate}
\item $ mult_p\rb{C} = 0 $ if and only if $ P\rb{p} = \partial^{\rb{0, 0, 0}} P\rb{p} \ne 0 $, that is if and only if $ p \notin C $.
\item $ mult_p\rb{C} = 1 $ if and only if $ P\rb{p} = \partial^{\rb{0, 0, 0}} P\rb{p} = 0 $ and there is $ \alpha $ with $ \abs{\alpha} = 1 $ such that $ \partial^{\alpha} P\rb{p} \ne 0 $. Since the only indices $ \alpha $ with $ \abs{\alpha} = 1 $ are $ \rb{1, 0, 0} $, $ \rb{0, 1, 0} $, and $ \rb{0, 0, 1} $, this happens if and only if at least one of $ P_{x_0}\rb{p} $, $ P_{x_1}\rb{p} $, or $ P_{x_2}\rb{p} $ is non-zero, if and only if $ p $ is not a singular point of $ C $. Thus, $ mult_p\rb{C} = 1 $ if and only if $ p $ is a smooth point of $ C $.
\item Similarly $ mult_p\rb{C} \ge 2 $ if and only if $ p $ is a singular point of $ C $.
\item $ mult_p\rb{C} \le \deg\rb{C} $. Why? Suppose $ \deg\rb{C} = n $. Pick a monomial $ x_{i_1}, \dots, x_{i_n} $ that appears with non-zero coefficient in $ P $. Then $ P_{x_{i_1} \dots x_{i_n}}\rb{p} \ne 0 $ for any $ p \in \P^2 $.
\end{enumerate}

\begin{example}
The curve with equation $ P\rb{x_0, x_1, x_2} = x_0^2x_2 - x_1^2\rb{x_1 + x_2} $ has multiplicity two at $ \sb{0, 0, 1} $. Indeed, $ P_{x_0} = 2x_0x_2 $, $ P_{x_1} = -3x_1^2 - 2x_1x_2 $, $ P_{x_2} = x_0^2 - x_1^2 $ all vanish at $ \rb{0, 0, 1} $, and $ P_{x_0, x_0} = 2x_2 $ does not vanish at $ \rb{0, 0, 1} $.
\end{example}

\begin{exercise}
\label{ex:32}
Let $ \overline{C} = \cb{P = 0} $ be a projective curve that does not contain the line $ \cb{x_2 = 0} $. Denote by $ f\rb{x, y} = P\rb{x, y, 1} $ and let $ C = \cb{f = 0} $ be the associated affine curve. Let $ \rb{a, b} \in C $. Then
$$ mult_{\sb{a, b, 1}}\rb{\overline{C}} = mult_{\rb{a, b}}\rb{f}. $$
\end{exercise}

\begin{remark}
\label{rem:11.4}
A consequence of Exercise \ref{ex:32} is that $ mult_p\rb{C} \le \deg\rb{C} $ for all $ p \in C $. Indeed, if $ f \in \C\sb{x, y} $ is a polynomial of degree $ d $, it is clear that $ mult_{\rb{0, 0}}\rb{f} \le d $, so that $ mult_{\sb{0, 0, 1}}\rb{C} \le d $. For any $ p \in \P^2 $, let $ f : \P^2 \to \P^2 $ be a projective transformation with $ f\rb{p} = \sb{0, 0, 1} $. Then $ mult_p\rb{C} = mult_{\sb{0, 0, 1}}\rb{C} $, and the result follows because $ f\rb{C} $ is a curve of degree $ \deg\rb{C} $.
\end{remark}

Another way to think about multiplicity of $ f \in \C\sb{x_0, \dots, x_n} $ at $ p \in \C^{n + 1} $. It is the number of factors of the first non-zero term of the Taylor expansion of $ f $ at $ p $,
$$ mult_p\rb{f} = \sum_{\alpha \in \N^{n + 1}} \dfrac{\partial^\alpha f}{\prod_{i = 0}^n \alpha_i!}\rb{p}\prod_{i = 0}^n \rb{x_i - p_i}^{\alpha_i}. $$
If $ p = \rb{0, \dots, 0} $, then this coincides with $ f $.

\begin{lemma}
\label{lem:11.5}
Let $ P \in \C\sb{x_0, x_1} $ be a homogeneous polynomial of degree $ d > 0 $, and denote $ \cb{P = 0} = \cb{p_1, \dots, p_k} \subseteq \P^1 $. Then, $ \sum_{i = 1}^k mult_{p_i}\rb{P} = d $, and, in particular, $ k \le d $.
\end{lemma}

\begin{example}
Let $ P = x_0^2 $ then $ \cb{P = 0} = \cb{\sb{0, 1}} $, then $ mult_{\sb{0, 1}}\rb{x_0^2} = 2 $. If $ a \ne 0 $ $ mult_{\sb{a, b}}\rb{x_0^2} = 0 $.
\end{example}

This is essentially the statement that a degree $ d $ polynomial in one variable has $ d $ roots when counted with multiplicity.

\begin{proof}
As in the proof of Lemma \ref{lem:6.2}, we may write
$$ P\rb{x_0, x_1} = \lambda x_0^{d - e} \cdot \prod_{i \in I} \rb{x_1 - a_ix_0}^{d_i}, $$
where $ \lambda \ne 0 $, $ d_i, e \in \N $ satisfy $ \sum_{i \in I} d_i = e $, and $ a_i \in \C $ are such that $ a_i \ne a_j $ and $ a_i \ne 0 $ for all $ i \ne j \in I $. Note that in the expression of $ P $, $ e $ is not necessarily distinct from zero, in which case $ I = \emptyset $ and the product is empty, or from $ d $. Let us denote
$$ \cb{p_1, \dots, p_k} = \begin{cases} \cb{\sb{1, a_i}}_{i \in I} \cup \cb{\sb{0, 1}} & 0 < d < e \\ \cb{\sb{0, 1}} & e = 0 \\ \cb{\sb{1, a_i}}_{i \in I} & e = d \end{cases}. $$
It is easy to check from the definition of multiplicity that, for each $ i $,
$$ mult_{\sb{1, a_i}}\rb{P} = d_i, \qquad mult_{\sb{0, 1}}\rb{P} = d - e, $$
and the claim follows from $ \sum_{i \in I} d_i = e $.
\end{proof}

\begin{exercise}
\label{ex:33}
Multiplicity behaves well under projective transformations. Assume that $ C = \cb{P = 0} $ is a projective curve and that $ \chi $ is a projective transformation. Show that
$$ mult_p\rb{C} = mult_{\chi\rb{p}}\rb{\chi\rb{C}}. $$
A hint is that partial derivatives are linear maps.
\end{exercise}

\marginpar{Lecture 16 \\ Monday \\ 12/11/18}

Multiplicity is the measure of the singularity of $ C $ at $ p $. Using multiplicities of curves, one can already improve Theorem \ref{thm:9.8}. We will not prove the following, but rather prove directly the strong version of Bézout's theorem, which will have the following as a corollary.

\begin{theorem}
\label{thm:11.6}
Let $ C_1 $ and $ C_2 $ be two plane projective curves with no common components of degree $ n $ and $ m $ respectively. Let $ C_1 \cap C_2 = \cb{p_1, \dots, p_k} $. Then
$$ \sum_{i = 1}^k mult_{p_i}\rb{C_1} \cdot mult_{p_i}\rb{C_2} \le n \cdot m. $$
In particular, it follows that $ k \le n \cdot m $, as in Theorem \ref{thm:9.8}.
\end{theorem}

Using multiplicities like this is still not enough to attain equality.

\begin{example}
For a line tangent to a conic at a point, the multiplicity of both curves at the intersection point is one, but the product of the degrees is two.
\end{example}

We have to define a multiplicity of intersection of the two curves, that on top of their singularities also keeps track of how they interact. To also detect tangencies, have to look at multiplicity of resultant. Let us start by proving the following.

\begin{lemma}
\label{lem:11.7}
Let $ P, Q \in \C\sb{x_0, x_1, x_2} $ be homogeneous polynomials such that
$$ P\rb{1, 0, 0}, \ Q\rb{1, 0, 0} \ne 0, $$
$ \deg\rb{P} = n $, and $ \deg\rb{Q} = m $. Let $ p = \sb{z_0, z_1, z_2} \in \P^2 $ be such that $ p \ne \sb{1, 0, 0} $ and let
$$ r = mult_p\rb{P}, \qquad s = mult_p\rb{Q}. $$
Then if $ R_{P, Q}\rb{x_1, x_2} = \det\rb{R\rb{x_1, x_2}} $ is the resultant of $ P $ and $ Q $, we have
$$ mult_{\sb{z_1, z_2}}\rb{R_{P, Q}} \ge r \cdot s. $$
\end{lemma}

\begin{proof}
After applying a projective transformation we may assume that $ p = \sb{0, 0, 1} $ by Exercise \ref{ex:33}. Note that $ mult_{\sb{0, 1}}\rb{R\rb{x_1, x_2}} = mult_0\rb{R\rb{y, 1}} $ since $ R $ is homogeneous. We may write
$$ P = \sum_{i = 0}^n a_i\rb{x_1, x_2}x_0^i, \qquad Q = \sum_{i = 0}^m b_i\rb{x_1, x_2}x_0^i, $$
where $ a_i $ and $ b_i $ are homogeneous polynomials. Let
$$ f\rb{x, y} = P\rb{x, y, 1} = \sum_{i = 0}^n a_i\rb{y, 1}x^i = \sum_{i = 0}^n \widetilde{a_i}\rb{y}x^i, $$
where $ \widetilde{a_i}\rb{y} = a_i\rb{y, 1} $ and similarly let
$$ g\rb{x, y} = Q\rb{x, y, 1} = \sum_{i = 0}^m b_i\rb{y, 1}x^i = \sum_{i = 0}^m \widetilde{b_i}\rb{y}x^i, $$
where $ \widetilde{b_i}\rb{y} = b_i\rb{y, 1} $. Then
$$ mult_{\rb{0, 0}}\rb{f} = r, \qquad mult_{\rb{0, 0}}\rb{g} = s. $$
In particular, since the minimum degree of any monomial that appears on $ f $ with non-zero coefficient is $ r $, it follows that
$$ i < r \qquad \implies \qquad mult_0\rb{\widetilde{a_i}\rb{y}} \ge r - i, $$
that is
$$ i < r \qquad \implies \qquad \widetilde{a_i}\rb{y} = y^{r - i} \cdot A_i\rb{y}, $$
and similarly
$$ i < s \qquad \implies \qquad \widetilde{b_i}\rb{y} = y^{s - i} \cdot B_i\rb{y}, $$
for some polynomial $ A_i, B_i $ in $ y $. Note that by Remark \ref{rem:11.4}, we have $ r \le n $ and $ s \le m $. By Definition \ref{def:9.5}, it follows that
$$ R\rb{y, 1} = \begin{pmatrix} y^r \cdot A_0\rb{y} & \dots & y^0 \cdot A_r\rb{y} & \dots & A_n\rb{y} & & 0 \\ & \ddots & & \ddots & & \ddots & \\ 0 & & y^r \cdot A_0\rb{y} & \dots & y^0 \cdot A_r\rb{y} & \dots & A_n\rb{y} \\ y^s \cdot B_0\rb{y} & \dots & y^0 \cdot B_s\rb{y} & \dots & B_m\rb{y} & & 0 \\ & \ddots & & \ddots & & \ddots & \\ 0 & & y^s \cdot B_0\rb{y} & \dots & y^0 \cdot B_s\rb{y} & \dots & B_m\rb{y} \end{pmatrix}. $$
Want to show $ mult_0\rb{\det\rb{R\rb{y, 1}}} \ge r \cdot s $. Let $ R_1\rb{y} $ be the matrix obtained by multiplying the $ i $-th row of $ R\rb{y, 1} $ by $ y^{s - i + 1} $ for any $ i \le s $ and multiplying its $ \rb{m + j} $-th row by $ y^{r - j + 1} $ for any $ j \le r $. Then we obtain
$$ R_1\rb{y} = \begin{pmatrix} y^{r + s} \cdot A_0\rb{y} & \dots & y^s \cdot A_r\rb{y} & \dots & y^s \cdot A_n\rb{y} & & 0 \\ & \ddots & & \ddots & & \ddots & \\ 0 & & y^{r + 1} \cdot A_0\rb{y} & \dots & y^1 \cdot A_r\rb{y} & \dots & y^1 \cdot A_n\rb{y} \\ y^{r + s} \cdot B_0\rb{y} & \dots & y^r \cdot B_s\rb{y} & \dots & y^r \cdot B_m\rb{y} & & 0 \\ & \ddots & & \ddots & & \ddots & \\ 0 & & y^{s + 1} \cdot B_0\rb{y} & \dots & y^1 \cdot B_s\rb{y} & \dots & y^1 \cdot B_m\rb{y} \end{pmatrix}. $$
Now let $ R_2\rb{y} $ be the matrix obtained by dividing the $ i $-th column of $ R_1\rb{y} $ by $ y^{r + s + 1 - i} $ for any $ i \le r + s $. Then we obtain
$$ R_2\rb{y} = \begin{pmatrix} A_0\rb{y} & \dots & A_r\rb{y} & \dots & y^s \cdot A_n\rb{y} & & 0 \\ & \ddots & & \ddots & & \ddots & \\ 0 & & A_0\rb{y} & \dots & A_r\rb{y} & \dots & y \cdot A_n\rb{y} \\ B_0\rb{y} & \dots & B_s\rb{y} & \dots & y^r \cdot B_m\rb{y} & & 0 \\ & \ddots & & \ddots & & \ddots & \\ 0 & & B_0\rb{y} & \dots & B_s\rb{y} & \dots & y \cdot B_m\rb{y} \end{pmatrix}. $$
In particular $ \det\rb{R_2} $ is a polynomial in $ y $ and it follows from linear algebra that
\begin{align*}
\det\rb{R\rb{y, 1}}
& = \det\rb{R_1\rb{y}} \cdot y^{-s} \cdot \dots \cdot y^{-1} \cdot y^{-r} \cdot \dots \cdot y^{-1} \\
& = \det\rb{R_1\rb{y}} \cdot y^{-\tfrac{s\rb{s + 1}+r\rb{r + 1}}{2}} \\
& = \det\rb{R_2\rb{y}} \cdot y^{r + s} \cdot \dots \cdot y^1 \cdot y^{-\tfrac{s\rb{s + 1} + r\rb{r + 1}}{2}} \\
& = \det\rb{R_2\rb{y}} \cdot y^{\tfrac{\rb{r + s + 1}\rb{r + s}}{2} - \tfrac{s\rb{s + 1}+r\rb{r + 1}}{2}} \\
& = \det\rb{R_2\rb{y}} \cdot y^{rs}.
\end{align*}
Since $ \det\rb{R_2\rb{y}} $ is a polynomial, it follows that
$$ mult_{\sb{0, 1}}\rb{R_{P, Q}} = mult_0\rb{\det\rb{R\rb{y, 1}}} \ge r \cdot s, $$
and the claim follows.
\end{proof}

Inspired by Lemma \ref{lem:11.7} above, we can now define the intersection number of two curves at a given point.

\begin{definition}
\label{def:11.8}
Let $ C = \cb{P = 0} $ and $ D = \cb{Q = 0} $ in $ \P^2 $ be projective curves defined by homogeneous polynomials $ P, Q \in \C\sb{x_0, x_1, x_2} $ with no repeated factor, and assume that $ \sb{1, 0, 0} \notin C \cup D $.
\begin{enumerate}
\item Assume that $ C $ and $ D $ have no common component, and let $ \cb{p_1, \dots, p_k} = C \cap D $, with $ \sb{a_i, b_i, c_i} = p_i $. Assume that $ \sb{1, 0, 0} $ lies on none of the lines through $ p_i $ and $ p_j $, where $ 1 \le i, j \le k $. The \textbf{intersection multiplicity} or \textbf{intersection number} of $ C $ and $ D $ at $ p = \sb{a, b, c} \in \P^2 $ is
$$ I\rb{p, C, D} = I\rb{p, P, Q} = \begin{cases} mult_{\sb{b, c}}\rb{R_{P, Q}\rb{x_1, x_2}} & p \in C \cap D \\ 0 & p \notin C \cap D \end{cases}. $$
\item If $ E $ is a common component of $ C $ and $ D $, set $ I\rb{p, C, D} = \infty $ for all $ p \in E $.
\end{enumerate}
\end{definition}

\begin{remark}
As in the proof of Theorem \ref{thm:9.8}, the assumption on $ \sb{1, 0, 0} $ guarantees that if $ p_i = \sb{a_i, b_i, c_i} \in C \cap D $, then $ \rb{b_i, c_i} \ne \rb{0, 0} $, so that $ \sb{b_i, c_i} \in \P^1 $, and that $ p_i $ is the only point of $ C \cap D $ with coordinates $ \sb{\lambda, b_i, c_i} $ for some $ \lambda \in \C $. In other words, it guarantees that $ \cb{\sb{b_i, c_i} \mid 1 \le i \le k} $ is a set of $ k $ distinct points of $ \P^1 $.
\end{remark}

Recall that given $ p $, $ C $, and $ D $, we may assume that the hypotheses of Definition \ref{def:11.8} are satisfied after a projective transformation $ \Psi : \P^2 \to \P^2 $.

\begin{exercise}
Let $ \Psi : \P^2 \to \P^2 $ be a projective transformation, and let $ C, D \subseteq \P^2 $ be projective curves and $ p \in \P^2 $. Assume that $ p \ne \sb{1, 0, 0} $, $ \Psi\rb{p} \ne \sb{1, 0, 0} $, and that $ \sb{1, 0, 0} \notin C \cup D \cup \Psi\rb{C} \cup \Psi\rb{D} $. Show that
$$ I\rb{p, C, D} = I\rb{\Psi\rb{p}, \Psi\rb{C}, \Psi\rb{D}}. $$
\end{exercise}

\begin{remark}
\label{rem:11.10}
Assume that $ C $ and $ D $ have no common component, and let $ \cb{p_i = \sb{a_i, b_i, c_i} \mid 1 \le i \le k} = C \cap D $. Then, by Theorem \ref{thm:9.3}, $ R_{P, Q}\rb{b_i, c_i} = 0 $ for all $ i = 1, \dots, k $, so that $ I\rb{p_i, C, D} > 0 $. This shows that $ I\rb{p, C, D} \ne 0 $ if and only if $ p \in C \cap D $.
\end{remark}

\begin{theorem}[Strong Bézout theorem]
\label{thm:11.11}
Let $ C, D \subseteq \P^2 $ be projective curves of degrees $ n $ and $ m $ that have no common component. Let $ C \cap D = \cb{p_i = \sb{a_i, b_i, c_i} \mid 1 \le i \le k} $. Then
$$ \sum_{i = 1}^k I\rb{p_i, C, D} = n \cdot m. $$
\end{theorem}

\begin{proof}
After projective transformation, we may assume that $ \sb{1, 0, 0} $ is contained in none of the lines $ L_{i, j} $ through $ p_i $ and $ p_j $ where $ 1 \le i < j \le k $. Let $ R\rb{x_1, x_2} $ be the resultant matrix of $ P $ and $ Q $. As is recalled in Remark \ref{rem:11.10}, the $ k $ points $ \sb{b_i, c_i} \in \P^1 $ are distinct and by definition
$$ I\rb{p_i, C, D} = mult_{\sb{b_i, c_i}}\rb{\det\rb{R\rb{x_1, x_2}}}. $$
By Theorem \ref{thm:9.6}, $ \det\rb{R\rb{x_1, x_2}} $ is a homogeneous polynomial of degree $ n \cdot m $, so that the claim follows from Lemma \ref{lem:11.5}.
\end{proof}

\marginpar{Lecture 17 \\ Thursday \\ 15/11/18}

\begin{example}
Consider the curves $ C_1 $ and $ C_2 $ defined by
$$ C_1 = \cb{x_0x_2 - x_1^2 = 0}, \qquad C_2 = \cb{x_0x_2 + x_1^2 = 0}. $$
We want to compute $ I\rb{p, C_1, C_2} $ for $ p = \sb{0, 0, 1} $. As $ \sb{1, 0, 0} \in C_1 \cap C_2 $, we first need to change variables. We consider the projective transformation
$$ \Psi\rb{\sb{x_0, x_1, x_2}} = \sb{x_1, x_0, x_2}, $$
so that
$$ \Psi\rb{C_1} = \cb{x_1x_2 - x_0^2 = 0}, \qquad \Psi\rb{C_2} = \cb{x_1x_2 + x_0^2 = 0}, $$
and $ \Psi\rb{C_1} \cap \Psi\rb{C_2} = \cb{\sb{0, 1, 0}, \sb{0, 0, 1}} $. Note that $ \Psi\rb{\sb{0, 0, 1}} = \sb{0, 0, 1} $. The resultant of $ \Psi\rb{C_1} $, $ \Psi\rb{C_2} $ is
$$ R\rb{x_1, x_2} = \det\begin{pmatrix} x_1x_2 & 0 & -1 & 0 \\ 0 & x_1x_2 & 0 & -1 \\ x_1x_2 & 0 & 1 & 0 \\ 0 & x_1x_2 & 0 & 1 \end{pmatrix} = 4x_1^2x_2^2. $$
Thus
$$ I\rb{\sb{0, 0, 1}, C_1, C_2} = I\rb{\sb{0, 0, 1}, \Psi\rb{C_1}, \Psi\rb{C_2}} = mult_{\sb{0, 1}}\rb{R\rb{x_1, x_2}} = 2, $$
and similarly, $ I\rb{\sb{1, 0, 0}, C_1, C_2} = 2 $. Bézout's theorem is satisfied in this case.
\end{example}

Now we can note that Theorem \ref{thm:11.6} is a corollary of the strong theorem.

\begin{proof}[Proof of Theorem \ref{thm:11.6}]
Follows directly from Theorem \ref{thm:11.11} and Lemma \ref{lem:11.7}.
\end{proof}

\begin{exercise}
Show that an irreducible projective curve $ C $ of degree $ d $ has at most
$$ \dfrac{d\rb{d - 1}}{2} $$
singular points.
\end{exercise}

\begin{exercise}
Let $ C $ be an irreducible projective curve of degree $ d $ with a point $ p \in C $ with multiplicity $ q = mult_p\rb{C} $. Show that there exists a line $ L \subseteq \P_\C^2 $ through $ p \in L $ and which intersects the rest of $ C $ in exactly $ d - q + 1 $ points. In particular, for any projective curve $ C $ of degree $ d $ there exists a line which meets $ C $ in $ d $ points.
\end{exercise}

\section{More about intersection multiplicities}

We now give an alternative description of the resultant of two polynomials in one variable.

\begin{lemma}
\label{lem:12.1}
Let $ p, q \in \C\sb{x} $ be monic polynomials of degrees $ n $ and $ m $, so that there are $ \lambda_1, \dots, \lambda_n \in \C $ and $ \mu_1, \dots, \mu_m  \in \C $ such that
$$ p\rb{x} = \prod_{i = 1}^n \rb{x - \lambda_i}, \qquad q\rb{x} = \prod_{j = 1}^m \rb{x - \mu_j}. $$
Then, $ R_{p, q} $, the resultant of $ p $ and $ q $, is a homogeneous polynomial of degree $ nm $ in the variables $ \lambda_1, \dots, \lambda_n $ and $ \mu_1, \dots, \mu_m $. More precisely,
$$ R_{p, q} = \prod_{i, j}\rb{\lambda_i - \mu_j} = \rb{-1}^{nm}\prod_{j = 1}^m p\rb{\mu_j} = \prod_{i = 1}^n q\rb{\lambda_i}. $$
\end{lemma}

\begin{proof}
Write $ p\rb{x} = \sum_{i = 1}^n a_i\rb{\lambda_1, \dots, \lambda_n} \cdot x^i $ and $ q\rb{x} = \sum_{j = 1}^m b_j\rb{\mu_1, \dots, \mu_m} \cdot x^j $, where $ a_i $ is a homogeneous polynomial of degree $ n - i $ in the variables $ \lambda_1, \dots, \lambda_n $ and $ b_j $ is a homogeneous polynomial of degree $ m - j $ in the variables $ \mu_1, \dots, \mu_m $. The proof that $ R_{p, q} $ is a homogeneous polynomial of degree $ nm $ in $ \lambda_1, \dots, \lambda_n, \mu_1, \dots, \mu_m $ is identical to the proof of Theorem \ref{thm:9.7}. We now see the expression for $ p $ as a homogeneous polynomial of degree $ n $ in $ \C\sb{x, \lambda_1, \dots, \lambda_n} $ and $ q $ as a homogeneous polynomial of degree $ m $ in $ \C\sb{x, \mu_1, \dots, \mu_m} $. By Theorem \ref{thm:9.3}, for each $ i, j $, if some $ \lambda_i = \mu_j $, $ R_{p, q} = 0 $, so that, as polynomials in $ \lambda_1, \dots, \lambda_n, \mu_1, \dots, \mu_m $, we have $ \rb{\lambda_i - \mu_j} \mid R_{p, q} $, so
$$ R_{p, q} = \prod_{i, j} \rb{\lambda_i - \mu_j}S, $$
where $ S $ is a homogeneous polynomial in $ \lambda_1, \dots, \lambda_n, \mu_1, \dots, \mu_m $. $ S $ is homogeneous by Lemma \ref{lem:4.10}. Since $ \deg\rb{R_{p, q}} = nm $, the degree of $ S $ is zero and $ S $ is a constant in $ \C^* $ with
$$ R_{p, q} = S\prod_{i, j} \rb{\lambda_i - \mu_j}. $$
The constant $ S = 1 $, as can be checked by computing the resultant of $ p\rb{x} = \rb{x - 1}^n $ and $ q\rb{x} = x^m $.
\end{proof}

\begin{remark}
More generally, if the leading coefficients of $ p $ and $ q $ are $ a_n \ne 0 $ and $ b_m \ne 0 $ then $ R_{p, q} = a_n^mb_m^n\prod_{i, j} \rb{\lambda_i - \mu_j} $.
\end{remark}

\begin{remark}
\label{rem:12.3}
Let $ f_1, f_2, g \in \C\sb{x} $ be monic polynomials. An immediate consequence of Lemma \ref{lem:12.1} is that $ R_{f_1f_2, g} = R_{f_1, g} \cdot R_{f_2, g} $.
\end{remark}

\begin{lemma}
\label{lem:12.4}
Let $ P_1, P_2, Q \in \C\sb{x_0, x_1, x_2} $ be homogeneous polynomials with $ P_i\rb{1, 0, 0}, Q\rb{1, 0, 0} \ne 0 $, and let $ P = P_1 \cdot P_2 $. Let $ R\rb{x_1, x_2} $ be the resultant of $ P $ and $ Q $ and $ R_i\rb{x_1, x_2} $ the resultant of $ P_i $ and $ Q $, for $ i = 1, 2 $. For all $ \sb{b, c} \in \P^1 $, we have
$$ mult_{\sb{b, c}}\rb{R} = mult_{\sb{b, c}}\rb{R_1} + mult_{\sb{b, c}}\rb{R_2}. $$
As homogeneous polynomials, $ R\rb{x_1, x_2} = R_1\rb{x_1, x_2}R_2\rb{x_1, x_2} $.
\end{lemma}

\begin{proof}
$ P_i\rb{1, 0, 0} \ne 0 $ gives $ a \cdot x_0^{\deg\rb{P_i}} $ is a monomial of $ P_i $ for $ a \ne 0 $. Let $ a_1, a_2, b \in \C^* $ be constants such that $ P_1 = a_1P'_1 $, $ P_2 = a_2P'_2 $, and $ Q = bQ' $, where $ P'_1, P'_2, Q' $ are homogeneous polynomials that are monic in $ x_0 $. Then, if $ m = \deg\rb{Q} $, $ P' = P'_1P'_2 $ and $ a = a_1a_2 \in \C^* $, the resultants are scalar multiples that satisfy
$$ R_{P_1, Q} = a_1^mb^{\deg\rb{P_1}}R_{P'_1, Q}, \qquad R_{P_2, Q} = a_2^mb^{\deg\rb{P_2}}R_{P'_2, Q}, \qquad R_{P, Q} = a^mb^{\deg\rb{P}}R_{P', Q}. $$
In particular, for any $ \sb{b, c} \in \P^1 $, the multiplicities of $ R_{P_1, Q} $, $ R_{P_2, Q} $, $ R_{P, Q} $ and of $ R_{P'_1, Q'} $, $ R_{P'_2, Q'} $, $ R_{P', Q'} $ coincide. We may therefore assume that $ P_1 $, $ P_2 $, and $ Q $ are monic with respect to $ x_0 $. Fix $ \sb{b, c} \in \P^1 $ and let $ f_i\rb{x} = P_i\rb{x, b, c} $, $ f\rb{x} = f_1\rb{x} \cdot f_2\rb{x} $, and $ g\rb{x} = Q\rb{x, b, c} $. The polynomials $ f_1 $, $ f_2 $, $ f $, and $ g $ are monic. (TODO Exercise) By Lemma \ref{lem:12.1} and the consequence in Remark \ref{rem:12.3},
$$ R_{f, g} = R_{f_1, g} \cdot R_{f_2, g}, $$
that is we have
$$ R_{P, Q}\rb{b, c} = R_{P_1, Q}\rb{b, c} \cdot R_{P_2, Q}\rb{b, c}. $$
As this holds for every $ \sb{b, c} \in \P^1 $,
$$ R_{P, Q} =  R_{P_1, Q} \cdot R_{P_2, Q} $$
in $ \C\sb{x_1, x_2} $, and $ mult_{\sb{b, c}}\rb{R} = mult_{\sb{b, c}}\rb{R_1} + mult_{\sb{b, c}}\rb{R_2} $ for all $ \sb{b, c} \in \P^1 $.
\end{proof}

The following proposition gathers a few important properties of intersection multiplicities.

\begin{proposition}
\label{prop:12.5}
Let $ C, D \subseteq \P^2 $ be projective curves defined by homogeneous polynomials $ P, Q \in \C\sb{x_0, x_1, x_2} $. Let $ p \in \P^2 $ be a point. The intersection number $ I\rb{p, P, Q} $ satisfies the following.
\begin{enumerate}
\item Intersection numbers are symmetric, so $ I\rb{p, C, D} = I\rb{p, D, C} $.
\item $ I\rb{p, C, D} = \infty $ if $ p $ lies on a common component of $ C $ and $ D $ and $ I\rb{p, C, D} $ is an integer otherwise.
\item $ I\rb{p, C, D} \ne 0 $ if and only if $ p \in C \cap D $.
\item If $ C $ and $ D $ are distinct lines and if $ \cb{p} = C \cap D $ then $ I\rb{p, C, D} = 1 $.
\item If $ P = P_1 \cdot P_2 $, for homogeneous polynomials $ P_1 $ and $ P_2 $, then
$$ I\rb{p, P_1 \cdot P_2, Q} = I\rb{p, P_1, Q} + I\rb{p, P_2, Q}. $$
\item If $ P = P_1 \cdot Q + P_2 $, for homogeneous polynomials $ P_1 $ and $ P_2 $, then
$$ I\rb{p, P, Q} = I\rb{p, P_2, Q}. $$
\end{enumerate}
\end{proposition}

\begin{proof}
\hfill
\begin{enumerate}
\item Since the effect of exchanging the rows of a matrix on its determinant is to multiply it by $ \rb{-1} $,
$$ R_{P, Q} = \pm R_{Q, P}, $$
and since multiplication by a constant does not affect the multiplicity of a polynomial, $ 1 $ follows.
\item $ 2 $ is by definition and is part of Definition \ref{def:11.8}.
\item $ 3 $ is by construction and was noted in Remark \ref{rem:11.10}.
\item For $ 4 $, after projective transformation, we may assume that $ C = \cb{x_0 = 0} $, $ D = \cb{x_0 + x_1 = 0} $, so that $ \sb{1, 0, 0} \notin C \cup D $ and $ p = C \cap D = \sb{0, 0, 1} $. Then, $ I\rb{p, C, D} = 1 $ by an easy computation. (TODO Exercise)
\item $ 5 $ is an application of the statement about multiplicities of resultants in Lemma \ref{lem:12.4}.
\item We now prove $ 6 $. Let $ R\rb{x_1, x_2} $ be the resultant of $ P_1 \cdot Q + P_2 $ and $ Q $ and let $ R'\rb{x_1, x_2} $ be the resultant of $ P_2 $ and $ Q $. Then, denote by
$$ P_2\rb{x_0, x_1, x_2} = \sum_{i = 1}^n a_i\rb{x_1, x_2} \cdot x_0^i, \qquad P_1\rb{x_0, x_1, x_2} = \sum_{i = 1}^n c_i\rb{x_1, x_2} \cdot x_0^i, $$
$$ Q\rb{x_0, x_1, x_2} = \sum_{i = 1}^n b_i\rb{x_1, x_2} \cdot x_0^i. $$
Then
$$ P_1 \cdot Q + P_2 = \sum_{i = 1}^n \rb{a_i + \sum_{j = 0}^i c_j \cdot b_{i - j}} \cdot x_0^i, $$
so
$$ R\rb{x_1, x_2} = \det\begin{pmatrix} a_0 + b_0c_0 & \dots & a_n + \sum_{j = 0}^n c_jb_{i - j} & & 0 \\ & \ddots & & \ddots & \\ 0 & & a_0 + b_0c_0 & \dots & a_n + \sum_{j = 0}^n c_jb_{i - j} \\ b_0 & \dots & b_n & & 0 \\ & \ddots & & \ddots & \\ 0 & & b_0 & \dots & b_n \end{pmatrix}, $$
so that the resultant matrix of $ R $ is obtained from the resultant matrix of $ R' $ by adding $ c_{j - 1} $ times the $ m + j + i - 1 $ row to the $ i $ row, for each $ j, i = 1, \dots, n $. Since performing these row operations do not affect the determinant, $ R = R' $ and $ 6 $ follows.
\end{enumerate}
This finishes the proof of Proposition \ref{prop:12.5}.
\end{proof}

\begin{example}
Consider the curves
$$ C_1 = \cb{x_0x_2 - x_1^2 = 0}, \qquad C_2 = \cb{x_0x_2 + x_1^2 = 0}, $$
and let $ p = \sb{0, 0, 1} \in C_1 \cap C_2 $. Then
\begin{align*}
I\rb{p, C_1, C_2}
& = I\rb{p, x_0x_2 - x_1^2, x_0x_2 + x_1^2} & \text{by definition}, \\
& = I\rb{p, 2x_0x_2, x_0x_2 + x_1^2} & \text{by } 6, \\
& = I\rb{p, x_0, x_0x_2 + x_1^2} + I\rb{p, x_2, x_0x_2 + x_1^2} & \text{by } 5, \\
& = I\rb{p, x_0, x_0x_2 + x_1^2} & \text{by } 3, \\
& = I\rb{p, x_0, x_1^2} & \text{by } 6, \\
& = 2 \cdot I\rb{p, x_0, x_1} & \text{by } 5, \\
& = 2 & \text{by } 4.
\end{align*}
\end{example}

\marginpar{Lecture 18 \\ Friday \\ 16/11/18}

What about when $ D $ is a line $ L $?

\begin{exercise}
Let
$$ C = \cb{x_0x_2^2 - x_1\rb{x_1 - x_0}\rb{x_1 + x_0} = 0} \subseteq \P^2, $$
$ p = \sb{0 : 0 : 1} $, and let
$$ L = \cb{ax_0 + bx_2 = 0} \subseteq \P^2, $$
for some $ a, b \in \C $ not both zero. Compute $ I\rb{p, C, L} $.
\end{exercise}

\begin{proposition}
Let $ C \subseteq \P^2 $ be a projective curve and $ p \in C $ a smooth point. If $ L = T_p\rb{C} $ is the tangent line to $ C $ at $ p $, $ I\rb{p, C, T_p\rb{C}} > 1 $.
\end{proposition}

(TODO Exercise: prove the converse)

\begin{proof}
Let $ C = \cb{P = 0} $ and $ p = \sb{a, b, c} $. We may assume that $ \sb{1, 0, 0} \notin C \cup T_p\rb{C} $ so that $ P_{x_0}\rb{a, b, c} \ne 0 $. The tangent to $ C $ at $ p $ is
$$ T_p\rb{C} = \cb{P_{x_0}\rb{a, b, c} \cdot x_0 + P_{x_1}\rb{a, b, c} \cdot x_1 + P_{x_2}\rb{a, b, c} \cdot x_2 = 0}, $$
so
$$ x_0 = -\dfrac{P_{x_1}\rb{a, b, c}x_1 + P_{x_2}\rb{a, b, c}x_2}{P_{x_0}\rb{a, b, c}}, $$
and from applying Lemma \ref{lem:12.1} to
$$ p\rb{x} = P\rb{x, b', c'}, \qquad q\rb{x} = x + \dfrac{P_{x_1}\rb{a, b, c}b' + P_{x_2}\rb{a, b, c}c'}{P_{x_0}\rb{a, b, c}}, $$
it follows that the resultant $ R_{C, L} $ is equal to a scalar multiple of the polynomial in $ x_1, x_2 $
$$ Q\rb{x_1, x_2} = P\rb{-\dfrac{P_{x_1}\rb{a, b, c}x_1 + P_{x_2}\rb{a, b, c}x_2}{P_{x_0}\rb{a, b, c}}, x_1, x_2}. $$
If $ P $ is not monic in $ x_0 $, there will be a non-zero constant factor, irrelevant when taking multiplicity. The multiplicity $ I\rb{p, C, T_p\rb{C}} > 1 $ if and only if $ Q $ has a repeated factor $ \rb{cx_1 - bx_2} $ at $ p $. This holds because
$$ Q\rb{b, c} = P\rb{a, b, c} = 0, \qquad Q_{x_1}\rb{b, c} = P_{x_0}\rb{a, b, c}\rb{-\dfrac{P_{x_1}\rb{a, b, c}}{P_{x_0}\rb{a, b, c}}} + P_{x_1}\rb{a, b, c} + 0 = 0, $$
and same for $ Q_{x_2}\rb{b, c} = 0 $ by the chain rule, so that $ \sb{b, c} $ is a common root of $ Q $ and $ Q_{x_i} $. Claim that this implies that $ Q\rb{x_1, x_2} = \rb{cx_1 - bx_2}^2 \cdot R\rb{x_1, x_2} $ for some polynomial $ R $. This concludes the proof.
\end{proof}

\begin{exercise}
Show that the claim holds, so $ \rb{cx_1 - bx_2}^2 \mid Q\rb{x_1, x_2} $.
\end{exercise}

One can actually define intersection multiplicities of projective curves $ C, D \subseteq \P^2 $ by the properties listed in Proposition \ref{prop:12.5}. We will not prove this, see Theorem 3.18 in Kirwan's book if you are interested.

\begin{proposition}
\label{prop:12.8}
Properties $ 1 $ to $ 6 $ in Proposition \ref{prop:12.5} determine uniquely and characterise completely $ I\rb{p, C, D} $ for any point $ p \in \P^2 $ and any projective curves $ C, D \subseteq \P^2 $.
\end{proposition}

\begin{remark}
Proposition \ref{prop:12.8} is another proof that the intersection multiplicity is defined independent of the choice of coordinates on $ \P^2 $, that is that $ I\rb{p, C, D} = I\rb{\Psi\rb{p}, \Psi\rb{C}, \Psi\rb{D}} $ for any projective transformation $ \Psi : \P^2 \to \P^2 $.
\end{remark}

\begin{remark}
Another consequence is that the intersection multiplicity $ I\rb{p, C, D} $ depends only on the components of $ C $ and $ D $ that contain the point $ p $.
\end{remark}

The intersection of two distinct lines at their point of intersection is one. The next question we could ask is, when do curves C and D intersect at a point $ p \in C \cap D $ with multiplicity one? The following proposition, that we state without proving, gives the answer.

\begin{proposition}
\label{prop:12.11}
Let $ C, D \subseteq \P^2 $ be projective curves and $ p $ a point of $ \P^2 $. Then $ I\rb{p, C, D} = 1 $ if and only if $ p \in C \cap D $ is a smooth point of $ C $ and of $ D $ and if the tangent lines to $ C $ and $ D $ at $ p $ are distinct.
\end{proposition}

$ C $ and $ D $ meet \textbf{transversely} at $ p $.

\begin{remark}
By Proposition \ref{prop:12.5}(2), (3), $ I\rb{p, C, D} $ is a non-zero finite number if and only if $ p \in C \cap D $ and $ p $ does not lie on a common component of $ C $ and $ D $, and by Lemma \ref{lem:11.7}, $ 1 \le mult_p\rb{C} \cdot mult_p\rb{D} \le I\rb{p, C, D} = 1 $ implies that $ mult_p\rb{C} = mult_p\rb{D} = 1 $, so that $ p $ is a smooth point of $ C $ and of $ D $. We would therefore have to show that if $ p \in C \cap D $ is a smooth point of $ C $ and of $ D $, $ I\rb{p, C, D} = 1 $ if and only if the tangent lines $ L_C $ and $ L_D $ of $ C $ and $ D $ at $ p $ are distinct.
\end{remark}

Proposition \ref{prop:12.11} implies immediately the following corollary.

\begin{corollary}
Let $ C, D \subseteq \P^2 $ be two projective curves with no common component, $ n = \deg\rb{C} $, and $ m = \deg\rb{D} $. Then for any $ p \in C \cap D $, $ C $ and $ D $ are smooth at $ p $ and have distinct tangent lines at $ p $ if and only if
$$ \#\cb{C \cap D} = n \cdot m. $$
\end{corollary}

\begin{proposition}
\label{prop:12.14}
Let $ L $ be a projective line and $ C \subseteq \P^2 $ a curve of degree $ d $. If $ p \in C \cap L $, $ I\rb{p, C, L} > 1 $ if and only if either $ C $ is singular at $ p $ or if $ L $ is the tangent line to $ C $ at $ p $. More generally, if $ C $ is singular at $ p $, then $ I\rb{p, C, L} > mult_p\rb{C} $ if and only if $ L $ is one of the higher tangent lines to $ C $ at $ p $.
\end{proposition}

\begin{proof}
See Exercise 6 of problem sheet 2.
\end{proof}

\begin{example}
Let $ y^2 = x^2\rb{x + 1} = x^3 + x^2 $. Lowest degree form is $ y^2 = x^2 $, so higher tangent lines at origin are $ y \pm x $.
\end{example}

\begin{example}
Let $ y^2 = x^3 $. Lowest degree form is $ y^2 = 0 $, so higher tangent lines at origin are $ y^2 = 0 $.
\end{example}

\begin{exercise}
Let $ C = \cb{x_0x_2^2 - x_1\rb{x_1 - x_0}\rb{x_1 + x_0} = 0} $, $ L = \cb{ax_0 + bx_1 = 0} $ for $ \rb{a, b} \ne \rb{0, 0} $ and $ p = \sb{0, 0, 1} $. Compute $ I\rb{p, C, L} $.
\end{exercise}

\section{Cubic curves}

In this section, we investigate the geometry of cubic curves. Up to projective transformation,
\begin{enumerate}
\item a projective line is $ L = \cb{x_0 = 0} $, and
\item a smooth irreducible conic is $ C = \cb{x_0^2 + x_1^2 + x_2^2 = 0} \subset \P^2 $.
\end{enumerate}
The equation of a smooth line or conic can be brought in a uniquely determined \textbf{standard form}. What about cubics? By contrast, we will see that the equation of a smooth cubic curve can be brought in standard form, but that standard form depends on a parameter. We prove the following.

\begin{theorem}
\label{thm:13.1}
Let $ C \subseteq \P^2 $ be a smooth projective cubic curve of degree three. Then, there exists a projective transformation $ \Psi : \P^2 \to \P^2 $ and $ \lambda \in \C \setminus \cb{0, 1} $ such that
$$ \Psi\rb{C} = \cb{x_1^2x_2 = x_0\rb{x_0 - x_2}\rb{x_0 - \lambda x_2}} \subset \P^2. $$
\end{theorem}

Dehomogenising with respect to $ x_2 $, $ y^2 = x\rb{x - 1}\rb{x - \lambda} $ is the Legendre form for elliptic curves. In order to prove Theorem \ref{thm:13.1}, we need to define inflection points of projective curves. These will generalise the points of inflection on graphs of functions.

\begin{definition}
\label{def:13.2}
Let $ P \in \C\sb{x_0, x_1, x_2} $ be a homogeneous polynomial. The \textbf{Hessian matrix} of $ P $ is the symmetric matrix whose entries are the second order differentials of $ P $
$$ H_P = \rb{\dfrac{\partial^2 P}{\partial x_i \partial x_j}}_{0 \le i, j \le 2} = \begin{pmatrix} P_{x_0, x_0} & P_{x_0, x_1} & P_{x_0, x_2} \\ P_{x_1, x_0} & P_{x_1, x_1} & P_{x_1, x_2} \\ P_{x_2, x_0} & P_{x_2, x_1} & P_{x_2, x_2} \end{pmatrix}. $$
The \textbf{Hessian} of $ P $ is $ \mathcal{H}_P\rb{x_0, x_1, x_2} = \det\rb{H_P} $. $ \mathcal{H}_P $ is a homogeneous polynomial of degree $ 3\rb{d - 2} $ when $ d \ge 3 $. An \textbf{inflection point} or \textbf{flex point} of $ C $ is a smooth point $ p = \sb{a, b, c} $ of $ C $ such that $ \mathcal{H}_P\rb{a, b, c} = 0 $.
\end{definition}

The only thing there is to check is that $ \mathcal{H}_P $ is a homogeneous polynomial of the specified degree. When $ d \ge 3 $, all non-zero entries of $ H_P $ are homogeneous polynomials of degree $ d - 2 $, and $ H_P $ is a $ 3 \times 3 $ matrix, so $ \mathcal{H}_P $ is indeed a homogeneous polynomial of degree $ 3\rb{d - 2} $. (TODO Exercise: an inflection point $ p $ is a smooth point $ p \in C $ whose intersection multiplicity with the tangent line to $ C $ at $ p $ is at least three)

\begin{lemma}
\label{lem:13.3}
Let $ P \in \C\sb{x_0, x_1, x_2} $ be a homogeneous polynomial of degree $ d > 1 $. Then
\begin{equation}
\label{eq:7}
x_2^2 \cdot \mathcal{H}_P = \rb{d - 1}^2 \cdot \det\begin{pmatrix} P_{x_0, x_0} & P_{x_0, x_1} & P_{x_0} \\ P_{x_1, x_0} & P_{x_1, x_1} & P_{x_1} \\ P_{x_0} & P_{x_1} & \tfrac{d}{d - 1} \cdot P \end{pmatrix}.
\end{equation}
\end{lemma}

\begin{remark}
Analogous formulae to $ \rb{\ref{eq:7}} $ can be stated for $ x_0 $, $ x_1 $.
\end{remark}

\marginpar{Lecture 19 \\ Monday \\ 19/11/18}

\begin{proof}
For ease of notation, let us label the rows and columns of the Hessian matrix as follows.
$$ H_P = \three{C_0}{C_1}{C_2} = \begin{pmatrix} R_0 \\ R_1 \\ R_2 \end{pmatrix}. $$
Each $ P_{x_i} $ is a homogeneous polynomial of degree $ d - 1 $, so that by the Euler relation Theorem \ref{thm:7.3}, we have
$$ \rb{d - 1}P_{x_i} = x_0P_{x_0, x_i} + x_1P_{x_1, x_i} + x_2P_{x_2, x_i}, $$
and
$$ x_2 \cdot \mathcal{H}_P = \det\three{C_0}{C_1}{x_0C_0 + x_1C_1 + x_2C_2} = \det\begin{pmatrix} P_{x_0, x_0} & P_{x_0, x_1} & \rb{d - 1}P_{x_0} \\ P_{x_1, x_0} & P_{x_1, x_1} & \rb{d - 1}P_{x_1} \\ P_{x_2, x_0} & P_{x_2, x_1} & \rb{d - 1}P_{x_2} \end{pmatrix}, $$
that is
$$ x_2 \cdot \mathcal{H}_P = \rb{d - 1}\det\begin{pmatrix} P_{x_0, x_0} & P_{x_0, x_1} & P_{x_0} \\ P_{x_1, x_0} & P_{x_1, x_1} & P_{x_1} \\ P_{x_2, x_0} & P_{x_2, x_1} & P_{x_2} \end{pmatrix} = \begin{pmatrix} R_0' \\ R_1' \\ R_2' \end{pmatrix}. $$
Similarly,
$$ x_2^2 \cdot \mathcal{H}_P = \rb{d - 1}\det\begin{pmatrix} R_0' \\ R_1' \\ x_0R_0' + x_1R_1' + x_2R_2' \end{pmatrix}. $$
Using the Euler relation for $ P_{x_0} $ and $ P_{x_1} $ and the Euler relation for $ P $
$$ dP = x_0P_{x_0} + x_1P_{x_1} + x_2P_{x_2}, $$
we find that this last matrix has the desired form.
\end{proof}

Note that for a line every point is an inflection point, because the Hessian matrix is zero.

\begin{lemma}
\label{lem:13.5}
Let $ C $ be a smooth projective curve of degree $ d $. If $ d \ge 3 $, $ C $ has at least one point of inflection.
\end{lemma}

\begin{proof}
Let $ C $ be a projective curve of degree $ d \ge 2 $. If $ d = 2 $, $ \mathcal{H}_P $ is a constant polynomial, and it is non-zero, as one can check directly, using the standard form for a conic that we have seen before. Therefore, $ C $ has no point of inflection, which agrees with Lemma \ref{lem:13.5}. We now assume that $ d > 2 $, so that either $ \mathcal{H}_P $ is a homogeneous polynomial of degree $ \deg\rb{\mathcal{H}_P} = 3\rb{d - 2} = 3 $ by Definition \ref{def:13.2}. Then it follows from the weak version of Bézout's Theorem \ref{thm:9.8} that there is at least a point in $ C \cap \cb{\mathcal{H}_P = 0} $, which is then an inflection point, since $ C $ is smooth.
\end{proof}

\begin{remark}
It also true that if $ d \ge 2 $, then $ \deg\rb{C} \cdot \deg\rb{\mathcal{H}_P} = 3d\rb{d - 2} $ so $ C $ has at most $ 3d\rb{d - 2} $ points of inflection. Note that this follows from the strong form Bézout's theorem, once we know that $ C $ and $ \cb{\mathcal{H}_P = 0} $ have no common component. Unfortunately this does not immediately follow from irreducibility of $ C $, because $ 3\rb{d - 2} > d $ as soon as $ d \ge 4 $, and $ \cb{\mathcal{H}_P = 0} $ could well be reducible, or have multiple components, for that matter. The way around this is to prove that if every smooth point of an irreducible curve is an inflection point, which would happen if $ C $ were a component of $ \cb{\mathcal{H}_P = 0} $, then $ C $ has to be a line, in Lemma 3.32 in Kirwan's book. (TODO Exercise: if $ \cb{\mathcal{H}_P = 0} $ and $ C $ have a common component, then $ C $ is a line)
\end{remark}

We now see that the presence of inflection points guarantees that the equation of every smooth cubic curve can be put in a very simple form.

\begin{proof}[Proof of Theorem \ref{thm:13.1}]
Let $ P $ be the irreducible homogeneous polynomial of degree three such that $ C = \cb{P = 0} $. By Lemma \ref{lem:13.5}, $ C $ has at least one inflection point. Up to projective transformation, we may thus assume that $ q = \sb{0, 1, 0} $ is an inflection point of $ C $, and that the tangent line to $ C $ at $ q $ is $ T_q\rb{C} = \cb{x_2 = 0} $. This implies that
$$ P\rb{0, 1, 0} = 0, \qquad P_{x_0}\rb{0, 1, 0} = P_{x_1}\rb{0, 1, 0} = 0, \qquad P_{x_2}\rb{0, 1, 0} \ne 0, \qquad \mathcal{H}_P\rb{0, 1, 0} = 0. $$
We may write the equation of $ P $ as
$$ P\rb{x_0, x_1, x_2} = Ax_1^3 + Bx_0x_1^2 + Cx_2x_1^2 + Dx_0^2x_1 + Ex_0x_2x_1 + Fx_2^2x_1 + \Phi\rb{x_0, x_2}, $$
where $ A, B, C, D, E, F \in \C $ and $ \Phi $ is a homogeneous polynomial of degree three in the variables $ x_0, x_2 $. Since $ P\rb{0, 1, 0} = 0 $, $ A = 0 $. Similarly, since $ P_{x_0}\rb{0, 1, 0} = 0 $, $ B = 0 $, and since $ P_{x_2}\rb{0, 1, 0} \ne 0 $, $ C \ne 0 $. Now we would like to say something about the other coefficients. This will involve some manipulation of $ \mathcal{H}_P $. By the analogous result of Lemma \ref{lem:13.3}, we have
$$ x_1^2 \cdot \mathcal{H}_P = \rb{d - 1}^2 \cdot \det\begin{pmatrix} P_{x_0, x_0} & P_{x_0} & P_{x_0, x_2} \\ P_{x_0} & \tfrac{d}{d - 1} \cdot P & P_{x_2} \\ P_{x_0, x_2} & P_{x_2} & P_{x_2, x_2} \end{pmatrix} = 4\det\begin{pmatrix} P_{x_0, x_0} & P_{x_0} & P_{x_0, x_2} \\ P_{x_0} & \tfrac{3}{2} \cdot P & P_{x_2} \\ P_{x_0, x_2} & P_{x_2} & P_{x_2, x_2} \end{pmatrix}, $$
so that
$$ 0 = \mathcal{H}_P\rb{0, 1, 0} = 4\det\begin{pmatrix} P_{x_0, x_0} & 0 & P_{x_0, x_2} \\ 0 & 0 & P_{x_2} \\ P_{x_0, x_2} & P_{x_2} & P_{x_2, x_2} \end{pmatrix} = -4\rb{P_{x_2}\rb{0, 1, 0}}^2P_{x_0, x_0}\rb{0, 1, 0}, $$
so that $ P_{x_0, x_0}\rb{0, 1, 0} = 0 $ since $ P_{x_2}\rb{0, 1, 0} $ is non-zero. Since $ P_{x_0, x_0}\rb{0, 1, 0} = 0 $, $ D = 0 $. We may thus write
\begin{align*}
P\rb{x_0, x_1, x_2}
& = x_1x_2\rb{Ex_0 + Cx_1 + Fx_2} + \Phi\rb{x_0, x_2} \\
& = C\rb{x_1^2 + \rb{\dfrac{Ex_0 + Fx_2}{C}}x_1}x_2 + \Phi\rb{x_0, x_2} \\
& = C\rb{x_1 + \dfrac{Ex_0 + Fx_2}{2C}}^2x_2 + \Phi'\rb{x_0, x_2},
\end{align*}
where $ \Phi' \in \C\sb{x_0, x_2} $ is a homogeneous polynomial of degree three. Consider the projective transformation
$$ \Psi_1 : \sb{x_0, x_1, x_2} \mapsto \sb{x_0, x_1 + \dfrac{Ex_0 + Fx_2}{2C}, x_2}. $$
Note that $ \Psi_1 $ is well-defined because $ C \ne 0 $. Then I get that the equation of $ \Psi_1\rb{C} $ is
$$ \Psi_1\rb{C} = \cb{x_1^2x_2 = \Phi''\rb{x_0, x_2}} \subset \P^2, $$
where $ \Phi'' \in \C\sb{x_0, x_2} $ is a homogeneous polynomial of degree three. By Lemma \ref{lem:6.2}, $ \Phi'' $ is a product of linear factors. Since $ C $ and hence $ \Psi_1\rb{C} $ are irreducible, $ x_2 $ does not divide $ \Phi'' $. Thus $ \Phi''\rb{x_0, x_2} = K\rb{x_0 - ax_2}\rb{x_0 - bx_2}\rb{x_0 - cx_2} $ for $ K \in \C^* $, $ a, b, c \in \C $. After a suitable diagonal projective transformation $ \Psi_2 $, $ K = 1 $ and the equation of $ \Psi_2 \circ \Psi_1\rb{C} $ is
$$ \Psi_2 \circ \Psi_1\rb{C} = \cb{x_1^2x_2 = \rb{x_0 - ax_2}\rb{x_0 - bx_2}\rb{x_0 - cx_2}} \subset \P^2. $$
Since $ C $ is non-singular, $ a $, $ b $, and $ c $ are distinct. (TODO Exercise: check) The map
$$ \Psi_3 : \sb{x_0, x_1, x_2} \mapsto \sb{\dfrac{x_0 - ax_2}{b - a}, \eta x_1, x_2}, $$
for $ 1 / \eta^2 = \rb{b - a}^3 $ is thus a well-defined projective transformation and
$$ \Psi_3 \circ \Psi_2 \circ \Psi_1\rb{C} = \cb{x_1^2x_2 = x_0\rb{x_0 - x_2}\rb{x_0 - \lambda x_2}} \subset \P^2, $$
for $ \lambda \in \C $. Note that $ \lambda \ne 0, 1 $ because $ C $ is non-singular. This is exactly the form we wanted.
\end{proof}

\marginpar{Lecture 20 \\ Thursday \\ 22/11/18}

Lecture 20 is a problem class.

\marginpar{Lecture 21 \\ Friday \\ 23/11/18}

\begin{corollary}
Let $ C \subseteq \P^2 $ be a smooth cubic curve. Then $ C $ has precisely nine points of inflection.
\end{corollary}

\begin{proof}
Let $ C = \cb{P = 0} $, where $ P $ is a homogeneous polynomial of degree three. Define $ D = \cb{\mathcal{H}_P = 0} $ be the curve defined by the Hessian of $ P $. Recall that $ \mathcal{H}_P $ may have repeated factors. Note that $ C $ and $ D $ do not have common components. Since $ C $ is irreducible, they would have to coincide, by Theorem \ref{thm:13.1} we can assume that $ C $ has equation $ x_1^2x_2 = x_0\rb{x_0 - x_2}\rb{x_0 - \lambda x_2} $, and it is easy to check that $ \sb{0, 0, 1} $ is not an inflection point of this particular cubic. (TODO Exercise) By Bézout's Theorem \ref{thm:11.11}, we then have
$$ 9 = \deg\rb{C} \cdot \deg\rb{D} = \sum_{p \in C \cap D} I\rb{p, C, D}. $$
It is thus enough to prove that for each inflection point $ p \in C \cap D $, $ I\rb{p, C, D} = 1 $. By Proposition \ref{prop:12.11}, this is equivalent to proving that $ p $ is a smooth point of $ C $ and $ D $ and that $ T_p\rb{C} \ne T_p\rb{D} $. Let $ p \in C \cap D $, then by Theorem \ref{thm:13.1}, up to projective transformation, we may assume that $ p = \sb{0, 1, 0} $, and that the equation of $ C $ is
$$ \cb{x_1^2x_2 = x_0\rb{x_0 - x_2}\rb{x_0 - \lambda x_2}}, $$
for $ \lambda \in \C \setminus \cb{0, 1} $. Then $ T_p\rb{C} = \cb{x_2 = 0} $, while
$$ \partial_{x_0}\mathcal{H}_P\rb{0, 1, 0} = 24, \qquad \partial_{x_1}\mathcal{H}_P\rb{0, 1, 0} = 0, \qquad \partial_{x_2}\mathcal{H}_P\rb{0, 1, 0} = 8\rb{\lambda - 1}, $$
so that $ p $ is a smooth point of $ D $ and $ T_p\rb{D} \ne T_p\rb{C} $. This finishes the proof.
\end{proof}

\section{Linear systems}

Before moving on to Riemann surfaces, we briefly turn our attention to the way curves behave in families. Here are two basic questions.
\begin{enumerate}
\item Given two irreducible curves $ C = \cb{P = 0} $ and $ D = \cb{Q = 0} $ of the same degree, we can consider a family of curves $ C_{\lambda, \mu} = \cb{\lambda P + \mu Q = 0} $ parametrised by $ \sb{\lambda, \mu} \in \P^1 $. The curves $ C $ and $ D $ are special members of the family. How do properties of $ C_{\sb{\lambda, \mu}} $ relate to properties of $ C $ and $ D $?
\item Let $ p_1, \dots, p_k $ be points in $ \P^2 $. When can we find a curve of degree $ d $ that passes through $ p_1, \dots, p_k $? When can we find a curve $ C $ of degree $ d $ with $ mult_{p_i}\rb{C} = q_i $ for a collection $ q_1, \dots, q_k \in \N $?
\end{enumerate}

\begin{example}
Let $ d = 1 $, $ k = 2 $, and $ q_1 = q_2 = 1 $. Then there is only one projective line. Let $ d = 1 $, $ k = 2 $, and $ q_1 = q_2 = q_3 = 1 $. Then it depends on whether $ p_1, p_2, p_3 $ are collinear or not.
\end{example}

\begin{example}
Let us consider projective lines in $ \P^2 $. Recall that the equation of every line $ L $ is given by
$$ L = \cb{ax_0 + bx_1 + cx_2 = 0} \subseteq \P^2, $$
where $ \rb{a, b, c} \in \C^3 $, and $ \rb{a, b, c} \ne \rb{0, 0, 0} $. Then $ \rb{a, b, c} $ and $ \rb{a', b', c'} $ define the same line if and only if there exists $ \lambda \in \C^* $ such that $ \rb{a, b, c} = \rb{\lambda a', \lambda b', \lambda c'} $, that is when $ \sb{a, b, c} = \sb{a', b', c'} $ as points of $ \P^2 $. This shows that
$$ \cb{L \subseteq \P^2} \cong \P^2. $$
Let $ p \in \P^2 $ be a point. Then the space of lines passing through $ p = \sb{z_0, z_1, z_2} $ for $ z_i \in \C $ is given by the points $ \sb{a, b, c} $ with $ a \cdot z_0 + b \cdot z_1 + c \cdot z_2 = 0 $, which are lines $ \P^1 $ defined by equations
$$ \cb{ax_0 + bx_1 + cx_2 = 0} \subseteq \P^2 $$
in $ \P^2 $. We may assume that $ p = \sb{0, 0, 1} $ after projective transformation. Then the set of lines $ L \subseteq \P^2 $ that contain $ p $ is the set of lines such that $ a \cdot 0 + b \cdot 0 + c \cdot 1 = c = 0 $. In other words,
$$ \cb{L \subseteq \P^2 \mid p \in L} \cong \cb{\sb{a, b, 0} \in \P^2} \cong \P^1. $$
We have seen that given two points $ p \ne q \in \P^2 $ there is a unique line $ L_{p, q} $ through $ p $ and $ q $. This shows that passage through $ q $ gives a single line
$$ \cb{L \subseteq \P^2 \mid p, q \in L} = \cb{L_{p, q}} \cong \P^0. $$
\end{example}

We can parametrise projective curves of degree $ d $ in a similar way. More precisely, we have seen that any curve of degree $ d $ $ C = \cb{P = 0} $ is defined by a homogeneous polynomial $ P \in \C\sb{x_0, x_1, x_2} $ of degree $ d $ with no repeated factors. Write
$$ P\rb{x_0, x_1, x_2} = \sum_{\rb{i, j, k} \in \N^3} a_{i, j, k}x_0^ix_1^jx_2^k $$
where the only non-zero coefficients $ a_{i, j, k} $ correspond to multi-indices $ \rb{i, j, k} $ with $ i + j + k = d $. The set $ I_d = \cb{\rb{i, j, k} \in \N^3 \mid i + j + k = d} $ has precisely $ \rb{d + 1}\rb{d + 2} / 2 $ elements, since the number of $ \rb{i, j} $ such that $ i + j \le e \le d $ is $ e + 1 $ and the number of $ \rb{i, j} $ such that $ i + j \le d $ is $ 1 + \dots + \rb{d + 1} $. We may order the triples $ \rb{i, j, k} \in I $ by lexicographic order. Recall from Remark \ref{rem:4.12} that if $ P, Q $ are two homogeneous polynomials with no repeated factors, $ C = \cb{P = 0} = \cb{Q = 0} $ precisely when $ P = \lambda Q $ for some $ \lambda \in \C^* $. This shows that there is a well-defined map
$$ \Psi_d : \cb{C = \cb{P = 0} \mid \deg\rb{C} = d} \mapsto \sb{a_{i, j, k}} = \sb{a_{0, 0, d}, \dots, a_{d, 0, 0}} \in \P^{N_d}, $$
where $ N_d = \rb{d + 1}\rb{d + 2} / 2 - 1 = d\rb{d + 3} / 2 $.

\begin{example}
$ d = 1 $ is $ N_1 = 1 \cdot 4 / 2 = 2 $, $ d = 2 $ is $ N_2 = 2 \cdot 5 / 2 = 5 $, and $ d = 3 $ is $ N_3 = 3 \cdot 6 / 2 = 9 $.
\end{example}

The map $ \Psi_d $ is not surjective when $ d > 1 $. In $ \P^{N_d} $, there are points that correspond to $ P $ with repeated factors.

\begin{example}
Indeed, the point $ \sb{1, 0, \dots, 0} \in \P^{N_d} $ corresponds to $ P\rb{x_0, x_1, x_2} = x_0^d $, so that it defines a line $ L = \cb{x_0 = 0} $ with multiplicity $ d $.
\end{example}

We will include this case in our description by counting that curve component with multiplicity $ d $.

\begin{definition}
Let $ \mathcal{L}_d $ denote the set of curves $ C \subseteq \P^2 $ defined by a homogeneous polynomial of degree $ d $, possibly with repeated factors. Then $ \Psi_d $ defines a bijection
$$ \Psi_d : \mathcal{L}_d \cong \P^{N_d}, \qquad \sum_{\rb{i, j, k} \in \N^3} a_{i, j, k}x_0^ix_1^jx_2^k \mapsto \sb{a_{i, j, k}}, \qquad N_d = \dfrac{d\rb{d + 3}}{2}. $$
\end{definition}

\begin{example}
We have seen that $ \mathcal{L}_1 \cong \P^2 $. Similarly, $ \mathcal{L}_2 \cong \P^5 $, and $ \mathcal{L}_2 $ contains smooth conics and a subspace $ F \subseteq \mathcal{L}_2 $ of fake conics that are defined by polynomials with repeated factors, so double lines in $ \P^2 $. The subset $ F $ parametrises lines counted with multiplicity two, $ F \cong \mathcal{L}_1 \cong \P^2 $. Last, $ \mathcal{L}_3 \cong \P^9 $, and $ \mathcal{L}_3 $ contains a subspace $ F_1 \cong \P^2 $ parametrising lines counted with multiplicity three and a subspace $ F_2 $ parametrising the union of a line counted with multiplicity two and a line counted with multiplicity one, $ F_2 \cong \P^2 \times \P^2 $.
\end{example}

\begin{remark}
We have proved in previous lectures that up to a projective transformation, there are only three kinds of conics. Here the space of conics $ \mathcal{L}_2 $ on the other hand has infinitely many elements, but in this space we are also remembering the way that the conic sits inside $ \P^2 $ via its equation. What relates the two different points of view, is that the group of projective transformations $ \Psi : \P^2 \to \P^2 $, that is usually denoted by $ PGL\rb{3, \C} $, acts on the space $ \mathcal{L}_2 $, and there are exactly three orbits for the action, described by the three conics $ x_0^2 = 0 $, $ x_0^2 + x_1^2 = 0 $, $ x_0^2 + x_1^2 + x_2^2 = 0 $, mentioned after Exercise \ref{ex:30}. For cubics something similar happens. $ PGL\rb{3, \C} $ acts on the space $ \mathcal{L}_3 $, but this time there are infinitely many orbits. The equation of Theorem \ref{thm:13.1} gives one such cubic for every $ \lambda \in \C \setminus \cb{1, 0} $, and although it is not true that these are all non-isomorphic, for every $ \lambda $ there is a finite number of $ \lambda' $, at most six, for which the two curves are isomorphic, that is one equation can be brought to the other via a projective transformation, so there is still an infinite number of non-isomorphic smooth cubics.
\end{remark}

\begin{lemma}
\label{lem:14.5}
Let $ d, q \in \N $. Let $ p \in \P^2 $, then
$$ \mathcal{S} = \cb{C \in \mathcal{L}_d \mid mult_p\rb{C} \ge q} \cong \P^{N_{d, q}}, $$
where $ N_{d, q} = d\rb{d + 3} / 2 - q\rb{q + 1} / 2. $
\end{lemma}

\begin{proof}
We will show that $ \Psi_d\rb{\mathcal{S}} \cong \P^{N_{d, q}} $, where as above, $ \Psi_d $ is the map
$$ \Psi_d : C = \cb{\sum_{\rb{i, j, k} \in I} a_{i, j, k}x_0^ix_1^jx_2^k = 0} \mapsto \sb{a_{i, j, k}} \in \P^{N_d}. $$
We denote by $ \sb{C} = \Psi_d\rb{C} $ for each $ C \in \mathcal{L}_d $. We show that $ C \in \mathcal{S} $ if and only if $ \sb{C} \in \P^{N_d} $ is in the subspace of solutions of a linear system of $ q\rb{q + 1} / 2 $ independent equations, that is a linear system defined by a matrix of size $ q\rb{q + 1} / 2 \times N_d $ of rank $ q\rb{q + 1} / 2 $. After projective transformation, we may assume that $ p = \sb{0, 0, 1} $. Let $ C \in \mathcal{L}_d $, and denote by $ P $ a homogeneous polynomial of degree $ d $ with $ C = \cb{P = 0} $. Let $ f\rb{x, y} = P\rb{x, y, 1} =  \sum_{\rb{i, j, k} \in I} a_{i, j, k}x^iy^j $. Then
$$ mult_p\rb{P} = mult_{\rb{0, 0}}\rb{f} \ge q $$
if and only if $ a_{i, j, k} = 0 $ for all $ \rb{i, j, k} \in I $ with $ i + j < q $. Denote by $ J \subseteq \cb{0, \dots, N_d} $ the set of indices in lexicographic order that correspond to the subset $ \cb{\rb{i, j, k} \in I \mid i + j < q} $ of $ I $. In $ \sb{a_{i, j, k}} $, a bunch of these are zero. As there are $ q\rb{q + 1} / 2 $ triples $ \rb{i, j, k} $ with $ i + j + k = d $ and $ i + j < q $, (TODO Exercise) the ones that are left are the coordinates of the $ \P^{N_{d, q}} $,
$$ \Psi_d\rb{\mathcal{S}} \cong \P^{N_{d, q}} = \cb{\sb{C} = \sb{C_0, \dots, C_{N_d}} \in \P^{N_d} \mid \forall j \in J, \ C_j = 0}. $$
\end{proof}

\begin{example}
Let $ \sb{x_1, x_2, x_3, x_4} \in \P^3 $. $ x_0 = 0 $ and $ x_1 = 0 $ gives $ \sb{0, 0, x_2, x_3} \mapsto \sb{x_2, x_3} \in \P^1 $ since $ 1 = 3 - 2 $.
\end{example}

\begin{example}
Let $ p \in \P^2 $, and consider the space
$$ \mathcal{S} = \cb{C \in \mathcal{L}_2 \mid mult_p\rb{C} \ge 2}. $$
Then, by Lemma \ref{lem:14.5}, $ \mathcal{S} \cong \P^2 $. Recall that an irreducible conic is always smooth by Corollary \ref{cor:10.4}. Therefore, the curves $ C \in \mathcal{S} $ parametrise degenerate conics of the form $ C = L \cup L' $, where $ L, L' $ are projective lines containing $ p $. The lines $ L, L' $ need not be distinct, as the points of $ \mathcal{L}_2 $ also parametrise double lines. It is easy to check by direct methods (TODO Exercise) that
$$ \cb{\rb{L, L'} \in \mathcal{L}_1 \mid p \in L \cap L'} \cong \P^2. $$
\end{example}

\marginpar{Lecture 22 \\ Monday \\ 26/11/18}

\begin{definition}
\label{def:14.7}
Let $ p_1, \dots, p_k $ be distinct points of $ \P^2 $, and fix $ d, q_1, \dots, q_k \in \Z_{> 0} $. Then
$$ \mathcal{S} = \cb{C \in \mathcal{L}_d \mid \forall i = 1, \dots, k, \ mult_{p_i}\rb{C} \ge q_i} $$
is the \textbf{linear system} of curves of degree $ d $ going through the points $ p_i $ with multiplicity at least $ q_i $ for $ i = 1, \dots, k $.
\end{definition}

This is a copy of $ \P^N $ inside $ \mathcal{L}_d $. Every condition on $ p_i $ gives some linear equations in the coefficients of a polynomial defining $ C $, but these equations might not be independent.

\begin{theorem}
\label{thm:14.8}
With the notation of Definition \ref{def:14.7},
$$ \mathcal{S} \cong \P^N, \qquad N \ge \dfrac{d\rb{d + 3}}{2} - \sum_{i = 1}^k \dfrac{q_i\rb{q_i + 1}}{2}. $$
The number $ d\rb{d + 3} / 2 - \sum_{i = 1}^k q_i\rb{q_i + 1} / 2 $ is the expected dimension of $ \mathcal{S} $, which is what you get if equations are independent, while $ N $ is its actual dimension.
\end{theorem}

\begin{proof}
TODO Exercise: prove this precisely.
\end{proof}

\begin{corollary}
Let $ p_1, \dots, p_k $ be distinct points of $ \P^2 $ and $ d, q_1, \dots, q_k \in \rb{\N^*}^{k + 1} $. If
$$ \dfrac{d\rb{d + 3}}{2} \ge \sum_{i = 1}^k \dfrac{q_i\rb{q_i + 1}}{2}, $$
there exists a curve $ C \in \mathcal{L}_d $ that passes through $ p_1, \dots, p_k $ with the assigned multiplicities $ q_1, \dots, q_k $.
\end{corollary}

\begin{proof}
This is an immediate consequence of Theorem \ref{thm:14.8}.
\end{proof}

\begin{example}
Let $ k = 1 $ and $ q_1 = 2 $. $ d\rb{d + 3} / 2 \ge 2 \cdot 3 / 2 $, so $ d\rb{d + 3} \ge 6 $. Thus $ d \ge 2 $.
\end{example}

Linear systems of dimension one have a special name.

\begin{definition}
A \textbf{pencil of curves} of degree $ d $ is a family of plane curves
$$ C_{\sb{\lambda_0, \lambda_1}} = \cb{\lambda_0 \cdot P_1 + \lambda_1 \cdot P_2 = 0} \subseteq \P^2, $$
where $ P_1 $ and $ P_2 $ are homogeneous polynomials of degree $ d $ with no common factor of $ C_1 = \cb{P_1 = 0} $ and $ C_2 = \cb{P_2 = 0} $ in $ \mathcal{S} $, and $ \sb{\lambda_0, \lambda_1} \in \P^1 $. In other words $ \cb{C_{\sb{\lambda_0, \lambda_1}}}_{\sb{\lambda_0, \lambda_1}} \in \P^1 $ is a family of curves of degree $ d $ of dimension one, that is parametrised by $ \P^1 $. If $ \mathcal{S} \cong \P^1 $ is a linear system of curves of degree $ d $, it defines naturally a pencil of curves of degree $ d $.
\end{definition}

\begin{exercise}
\label{ex:40}
Let $ p_1, p_2, p_3, p_4 \in \P^2 $ be four non-collinear points. Prove that
$$ \mathcal{S} = \cb{C \in \mathcal{L}_2 \mid p_1, \dots, p_4 \in C} \cong \P^1. $$
\end{exercise}

\begin{example}
By Exercise \ref{ex:40}, $ \mathcal{S} $ defines a pencil of conics, the pencil of conics through $ p_1, \dots, p_4 $. Let $ C_1, C_2 \in \mathcal{S} $ be distinct conics in $ \mathcal{S} $ and assume that $ C_1 = \cb{P_1 = 0} $ and $ C_2 = \cb{P_2 = 0} $ for homogeneous polynomials $ P_1 $ and $ P_2 $ of degree two. Then every other conic is $ \lambda \cdot P_1 + \mu \cdot P_2 = 0 $. How many reducible conics are there in $ \mathcal{S} $? Since $ p_1, \dots, p_4 $ are non-collinear, neither $ C_1 $ nor $ C_2 $ is a line with multiplicity two, and $ P_1, P_2 $ are polynomials with no repeated factor. Further $ P_1, P_2 $ have no common factor. The pencil of conics through $ p_1, \dots, p_4 $ parametrises the family of curves
$$ C_{\sb{\lambda, \mu}} = \cb{\lambda \cdot P_1 + \mu \cdot P_2 = 0} \subseteq \P^2. $$
Recall that $ C_{\sb{\lambda, \mu}} $ is smooth if and only if it is irreducible, that is if and only if
$$ F\rb{\lambda, \mu} = \det\rb{M_{\sb{\lambda, \mu}}} = \det\rb{\lambda \cdot M_{P_1} + \mu \cdot M_{P_2}} \ne 0, $$
where $ M_{\sb{\lambda, \mu}} $, $ M_{P_1} $, and $ M_{P_2} $ are the matrices associated to $ C_{\sb{\lambda, \mu}} $, $ C_1 $, and $ C_2 $ as in Exercise \ref{ex:31}. The polynomial $ F \in \C\sb{\lambda, \mu} $ is homogeneous of degree three in $ \lambda, \mu $, so that it is either identically zero, and one can prove that this does not happen, or it has at least one and at most three roots by Lemma \ref{lem:6.2}, as the geometry says. So the pencil of conics through $ p_1, \dots, p_4 $ has at least one and at most three reducible elements.
\end{example}

\begin{proposition}
Let $ p_1, \dots, p_8 \in \P^2 $ be eight distinct points and suppose that no four of the points lie on a line and no seven on a conic. Then
$$ \mathcal{S} = \cb{C \in \mathcal{L}_3 \mid p_1, \dots, p_8 \in C} \cong \P^1. $$
\end{proposition}

\begin{corollary}
Let $ C_1, C_2 $ be two cubic curves whose intersection consists of nine distinct points
$$ C_1 \cap C_2 = \cb{p_1, \dots, p_9}. $$
Then any cubic curve $ D \subseteq \P^3 $ that contains $ p_1, \dots, p_8 $ passes through $ p_9 $.
\end{corollary}

\begin{proof}
One proves that $ p_1, \dots, p_8 $ satisfy the assumptions of the previous theorem, so that cubics through these points form a pencil. Then if $ P_1, P_2 $ are homogeneous polynomials of degree three with $ C_1 = \cb{P_1 = 0} $ and $ C_2 = \cb{P_2 = 0} $, $ P_1, P_2 $ form a basis of the two-dimensional vector space of homogeneous polynomials defining a curve in $ \mathcal{S} $. In other words,
$$ \mathcal{S} = \cb{P_{\sb{\lambda, \mu}} = \cb{\lambda \cdot P_1 + \mu \cdot P_2 = 0} \mid \sb{\lambda, \mu} \in \P^1}. $$
It follows that if $ p_9 \in C_1 \cap C_2 $, $ P_1\rb{p_9} = P_2\rb{p_9} = 0 $, then $ P_{\sb{\lambda, \mu}}\rb{p_9} = 0 $ and $ p_9 \in P $ for every $ P \in \mathcal{S} $.
\end{proof}

\section{Riemann surfaces}

In this section, we will see that a smooth projective curve has a single topological invariant that characterises its topology, its genus, and we will introduce the formalism of Riemann surfaces to study this genus. We have seen that a projective line $ \P^1 $ is homeomorphic to the sphere $ S^2 \subseteq \R^3 $. We also saw that stereographic projection defined a homeomorphism (TODO Exercise: check that it is in fact a diffeomorphism) between any smooth conic $ C $ and $ \P^1 $, by a bijection between $ C $ and lines through $ p_0 $, or a bijection to any line $ L_0 $ not containing $ p_0 $, so that a smooth conic is also homeomorphic to a sphere $ S^2 \subseteq \R^3 $. If $ C \subseteq \P^2 $ is a smooth projective plane curve of degree $ d \ge 3 $, can do a similar projection. We will see that $ C $ is a homeomorphic to a sphere with $ g $ handles, where $ g $ is the genus of $ C $ and can be determined in terms of $ d $. In these cases, stereographic projection turns out to be a bit more complicated than for smooth conics. We now look at the case $ d = 3 $, where the study of stereographic projection gives us a good grasp of the topology of a smooth cubic. Stereographic projection of $ C $ with respect to $ p_0 \in C $ is defined as follows. Fix a smooth point $ p_0 \in C $, recall that $ \mathcal{S} = \cb{L \in \mathcal{L}_1 \mid p_0 \in L} \cong \P^1 $, and we can identify $ \mathcal{S} $ with any line $ L_0 \subseteq \P^2 $ that does not contain $ p_0 $ by the bijection $ L \in \mathcal{S} \mapsto L \cap L_0 \in L_0 $. Then, the stereographic projection is the map
$$ \pi : p \in C \mapsto L_{p, p_0} \cap L_0, $$
where $ L_{p, p_0} $ is the unique line through $ p $ and $ p_0 $ if $ p \ne p_0 $ or $ T_{p_0}\rb{C} $ if $ p = p_0 $, which is not one to one anymore. Equivalently, $ L_{p, p_0} $ is the line through $ p_0 $ and $ \pi\rb{p} $. By Proposition \ref{prop:12.14} and Bézout's Theorem \ref{thm:11.11}, $ \pi^{-1}\rb{\pi\rb{p}} \subseteq C $ consists of $ d - 1 $ points unless $ L_{p, p_0} $ is tangent to $ C $ at some point in $ \pi^{-1}\rb{p} $. $ \pi $ is a $ d - 1 $ to one covering space of $ \P^1 $.

\end{document}